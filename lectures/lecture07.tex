\chapter{Jan.~28 --- Kummer's Theorem}

\section{Kummer's Theorem}

\begin{lemma}
  Let $\theta \in \OO_K$ and assume
  $p \nmid [\OO_K : \Z[\theta]]$. Then
  \[
    \OO_K / p \OO_K \cong \Z[\theta] / p \Z[\theta].
  \]
\end{lemma}

\begin{proof}
  Consider the map $\psi: \Z[\theta] \hookrightarrow \OO_K \twoheadrightarrow \OO_K / p \OO_K$.
  Note that we have $p\Z[\theta] \subseteq \ker \psi$,
  so $\psi$ induces a map $\overline{\psi}: \Z[\theta] / p \Z[\theta] \to \OO_K / p \OO_K$
  on the quotient. We will show that
  $\overline{\psi}$ is an isomorphism, by checking:
  \begin{enumerate}
    \item $\ker \psi = p \Z[\theta]$.

  Let $\alpha \in \ker \psi$.
  Then $\alpha \in \Z[\theta] \cap p \OO_K$, so
  $\alpha = p \beta$ for some $\beta \in \OO_K$.
  Then $\overline{\beta} \in \OO_K / \Z[\theta]$
  has order dividing $p$ since $p \overline{\beta} = \overline{\alpha} = 0$.
  Therefore $\overline{\beta} = 0$, so
  $\beta \in \Z[\theta]$. This gives $\alpha \in p\Z[\theta]$, so
  $\ker \psi = p \Z[\theta]$.

  \item $\psi$ is surjective.

  Note that if $(|G|, p) = 1$ where $G$ is a
  finite abelian group, then $[p] : G \to G$ is injective
  implies that it is bijective. So let $\gamma \in \OO_K$,
  so that $\overline{\gamma} \in \OO_K / \Z[\theta]$ is
  a multiple of $p$, i.e. $\overline{\gamma} = p \overline{\gamma'}$ for some $\gamma' \in \OO_K$.
  Then $\gamma - p \gamma' \in \Z[\theta]$,
  so $\psi(\gamma - p\gamma') = \gamma$.
  Since $\gamma - p \gamma' \in \Z[\theta]$, this
  shows that $\psi$ is surjective.
  \end{enumerate}

  Thus $\overline{\psi}$ is bijective, so it
  is an isomorphism $\Z[\theta] / p \Z[\theta] \to \OO_K / p \OO_K$.
\end{proof}

\begin{theorem}[Kummer]
  Let $K = \Q(\theta)$ and $p$ be a prime.
  Assume that $p \nmid [\OO_K : \Z[\theta]]$, and let
  $g(x)$ be the minimal polynomial of $\theta$.
  Write (let $\overline{g}$ denote the reduction of
  $g$ modulo $p$)
  \[
    \overline{g} = \prod_{i = 1}^r (\overline{g_i})^{e_i},
  \]
  where $g_i(x) \in \Z[x]$ and $\overline{g_i} \in \F_p[x]$
  is irreducible and monic, with
  $g_1, \dots, g_r$ distinct. Then
  \[
    p \OO_K = \p_1^{e_1} \dots \p_r^{e_r},
  \]
  where $N(\p_i) = p^{f_i}$ for $f_i = \deg g_i$, and
  $\p_i = (p, g_i(\theta))$ are distinct prime ideals.
\end{theorem}

\begin{proof}
  Let $\p_i = (p, g_i(\theta))$ as in the statement.
  Then
  \[
    \OO_K / \p_i = \OO_K / (p, g_i(\theta))
    \cong \Z[\theta] / (p, g_i(\theta))
    \cong \Z[x] / (p, g_i(x))
  \]
  The first isomorphism follows from the lemma, which
  holds only when $p \nmid [\OO_K : \Z[\theta]]$.
  Note that
  \[
    \F_p[\theta] / (\overline{g_i}(\theta)) \cong \Z[\theta] / (p, g_i(\theta))
    \cong \Z[x] / (p, g_i(x))
    \cong \F_p[x] / (\overline{g_i}(x)).
  \]
  Since $\overline{g_i}$ is irreducible of
  degree $f_i$, the quotient is a field of size
  $p^{f_i}$. This proves that $N(\p_i) = p^{f_i}$
  and also that $\p_i$ is a maximal ideal (so in
  particular, a prime ideal).
  Now if $n = [K : \Q]$, then
  \[
    \sum_{i = 1}^r e_i f_i = \deg \overline{g} = n.
  \]
  Check as an exercise that the $\p_i$ are distinct
  (use the fact that $\overline{g_i}$ and
  $\overline{g_j}$ are relatively prime, so that
  $(\overline{g_i}, \overline{g_j}) = 1$ in $\F_p[x]$).
  Now we will show that $p \OO_K \cong \p_1^{e_1} \dots \p_r^{e_r}$.
  First observe that
  \[
    \p_1^{e_1} \dots \p_r^{e_r} = (p, g_1(\theta))^{e_1} \dots (p, g_r(\theta))^{e_r}
    \subseteq (p, g_1(\theta)^{e_1} \dots g_r(\theta)^{e_r}) = (p).
  \]
  (Check the above inclusion as an exercise. Note that
  for $\overline{g}(x) = \overline{g_1}(x) \overline{g_2}(x)$, we can find $h$ such that $g_1 + g_2 = 1 + ph$, so
  that $(p, g_1(\theta))(p, g_2(\theta)) = (p^2, p g_1(\theta), p g_2(\theta), g_1(\theta) g_2(\theta)) = (p)$ as
  $p(1 + ph) = p + p^2 h$ and $g(\theta) = 0$).
  Thus $p \OO_K | \p_1^{e_1} \dots \p_r^{e_r}$, which
  implies that
  \[
    p \OO_K = \p_1^{e_1'} \dots \p_r^{e_r'}
  \]
  with $0 \le e_i' \le e_i$. But $n = \sum e_i' f_i = \sum e_i f_i$, so $e_i' = e_i$ for all $i$,
  which completes the proof.
\end{proof}

\section{Ramification}

\begin{definition}
  Let $\p_i, e_i, f_i, r$ be defined as in the statement
  of the previous
  theorem. We say that the $\p_i$ are prime ideals
  \emph{lying over} $p$, and $e_i$ is called
  the \emph{ramification index} of $\p_i$ over $p$.

  If $e_i = 1$, then we say that $\p_i$ is
  \emph{unramified} over $p$. Otherwise if
  $e_i > 1$, we say that $p$ is \emph{ramified}.
  Finally if $e_i = n$, then we say that $\p$ is
  \emph{totally ramified}, i.e. $p\OO_K = \p^n$.

  If $p \OO_K$ is prime, then we say that $p$ is
  \emph{inert}. If $r = n$, i.e. if
  $p \OO_K = \p_1 \dots \p_n$ for distinct $\p_i$,
  then we say that $p$ \emph{splits completely} in
  $\OO_K$ (or in $K$). The $f_i$
  is called the \emph{residue degree}.
\end{definition}

\begin{corollary}
  If the minimal polynomial of $\theta \in \OO_K$ is
  Eisenstein at $p$ and $K = \Q(\theta)$, then 
  $p$ is totally ramified in $\OO_K$.
\end{corollary}

\begin{proof}
  We have previously shown that
  $p \nmid [\OO_K : \Z[\theta]]$, so Kummer's theorem
  applies. Let $g(x)$ be the minimal polynomial of
  $\theta$. Then $\overline{g(x)} = x^n$ in
  $\F_p[x]$ since $g(x)$ is Eisenstein at $p$. Thus
  by Kummer's theorem, we have
  $p \OO_K = \p^n$ where $\p = (p, \theta)$, i.e.
  $p$ is totally ramified.
\end{proof}

\begin{corollary}
  Only finitely many primes ramify in any number
  field $K / \Q$. More specifically, if
  $p \nmid [\OO_K : \Z[\theta]]$ for some $\theta \in \OO_K$, then
  $p$ ramifies in $K$ if and only if
  $p | \Delta_K$.
\end{corollary}

\begin{proof}
  Note that $p$ ramifies in $\OO_K$ if and only if
  $\overline{g}$ has a multiple root in $\F_p[x]$,
  if and only if $\Delta(\overline{g}) = 0$ in $\F_p[x]$.
  Now recall that we have
  \[
    \Delta_K = \frac{\Delta_{\Z[\theta]}}{[\OO_K : \Z[\theta]]^2},
  \]
  so $p | \Delta_K$ if and only if
  $p | \Delta_{\Z[\theta]}$ (since
  $p \nmid [\OO_K : \Z[\theta]]$ by hypothesis).
  So we look at $\Delta_{\Z[\theta]}$ instead.
  Taking $g$ to be the minimal polynomial of $\theta$,
  this happens if and only if $p | \Delta(g) \equiv \Delta_K$.
\end{proof}

\begin{remark}
  The above corollary holds in greater generality
  (without the hypothesis that $p \nmid [\OO_K : \Z[\theta]]$),
  but the proof requires some more advanced tools.
\end{remark}

\section{More Computations of Rings of Integers}

\begin{remark}
  Recall that $\OO_{\Q(\sqrt[3]{2})} = \Z[\sqrt[3]{2}]$.
  We will now generalize this result.
\end{remark}

\begin{theorem}
  Let $p$ be a prime and $a \ne 0, \pm 1$ be a
  square-free integer such that $(p, a) = 1$.
  Let $\theta = \sqrt[p]{a}$. Letting
  $K = \Q(\theta) = \Q[x] / (x^p - a)$, we have
  $\OO_K = \Z[\theta]$
  if and only if $a^{p - 1} \not\equiv 1 \Pmod{p^2}$.
\end{theorem}

\begin{proof}
  $(\Leftarrow)$
  Let $K = \Q(\theta)$ and assume that
  $a^p \not\equiv a \Pmod{p^2}$. The discriminant
  of $x^p - a$ is
  \[
    \Delta(\theta) = \pm p^p a^{p - 1}
    = \Delta_K \cdot [\OO_K : \Z[\theta]]^2.
  \]
  Note that $x^p - a$ is Eisenstein at every prime
  dividing $a$. Now observe that
  \[
    (x + a)^p - a
  \]
  is Eisenstein at $p$ (since $p^2 \nmid (a^p - a)$
  by hypothesis), and $\Z[\theta] = \Z[\theta - a]$,
  so $p \nmid [\OO_K : \Z[\theta]]$.

  $(\Rightarrow)$ Suppose that $\OO_K \ne \Z[\theta]$.
  Kummer's theorem implies that $p \OO_K = \p^p$ where
  $\p = (p, \theta - a)$. Note that
  \[
    x^p - a \equiv (x - a)^p \pmod{p}
  \]
  by Fermat's little theorem, and that $N(\p) = p$.
  Now $\theta - a \in \p$ (and $\theta - a \notin \p^2$), so
  \[
    p \in \p^2 = (p^2, p(\theta - a), (\theta - a)^2)
  \]
  since $\p^2 | (p) = \p^p$ and $p \ge 2$, so
  $(p) \subseteq \p^2$. Thus we have
  $(\theta - a) = \p \mathfrak{a}$
  for some ideal $\mathfrak{a}$ which is relatively
  prime to $\p$. Now
  $(p, N(\mathfrak{a})) = 1$ since
  $N(\mathfrak{a}) = \prod q_i^{e_i f_i}$ where
  $\mathfrak{a} = \q_1^{e_1} \dots \q_r^{e_r}$ for
  $\q_i \ne \p$, so $q_i \ne p$. Then
  \[
    a^p - a = |N(\theta - a)| = N(\p \mathfrak{a})
    = p N(\mathfrak{a})
  \]
  where $N(\mathfrak{a})$ is relative prime to $p$,
  so $p^2$ does not divide $a^p - a$.
\end{proof}

\begin{remark}
  Next class, we will show that $\Q(\sqrt{-5})$
  has class number $2$, which we will use to solve
  the Diophantine equation $y^2 = x^3 - 5$ in $\Z$
  via arithemtic in $\Z[\sqrt{-5}]$, by writing
  \[
    x^3 = y^2 + 5 = (y + \sqrt{-5})(y - \sqrt{-5}).
  \]
  We will fix our previous issue by arguing
  that if a product of ideals is
  a cube, then each ideal is a cube when the class
  number is not a multiple of $3$. This is
  similar to Kummer's ideas for Fermat's last theorem.
\end{remark}
