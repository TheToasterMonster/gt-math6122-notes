\chapter{Feb.~20 --- Localization}

\section{Motivation for Localization}

\begin{remark}
So far we have been dealing with the following
scenario of a number field $K$:
\begin{center}
  \begin{tikzcd}
    K \ar[dash, r] \ar[dash, d, "n"] & \OO_K \ar[dash, d, "n"] \\
    \Q \ar[dash, r] & \Z
  \end{tikzcd}
\end{center}
We would now like to consider relative extensions,
for a finite extension $L / K$:
\begin{center}
  \begin{tikzcd}
    L \ar[dash, r] \ar[dash, d, "n"] & \OO_L \ar[dash, d, "n"] \\
    K \ar[dash, r] & \OO_K
  \end{tikzcd}
\end{center}
For instance, that if prime ideals $\q_1, \dots, \q_r \subseteq \OO_L$ lie
over $\p \subseteq \OO_K$, then $[L : K] = \sum_{i = 1}^r e_i f_i$.
One of our goals will be to show that a Noetherian
domain $R$ is Dedekind if and only if $R_\p$ is a
PID for every prime ideal $\p$.
The second goal will be to prove
``Dirichlet's $S$-unit theorem.''
\end{remark}

\section{Localization}

\begin{definition}
  Let $R$ be a domain and $K$ its field of fractions.
  Let $S$ be a \emph{multiplicative subset} of $R$,
  i.e. $0 \notin S$, $1 \in S$, and
  $a, b \in S$ implies $ab \in S$. Then the
  \emph{localization of $R$ by $S$} is the subring
  \[
    S^{-1} R = \left\{ \frac{a}{b} : a \in R, b \in S \right\} \subseteq K.
  \]
\end{definition}

\begin{remark}
  The equivalence relation of $a / b \sim c / d$
  if $ad = bc$ is included by default since
  $S^{-1} R \subseteq K$.
\end{remark}

\begin{remark}
  Note that $R \subseteq S^{-1} R \subseteq K$. The
  idea is that $S^{-1} R$ will have some of the nice
  properties of $K$ while still retaining enough
  of the information of $R$.
\end{remark}

\begin{example}
  For $S = R \setminus \{0\}$, we have
  $S^{-1} R = K$, and for $S = \{1\}$, we have
  $S^{-1} R = R$.
\end{example}

\begin{example}
  If $\p \subseteq R$ is a prime ideal, then
  $S = R \setminus \p$ is a multiplicative subset.
  Thus we can define $R_\p = S^{-1} R$, which we
  will call the \emph{localization of $R$ at $\p$}.
\end{example}

\begin{remark}
  One can definition localization in a more general
  context, e.g. for a ring which is not a domain.
  This is necessary for algebraic geometry, but we do
  not need this, so we will avoid it.
\end{remark}

\begin{prop}
  The prime ideals of $S^{-1} R$ are in
  (inclusion-preserving) bijection with
  the prime ideals of $R$ disjoint from $S$.
\end{prop}

\begin{proof}
  Denote by $\Spec R$ the set of prime ideals of $R$.
  We will show the following bijection:
  \begin{align*}
    \{\q \in \Spec(R) : \q \cap S = \varnothing\}
    &\longleftrightarrow \Spec(S^{-1} R) = \Spec(R') \\
    \q &\longmapsto S^{-1} \q = \left\{\frac{a}{b} : a \in \q, b \in S\right\} \\
    \q' \cap R &\mathrel{\reflectbox{\ensuremath{\longmapsto}}} \q'
  \end{align*}
  First we claim that $S^{-1} \q$ is a prime ideal in
  $R'$. Note that $\q \cap S = \varnothing$ is
  equivalent to $1 \notin S^{-1} \q$. Check as an
  exercise that $S^{-1} \q$ is in fact an ideal
  in $R'$. To see that it is prime, suppose that
  $(a_1 / b_1)(a_2 / b_2) \in S^{-1} \q$, where
  $a_1, a_2 \in R$ and $b_1, b_2 \in S$. Then
  we can see that
  \[
    \frac{a_1 a_2}{b_1 b_2} = \frac{a}{b}, \quad a \in \q, b \in S,
  \]
  so $a_1 a_2 b = a b_1 b_2 \in \q$. As
  $b \in S$ and $q \cap S = \varnothing$, we have
  $b \notin \q$. Thus $a_1 \in \q$ or $a_2 \in \q$, i.e.
  $S^{-1} \q$ is prime.

  The other direction is easier and is mostly
  left as an exercise. To see that
  $(\q' \cap R) \cap S = \varnothing$, suppose that
  $s \in S \cap \q'$. Then $s \cdot (1 / s) = 1 \in \q'$,
  where $s \in \q'$ and $(1 / s) \in R$, which is
  impossible since $\q' \ne R$.

  It only remains to show that these maps are
  inverses of each other, which is left as an exercise.
\end{proof}

\begin{corollary}
  The prime ideals of $R_\p$ are in bijection
  with the prime ideals of $R$ contained in $\p$.
\end{corollary}

\begin{corollary}
  The localization $R_\p$ is a local ring,
  i.e. it has a unique maximal ideal.
\end{corollary}

\begin{example}
  We can write the localization $\Z_{(2)}$ as
  \[
    \Z_{(2)} = \left\{ \frac{a}{b} \in \Q : \text{$b$ is odd} \right\}.
  \]
  The unique maximal ideal is $2 \Z_{(2)} = \{2 a / b : \text{$b$ is odd}\}$, and
  $\Z_{(2)} \setminus 2\Z_{(2)} = \Z_{(2)}^\times$. We
  also have
  \[
    \Z_{(2)} / 2\Z_{(2)} = \Z / 2\Z.
  \]
  In general, we will see that
  $\OO_K / \p^m \OO_K \cong (\OO_K)_\p / \p^m (\OO_K)_\p$,
  so we can study the localization instead.
\end{example}

\begin{lemma}
  Let $R$ be a ring and $\m$ a maximal ideal in $R$.
  If $s \notin \m$, then $\m^n + (s) = (1)$ for all
  $n \ge 1$.
  Equivalently, $\overline{s}$ is a unit in
  $R / \m^n$ for every $n \ge 1$.
\end{lemma}

\begin{proof}
  We induct on $n$. The case $n = 1$ is clear since
  $\m$ is maximal. Now suppose
  $(1) = \m^{n - 1} + (s)$, so
  \[
    \m = \m^n + s \m \subsetneq \m^n + (s)
  \]
  since $s \notin \m$. But $\m$ is a maximal ideal, so
  we must have $\m^n + (s) = (1)$.
\end{proof}

\begin{remark}
  We will denote $\m_\p = \p R_\p$, the unique
  maximal ideal in the localization $R_\p$.
\end{remark}

\begin{lemma}
  Let $R$ be an integral domain and $\p$ a maximal
  ideal. Then for all $n \ge 1$, the natural map
  \[
    \phi : R / \p^n \to R_\p / \m_\p^n
  \]
  is an isomorphism. (In particular,
  $R / \p \cong R_\p / \m_\p$.)
\end{lemma}

\begin{proof}
  The natural map is the map induced by the
  inclusion $R \hookrightarrow R_\p$ on the quotient
  $R / \p^n$.

  First we show that $\phi$ is injective.
  Suppose $x \in R \cap \m_\p^n$, and we will show
  that $x \in \p^n$. Write
  \[
    x = \frac{y}{s}, \quad y \in \p^n, s \notin \p.
  \]
  By the lemma, we have
  $\overline{s} \in (R / \p^n)^\times$. But
  $sx = y \in \p^n$, which implies that $\overline{s} \cdot \overline{x} = 0$
  in $R / \p^n$. Since $\overline{s}$ is a unit,
  this means that $\overline{x} = 0$, so $x \in \p^n$.
  This shows that $\phi$ is injective.

  Now we show that $\phi$ is surjective.
  Let $r / s \in R_\p$ with $r \in R$ and
  $s \notin \p$. By the lemma, there exists $r' \in R$
  such that $r \equiv r' s \Pmod{\p^n}$ (e.g.
  take $r' = s^{-1} r \in R / \p^n$). Then
  $r / s \equiv r' \Pmod{\m_\p^n}$, so we have
  \[
    \phi(r') = \frac{r}{s} + \m_\p^n.
  \]
  Since $r / s \in R_\p$ was arbitrary, this shows
  surjectivity. Thus $\phi$ is an isomorphism.
\end{proof}

\section{Dedekind Domains and Localization}

\begin{lemma}
  If $R$ is a Noetherian domain, then so is
  $S^{-1} R$.
\end{lemma}

\begin{proof}
  The key observation is that every ideal of
  $S^{-1} R$ is of the form $S^{-1} I$ for some ideal
  $I$ of $R$ (check this as an exercise).
  So if $I$ is generated by $x_1, \dots, x_r$,
  then $S^{-1} I$ is generated by $x_1 / 1, \dots, x_r / 1$.
\end{proof}

\begin{lemma}
  If $R$ is integrally closed, then $S^{-1} R$ is also
  integrally closed.
\end{lemma}

\begin{proof}
  Suppose $\alpha = a / b \in K$ satisfies
  a monic polynomial $f \in (S^{-1} R)[x]$. Then we need
  to show that $\alpha \in S^{-1} R$. Write
  \[
    f(x) = x^n + a_{n - 1} x^{n - 1} + \dots + a_0,
    \quad a_i = \frac{r_i}{s_i} \in S^{-1} R.
  \]
  Let $s = s_0 s_1 \dots s_{n - 1}$ and multiply the
  equation $f(\alpha) = 0$ by $s^n$ to get
  \[
    (s\alpha)^n + a_{n - 1} s (s \alpha)^{n - 1}
    + \dots + a_1 s^{n - 1} (s\alpha) + a_0 s^n = 0.
  \]
  The coefficients $a_{n - 1} s, \dots, a_0 s^n$
  are in $R$, so $s \alpha$ is integral over $R$.
  But $R$ is integrally closed, so $s \alpha \in R$.
  Since $s \in S$, we have $\alpha \in S^{-1} R$, which
  proves that $S^{-1} R$ is integrally closed.
\end{proof}

\begin{prop}
  If $R$ is a Dedekind domain and $S$ is a multiplicative
  set, then $S^{-1} R$ is either a Dedekind domain or a field.
\end{prop}

\begin{proof}
  From the above lemmas, it suffices to show
  $\dim(S^{-1} R) \le 1$. There is an bijection from
  chains of prime ideals in $S^{-1} R$ to
  chains of prime ideals in $R$ disjoint from $S$, so
  $\dim(S^{-1} R) \le \dim R = 1$.
\end{proof}

\begin{corollary}
  If $R$ is Dedekind, then $R_\p$ is a PID
  for every nonzero prime ideal $\p$.
\end{corollary}

\begin{proof}
  Note that $R_\p$ is a local Dedekind ring with
  prime ideals $(0)$ and $\m_\p$. By unique
  factorization, every nonzero ideal of $R_\p$ is
  of the form $\m_\p^k$. So it suffices to show that
  $\m_\p$ is principal.

  To do this, choose any $\pi \in \p \setminus \p^2$,
  and we claim that $\m_\p = \pi R_\p$.
  Since $\pi R_\p$ is an ideal, we have
  $\pi R_\p = \m_\p^k$ for some $k \ge 1$.
  If $k \ge 2$, then $\pi \in \m_\p^2 \cap R = (\m_\p \cap R)^2 = \p^2$,
  a contradiction. So $k = 1$.
\end{proof}

\begin{definition}
  A local PID is called a \emph{discrete valuation ring (DVR)}.
\end{definition}
