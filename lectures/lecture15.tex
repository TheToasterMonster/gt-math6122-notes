\chapter{Feb.~25 --- Localization, Part 2}

\section{Valuations}

\begin{example}
  Recall that $R_\p$ is a local PID, also known
  as a discrete valuation ring. Consider
  \[
    \Z_{(p)} = \left\{
      \frac{a}{b} \in \Q : p \nmid b
    \right\} \subseteq \Q.
  \]
  There is a \emph{$p$-adic valuation}
  $v_p : \Z_{(p)} \to \N \cup \{\infty\}$ given by
  $v_p(x) = k$ if $x = p^k \cdot a / b$ where
  $a, b \in \Z$, $p \nmid a$, $p \nmid b$, and
  $v_p(0) = \infty$.
  Now $v_p$ extends to $v_p : \Q \to \Z \cup \{\infty\}$,
  and we can recover $\Z_{(p)}$ as
  \[
    \Z_{(p)} = \{x \in \Q : v_p(x) \ge 0\}.
  \]
  This is known as the \emph{valuation ring} associated
  to $v_p$. This is discrete since
  $\Z \cup \{\infty\}$ is discrete.
\end{example}

\begin{definition}
  A \emph{(real) valuation} on a field $k$ is a function
  $v : k \to \R \cup \{\infty\}$ such that
  \begin{itemize}
    \item $v(x) = \infty$ if and only if $x = 0$;
    \item $v(ab) = v(a) + v(b)$;
    \item $v(a + b) \ge \min\{v(a), v(b)\}$.
  \end{itemize}
\end{definition}
Given such a valuation $v$, there
\emph{valuation ring associated to $v$} is
$\OO_v = \{x \in k : v(x) \ge 0\}$.

\begin{lemma}
  For all $x \in k$, either $x \in \OO_v$ or
  $x^{-1} \in \OO_v$.
\end{lemma}

\begin{definition}
  A \emph{valuation ring} is a subring $\OO$
  of a field $k$ such that for all $x \in k$,
  $x \in \OO$ or $x^{-1} \in \OO$.
\end{definition}

\section{Dedekind Domains and Localization, Continued}

\begin{lemma}\label{lem:int-int-closed}
  If $R$ is an integral domain, then
  \[
    R = \bigcap_{\p \text{ prime}} R_\p
    = \bigcap_{\m \text{ maximal}} R_\m.
  \]
\end{lemma}

\begin{proof}
  Since every maximal ideal is prime, we prove
  the statement for maximal ideals. We need to show that
  $\bigcap_\p R_\p \subseteq R$. Choose
  $a / b \in \bigcap_\p R_\p$. Define the ideal
  \[
    I = \{y \in R : ay \in bR\}
  \]
  We will show that $I = R$, which implies that
  $a \in b R$ and thus $a / b \in R$ since
  $1 \in I$. Check as an exercise that $I$ is indeed
  an ideal. Now since $a / b \in R_\p$, we can
  write $a / b = x / y$ with $x, y \in R$ and
  $y \notin \p$. Then $ay = bx$, so $y \in I$.
  Since $y \notin \p$, this means that $I \nsubseteq \p$.
  We can do this for every maximal ideal $\p$. But every
  $I \ne R$ is contained in a maximal ideal, so 
  we must have $I = R$.
\end{proof}

\begin{theorem}
  If $R$ is a Noetherian integral domain, then $R$ is
  Dedekind if and only if $R_\p$ is a PID for
  every nonzero prime ideal $\p$ of $R$.
\end{theorem}

\begin{proof}
  $(\Rightarrow)$ This was Corollary \ref{cor:local-pid}.

  $(\Leftarrow)$ The tricky part to
  show is that $R$ is integrally closed. This part
  follows from Lemma \ref{lem:int-int-closed} since the
  intersection
  of integrally closed subrings is again
  integrally closed.
\end{proof}

\section{\texorpdfstring{$S$}{S}-Integers and \texorpdfstring{$S$}{S}-Units}

\begin{definition}
  Let $S$ be a finite set of nonzero prime ideals in
  a domain $R$ and $K = \Frac R$. Define
  \[
    R^S = \left\{
      \frac{x}{y} \in K : x, y \in R,\, y \notin \p \text{ for all } \p \notin S
    \right\}
    = T^{-1} R,
  \]
  where $T = \{x \in R : x \notin \bigcup_{\p \notin S} \p\}$.
\end{definition}

\begin{remark}
  If $R$ is Dedekind, this means $(y)$ is divisible
  only by primes in $S$.
\end{remark}

\begin{example}
  Let $S = \{(2), (3)\} \subseteq \Spec \Z$. Then
  \[
    \Z^S = \left\{
      \frac{x}{y} \in \Q : x, y \in \Z,\, y = \pm 2^a 3^b
    \right\}.
  \]
  In some sense, $\Z_{(2)}$ is close to $\Q$ while
  $\Z^{(2)}$ is close to $\Z$. Note that
  $(\Z^{(2)})^\times = \pm \{2^k\}$, which is of
  rank $1$.
\end{example}

\begin{remark}
  The following are some facts and definitions
  regarding $R^S$:
  \begin{itemize}
    \item $R^S$ is Dedekind.
    \item $\Cl(R^S)$ is finite.
    \item $(R^S)^\times$ is finitely generated of
      rank $|S| + r_1 + r_2 - 1$.
  \end{itemize}
\end{remark}

\begin{exercise}
  Let $K$ be a number field, $R = \OO_K$, and $S$ a
  finite set of nonzero prime ideals. Show that
  \begin{center}
    \begin{tikzcd}
      1 \ar[r] & R^\times \ar[r] & (R^S)^\times \ar[r] & \bigoplus_{\p \in S} (K^\times / R_\p^\times) \ar[r] & \Cl(R) \ar[r] & \Cl(R^S) \ar[r] & 0
    \end{tikzcd}
  \end{center}
  is an exact sequence, where
  $K^\times / R_\p^\times \cong \Z$.
\end{exercise}

\begin{remark}
  Note that the above shows that $\Cl(R^S)$
  is in fact a quotient of $\Cl(R)$, hence it must be
  finite.
\end{remark}

\begin{prop}
  If $K$ is a number field, then there exists a finite
  set $S$ of nonzero prime ideals such that
  $\OO_K^S$ is a PID (equivalently, $\Cl(\OO_K^S) = \{0\}$).
\end{prop}

\begin{proof}
  There is a map $\rho : \Cl(R) \to \Cl(R^S)$ given by
  $[I] \mapsto [I R^S]$, which one can verify as
  an exercise is surjective. Let $I_1, \dots, I_t$ be
  ideals which generate $\Cl(R)$. Define
  \[
    S = \text{all prime ideals of $R$ dividing some $I_k$}.
  \]
  For $\p \in S$, we have $\p R^S = (1)$. So
  $\rho([I_k]) = 0$ for all $k$. Thus we must have
  $\Cl(R^S) = 0$.
\end{proof}

\begin{remark}
  The above proposition says that we may get a PID
  $\OO_K^S$ in place of $\OO_K$ (which is often not a
  PID), at the cost of increasing the rank of the unit
  group by $|S|$.
\end{remark}

\section{Applications to Elliptic Curves}

\begin{theorem}[Siegel]
  Let $K$ be a number field and $S$ a finite set
  of nonzero primes in $\OO_K$. Let $f(x) \in K[x]$
  be a separable polynomial of degree $\ge 3$.
  Then the Diophantine equation
  \[
    Y^2 = f(X)
  \]
  has only finitely many solutions with
  $X, Y \in \OO_K^S$. (As a special case, this holds when
  $S = \varnothing$, so that $\OO_K^S = \OO_K$. In particular, if $K = \Q$ and $S = \varnothing$, then $\OO_K^S = \Z$. The case $\deg f = 3$ is an elliptic curve.)
\end{theorem}

\begin{proof}
  We will use the following result without proof:
  \begin{quote}
    \textbf{Theorem} (Siegel, Mahler)\textbf{.}
    Let $K, S$ as before. Then the equation $x + y = 1$
    has only finitely many solutions with $x, y \in (\OO_K^S)^\times$.\footnote{For instance, $x + y = 1$ has finitely solutions with $x, y \in \pm \{2^a 3^b\}$. This is not obvious, e.g. $1 = 3^2 - 2^3 = 3 - 2 = 2^2 - 3 = \dots$.}
  \end{quote}
  Without loss of generality, assume that $f$ splits over
  $K$ (by enlarging $K$ if necessary), so
  \[
    f(X) = a(X - \alpha_1) \dots (X - \alpha_n), \quad a, \alpha_j \in K
  \]
  for distinct $\alpha_1, \dots, \alpha_n$ and
  $n \ge 3$. Let $K^S = (\OO_K^S)^\times$.
  By enlarging $S$ if necessary, we
  can assume:
  \begin{enumerate}
    \item $a \in K^S$, $\alpha_1, \dots, \alpha_n \in \OO_K^S$;
    \item $\alpha_i - \alpha_j \in K^S$ for all $i \ne j$;
    \item $\OO_K^S$ is a PID.
  \end{enumerate}
  By Dirichlet's $S$-unit theorem, $K^S$ is
  finitely generated, and so $K^S / (K^S)^2$ is finite.
  So if
  \[
    L = \text{compositum of $K(\sqrt{u})$ for all $u \in K^S$},
  \]
  then $L / K$ is finite. Now define the set
  \[
    T = \text{prime ideals of $\OO_L$ containing some element of $S$}.
  \]
  We will work with $\OO_L^T$, the
  $T$-integers in $\OO_L$. Now suppose
  $y^2 = f(x)$ with $x, y \in \OO_K^S$, so that
  \[
    y^2 = a(x - \alpha_1) \dots (x - \alpha_n).
  \]
  Then any prime ideal $\p$ of $\OO_K^S$ divides
  at most one of these terms (otherwise it divides
  $\alpha_i - \alpha_j$ for $i \ne j$, which we assumed
  lies in $K^S$). Since $\p$ must divides $y^2$, it
  must divide $y^2$ twice, and since
  \[
    (x - \alpha_1) \dots (x - \alpha_n) = a^{-1} y^2,
  \]
  we must have $x - \alpha_i = u_i a_i^2$
  for each $i$, where $u_i \in K^S$ and
  $a_i \in \OO_K^S$. Then
  $(x - \alpha_i) = \mathfrak{a}_i^2$, where $\mathfrak{a}_i = (a_i)$.
  Let $u_i = v_i^2$ with $v_i \in L^T$, then
  \[
    x - \alpha_i = v_i^2 a_i^2 = w_i^2, \quad w_i \in \OO_L^T.
  \]
  The trick is then the following: We can write
  \[
    \alpha_j - \alpha_i = w_i^2 - w_j^2
    = (w_i - w_j)(w_i + w_j).
  \]
  Since $\alpha_j - \alpha_i \in K^S$, we must have
  $w_i - w_j, w_i + w_j \in L^T$. We have the identities
  \[
    \frac{w_1 - w_2}{w_1 - w_3} + \frac{w_2 - w_3}{w_1 - w_3} = 1 \quad \text{and} \quad
    \frac{w_1 + w_2}{w_1 - w_3} - \frac{w_2 + w_3}{w_1 - w_3} = 1.
  \]
  All of the above quotients lie in $L^T$, so there
  are only finitely many choices for the quotients
  by the theorem of Siegel and Mahler.
  One can show that there are then only finitely
  many choices for $w_1$, hence there can only be
  finitely many choices for $x = \alpha_1 + w_1^2$,
  since $\alpha_1$ is fixed.
\end{proof}
