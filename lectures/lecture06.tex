\chapter{Jan.~23 --- Computing Rings of Integers}

\section{More on Computing of Rings of Integers}
\begin{lemma}
  Let $\alpha_1, \dots, \alpha_n \in \OO_K$ be a
  basis for $K / \Q$, and suppose that
  $\OO_K / (\Z \alpha_1 \oplus \dots \oplus \Z \alpha_n)$
  has exponent $m$, i.e. $m \alpha \in \Z \alpha_1 \oplus \dots \oplus \Z \alpha_n$
  for every $\alpha \in \OO_K$. Then
  \[
    \OO_K \subseteq \Z \frac{\alpha_1}{m} \oplus \dots \oplus \Z \frac{\alpha_n}{m}.
  \]
  Moreover, if $\OO_K \ne \Z \alpha_1 \oplus \dots \oplus \Z \alpha_n$,
  then there exist $0 \le m_i \le m - 1$, not all
  zero, such that
  \[
    m_1 \frac{\alpha_1}{m} + \dots + m_n \frac{\alpha_n}{m} \in \OO_K.
  \]
\end{lemma}

\begin{proof}
  The idea is the following: Let
  $M = \Z \alpha_1 \oplus \dots \oplus \Z \alpha_n$,
  so $m \OO_K \subseteq M$. Then
  $\OO_K \subseteq (1 / m) M$, which proves the
  first part. Now if $\OO_K \ne M$, then there
  exists a nonzero element of $(1 / m) M$ which is
  not in $M$. This gives a nonzero element in the
  quotient:
  \[
    \frac{(1 / m) M}{M}
    = \left[m_1 \frac{\alpha_1}{m} + \dots + m_n \frac{\alpha_n}{m}\right]
  \]
  for some $m_i$, as in the statement.
\end{proof}

\begin{example}
  We can apply this lemma to our example from last
  lecture: Let $K = \Q(\sqrt{d})$, where
  $d$ is square-free. Then $\Delta(1, \sqrt{d}) = 4d$,
  so $\Delta_K | 4d$. Then we have
  \[
    \frac{4d}{\Delta_K} = [\OO_K : (\Z \oplus \Z \sqrt{d})]^2
    = 1^2 \text{ or } 2^2.
  \]
  Thus by the lemma, either $\OO_K = \Z[\sqrt{d}]$
  or one of
  \[
    \frac{1}{2}, \quad \frac{\sqrt{d}}{2}, \quad \frac{1 + \sqrt{d}}{2}
  \]
  is in $\OO_K$. The first two are obvious not in
  $\OO_K$, and $(1 + \sqrt{d}) / 2 \in \OO_K$ if
  and only if $d \equiv 1 \Pmod{4}$. Thus if
  $d \equiv 1 \Pmod{4}$, then
  $1, (1 + \sqrt{d}) / 2$ is an integral basis for
  $\OO_K$.
\end{example}

\begin{prop}
  Let $[K : \Q] = n$ and $\alpha \in \OO_K$
  with minimal polynomial of degree $n$. Suppose
  further that the minimal polynomial of $\alpha$ is
  Eisenstein at $p$. Then
  $p \nmid [\OO_K : \Z[\alpha]]$.\footnote{Recall that $\Z[\alpha] = \Z \oplus \Z \alpha \oplus \dots \oplus \Z \alpha^{n - 1}$.}
\end{prop}

\begin{proof}
  Let the minimal polynomial of $\alpha$ be
  \[
    f(x) = x^n + a_{n - 1} x^{n - 1} + \dots + a_1 x + a_0,
  \]
  with $p | a_i$ for every $i$ and $p^2 \nmid a_0$
  (since $f$ is Eisenstein at $p$). Suppose
  otherwise that $p | [\OO_K : \Z[\alpha]]$. Then
  by Cauchy's theorem, there exists  $\xi \in \OO_K$
  such that $[\xi] \in \OO_K / \Z[\alpha]$ has
  order $p$. Then
  \[
    p \xi = b_0 + b_1 \alpha + \dots + b_{n - 1} \alpha^{n - 1}
  \]
  where $b_i \in \Z$, not all divisible by $p$.
  Let $j$ be the smallest index such that $p \nmid b_j$.
  Then
  \[
    \OO_K \ni \eta = \xi - \left(\frac{b_0}{p} + \frac{b_1}{p} \alpha + \dots + \frac{b_{j - 1}}{p} \alpha^{j - 1}\right)
    = \frac{b_j}{p} \alpha^j + \frac{b_{j + 1}}{p} \alpha^{j + 1} + \dots + \frac{b_n}{p} \alpha^n.
  \]
  So we have
  \[
    \OO_K \ni \eta \alpha^{n - j - 1} = \frac{b^j}{p} \alpha^{n - 1} + \frac{\alpha^n}{p} (b_{j + 1} + b_{j + 2} \alpha + \dots).
  \]
  Also notice that
  \[
    \frac{\alpha^n}{p} = -\frac{a_0 + a_1 \alpha + \dots + a_{n - 1} \alpha^{n - 1}}{p} \in \OO_K
  \]
  since $f$ was Eisenstein at $p$. Since
  $(b_{j + 1} + b_{j + 2} \alpha + \dots) \in \OO_K$,
  we see that $b_j \alpha^{n - 1} / p \in \OO_K$
  and $p \nmid b_j$. So
  \[
    \Z \ni N^K_\Q\left(\frac{b_j}{p} \alpha^{n - 1}\right)
    = \frac{b_j^n}{p^n} N(\alpha^{n - 1})
    = \frac{b_j^n}{p^n} a_0^{n - 1}.
  \]
  Now $p \nmid b_j$ and $p^2 \nmid a_0$ so we
  have at most $n - 1$ factors of $p$ in the numerator,
  a contradiction.
\end{proof}

\begin{prop}
  For $K = \Q(\sqrt[3]{2})$, we have
  $\OO_K = \Z[\sqrt[3]{2}]$.
\end{prop}

\begin{proof}
  Let $\alpha = \sqrt[3]{2}$ and
  $M = \Z[\alpha] = \Z \oplus \Z \alpha \oplus \Z \alpha^2$.
  Let $m = |\OO_K / M|$. Then
  \[
    m^2 \Delta(\OO_K)
    = \Delta(1, \alpha, \alpha^2)
    = \Delta(f_\alpha),
  \]
  where $f_\alpha$ is the minimal polynomial of
  $\alpha$ (check that $\Delta(1, \alpha, \alpha^2) = \Delta(f_\alpha)$).
  Recall that up to signs,
  \[
    \Delta(f) = \prod_{\text{roots } \alpha_i} f(\alpha_i).
  \]
  For a cubic polynomial $f(x) = x^3 + ax + b$, the
  discriminant is $\Delta f = -4 a^3 - 27b^2$ (for a quadratic
  $f(x) = x^2 + bx + c$, it is $\Delta f = b^2 - 4c$).
  Thus for $f_\alpha(x) = x^3 - 2$, we have
  \[
    m^2 \Delta(\OO_K) = \Delta(f_\alpha) = -108
    = -6^2 \cdot 3.
  \]
  Thus the index $m$ divides $6$, and since
  $f_\alpha$ is Eisenstein at $2$, we have
  $2 \nmid m$. Now notice that
  \[
    \Z[\alpha] = \Z[\beta], \quad \text{where } \beta = \alpha - 2.
  \]
  The minimal polynomial of $\beta$ is
  $g(x) = (x + 2)^3 - 2 = x^3 + 6x^2 + 12x + 6$. Then
  $g$ is Eisenstein at $3$, so $3 \nmid m$. Thus
  we must have $m = 1$, which proves that
  $\OO_K = \Z[\alpha]$.
\end{proof}

\begin{remark}
  Later on, we will show that if
  $K = \Q(\zeta_n)$ for some $n \ge 1$ (where
  $\zeta_n$ is a primitive $n$th root of unity), then
  $\OO_K = \Z[\zeta_n]$. The proof will largely
  involve similar types of ideas.
\end{remark}

\begin{remark}
  In general if $K = \Q(\theta)$ and we only work
  with $\Z[\theta]$ instead of $\OO_K$, we may run
  into trouble since $\Z[\theta]$ may not be
  integrally closed (hence we may not have unique
  factorization of ideals).
\end{remark}

\section{Computing Factorizations of Ideals}
\begin{prop}
  Let $K = \Q(\sqrt{d})$, where $d$ is square-free.
  Let $p$ be an odd prime with $p \nmid d$. Then:
  \begin{enumerate}[(a)]
    \item if $(\frac{d}{p}) = 1$,
      then $p \OO_K = \p_1 \p_2$ where
      $\p_1 = (p, a + \sqrt{d}) \ne \p_2 = (p, a - \sqrt{d})$, and $a^2 \equiv d \Pmod{p}$;
    \item if $(\frac{d}{p}) = -1$, then $p \OO_K = \p$
      is prime in $\OO_K$.
  \end{enumerate}
  In the above, $(\frac{d}{p})$ is the
  \emph{Legendre symbol}.
\end{prop}

\begin{proof}
  $(a)$ Note that
  \[
    \p_1 \p_2 = (p^2, p(a + \sqrt{d}), p(a - \sqrt{d}), a^2 - d) \subseteq (p)
  \]
  since each of the above terms is divisible by $p$.
  But $p^2 \in \p_1 \p_2$ and
  $p(a + \sqrt{a}) + p(a - \sqrt{d}) = 2ap$ so
  \[
    (p) = (\gcd(p^2, 2ap)) \subseteq \p_1 \p_2.
  \]
  This gives the equality $\p_1 \p_2 = (p)$. Now we
  show that $\p_i$ is prime. Note that
  \[
    N(\p_1) N(\p_2) = N(p) = p^2.
  \]
  It is enough to show that $a + \sqrt{d} \notin (p)$
  (this implies $\p_1 \ne (p)$, so
  $N(\p_1) = |\OO_K / \p_1| < |\OO_K / (p)| = p^2$ and
  we must have $N(\p_1) = p$).
  Now if $p | (a + \sqrt{d})$, then
  $p | (a - \sqrt{d})$ as well, so $p | 2a$. This is a
  contradiction. Thus $N(\p_i) = p$, which implies
  that $\p_i$ is a prime ideal (otherwise the norm
  would also factor). It only remains to show that
  $\p_1 \ne \p_2$, which is left as an exercise.

  $(b)$ It is enough to show that there is no prime
  ideal $\p \subseteq \OO_K$ such that $N(\p) = p$.
  Equivalently, it suffices to show that if $\p$ is a
  prime ideal, then
  $\OO_K / \p \ncong \Z / p\Z$.  Note that $x^2 - d$
  has a root in $\OO_K$ and thus in $\OO_K / \p$, so $\OO_K / \p \cong \Z / p\Z$ would imply that
  $d$ is a square modulo $p$, contradicting $(\frac{d}{p}) = -1$.
\end{proof}

\begin{exercise}
  Show that if $p | d$, then $p \OO_K = \p^2$ for some
  prime ideal $\p$.
\end{exercise}

\begin{theorem}[Kummer]
  Let $K = \Q(\theta)$ with $\theta \in \OO_K$.
  Suppose $p$ is a prime such that $p \nmid [\OO_K : \Z[\theta]]$.
  Let $g$ be the minimal polynomial of $\theta$.
  Factor $g \Mod p$ as
  \[
    g(x) \equiv g_1(x)^{e_1} \dots g_r(x)^{e_r} \Pmod{p},
  \]
  where $g_i(x) \in \Z[x]$, $\overline{g_i(x)}$ is
  irreducible over $\mathbb{F}_p$, and the
  $\overline{g_i}$ are pairwise distinct.\footnote{Here $\overline{g(x)}$ denotes the reduction of $g(x)$ modulo $p$.} Then
  \[
    p \OO_K = \p_1^{e_1} \dots \p_r^{e_r},
  \]
  where $\p_i = (p, g_i(\theta))$ is a prime ideal,
  $N(\p_i) = p^{f_i}$ where
  $f_i = \deg g_i$, and the $\p_i$ are distinct.
\end{theorem}

\begin{remark}
  Note that this generalizes the quadratic case:
  $x^2 - d \Mod p$ factors if and only if $d$ is a
  square modulo $p$, and
  the ideals are $(p, g_i(\theta))$ for
  $g_1 = x - a$ and $g_2 = x + a$.
\end{remark}
