\chapter{Jan.~16 --- Ideal Class Group}

\section{Unique Factorization of Ideals, Continued}
The following is a corollary of Proposition
\ref{prop:dedekind}:

\begin{corollary}
  If $R$ is Dedekind and
  $\p \ne 0$ is a prime ideal, then
  $\p^{-1} \p = R = (1)$.
\end{corollary}

\begin{proof}
  First note that we have $\p \subseteq \p^{-1} \p \subseteq R$
  since $R \subseteq \p^{-1}$ by the definition
  of $\p^{-1}$. Furthermore, $\p^{-1}$ is an
  $R$-submodule of $K$, so $\p^{-1} \p$ is an
  $R$-submodule of $R$, i.e. an ideal of $R$. Also,
  by Proposition \ref{prop:dedekind},
  $\p^{-1} \p \ne \p$. Now $R$ being $1$-dimensional
  implies that $\p$ is maximal, so we must have
  $\p^{-1} \p = R$.
\end{proof}

\begin{prop}
  A Dedekind domain $R$ admits unique factorization
  of ideals into prime ideals.
\end{prop}

\begin{proof}
  For uniqueness, suppose that
  $I = \p_1 \cdots \p_r = \q_1 \cdots \q_s$.
  Then $\q_1 \dots \q_s \subseteq \p_1$, so we must
  have some $\q_i \subseteq \p_1$. Without loss
  of generality, assume $\q_1 \subseteq \p_1$, so
  that $\q_1 = \p_1$. Now multiplying by
  $\p_1^{-1}$, we get
  \[
    \p_2 \dots \p_r = \q_2 \dots \q_s.
  \]
  Proceeding by induction finishes the proof for
  uniqueness.

  Now we argue for existence. Let $\Sigma$ be the set
  of all proper ideals of $R$ which cannot be written
  as a product of prime ideals. If $\Sigma$ is
  nonempty, then the Noetherian property of $R$ implies
  that $\Sigma$ has a maximal element $J$. Then
  $J \subsetneq \p$ for some maximal ideal $\p$, which
  is equivalently a nonzero prime ideal since
  $R$ is one-dimensional. Since
  $R \subseteq \p^{-1}$, we have the chain of inclusions
  \[
    J \subsetneq J \p^{-1} \subsetneq \p \p^{-1} = R.
  \]
  Since $J$ was maximal in $\Sigma$, we must have
  $J \p^{-1} \notin \Sigma$, so we can write
  $J \p^{-1} = \p_1 \p_2 \dots \p_r$.
  But then we have $J = \p \p_1 \p_2 \dots \p_r$
  which is a contradiction with $J \in \Sigma$.
\end{proof}

\section{Ideal Class Group}

\begin{prop}
  In a Dedekind ring $R$, to contain is to divide,
  i.e. $I \subseteq J$ if and only if $J | I$.\footnote{We say that $J$ \emph{divides} $I$, written $J | I$, if $I = J J'$ for some ideal $J'$.}
\end{prop}

\begin{proof}
  $(\Rightarrow)$ If $I \subseteq J$, then
  $I J^{-1} \subseteq J J^{-1} = R$.\footnote{Note that we have technically only proved this property for prime ideals, but any ideals factors as prime ideals and we can argue via this factorization.} Then $J' = IJ^{-1}$
  is an ideal and satisfies $I = J J'$.

  $(\Leftarrow)$ This is the easier direction, verify
  this as an exercise.
\end{proof}

\begin{definition}
  Let $R$ be an integral domain.
  A \emph{fractional ideal} of $R$ is an $R$-submodule
  $J$ of $K$ such that $aJ$ is an ideal for some
  $a \in R$.
\end{definition}

\begin{exercise}
  If $I \subseteq R$ is an ideal, then show that
  $I^{-1}$ is a fractional ideal.
\end{exercise}

\begin{exercise}
  If $J$ is an $R$-submodule of $K$, then show that
  $J$ is a fractional ideal if and only if $J$ is
  finitely generated as an $R$-module.
\end{exercise}

\begin{exercise}
  Show that set of nonzero fractional ideals in a
  Dedekind domain $R$ forms a group under multiplication.
\end{exercise}

\begin{remark}
  In fact, one can actually show that
  \[
    I(R) = \{\text{nonzero fractional ideals}\}
    = \{\p_1^{k_1} \p_2^{k_2} \dots \p_r^{k_r} \mid k_i \in \Z\}.
  \]
  Due to unique factorization, this is actually the
  free abelian group on the set of nonzero prime ideals.
  We can also define
  \[
    P(R) = \{\text{principal fractional ideals}\}
    = \{aR \mid a \in K\}.
  \]
\end{remark}

\begin{definition}
  The \emph{ideal class group} of a Dedekind domain $R$
  is 
  the quotient $\Cl(R) = I(R) / P(R)$.\footnote{As a shorthand, we may write the class group of
  a number field $K$ to mean $\Cl(\mathcal{O}_K)$.}
\end{definition}

\begin{exercise}
  Show that $\Cl(R)$ is also the equivalence classes
  of ideals under $\sim$, where $I \sim J$ if
  there exist $a, b \in R$ such that $aI = bJ$.
\end{exercise}


\begin{remark}
  Our goal now will be to show that if $R = \OO_K$
  and $[K : \Q] < \infty$, then $\Cl(R)$ is finite.
  The key tool will be the norm
  $N : \{\text{ideals of R}\} \to \N$, where
  $\N$ contains $0$.
\end{remark}

\begin{definition}
  We define the \emph{norm} of an ideal $I \subseteq R$
  to be $N(I) = |R/I|$.
\end{definition}

\begin{remark}
  To prove the finiteness of $\Cl(\OO_K)$ where
  $K$ is a number field,
  we will need to show the following properties of the
  norm $N$:
  \begin{itemize}
    \item $N((\alpha)) = N^{K}_\Q(\alpha)$.
    \item $N(IJ) = N(I) N(J)$.
  \end{itemize}
  Then, we will proceed to show the following:
  \begin{itemize}
    \item There exists $M \ge 0$ such that
      $\{\text{ideals } I \mid N(I) \le M\}$ is finite.
    \item Letting $\nu(I) = \min_{\alpha \in I} \{N(I) / N(\alpha)\}$,
      there exists $M$ such that
      $\nu(I) \le M$ for every $I$.
      Moreover, $\nu(I) = 1$ if and only if
      $I$ is principal.
      Note that
      $\nu(I) \in \Z$ by the multiplicative property
      of $N$.
  \end{itemize}
\end{remark}

\section{Discriminants}

\begin{definition}
  Let $L / K$ be a finite separable field extension,
  where $[L : K] = n$.
  Fix a Galois closure $M$ of $L / K$, so there are
  $n$ distinct embeddings
  $\sigma_1, \dots, \sigma_n : L \to M$ fixing $K$. The
  \emph{norm} of $\alpha \in L$ is
  \[
    N^L_{K} = \sigma_1(\alpha) \dots \sigma_n(\alpha) \in K.
  \]
  Now let $\alpha_1, \dots, \alpha_n \in L$.
  The \emph{discriminant} of $\alpha_1, \dots, \alpha_n$
  is
  \[
    \Delta(\alpha_1, \dots, \alpha_n) = \det
    \begin{bmatrix}
      \sigma_1(\alpha_1) & \cdots & \sigma_1(\alpha_n) \\
      \vdots & \ddots & \vdots \\
      \sigma_n(\alpha_1) & \cdots & \sigma_n(\alpha_n)
    \end{bmatrix}^2 = (\det T)^2.
  \]
\end{definition}

\begin{lemma}
  For $\alpha_1, \dots, \alpha_n \in L$,
  the discriminant
  $\Delta(\alpha_1, \dots, \alpha_n) \in K$
  and is nonzero if and
  only if $\alpha_1, \dots, \alpha_n$ form a basis
  for $L / K$.
\end{lemma}

\begin{proof}
  $(\Rightarrow)$ One can show the contrapositive
  that if $\alpha_1, \dots, \alpha_n$ are linearly
  dependent, then $\Delta = 0$.

  $(\Leftarrow)$ Let $\alpha_1, \dots, \alpha_n$
  be a basis for $L / K$.
  By the primitive element theorem, there exists
  $\theta \in L$ such that $L = K(\theta)$, so
  that $1, \theta, \theta^2, \dots, \theta^{n - 1}$
  form a basis for $L / K$. Then we have
  \[
    \begin{bmatrix}
      \alpha_1 \\ \vdots \\ \alpha_n
    \end{bmatrix}
    = M
    \begin{bmatrix}
      1 \\ \vdots \\ \theta^{n - 1}
    \end{bmatrix}
  \]
  for some matrix $M \in M_{n \times n}(K)$ with
  $\det M \ne 0$.
  This implies that
  \[
    \begin{bmatrix}
      \sigma_i(\alpha_1)  \\ \vdots \\ \sigma_i(\alpha_n)
    \end{bmatrix}
    = M
    \begin{bmatrix}
      1 \\ \vdots \\ \sigma_i(\theta^{n - 1})
    \end{bmatrix}.
  \]
  Thus if we define
  \[
    T' =
    \begin{bmatrix}
      \sigma_1(1) & \cdots & \sigma_1(\theta^{n - 1}) \\
      \vdots & \ddots & \vdots \\
      \sigma_n(1) & \cdots & \sigma_n(\theta^{n - 1})
    \end{bmatrix}
    =
    \begin{bmatrix}
      \sigma_1(1) & \cdots & \sigma_1(\theta)^{n - 1} \\
      \vdots & \ddots & \vdots \\
      \sigma_n(1) & \cdots & \sigma_n(\theta)^{n - 1}
    \end{bmatrix}
  \]
  and $\Delta' = (\det T')^2$, then
  $T = T' M^t$ implies $\Delta = \Delta' (\det M)^2$.
  Now $T'$ is a Vandermonde matrix, so
  \[
    (\det T')^2 = \prod_{i \ne j} (\sigma_i(\theta) - \sigma_j(\theta)) \ne 0.
  \]
  We can also see $\Delta' = (\det T')^2 \in K^\times$ (via Galois theory)
  and $(\det M)^2 \in K^\times$, so
  $\Delta \in K^\times$ as well.
\end{proof}

\begin{theorem}
  Let $K$ be a number field and $\alpha \in \OO_K$.
  Then $N((\alpha)) = N(\alpha)$.
\end{theorem}

\begin{proof}
  Let $\omega_1, \dots, \omega_n$ be a $\Z$-basis for
  $\OO_K$, and let
  $\alpha_i = \alpha \omega_i$. Then
  $\alpha_1, \dots, \alpha_n$ is a $\Z$-basis for
  $\mathfrak{a} = (\alpha)$. Thus we may write
  \[
    \begin{bmatrix}
      \alpha_1 \\ \vdots \\ \alpha_n
    \end{bmatrix}
    = A
    \begin{bmatrix}
      \omega_1 \\ \vdots \\ \omega_n
    \end{bmatrix}
  \]
  for some matrix $A \in M_{n \times n}(\Z)$.
  Now the theory of finitely generated modules over
  a PID implies that $N(\mathfrak{a}) = |{\det A}|$. (This
  is because we have two free $\Z$-modules of
  rank $n$: $\mathcal{O}_K$ and $\mathfrak{a} \subseteq \OO_K$. So if $A \sim A'$ where $A' = \diag(d_1, \dots, d_n)$ is in Smith normal form, then
  $|\OO_K / \mathfrak{a}| = |(\Z / d_1) \times \dots \times (\Z / d_n)|$, so we see that
  $N(\mathfrak{a}) = |\OO_K / \mathfrak{a}| = |d_1 \dots d_n| = |{\det A'}| = |{\det A}|$.)
  Thus we have
  \[
    \Delta(\alpha_1, \dots, \alpha_n)
    = (\det A)^2 \Delta(\omega_1, \dots, \omega_n).
  \]
  But we can also see that
  \begin{align*}
    \Delta(\alpha_1, \dots, \alpha_n)
    = \Delta(\alpha \omega_1, \dots, \alpha \omega_n)
    &= \det
    \begin{bmatrix}
      \sigma_1(\alpha \omega_1) & \cdots & \sigma_1(\alpha \omega_n) \\
      \vdots & \ddots & \vdots \\
      \sigma_n(\alpha \omega_1) & \cdots & \sigma_n(\alpha \omega_n)
    \end{bmatrix}^2 \\
    &= (\sigma_1(\alpha) \dots \sigma_n(\alpha))^2 \Delta(\omega_1, \dots, \omega_n)
    = N(\alpha)^2 \Delta(\omega_1, \dots, \omega_n).
  \end{align*}
  This shows that $N(\mathfrak{a})^2 = (\det A)^2 = N(\alpha)^2$,
  so that $N(\mathfrak{a}) = N(\alpha)$
  since these values are positive.
\end{proof}
