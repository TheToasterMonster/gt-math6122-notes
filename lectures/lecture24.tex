\chapter{Apr.~10 --- Guest Lecture: Gauss Sums}

\section{Quadratic Gauss Sums}

\begin{definition}
  Let $p$ be a prime. A \emph{quadratic Gauss sum}
  is a sum of the form
  \[
    G_p = \sum_{n = 1}^p \left(\frac{n^2}{p}\right), \quad e(x) = e^{2\pi ix},\, x \in \R.
  \]
\end{definition}

\begin{exercise}
  Show that $|G_{p}|^2 = p$.
\end{exercise}

\begin{remark}
  Gauss famously determined that $G_p$
  only takes two values:
  \[
    G_p =
    \begin{cases}
      \sqrt{p} & \text{if $p \equiv 1 (4)$}, \\
      i \sqrt{p} & \text{if $p \equiv 3 (4)$}.
    \end{cases}
  \]
  Note that $G_p$ are really $p$-adic
  analogues of the Gaussian integral:
  \[
    \int_0^\infty e\left(\frac{a}{2} u^2 + bu\right) du, \quad a \in \Z \setminus \{0\},\, b \in \Z.
  \]
  The Gaussian integral is at the infinite
  place, whereas $G_p$ are at the $p$-adic
  places. Dirichlet used Poisson summation
  (from Fourier analysis) to evaluate $G_p$
  using the Gaussian integral.
\end{remark}

\section{Cubic Gauss Sums}

\begin{definition}
  Let $p \equiv 1 \Pmod{3}$ be a prime.
  A \emph{cubic Gauss sum} is a sum of the form
  \[
    S_p = \sum_{n = 1}^p e\left(\frac{n^3}{p}\right).
  \]
  We work in the Eisenstein fields
  $\Z[\omega] \subseteq \Q(\omega)$, where
  $\omega = e^{2\pi i / 3}$ satisfies
  $\omega^2 + \omega + 1 = 0$. Note that
  \[
    p = \pi \overline{\pi}, \quad \pi \text{ a prime in } \Q(\omega).
  \]
\end{definition}

\begin{remark}
  One can see that
  \[
    \frac{S_p}{2\sqrt{p}}
     = \re \widetilde{g}_3(\pi), \quad
     \widetilde{g}_3(\pi) = \frac{1}{\sqrt{N(\pi)}} \sum_{x \Pmod{\pi}} \left(\frac{x}{\pi}\right)_3 \check{e}\left(\frac{x}{\pi}\right),\, \check{e}(x) = e^{2\pi i (z + \overline{z})}.
  \]
  Kummer in 1846 observed that for $p \le 500$,
  the values of $\widetilde{g}_3(\pi)$
  are distributed in a ratio of $1 : 2 : 3$
  for the left quarter of the unit circle,
  the top and bottom quarters, and the right quarter.
  After Kummer, however, the problem remained
  untouched for around 100 years.

  Later on, the ENIAC (a 1700-vacuum tube
  computer) at the IAS changed the game.
  In 1953, von Neumann, Goldstine, and Selberg
  computed the ratio to be $2 : 3 : 4$.
  Lehmer in 1956 changed this to $3 : 4 : 5$,
  and Cassels in 1969 led mathematicians
  to believe that the values were equidistributed.
\end{remark}

\begin{conjecture}[Folklore]
  The cubic Gauss sums $\widetilde{g}_3(\pi)$
  are equidistributed on the unit circle
  as $N(\pi) \to \infty$, for
  $\pi \equiv 1 \Pmod{3}$ prime.
\end{conjecture}

\begin{conjecture}[Patterson, 1978]
  One has
  \[
    \sum_{\substack{\pi \in \Z[\omega] \\ \pi \equiv 1 (3), N(\pi) \le x}} \widetilde{g}_3(\pi)
    \sim \frac{2(2\pi)^{2 / 3}}{5T(2 / 3)} \cdot \frac{x^{5 / 6}}{\log x}.
  \]
\end{conjecture}

\begin{remark}
  In the same paper, Patterson established
  the $x^{5 / 6}$-bias over all Eisenstein
  integers.
\end{remark}

\section{Automorphic Forms}

\begin{remark}
  One can see a connection between
  Gauss sums and automorphic forms. Kubota
  (1969, 1971) considered
  automorphic forms on the $3$-fold cover
  of $\mathrm{SL}_2(\mathbb{A}_{\Q(\omega)})$,
  i.e. functions $F : \mathbb{H}^3 \to \C$
  with $\Delta F = \lambda F$ and
  \[
    F\left(
      \begin{pmatrix}
        a & b \\
        c & d
      \end{pmatrix} w
    \right)
    \approx \left(\frac{c}{a}\right)_3 F(w), \quad
    \begin{pmatrix}
      a & b \\
      c & d
    \end{pmatrix} \in \mathrm{SL}_2(\mathbb{Z}[\omega]),
  \]
  such that $F$ is an eigenfunction
  of the Hecke operator:
  $T_{\pi^3} F = \lambda_{\pi^3} F$.
  Also let $w = (z, v) \in \mathbb{H}^3$.
\end{remark}

\begin{example}
  Consider cubic Eisenstein series $E_3(w, s)$, which
  have a pole at $s = 4 / 3$, and define
  \[
    \Theta_s(w) = \Res_{s = 4 / 3} E_3(w, s)
  \]
  to be the \emph{cubic theta function}.
  One can expand this function via
  \[
    \Theta_3(w) = \frac{3^{5 / 2}}{2} v^{2 / 3}
    + \sum_{\mu \in \Z[\omega]} \tau_3(\mu) v K_{1 / 3} (4\pi |\mu| v) \check{e}(\mu z)
  \]
  Patterson (1977) determined that $\tau_3(\mu) = \widetilde{g}_3(\mu)$
  via a Hecke converse argument. Note that this
  gives access to a functional equation.
  The $5 / 6$-bias is a consequence of the
  translation $4 / 3 - 1 / 2 = 5 / 6$.
\end{example}

\begin{theorem}[Heath-Brown-Patterson, 1979]
  The Folklore conjecture is true.
\end{theorem}

\begin{theorem}[Heath-Brown, 2000]
  For any $\epsilon > 0$, we have
  \[
    \sum_{\substack{\pi \in \Z[\omega] \\ \pi \equiv 1 (3), N(\pi) \le x}} \widetilde{g}_3(\pi)
    \ll_{\epsilon} x^{5 / 6 + \epsilon}, \quad
    \text{as $x \to \infty$}.
  \]
\end{theorem}

\begin{remark}
  Note that the Heath-Brown result falls short
  of the Patterson conjecture by $x^\epsilon$.
\end{remark}

\begin{theorem}[Dunn-Radziwitt, 2021/2024]
  Patterson's conjecture is true under GRH
  for Hecke $L$-functions over $\Q(\omega)$.
\end{theorem}

\section{Detecting Primes}
\begin{remark}
  \emph{Vaughan's identity} says that
  \[
    -\frac{\zeta'}{\zeta} = F - \zeta FG - \zeta' G + \left(-\frac{\zeta'}{\zeta} - F\right)(1 - \zeta G),
  \]
  where
  \[
    F(s) = \sum_{m \le u} \Lambda(m) m^{-s},
    \quad G(s) = \sum_{d \le v} \mu(d) d^{-s}.
  \]
  where $\Lambda$ is the \emph{von-Mangoldt function}.
  One can use this to write
  \[
    \Lambda = \Lambda_{\le u} - u_{\le v} * \Lambda_{\le u} * 1
    + \mu_{\le v} * \log + \mu_{> v} * \Lambda_{> u} * 1.
  \]
  This means that we can convert sums over
  primes to multilinear sums, i.e. sums of the
  form
  \[
    \sum^*_{\substack{N(a) \sim A \\ N(b) \sim B}} \alpha_a \beta_b \widetilde{g}_3(ab),
    \quad \alpha_a, \beta_b \in \C,\, |\alpha_a|, |\beta_b| \le 1.
  \]
  Note that $\widetilde{g}_3$ is not multiplicative, but we have
  \[
    \widetilde{g}_3(ab) = \widetilde{g}_3(a) \widetilde{g}_3(b) \overline{\left(\frac{a}{b}\right)_3}.
  \]
  By absorbing $\widetilde{g}_3(a), \widetilde{g}_3(b)$, we can convert the sums to be of the
  form
  \[
    \sum^*_{\substack{N(a) \sim A \\ N(b) \sim B}} \alpha_a \beta_b \left(\frac{a}{b}\right)_3.
  \]
  Applying Cauchy-Schwarz, one ends up with a
  sum of the form
  \[
    \sum_{N(a) \sim A}^* \left|\sum_{N(b) \sim B}^* \beta_b \left(\frac{a}{b}\right)_3\right|^2.
    \tag{$*$}
  \]
  One can approach this with a \emph{cubic large sieve}, and
  Heath-Brown showed that
  \[
    \sum_{N(a) \sim A}^* \left|\sum_{N(b) \sim B}^* \beta_b \left(\frac{a}{b}\right)_3\right|^2
    \ll (AB)^{\epsilon} (A + B + (AB)^{2 / 3}) \|\beta\|_{2}^2,
  \]
  where the $A$ in the middle term comes from
  square-root cancellation and $B$ comes
  from orthogonality. The $(AB)^{2 / 3}$
  term was the reason that Heath-Brown was
  stuck at $5 / 6 + \epsilon$. For a long
  time, people believed that
  $(AB)^{2 / 3}$ could be improved, but
  it turns out that it is actually sharp.
  Instead, to achieve $5 / 6$, one must
  remove a main term in the inner sum of $(*)$.
\end{remark}
