\chapter{Mar.~4 --- More Galois Theory}

\section{Frobenius Elements}
\begin{remark}
  Recall that last class, we showed that
  $|D_{\q}| = e_{\q} f_{\q} = ef$,
  $|I_{\q}| = e_{\q} = e$, and
  $D_{\q} / I_{\q} \cong \Gal(\ell / k)$.
  This is
  useful because $\Gal(L / K)$ is in general a complicated
  group; in fact, any finite group can
  arise as $\Gal(L / K)$ for some extension
  $L / K$. On the other hand, $\ell, k$ are
  finite fields, and thus $\Gal(\ell / k)$
  is cyclic. In particular, $\Gal(\ell / k) = \langle \tau \rangle$
  where $\tau(x) = x^{|k|}$ is the
  \emph{Frobenius automorphism}.
\end{remark}

\begin{corollary}
  If $\q / \p$ is unramified (i.e.
  $p \nmid \Delta_{L / K}$, or $e_{\q} = 1$), then
  $D_{\q}$ is cyclic.
\end{corollary}

\begin{definition}
  The generator of $D_{\q}$ is called
  the \emph{Frobenius element} and is denoted
  $\Frob_{\q}$.\footnote{Note that $\Frob_{\q}$ depends on the choice of $\q$ and only makes sense when $\q$ is unramified.}
\end{definition}

\begin{remark}
  What is the dependence of $\Frob_{\q}$ on
  the choice of $\q / \p$?
\end{remark}

\begin{lemma}
  Suppose $\q, \q'$ are prime ideals of $\OO_L$ lying
  over $\p$. Choose $\sigma \in G$ such that
  $\sigma \q = \q'$ Then
  \[
    D_{\q'} = \sigma D_{\q} \sigma^{-1}
    \quad \text{and} \quad
    I_{\q'} = \sigma I_{\q} \sigma^{-1}.
  \]
  If $\q / \p$ is unramified, then so is
  $\q' / \p$ and
  $\Frob_{\q'} = \sigma \Frob_{\q} \sigma^{-1}$.
\end{lemma}

\begin{proof}
  We will prove the statement only for $D_{\q'}$, the
  rest is left as an exercise. We can write
  \begin{align*}
    D_{\q'}
    = \{\tau \in G : \tau \q' = \q'\}
    &= \{\tau \in G : \tau \sigma \q = \sigma \q\}
    = \{\tau \in G : (\sigma^{-1} \tau \sigma) \q = \q\} \\
    &= \sigma \{\sigma^{-1} \tau \sigma : \tau \in G\} \sigma^{-1}
    = \sigma D_{\q} \sigma^{-1}.
  \end{align*}
  The proof of the other statements are similar.
\end{proof}

\begin{remark}
  When $L / K$ is Galois and $\p$ is unramified, the
  above lemma allows us to define
  the Frobenius conjugacy class $\Frob_{\p} \subseteq \Gal(L / K)$ (the set of all $\Frob_{\q}$ where $\q / \p$
  is unramified).
\end{remark}

\begin{example}
  Consider $K / \Q$, let $G = \Gal(K / \Q)$ and
  $\Frob_p$ be the conjugacy class corresponding
  to a prime $p$ not dividing $\Delta_K$. The
  \emph{Chebotarev density theorem} then says that
  if we fix a conjugacy class $C \subseteq G$, then
  the density of primes $p$ with $\Frob_p = C$ is
  $|C| / |G|$.
\end{example}

\begin{example}
  Consider $K = \Q(\zeta_n)$, so $G \cong (\Z / n\Z)^\times$
  is abelian (so that the Frobenius conjugacy classes
  are singletons and can therefore be identified
  with elements). If we fix $\sigma_a \in G$
  with $(a, n) = 1$, then we will show shortly
  that $\Frob_p = \sigma_p$ for $p \nmid n$.
  So if we fix a conjugacy class (element)
  $a \in (\Z / n\Z)^\times$, then $\Frob_p = a$
  if and only if $p \equiv a \Pmod{n}$. The Chebotarev
  density theorem then says that the density of
  such $p$ is $1 / \phi(n)$ (this is
  Dirichlet's theorem on primes in arithmetic
  progressions).
\end{example}

\section{Fixed Fields}

\begin{remark}
  Let $L / K$ be Galois and fix $\q / \p$.
  Let $D = D_{\q / \p}$ and $I = I_{\q / \p}$. Then for
  an intermediate field $K \subseteq K' \subseteq L$
  (e.g. $K = L^H$, the fixed field of some subgroup
  $H \subseteq G$), we have some $\p'$ lying over $\p$:
  \begin{center}
    \begin{tikzcd}
      L \ar[d, no head] & \q \ar[d, no head] \\
      L^H = K' \ar[d, no head] & \p' = \q \cap \OO_{K'} \ar[d, no head] \\
      K & \p
    \end{tikzcd}
  \end{center}
\end{remark}

\begin{lemma}
  Let $D' = D_{\q / \p'}$ and
  $I' = I_{\q / \p'}$.
  Then $D' = D \cap H$ and $I' = I \cap H$.
\end{lemma}

\begin{proof}
  We only prove the statement for $D'$, the other for
  $I'$ is similar. Note that
  \[
    D = \text{stabilizer of $\q$ in $G$} \quad \text{and} \quad
    D' = \text{stabilizer of $\q$ in $H$},
  \]
  from which it is clear that $D' = D \cap H$.
\end{proof}

\begin{remark}
  The following is a fact from Galois theory that
  we will use:
  \begin{quote}
    If $H, H'$ are subgroups of $G = \Gal(L / K)$,
    then $L^{H \cap H'} = L^H L^{H'}$.
  \end{quote}
  In the above, $L^H L^{H'}$ is compositum of
  $L^H$ and $L^{H'}$.
\end{remark}

\begin{prop}
  Call $L^D$ the \emph{decomposition field} and
  $L^I$ the \emph{inertia field}. Then
  \begin{enumerate}
    \item $L^D$ is the largest intermediate field $K'$
      such that $e(\p' / \p) = f(\p' / \p) = 1$;
    \item $L^I$ is the largest intermediate field $K'$
      such that $e(\p' / \p) = 1$.
  \end{enumerate}
\end{prop}

\begin{proof}
  Again, we only prove the statement for $L^D$.
  We first claim that if $K' = L^D$, then we in fact
  have
  $e(\p' / \p) = f(\p' / \p) = 1$. By the lemma,
  we have $D' = D$, so we have
  \[
    e(\q / \p') f(\q / \p')
    = e(\q / \p) f(\q / \p).
  \]
  But by a homework problem, $e$ and $f$ are
  multiplicative in towers, so
  $e(\p' / \p) = f(\p' / \p) = 1$.

  To see that it is the largest such extension, let
  $K'$ be any intermediate field with
  $e(\p' / \p) = f(\p' / \p) = 1$. We want to show
  that $K' \subseteq L^D$. We can write $K' = L^H$
  for some $H \le G$. Then $D' = D \cap H$, so
  \[
    L^{D'} = L^D K'
  \]
  by Galois theory. The hypothesis implies $|D| = |D'|$,
  which combined with $D' \subseteq D$ implies
  $D' = D$. Then we have
  $L^D = L^{D'} = L^D K'$, which implies that
  $K' \subseteq L^D$.
  This proves the result.
\end{proof}

\begin{corollary}
  Let $\p \subseteq \OO_K \subseteq K \subseteq M, M' \subseteq L$ (not necessarily Galois).
  Then $\p$ splits completely (resp. is unramified)
  in both $M$ and $M'$ if and only if
  $\p$ splits completely (resp. is unramified) in $M M'$.
\end{corollary}

\begin{proof}
  Take a Galois closure $L'$ of $L / K$. If
  $\p$ splits completely in $L$, then $e = 1$ in
  both $K$ and $L$, so $e = 1$ in $M, M'$ as well since
  $e$ is multiplicative in towers.
  On the other hand, if $\p$ splits completely
  in both $M$ and $M'$, then $(L')^{D}$ contains
  both $M$ and $M'$, so $M M' \subseteq (L')^{D}$
  and $\p$ splits completely in $M M'$.
\end{proof}

\begin{corollary}
  Suppose $L / K$ is finite (but not necessarily Galois).
  Let $M$ be the Galois closure of $L / K$.
  Then $\p$ is unramified (resp. splits completely)
  in $L$ if and only if $\p$ is unramified (resp. splits
  completely) in $M$.
\end{corollary}

\begin{proof}
  This is because $M$ is the compositum of $L^\sigma$
  over $\sigma \in \Gal(M / K)$.
\end{proof}

\begin{remark}
  The above corollaries imply that we can reduce
  many questions to the Galois case (many extensions
  of number fields, e.g. $\Q(\sqrt[3]{2})$, are not
  Galois, but we can still work with them by embedding
  them in a Galois closure).
\end{remark}

\section{A Non-Monogenic Number Ring}
\begin{remark}
  We will show that if $K = \Q(\sqrt{7}, \sqrt{10})$,
  then $\OO_K$ is not monogenic, i.e. there is
  no $\alpha \in \OO_K$ such that $\OO_K = \Z[\alpha]$.
  In fact, we will show that for all $\alpha \in \OO_K$
  such that $K = \Q(\alpha)$, we have
  $3 \mid [\OO_K : \Z[\alpha]]$. In particular, this
  means that Kummer's theorem will not help us factor
  $(3)$ in $\OO_K$, even when trying some change of
  variables.

  We will actually do this more generally:
  Let $K_1 = \Q(\sqrt{d_1})$ and $K_2 = \Q(\sqrt{d_2})$
  with $d_1, d_2 \equiv 1 \Pmod{3}$. Kummer's theorem
  says that $3$ splits completely in $K_i$ ($i = 1, 2$)
  if and only if
  \[x^2 - d_i \equiv x^2 - 1 \Pmod{3}\]
  splits into linear factors (which it does). By our
  previous results, this
  implies that $3$ splits completely in the compositum
  $K = K_1 K_2$.

  Now suppose otherwise that $\OO_K = \Z[\alpha]$
  (or even $3 \nmid [\OO_K : \Z[\alpha]]$).
  Then Kummer's theorem applies to $K$ as well, so
  the minimal polynomial $f_\alpha$ of $\alpha$
  over $\Z$ factors into distinct linear factors mod $3$.
  But $\deg f_{\alpha} = 4$ and there are only
  $3$ linear polynomials in $\F_3[x]$, so this cannot
  possibly happen.
\end{remark}
