\chapter{Mar.~11 --- The \texorpdfstring{$p$}{p}-adic Numbers}

\section{The \texorpdfstring{$p$}{p}-adic Absolute Value}

\begin{definition}
  Let $p$ be a prime number.
  For $x = a / b \in \Q$ with $a, b \in \Z$, define
  the \emph{$p$-adic valuation}
  \[
    \ord_p(x) = \ord_p(a) - \ord_p(b).
  \]
  Recall that $\ord_p(a) = k$ if and only if
  $a = p^k m$ with $p \nmid m$.
  This defines a \emph{valuation} $v_p = \ord_p$ on $\Q$:
  \begin{enumerate}
    \item $v_p(0) = \infty$ and $v_p(x) \in \Z$ for
      $x \ne 0$;
    \item $v_p(xy) = v_p(x) + v_p(y)$;
    \item $v_p(x + y) \ge \min\{v_p(x), v_p(y)\}$.
  \end{enumerate}
  Also define the \emph{$p$-adic absolute value}
  $| \cdot |_{p} : \Q \to \R_{\ge 0}$ by
  $|x|_p = p^{-\ord_p(x)}$, which satisfies
  \begin{enumerate}
    \item $|0|_p = 0$ and $|x|_p > 0$ for $x \ne 0$;
    \item $|xy|_p = |x|_p |y|_p$;
    \item $|x - y|_p \le \max\{|x|_p, |y|_p\}$
      (the \emph{ultrametric inequality} or
      \emph{non-Archimedean triangle equality}).
  \end{enumerate}
  The above properties define an \emph{ultrametric}, and
  a set with an ultrametric is called an
  \emph{ultrametric space}.
\end{definition}

\begin{example}
  Let $p = 3$, then
  $|1 / 54|_{3} = 27$ and
  $|54|_3 = 1 / 27$.
\end{example}

\begin{exercise}
  Check the following properties:
  \begin{enumerate}
    \item If $|x|_p \ne |y|_p$, then
      $|x - y|_p = \max\{|x|_p, |y|_p\}$.
    \item In any ultrametric space, all triangles
      are isosceles (in fact, the maximum occurs
      at least twice).
    \item Any two closed discs in an ultrametric space
      are either disjoint or one contains the other.
  \end{enumerate}
\end{exercise}

\section{Completion of a Metric Space}

\begin{remark}
  Recall that if $(X, d)$ is a metric space, a
  \emph{Cauchy sequence} is a sequence $\{x_n\}$
  such that for every $\epsilon > 0$, there exists
  $N$ such that if $n, m \ge N$, then
  $d(x_n, x_m) < \epsilon$.
\end{remark}

\begin{definition}
  A metric space $X$ is \emph{complete} if every
  Cauchy sequence converges.
\end{definition}

\begin{definition}
  Define two Cauchy sequences $\{x_n\}$ and $\{y_n\}$
  to be \emph{equivalent} if they become arbitrarily
  close to each other, i.e.
  $d(x_n, y_n) \to 0$ as $n \to \infty$.
\end{definition}

\begin{definition}
  The \emph{completion} $(\widehat{X}, d)$
  of a metric space $(X, d)$ is the set of
  equivalence classes of Cauchy sequences in $X$.
  There is a natural (continuous)
  embedding $X \hookrightarrow \widehat{X}$ by
  $x \mapsto [(x, x, x, \dots)]$.
\end{definition}

\begin{prop}
  If $X = R$ is a ring and
  $|\cdot|$ is a norm on $R$ inducing $d$,
  i.e. $d(x, y) = |x - y|$ and $|xy| = |x| |y|$, then
  $\widehat{X}$ is naturally a ring $\widehat{R}$
  and $R \to \widehat{R}$ is a ring homomorphism.
\end{prop}

\begin{prop}[University property of the completion]
  Let $X$ be a metric space and $i : X \to \widehat{X}$
  be the natural embedding. If $Y$ is any complete
  metric space and $f : X \to Y$ is a continuous map,
  then there exists a unique continuous map
  $\widetilde{f} : \widehat{X} \to Y$ such that
  the following diagram commutes:
  \begin{center}
    \begin{tikzcd}
      & Y \\
      X \ar[r, swap, "i"] \ar[ur, "f"] & \widehat{X} \ar[u, dashed, swap, "\widetilde{f}"]
    \end{tikzcd}
  \end{center}
\end{prop}

\section{The \texorpdfstring{$p$}{p}-adic Numbers}

\begin{definition}
  Define $\Z_p = \widehat{(\Z, |\cdot|_p)}$ to be
  the ring of \emph{$p$-adic integers} and
  $\Q_p = \widehat{(\Q, |\cdot|_p)}$ to be the
  field of \emph{$p$-adic numbers}.
\end{definition}

\begin{theorem}
  Every $p$-adic integer $x \in \Z_p$ can be written
  uniquely as
  \[
    x = b_0 + b_1 p + b_2 p^2 + \dots, \quad
    b_i \in \{0, 1, \dots, p - 1\}.
  \]
\end{theorem}

\begin{example}
  Consider
  \[
    x = 1 + 1 \cdot 3 + 2 \cdot 3^2 + 2 \cdot 3^4 + O(3^5) \quad \text{and} \quad
    y = 1 \cdot 3 + 2 \cdot 3^2 + 1 \cdot 3^4 + O(3^5).
  \]
  Then we can compute
  \[
    x + y = 1 + 2 \cdot 3 + 1 \cdot 3^2 + 1 \cdot 3^3
    + O(3^5).
  \]
  Multiplication is similar:
  \[
    xy = (1 + 1 \cdot 3 + 2 \cdot 3^2 + O(3^4))(1 \cdot 3 + 2 \cdot 3^2 + O(3^4))
    = 1 \cdot 3 + 2 \cdot 3^3 + O(3^4).
  \]
\end{example}

\begin{theorem}
  Every $p$-adic number $x \in \Q_p$ can be written
  uniquely as
  \[
    x = b_{-k} p^{-k} + \dots + b_{-1} p^{-1}
    + b_0 + b_1 p + b_2 p^2 + \dots,
    \quad b_i \in \{0, 1, \dots, p - 1\},
  \]
  where $k \in \Z$ and $b_{-k} \ne 0$. In this form,
  we have $v_p(x) = -k$ and $|x|_p = p^k$.
\end{theorem}

\begin{definition}
  Consider an \emph{inverse system of homomorphisms}
  \begin{center}
    \begin{tikzcd}
      \cdots \ar[r] & R_3 \ar[r, "f_3"] & R_2 \ar[r, "f_2"] & R_1
    \end{tikzcd}
  \end{center}
  Then the \emph{inverse limit} (or \emph{projective limit}, or \emph{limit}) is the set of
  \emph{coherent sequences}
  \[\widehat{R} = \varprojlim R_n
  = \{\{x_n\}_{n \ge 1} : x_i \in R_i, f_i(x_i) = x_{i - 1}\}.\]
\end{definition}

\begin{remark}
  There is also a notion of \emph{direct limit}
  (or \emph{inductive limit}, or \emph{colimit}),
  denoted $\varinjlim$.
\end{remark}

\begin{remark}
  There is an obvious inverse system of homomorphisms (reduction mod $p^k$)
  \begin{center}
    \begin{tikzcd}
      \cdots \ar[r] & \Z / p^3 \Z \ar[r] & \Z / p^2 \Z \ar[r] & \Z / p \Z
    \end{tikzcd}
  \end{center}
  Then $\Z_p$ (as previously defined) coincides
  with the inverse limit:
  $\Z_p = \varprojlim (\Z / p^n \Z)$. We can also
  define $\Q_p = \Frac(\Z_p) = \Z_p \otimes_{\Z} \Q$,
  but $\Q_p$ is not an inverse limit.
\end{remark}

\begin{lemma}
  $\Z_p = \{x \in \Q_p : |x|_p \le 1\} = \{x \in \Q_p : v_p(x) \ge 0\}$, i.e.
  $\Z_p$ is the valuation ring of $\Q_p$.
\end{lemma}

\begin{lemma}
  $\Z_p$ is a local ring with maximal ideal
  $p \Z_p = \{x \in \Z_p : |x|_p < 1\} = \{x \in \Z_p : |x|_p \le 1 / p\}$.
  In particular, $p \Z_p$ is both open and closed
  (clopen) in $\Z_p$.
\end{lemma}

\begin{remark}
  One can think of $\Z_p$ as a tree. For
  $p = 2$, we can think of $\Z_2$ as an infinite binary
  tree, where the edges on the $i$th level of the tree
  corresponds to residues mod $2^i$. The
  $2$-adic integers are then the \emph{ends} of this
  tree, i.e. an infinite rooted path. Viewed in
  this way, the $2$-adic integers are homeomorphic
  to the (middle-thirds) Cantor set, hence $\Z_2$ is
  totally disconnected.
\end{remark}
