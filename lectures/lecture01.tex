\chapter{Jan.~7 --- Motivation for Algebraic Number Theory}

\section{Motivation: Fermat's Last Theorem}

\begin{theorem}[Fermat's last theorem\footnote{This problem was finally resolved by Wiles-Taylor in 1995.}]
  $x^n + y^n = z^n$
  has no nonzero integer solutions when $n \ge 3$.
\end{theorem}

\begin{remark}
  The $n = 3$ case was likely solved by Fermat, and
  Euler and Gauss had work for $n = 4$. So we will
  assume
  $n \ge 5$. We can also assume $n$ is prime, since
  if $n = pm$, then we can instead consider
  \[
    (x^m)^p + (y^m)^p = (z^m)^p.
  \]
  Thus any nonzero solution to $x^n + y^n = z^n$
  also yields a nonzero solution to $x^p + y^p = z^p$.
  So let $p \ge 5$ be prime, and let
  $\zeta = \zeta_p$ be a primitive $p$th root of $1$.
  Then consider
  \[
    x^p + y^p = (x + y) (x + \zeta y) (x + \zeta^2 y) \dots (x + \zeta^{p-1} y) = z^p.
  \]
  Note that $x + \zeta^j y \in \Z[\zeta] \subseteq \C$.
  Let us pretend for the moment that $\Z[\zeta]$ is
  a UFD.\footnote{It is far from it, and this is likely the mistake that Fermat originally made.}
  One can check that
  \[
    \gcd(x + \zeta^j y, x + \zeta^k y) = 1
  \]
  whenever $j \ne k$. If $\Z[\zeta]$ were a UFD, then
  we could conclude that
  \[
    x + y \zeta = u \alpha^p
  \]
  for some $u \in \Z[\zeta]^\times$ and
  $\alpha \in \Z[\zeta]$.\footnote{In a UFD, if a product of relatively prime elements is a $p$th power, then each factor must itself be a $p$th power.}
  For the sake of illustration, suppose $u = \pm \zeta^j$
  for some $j$.
  Then
  \[
    \alpha = a_0 + a_1 \zeta + \dots + a_{p - 2} \zeta^{p - 2}
  \]
  for $a_i \in \Z$. This gives
  \[
    \alpha^p = a_0 + a_1 + \dots + a_{p - 2} \pmod{p},
  \]
  using Fermat's little theorem, $\zeta^p = 1$,
  and the binomial theorem. So
  $\alpha^p = a \text{ (mod $p$)}$ with $z \in \Z$,
  and
  \[
    x + y \zeta = \pm a \zeta^j \pmod{p}
  \]
  for some $0 \le j \le p - 1$. Note that
  $\zeta^{p - 1} = -(1 + \zeta + \dots + \zeta^{p - 2})$,
  and one can check as an exercise that this implies
  $p | x$ or $p | y$. This would have proved the
  ``first case'' of Fermat's last theorem.
\end{remark}

\begin{remark}
  However, Kummer (c.~1850) observed that $\Z[\zeta]$ is
  rarely a UFD (in fact, $\Z[\zeta]$ is a UFD if and
  only if $p \le 19$).\footnote{Kummer made the first real progress on Fermat's last theorem in a long time.} Also, when
  $p \ge 5$, the unit group of $\Z[\zeta]$ is
  always infinite (so that $\Z[\zeta]^\times \ne \{\pm \zeta^j\}$).
\end{remark}

\begin{theorem}[Kummer]
  Fermat's last theorem holds for all ``regular'' primes.\footnote{A prime $p$ is \emph{regular} if $p$ does not divide the order of the \emph{ideal class group} of $\Z[\zeta]$.}
\end{theorem}

\begin{remark}
  The first irregular prime is $37$, so Kummer's method
  works for $3 \le n \le 36$.
\end{remark}

\section{Algebraic Integers}

\begin{remark}
  To resolve these issues,
  Kummer realized that one can replace elements
  of $\Z[\zeta]$ by ``ideal elements.'' Later on,
  Dedekind look at Kummer's work and introduced
  the modern notion of an ideal. We will be working
  towards the \emph{unique factorization of ideals into
  prime ideals} in certain cases.
\end{remark}

\begin{remark}
  We will work at the level of generality of Dedekind
  rings (as opposed to just number rings). This is
  because there is an analogue of such a unique
  factorization of ideals
  for function fields of curves in algebraic geometry,
  and this framework is general enough to capture
  both cases.
\end{remark}

\begin{definition}
  Let $K / \Q$ be a finite extension (i.e. a \emph{number field}).
  Then $\alpha \in K$ is an \emph{algebraic integer}
  if there exists a monic polynomial $f \in \Z[x]$
  such that $f(\alpha) = 0$.
\end{definition}

\begin{theorem}
  Let $A \subseteq B$ be rings and let $b \in B$.
  Then the following are equivalent:
  \begin{enumerate}
    \item $b$ is integral over $A$ (i.e. there exists
      a monic $f \in A[x]$ such that $f(b) = 0$).
    \item $A[b]$ is a finitely generated $A$-module.\footnote{Here $A[b]$ is the smallest subring of $B$ containing $A$ and $b$, so $A[b] = \{a_0 + a_1 b + a_2 b^2 + \dots + a_k b_k : a_i \in A\}$.}
    \item $A[b]$ is contained in a subring $C \subseteq B$
      which is finitely generated as an $A$-module.
  \end{enumerate}
\end{theorem}

\begin{proof}
  $(1 \Rightarrow 2)$ This direction is standard, one
  only needs powers up to $\deg f$ since $f(b) = 0$.

  $(2 \Rightarrow 3)$ This direction is clear
  since $A[b]$ itself satisfies the desired conditions.

  $(3 \Rightarrow 1)$ The idea is to argue via
  determinants and use the Cayley-Hamilton theorem for
  modules.
\end{proof}

\begin{corollary}
  Integrality is transitive, i.e. if $B$ is integral
  over $A$ and $C$ is integral over $B$, then $C$ is
  integral over $A$.\footnote{We say that $B$ is \emph{integral over $A$} if every $b \in B$ is integral over $A$.}
\end{corollary}

\begin{proof}
  A finitely generated module over a finitely
  generated module is finitely generated.
\end{proof}

\begin{corollary}
  If $\alpha, \beta$ are integral over $A$, then
  $\alpha \pm \beta, \alpha \beta$ are also integral
  over $A$.
\end{corollary}

\begin{proof}
  This is because
  $\alpha \pm \beta, \alpha \beta \subseteq C = A[\alpha][\beta]$.
\end{proof}

\begin{theorem}
  The set of all algebraic integers in $K$ (denoted
  $\mathcal{O}_K$) forms a
  subring of $K$.\footnote{This theorem is not obvious: Given $f(\alpha) = 0$ and
  $g(\beta) = 0$, one must find a polynomial $h$ such
  that $h(\alpha + \beta) = 0$.}
\end{theorem}
