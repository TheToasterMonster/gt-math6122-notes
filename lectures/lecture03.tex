\chapter{Jan.~14 --- Unique Factorization of Ideals}

\section{Norm in a Field}

\begin{remark}
  Let $K / \Q$ be a finite extension of degree $n$.
  Our goal will be to define a \emph{norm}
  $N_{K / \Q} : K \to \Q$ which also sends
  $\mathcal{O}_K \to \Z$. Note that there are $n$
  distinct embeddings $\sigma_1, \dots, \sigma_n : K \to \C$, e.g.
  choose a primitive element $\theta \in K$
  (so that $K = \Q(\theta)$) with minimal polynomial
  $f$ of degree $n$ and define
  $\sigma : K \to \C$ by sending $\theta$ to
  some root of $f$, of which there are $n$ choices.\footnote{As an example of having $n$ embeddings, consider $\Q(\sqrt{2}) \subseteq \R \subseteq \C$, where we can send $\sqrt{2} \mapsto \pm \sqrt{2}$.}
\end{remark}

\begin{definition}
  Given a finite extension $K / \Q$, define the
  \emph{norm} $N_{K / \Q} : K \to \Q$ by
  \[
    N_{K / \Q}(x) = \prod_{i = 1}^n \sigma_i(x),
  \]
  where $\sigma_1, \dots, \sigma_n : K \to \C$
  are the $n$ distinct embeddings of $K$ into $\C$.
\end{definition}

\begin{exercise}
  Show that $N_{K / \Q}(\gamma) \in \Q$. (Hint: One way is via Galois theory.)
\end{exercise}

\begin{exercise}
  Define $[\gamma] : K \to K$ by $x \mapsto \gamma x$,
  which is a $\Q$-linear map. Show that
  $N_{K / \Q}(\gamma) = \det [\gamma]$.
\end{exercise}

\begin{prop}
  We have the following properties of the norm $N = N_{K / \Q}$:
  \begin{enumerate}
    \item $N(\gamma) = 0$ if and only if $\gamma = 0$;
    \item if $\gamma \in \mathcal{O}_K$, then
      $N(\gamma) \in \Z$.
  \end{enumerate}
\end{prop}

\begin{proof}
  Check these properties as an exercise.
\end{proof}

\begin{theorem}
  A number ring $\mathcal{O}_K$ is a complete
  lattice in $K \cong \Q^n \subseteq \R^n$.
\end{theorem}

\begin{proof}
  We need to show that $\mathcal{O}_K$ is discrete.
  Note that there exists a basis $\alpha_1, \dots, \alpha_n$
  for $K / \Q$ such that $\alpha_i \in \OO_K$
  for every $i$.
  Now suppose otherwise that $\OO_K$ is not discrete,
  so there
  are arbitrarily small $\lambda_1, \dots, \lambda_n \in \Q$
  such that $\alpha = \sum \lambda_i \alpha_i$
  is nonzero and in $\OO_K$. Then
  \[
    N_{K / \Q}(\alpha)
    = \phi(\lambda_1, \dots, \lambda_n)
  \]
  for some homogeneous polynomial $\phi$ of
  degree $n$ (since each
  $\sigma(\alpha) = \sum \lambda_i \sigma(\alpha_i)$).
  Thus if $|\lambda_i| \ll 1$, the polynomial
  $\phi$ also gets small and we can obtain
  $0 < |N_{K / \Q}(\alpha)| < 1$, a contradiction
  since $N_{K / \Q}(\alpha) \in \Z$.
\end{proof}

\begin{corollary}
  If $I \subseteq \OO_K$ is a nonzero ideal, then
  $I$ is also a complete lattice in $\R^n$.
\end{corollary}

\begin{proof}
  One needs to show that $I$ contains a basis for
  $K / \Q$. Choose any nonzero $c \in I$ and consider
  $c \alpha_1, \dots, c \alpha_n \in I$ (since $I$
  is an ideal). This will also be a basis for
  $K / \Q$ since $c \ne 0$.
\end{proof}

\begin{corollary}
  We have $|\OO_K / I| < \infty$ for every
  nonzero ideal $I \subseteq \OO_K$.
\end{corollary}

\begin{proof}
  This is because $\OO_K \cong I \cong \Z^n$
  as $\Z$-modules, so the result follows by the
  structure theorem.
\end{proof}

\begin{remark}
  These details complete the proof from last time that
  $\OO_K$ is a Dedekind domain.
\end{remark}

\begin{remark}
  The following is a preview of what we will do
  later in the class: We will define the
  \emph{norm} of an ideal to be $N(I) = |\OO_K / I|$.
  One can show that if $I = (\gamma)$, then
  $N(I) = N(\gamma)$. An extension of the previous
  techniques then lead to a proof of the finiteness
  of the \emph{ideal class group}.
\end{remark}

\section{Unique Factorization of Ideals}

\begin{remark}
  Recall that for ideals $I = (\alpha_1, \dots, \alpha_k)$
  and $J = (\beta_1, \dots, \beta_\ell)$, their
  \emph{product} is
  $IJ = (\alpha_i \beta_j)_{i, j}$.
\end{remark}

\begin{example}
  Consider $R = \Z[\sqrt{-5}]$, which is the
  ring of integers $\OO_K$ in $K = \Q(\sqrt{-5})$.
  Note that
  \[
    6 = 2(3) = (1 + \sqrt{-5})(1 - \sqrt{-5})
  \]
  and these elements are irreducible and not associates,
  so $R$ is not a UFD. However, let
  \[
    \p_1 = (2, 1 + \sqrt{-5}),
    \quad \p_2 = (2, 1 - \sqrt{-5}),
    \quad \p_3 = (3, 1 + \sqrt{-5}),
    \quad \p_4 = (3, 1 - \sqrt{-5}).
  \]
  None of these ideals are principal, but they are
  all prime ideals. One can check that
  \[
    \p_1 \p_2 = (4, 2 - 2\sqrt{-5}, 2 + 2 \sqrt{-5}, 6) = (2),
  \]
  that
  $\p_3 \p_4 = (3)$, that
  \[
    \p_1 \p_3 = (6, 2 + 2 \sqrt{-5}, 3 + 3\sqrt{-5}, 6)
    = (1 + \sqrt{-5}),
  \]
  and finally that $\p_2 \p_4 = (1 - \sqrt{-5})$. At
  the level of ideals, the original equation then becomes
  \[
    (6) = (2)(3) = (\p_1 \p_2) (\p_3 \p_4)
    = (\p_1 \p_3) (\p_2 \p_4)
    = (1 + \sqrt{-5})(1 - \sqrt{-5}).
  \]
  In fact, the previous nonunique factorization is
  now the same factorization in the language of ideals.
\end{example}

\begin{lemma}
  Let $I_1, \dots, I_n$ be ideals in a commutative
  ring $R$, and let $\p$ be a prime ideal. Suppose that
  $I_1 I_2 \dots I_n \subseteq \p$. Then
  $I_j \subseteq \p$ for some $j$.
\end{lemma}

\begin{proof}
  Check this as an exercise, it follows from the
  definition of a prime ideal.
\end{proof}

\begin{lemma}
  Let $R$ be a Noetherian ring, and $I \subseteq R$
  be a nonzero ideal. Then there exist
  nonzero prime ideals $\p_1, \dots, \p_r$ such that
  $\p_1 \p_2 \dots \p_r \subseteq I$.
\end{lemma}

\begin{proof}
  Let $\Sigma$ be the set of all $I$ for which
  the lemma is false. If $\Sigma \ne \varnothing$, then
  since $R$ is Noetherian, $\Sigma$ has a maximal
  element (pick $I_1 \in \Sigma$, if it is not maximal,
  then we can find $I_2 \in \Sigma$ with
  $I_1 \subsetneq I_2$, and we obtain
  $I_1 \subsetneq I_2 \subsetneq \dots$ by continuing;
  this chain must terminate since $R$ is Noetherian).
  Let $J$ be this maximal element. Now $J$ cannot
  be prime, so there exist $a, b \in R$ such that
  $ab \in J$ but $a, b \notin J$. Let
  \[
    \mathfrak{a} = (J, a) \supsetneq J \quad
    \text{and} \quad \mathfrak{b} = (J, b) \supsetneq J.
  \]
  Then $\mathfrak{a} \supseteq \p_1 \p_2 \dots \p_m$
  and $\mathfrak{b} \supseteq \mathfrak{q}_1 \mathfrak{q}_2 \dots \mathfrak{q}_n$.
  Since $\mathfrak{a} \mathfrak{b} = (J^2, Ja, Jb, ab) \subseteq J$, we obtain
  \[
    J \supseteq \mathfrak{a} \mathfrak{b} \supseteq \p_1 \dots \p_m \mathfrak{q}_1 \dots \mathfrak{q}_n,
  \]
  which is a contradiction. Thus we must have
  $\Sigma = \varnothing$, so the lemma holds for
  every nonzero ideal $I$.
\end{proof}

\section{Inverse Ideals}

\begin{example}
  Consider the problem of finding $(2)^{-1}$ in $\Z$.
  Logically, the answer should be something like $(1 / 2) = (1 / 2) \Z \subseteq \Q$,
  which is not an ideal in $\Z$.\footnote{Note that this is not an ideal of $\Q$ either since it is not closed under multiplication by elements of $\Q$. The inverse ideal $(2)^{-1}$ is instead a $\Z$-submodule of $\Q$.}
  This will satisfy $2 ((1 / 2)\Z) = \Z$.
\end{example}

\begin{definition}
  Let $R$ be an integral domain with fraction field $K$,
  and let $I$ be a nonzero ideal in $R$. Then
  the \emph{inverse ideal} $I^{-1}$ of $I$ is
  \[
    I^{-1} = \{x \in K \mid xI \subseteq R\}.
  \]
\end{definition}

\begin{example}
  Let $R = \Z$ and $I = (2)$. Then we can
  see that
  \[
    I^{-1} = \{x \in \Q \mid x (2) \subseteq \Z\}
    = \frac{1}{2} \Z.
  \]
\end{example}

\begin{remark}
  Our goal at this point is to show that if $R$ is
  Dedekind, then $I I^{-1} = R$. Note that if
  $M, N$ are two $R$-submodules of $K$, then their
  product is well-defined:
  \[
    MN = \text{$R$-submodule of $K$ generated by } \{xy \mid x \in M, y \in N\},
  \]
  e.g. $((1 / 2)\Z)((1 / 3)\Z) = (1 / 6)\Z$. This is
  how we will make sense of the product $II^{-1}$.
\end{remark}

\begin{lemma}
  If $I = (a)$, then $I^{-1} = (a^{-1})$ and
  $I I^{-1} = (1) = R$.
\end{lemma}

\begin{proof}
  Check this as an exercise.
\end{proof}

\begin{prop}
  If $R$ is Dedekind, $I \ne 0$ is an ideal, and
  $\p \ne 0$ is a prime ideal, then $\p^{-1} I \ne I$.
\end{prop}

\begin{proof}
  First consider the special case $I = R$, and we
  want to show that $\p^{-1} \ne R$. We will find
  $x \in \p^{-1}$ which is not in $R$. To do this,
  we will take $x = a^{-1} b = b / a$ for some $a, b \in R$.
  We want $(b / a) \p \subseteq R$, so we should look
  for $b \p \subseteq (a)$ with $b \notin (a)$.
  Let $a \in \p$ be any nonzero element, and
  we will find a suitable $b$.

  Since $R$ is Noetherian, there exists $\p_i$ such that
  $\p_i \dots \p_r \subseteq (a) \subseteq \p$.
  Without loss of generality, we can assume $r$ is minimal.
  This then implies that $\p_i \subseteq \p$ for
  some $i$, which implies $\p_i = \p$ since
  $R$ is $1$-dimensional. Assume without loss of
  generality that $i = 1$, so $\p_1 = \p$.

  If $r = 1$, then $\p = (a)$, so that
  $\p^{-1} = (a^{-1}) \ne R$ since $a$ is not a unit.
  So now assume $r \ge 2$. Then
  \[
    \p_2 \dots \p_r \not\subseteq (a)
  \]
  by the minimality of $r$, so there exists
  $b \in \p_2, \dots, \p_r$ such that $b \notin (a)$.
  But $b \p = b \p_1 \subseteq (a)$, so the element
  $x = b / a \in \p^{-1}$
  but is not in $R$. This proves that $I = R$.

  In the general case, using the hypothesis that $R$
  is Noetherian, we can write $I = (\alpha_1, \dots, \alpha_n)$.
  Assume otherwise that $\p^{-1} I = I$.
  Then for $x \in \p^{-1}$, we can write
  \[
    x \alpha_i = \sum_{j = 1}^n a_{ij} \alpha_j,
    \quad a_{ij} \in R.
  \]
  Let $A = (a_{ij})$ and define
  $T = x I_n - A$. Check as an exercise that
  $\det T = 0$. Since $\det T$ is a monic polynomial
  in $x$ with coefficients in $R$, we see that
  $x$ is integral over $R$. Since $R$ is integrally
  closed, we must have $x \in R$, so we get
  $\p^{-1} = R$. This contradicts the above special
  case.
\end{proof}

\begin{remark}
  The key idea of the proof is Cayley-Hamilton for
  modules: Let $R$ be a commutative ring and $M$ a
  finitely generated $R$-module. Then if
  $JM = M$, there exists $a$ with $1 - a \in J$
  such that $aM = M$. The proof above uses a similar
  strategy to the proof of this statement.
\end{remark}
