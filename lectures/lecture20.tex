\chapter{Mar.~13 --- Absolute Values on \texorpdfstring{$\Q$}{Q}}

\section{Absolute Values}

\begin{definition}
  Let $K$ be a field. An \emph{absolute value} on $K$
  is a map $|\cdot| : K \to \R_{\ge 0}$ such that
  \begin{enumerate}
    \item $|x| = 0$ if and only if $x = 0$;
    \item $|xy| = |x| |y|$;
    \item $|x + y| \le |x| + |y|$.
  \end{enumerate}
  If we also have $|x + y| \le \max\{|x|, |y|\}$,
  then we call $|\cdot|$
  \emph{non-Archimedean}.
\end{definition}

\begin{example}
  We have the usual absolute value $|\cdot|_{\infty}$
  given by
  $|x|_{\infty} = \max\{x, -x\}$. More generally, we
  have absolute values $|\cdot|_{\infty}^t$ for
  $0 \le t \le 1$. When $t = 0$, we have the
  trivial absolute value $|\cdot|_0$:
  \[
    |x|_0 =
    \begin{cases}
      1 & \text{if } x \ne 0, \\
      0 & \text{if } x = 0.
    \end{cases}
  \]
  For $0 < t \le 1$, the
  $|\cdot|_{\infty}^t$ are all equivalent, in the
  sense that they give the same topology on $\Q$.
\end{example}

\begin{definition}
  Two absolute values
  $|\cdot|$ and $|\cdot|'$ on $\Q$ are \emph{equivalent},
  denoted $|\cdot| \sim |\cdot|'$, if they define the
  same topology on $\Q$.
\end{definition}

\begin{example}
  We have the $p$-adic absolute values $|\cdot|_p$ for
  some prime $p$, and $|\cdot|_p^t$ for $0 < t < \infty$.
  Each of the $|\cdot|_p^t$ are equivalent to
  $|\cdot|_p$.
\end{example}

\begin{remark}
  Note that we only have $p$-adic absolute values on
  $\Q$ for prime $p$, e.g.
  $\Z_{10} = \varprojlim \Z / 10^k \Z$ is
  not an integral domain, so we cannot take its
  field of fractions. In fact,
  $\Z_{10} \cong \Z_2 \times \Z_5$, so it suffices
  to study $\Z_2$ and $\Z_5$. There is also no need
  to consider prime powers, since $\Z_9 = \Z_3$, for
  instance.
\end{remark}

\section{Classification of Absolute Values}

\begin{definition}
  A \emph{place} of $K$,
  is an equivalence class of non-trivial
  absolute values on $K$. The set of places
  of $K$ is denoted $M_K$.
\end{definition}

\begin{lemma}
  Let $K$ be a field and
  $|\cdot|, |\cdot|'$ be two non-trivial absolute values on $K$.
  Then the following are equivalent:
  \begin{enumerate}
    \item $|\cdot| \sim |\cdot|'$;
    \item $|x| < 1$ if and only if $|x|' < 1$;
    \item $|\cdot|' = |\cdot|^s$ for some $s > 0$.
  \end{enumerate}
\end{lemma}

\begin{proof}
  $(3 \Rightarrow 1)$ This is clear.

  $(1 \Rightarrow 2)$ Note that
  $|x| < 1$ if and only if $x^n \to 0$ with respect
  to $|\cdot|$.

  $(2 \Rightarrow 3)$ Fix some $y \in K^\times$ such
  that $|y| < 1$ (we can do this since the
  $|\cdot|$ is multiplicative and non-trivial).
  Note that for all $x \in K^\times$, we have
  $|x| = |y|^\alpha$ for some $\alpha \in \R$
  depending on $x$. Choose $m_i / n_i \to \alpha$
  from above, with $n_i > 0$. Then
  $|x| = |y|^\alpha < |y|^{m_i / n_i}$, so by
  $(2)$, we have
  \[
    \left|\frac{x^{n_i}}{y^{m_i}}\right| < 1
    \implies 
    \left|\frac{x^{n_i}}{y^{m_i}}\right|'' < 1
    \implies |x|' < (|y|')^{m_i / n_i}
    \text{ for all $i$}
    \implies |x|' \le (|y|')^{\alpha}.
  \]
  Letting $m_i / n_i \to \alpha$ from below,
  we get $|x|' = (|y|')^{\alpha}$. Then setting
  \[
    s = \frac{\log |y|'}{\log |y|} = \frac{\log |x|'}{\log |x|},
  \]
  we shows that $|x|' = |x|^s$ for all $x$. It is
  clear that $s > 0$ since $|y|, |y|' < 1$.
\end{proof}

\begin{remark}
  Recall the Archimedean property of $|\cdot|_{\infty}$
  (the completion with respect to which is $\R$):
  If $x \in \Q^\times$, then the sequence
  $\{|nx|\}_{n \in \N}$
  is unbounded. In the $p$-adic norm, however, we have
  \[
    |nx|_{p} = |n|_p |x|_p \le |x|_p,
  \]
  so this is a bounded sequence for any $x \in \Q^\times$.
\end{remark}

\begin{definition}
  We say that an absolute value $|\cdot|$ on a field
  $K$ is \emph{Archimedean} if $\{|n|\}_{n \in \N}$
  is unbounded, and we call $|\cdot|$ \emph{non-Archimedean}
  otherwise.
\end{definition}

\begin{lemma}
  An absolute value $|\cdot|$ on $K$ satisfies the
  ultrametric inequality $|x + y| \le \max\{|x|, |y|\}$
  if and only if $\{|n|\}_{n \in \N}$ is bounded.
\end{lemma}

\begin{proof}
  $(\Rightarrow)$ This is clear by induction, as
  $|1 + 1| \le \max\{|1|, |1|\} = |1|$.

  $(\Leftarrow)$ Suppose that $|n| \le N$, and
  note that we always have
  $|x + y| \le |x| + |y| \le 2 \max\{|x|, |y|\}$. Then
  \[
    |x + y|^n
    = \left|\sum_{k = 0}^n \binom{n}{k} x^k y^{n - k}\right|
    \le N (n + 1) \max\{|x|, |y|\}^n
  \]
  since $\binom{n}{k} \in \N$.
  Taking $n$th roots and letting $n \to \infty$, we
  get $|x + y| \le \max\{|x|, |y|\}$.
\end{proof}

\begin{theorem}[Ostrowski]
  The set of places of $\Q$ are precisely
  $|\cdot|_{\infty}$ and $|\cdot|_p$ for prime $p$, i.e.
  \[
    M_{\Q}
    = \{|\cdot|_{\infty}, |\cdot|_{p} \text{ for $p$ prime}\}
    \longleftrightarrow \{\mathrm{primes}\} \cup \{\infty\}.
  \]
\end{theorem}

\begin{proof}
  Let $|\cdot|$ be a non-trivial absolute value on $\Q$.

  First suppose that $|\cdot|$ is non-Archimedean.
  Then $|n| \le 1$ for all $n \ge 1$, and there
  exists a prime $q$ such that $|q| < 1$ (if $|q| = 1$ for
  all primes $q$,
  then $|n| = 1$ for all $n \ge 1$ and $|\cdot|$ is
  trivial). Consider the set
  \[
    I = \{n \in \Z : |n| < 1\},
  \]
  which is an ideal in $\Z$ (it is obviously closed
  under multiplication and the ultrametric inequality
  implies it is closed under addition). Note that
  $1 \notin I$ and $q \in Z$, so $I = p\Z$ for some prime $p$.
  For $n = mp^k$ with $p \nmid m$, we have
  $|m| = 1$ since $m \notin I$, so $|n| = |p|^k$.
  This gives $|n| = |n|_p^s$ for
  $s = -\log_p |p|$, so $|\cdot| \sim |\cdot|_p$.

  Now let $|\cdot|$ be Archimedean. Then we can
  choose $n, m > 1$ such that $|n|, |m| > 1$. Write $m$
  in base $n$:
  \[
    m = a_0 + a_1 n + \dots a_k n^k, \quad a_i \in \{0, \dots, n - 1\},
  \]
  where $k \le \log_n m$, and $|a_i| \le a_i |1| < n$
  since $a_i < n$. So $|n^i| < |n^k|$ for $1 \le i \le k$, and we have
  \[
    |m| \le \sum_{i = 1}^k |a_i| |n|^k
    \le (1 + \log_n m) \cdot n \cdot |n|^{\log_n m}.
  \]
  Replacing $m$ by $m^t$, taking $t$th roots and
  letting $t \to \infty$, one finds
  $|m| \le |n|^{\log_n m}$. By switching $m$ and $n$,
  we get $|m| = |n|^{\log_n m}$, so
  $|m|^{1 / \log m} = |n|^{1 / \log n}$. Thus we can
  define $s = (\log |n|) / (\log n)$, which is
  independent of $n$. If $x \in \Q$, write
  $x = m / n$, where without loss of generality
  we can assume $|m|, |n| > 1$ (by replacing
  $m, n$ with $mt, nt$, if necessary). Then we have
  \[
    |x|_{\infty}^s = e^{s \log x}
    = \frac{e^{s \log m}}{e^{s \log n}}
    = \frac{e^{\log |m|}}{e^{\log |n|}}
    = \frac{|m|}{|n|} = |x|,
  \]
  which shows that $|\cdot| = |\cdot|_{\infty}^s \sim |\cdot|_\infty$.
\end{proof}

\begin{theorem}[Ostrowski for number fields]
  Let $K$ be a number field. Then every non-trivial
  absolute value on $K$ is equivalent to one of the
  following:
  \begin{enumerate}
    \item $\p$-adic absolute value
      $|\cdot|_\p$ for some prime ideal
      $\p \subseteq \OO_K$ (defined by
      $|\alpha|_\p = (N \p)^{-\ord_\p(\alpha)}$);
    \item the absolute value given by
      $K \overset{\sigma}{\hookrightarrow} \C \overset{|\cdot|}{\to} \R_{\ge 0}$
      for some $\sigma$.
  \end{enumerate}
  In particular, we have a bijection
  \[M_K \longleftrightarrow \{\text{nonzero prime ideals of $\OO_K$}\} \cup \{\text{real or pairs of complex embeddings of $K$}\}\]
\end{theorem}

\begin{theorem}[Product formula]
  For $x \in \Q^\times$, we have
  $\prod_{v \in M_{\Q}} |x|_v = 1$. More generally,
  for a general number field $K$ and $x \in K^\times$,
  we have $\prod_{v \in M_K} |x|_v = 1$, where
  $|\cdot|_v = |\cdot|_{\C}^2$ if $v$ is complex.
\end{theorem}

\begin{theorem}[Weak approximation]
  Let $K$ be any field with pairwise inequivalent
  absolute values $|\cdot|_1, \dots, |\cdot|_n$, and
  let $a_1, \dots, a_n \in K$. Then for all
  $\epsilon > 0$, there exists $x \in K$ such that
  \[
    |x - a_i|_{i} < \epsilon, \quad \text{for all $1 \le i \le n$}.
  \]
\end{theorem}

\begin{proof}
  We only sketch the proof. The idea is the following:
  We know that there exists $\alpha \in K$ such that
  $|\alpha|_1 < 1$ and $|\alpha_n| \ge 1$ (since
  the absolute values are inequivalent). Similarly,
  there exists $\beta \in K$ such that
  $|\beta|_n < 1$ and $|\beta|_1 \ge 1$. Let
  $y = \beta / \alpha$, so that
  $|y|_1 > 1$ and $|y|_n < 1$.

  Now we induct on $n$: The above argument is the base
  case, so assume there exists $w \in K$ such that
  $|w|_1 > 1$, and $|w_j < 1$ for $j = 2, \dots, n - 1$.
  Then if $|w_n| \le 1$, take $z = w^m y$ with
  $m \gg 0$, and if $|w_n| > 1$, take $z = w^m y / (1 + w^m)$
  with $m \gg 0$. This $z$ satisfies
  $|z|_1 > 1$ and $|z|_j < 1$ for $j = 2, \dots, n$.

  Taking $m \to \infty$, we get
  \[
    \frac{z^m}{1 + z^m} \xrightarrow[m \to \infty]{}
    \begin{cases}
      1 & \text{in $|\cdot|_1$-norm}, \\
      0 & \text{in $|\cdot|_j$-norm, $j = 2, \dots, n$}, \\
    \end{cases}
  \]
  Doing this and getting $z_i$ for each $i$, we can take
  $x = a_1 z_1 + \dots + a_n z_n$.
\end{proof}
