\chapter{Apr.~22 --- Class Field Theory, Part 3}

\section{Ray Class Fields}

\begin{definition}
  Let $K$ be a number field.
  A \emph{modulus} is a formal product
  \[
    \m = \prod_{v \in M_K} v^{e_v},
    \quad e_v \in \Z_{\ge 0},
  \]
  where $e_v = 0$ for almost all $v$,
  $e_v = 0$ for $v$ complex, and
  $e_v \in \{0, 1\}$ for $v$ real.
  So $\m$ corresponds to a
  pair $(I, S)$, where $I = \prod_\p \p^{e_\p}$
  and $S$ is a subset of the real places.
\end{definition}

\begin{remark}
  Fix a modulus $\m$, and we
  will define a subgroup $U_\m \le J_K$.
  This will give a subgroup
  \[
    U_{\m}' \le C_K = J_K / K^\times,
  \]
  which will be an open subgroup of
  finite index. Thus $U_{\m}'$ corresponds
  to a finite abelian extension
  $R_\m / K$, which is called the
  \emph{Ray class field} of modulus $\m$.
  By construction, $R_\m / K$ will ramify
  only at places of $K$ in $\supp(\m)$, and
  $\Gal(R_\m / K) \cong C_K / U_{\m}'$,
  which is some kind of \emph{generalized ideal class group}.
  For $\m = 1$, the Ray class field
  $R_1$ is just the Hilbert class field,
  and $C_K / U_1' = \Cl(\OO_K)$.
\end{remark}

\begin{definition}
  Let $\m$ be a modulus. Define $U_{\m, v}$
  as follows:
  \begin{enumerate}
    \item if $e_v = 0$, then set
      $U_{\m, v} = \OO_v^\times$;
    \item if $e_v > 0$ and
      $v$ is non-Archimedean, then
      set $U_{\m, v} = 1 + \p_v^{e_v}$,
      where $\p_v$ is the maximal ideal
      in $\OO_v$;
    \item if $e_v > 0$ and $v$ is
      Archimedean (so $v$ is real), then
      set $U_{\m, v} = \R_{> 0} \subseteq K_v^\times = \R^\times$.
  \end{enumerate}
  Define $U_\m = \prod_v U_{\m, v}$.
\end{definition}

\begin{lemma}
  Every open subgroup of finite index
  in $C_K$ contains some $U_\m'$.
\end{lemma}

\begin{corollary}
  Every finite abelian extension of
  $K$ is contained in some Ray
  class field.
\end{corollary}

\begin{proof}
  Any finite abelian extension $L / K$
  corresponds to a subgroup $H \le C_K$.
  Then $H$ contains $U_\m'$, so
  $L / K$ lies in the corresponding
  Ray class field $R_\m / K$.
\end{proof}

\begin{corollary}[Kronecker-Weber]
  Every finite abelian extension of $\Q$
  lies in a cyclotomic field.
\end{corollary}

\begin{proof}
  If $\m = m \ge 1$, then
  $R_\m = \Q(\zeta_m)$, and if
  $\m = m \cdot \infty$, then
  $R_\m = \Q(\zeta_m)^+$,
\end{proof}

\begin{definition}
  Define $I_\m$ to be the fractional
  ideals coprime to $\m$ and
  $P_\m = \{(a) : a \equiv 1 \modstar{\m}\}$,
  where $a \equiv 1 \modstar{\m}$ means
  that $a \in U_{\m, v}$ for all $v \in \supp(\m)$.
  Define $\Cl_\m(\OO_K) = I_\m / P_\m$.
\end{definition}

\begin{prop}
  $\Cl_\m(\OO_K) \cong C_K / U_\m'$.
\end{prop}

\section{Artin Reciprocity}

\begin{definition}
  Let $K$ be a number field and $L / K$
  a finite abelian extension. Let
  $S$ be the set of ramified places, and
  let $I_S$ be the set of fractional ideals
  coprime to $S$. Define the \emph{Artin map}
  by
  \begin{align*}
    \Theta_{L / K} : I_S
    &\longrightarrow \Gal(L / K) \\
    \p &\longmapsto
    \Frob_\p
  \end{align*}
\end{definition}

\begin{prop}[Artin reciprocity]
  There exists a modulus $\m = \m(L / K)$
  with
  $\supp(\m) = S$ such that
  $P_\m \subseteq \ker \Theta_{L / K}$.
\end{prop}

\begin{remark}
  The primes of $\OO_K$ that split
  completely in $R_\m$ (i.e. $\Frob_\p$ is
  trivial) are exactly those
  in $P_\m$.
\end{remark}

\begin{prop}[Existence theorem]
  Given a modulus $\m$ and
  $P_\m \subseteq H \subseteq I_\m$, then
  there exists an abelian extension $L / K$
  which is unramified outside of $\supp(\m)$,
  such that $\ker \Theta_{L / K} = H$.
\end{prop}

\begin{example}
  Let $K$ be a number field and
  $L / K$ an abelian extension. Then
  $K \subseteq L \subseteq R_\m$ for some
  Ray class field $R_\m$, and
  how a prime $\p$ in $K$ splits in $L$
  and $R_\m$ can be related to produce
  a reciprocity law. Note that this is
  similar to how we proved quadratic
  reciprocity, where we compared how a
  prime splits in a quadratic extension and
  a cyclotomic extension.
\end{example}

\begin{definition}
  Let $p$ be a prime or $\infty$, and $a, b \in \Q_p^\times$ (where $\Q_\infty = \R$).
  Define the \emph{Hilbert symbol}
  \[
    (a, b)_p =
    \begin{cases}
      1 & \text{if $z^2 = ax^2 + by^2$ has a nontrivial solution in $\Q_p$}, \\
      -1 & \text{otherwise}.
    \end{cases}
  \]
  This defines a map
  $(a, b)_p : \Q_p^\times / (\Q_p^\times)^2 \times \Q_p^\times / (\Q_p^\times)^2 \to \{\pm 1\}$.
  See Serre's \emph{A Course in Arithmetic}.
\end{definition}

\begin{theorem}
  The pairing $(a, b)_p$ is bi-multiplicative
  (i.e. $(a, b_1 b_2)_p = (a, b_1)_p (a, b_2)_p$), symmetric, and non-degenerate (i.e.
  if $(a, b)_p = 1$ for every $b$,
  then $a \in (\Q_p^\times)^2$).
\end{theorem}

\begin{corollary}
  $b \in \Q_p^\times$
  is a norm from all quadratic extensions
  of $\Q_p$ if and only if
  $b \in (\Q_p^\times)^2$.
\end{corollary}

\begin{proof}
  Suppose $b$ is a norm from
  $\Q_p[\sqrt{a}]$ for all
  $a \in \Q_p^\times$, so
  $b = z^2 - ax^2$ has a solution in $\Q_p$.
  Then
  \[
    z^2 = ax^2 + by^2
  \]
  has a nonzero solution, which means
  $(a, b)_p = 1$ for all $a \in \Q_p^\times$.
  By non-degeneracy, $b \in (\Q_p^\times)^2$.
\end{proof}

\begin{remark}
  Using quadratic reciprocity, Serre
  shows that $\prod_{p \le \infty} (a, b)_p = 1$
  for all $a, b \in \Q^\times$. In fact,
  this statement is equivalent to
  quadratic reciprocity: Suppose
  $p \ne q$ are distinct odd primes. Then
  \[
    (p, q)_p = \left(\frac{q}{p}\right)
    \quad \text{and} \quad
    (p, q)_q = \left(\frac{p}{q}\right).
  \]
  Furthermore, $(p, q)_\ell = 1$ if
  $\ell \ne 2, p, q, \infty$,
  $(p, q)_\infty = 1$, and $(p, q)_2 = (-1)^{(p - 1)(q - 1) / 4}$.
\end{remark}

\begin{theorem}
  Let $K$ be a local field containing
  $\mu_n$. If $b \in K^\times$ is
  contained in $N_{L / K}(L^\times)$ for
  every
  cyclic extension $L / K$ of degree
  dividing $n$, then $b \in (K^\times)^n$.
\end{theorem}

\begin{remark}
  The modern interpretation of the
  above theorem is that if $L / K$ is
  cyclic, then
  \[
    K^\times / N_{L / K}(L^\times) \cong
    H^2(\Gal(L / K), L^\times).
  \]
\end{remark}

\begin{definition}
  Let $K$ be a local field and
  $a, b \in K^\times$. Then define the
  \emph{local Hilbert symbol}
  \[
    (a, b)_K = \frac{\Theta_K(b)(\sqrt[n]{a})}{\sqrt[n]{a}} \in \mu_n,
  \]
  where $\Theta_K(b) \in \Gal(K^\mathrm{ab} / K)$.
\end{definition}

\begin{theorem}
  The Hilbert symbol $(a, b)_K$
  satisfies the
  following properties:
  \begin{enumerate}
    \item $(b, a)_K = (a, b)_K^{-1}$;
    \item $(a, b)_K$ is bi-multiplicative;
    \item $(a, b)_K$ is non-degenerate.
  \end{enumerate}
\end{theorem}

\begin{definition}
  Let $K$ be a number field.
  The \emph{global Hilbert symbol} is given
  by
  \[
    (a, b)_K = \prod_v (a, b)_{K_v} \in \mu_n,
  \quad a, b \in K^\times / (K^\times)^n.
  \]
\end{definition}

\begin{theorem}[Hilbert reciprocity]
  $(a, b)_K = 1$ for all $a, b \in K^\times$,
  i.e. $\prod_v (a, b)_{K_v} = 1$.
\end{theorem}

\begin{prop}
  Let $p$ be an odd prime, $n = 2$, and
  $a, b \in \Q_p^\times$. Then the
  following are equivalent:
  \begin{enumerate}
    \item $(a, b)_p = 1$ in Serre's sense;
    \item $X^2 - aY^2 - bZ^2$ represents $0$
      in $\Q_p$;
    \item $b$ is a norm from $\Q_p(\sqrt{a})$;
    \item $\Theta_{\Q_p(\sqrt{a}) / \Q_p}(b) = 1$;
    \item $X^2 - aY^2 - bZ^2 + abW^2$
      represents $0$ in $\Q_p$;
    \item the \emph{quaternion algebra}
      $A_{-1}(a, b)$ is \emph{split}, i.e.
      $A_{-1}(a, b) \cong M_2(\Q_p)$.
  \end{enumerate}
\end{prop}

\section{Brauer Groups}

\begin{definition}
  Given $a, b \in K_v^\times$ and a fixed
  $\zeta \in \mu_n$, define a
  \emph{central simple algebra} $A_\zeta(a, b)$
  over $K_v$ with generators
  $i, j$ and relations $i^n = a$, $j^n = b$,
  and $ij = \zeta ji$.
\end{definition}

\begin{example}
  The algebra $A_{-1}(-1, -1)$ is the
  usual Hamilton quaternion algebra
  $\mathbb{H}_\R$.
\end{example}

\begin{definition}
  Define the \emph{Brauer group} as
  $\mathrm{Br}(K_v) = \{\text{central simple algebras over $K_v$}\} / {\sim}$, and
  \[
    (a, b)_v = \zeta^{-[A_\zeta(a, b)]}.
  \]
\end{definition}

\begin{remark}
  One has
  $\mathrm{Br}(K_v) \cong H^2(\Gal(K_v^\mathrm{ab} / K_v), (K_v^\mathrm{ab})^\times)$.
\end{remark}

\begin{remark}
  There is a cohomological interpretation
  of the statement $\prod_v (a, b)_v = 1$
  via Brauer groups.
\end{remark}

\section{Applications to Fermat's Last Theorem}

\begin{remark}
  Let $p$ be an odd prime and
  $\zeta = \zeta_p$. Factor the
  equation $x^p + y^p = z^p$ (let
  $x, y, z$ be coprime and $p \nmid xyz$;
  recall that this is the first case of
  Fermat's last theorem) as
  \[
    \prod_{i = 0}^{p - 1} (x + \zeta^i y)
    = (z)^p.
  \]
  Then the ideal $(\alpha)$ is a $p$th
  power, where
  \[
    \alpha = \frac{x + \zeta y}{x + y}
    = 1 - \frac{y\pi}{x + y}, \quad
    \pi = 1 - \zeta.
  \]
\end{remark}

\begin{theorem}
  If $q | xyz$ is prime, then
  $q^{p - 1} \equiv 1 \Pmod{p^2}$.
\end{theorem}

\begin{proof}
  Let $K = \Q(\zeta_p)$, $\beta = q^{p - 1}$,
  and $\alpha = 1 - y\pi / (x + y)$.
  Then the \emph{power reciprocity law}
  says
  \[
    1 = \left(\frac{\beta}{\alpha}\right)_p
    \left(\frac{\alpha}{\beta}\right)_p^{-1}
    = \zeta_p^{\Tr_{K / \Q}(\eta)},
    \quad \eta = \frac{\beta - 1}{p} \cdot \frac{\alpha - 1}{\pi}
  \]
  where both symbols are $1$ in this case.
  One can compute that
  \[
    \Tr_{K / \Q}(\eta)
    = \frac{q^{p - 1} - 1}{p} \cdot (p - 1)\frac{-y}{x + y},
  \]
  so $p$ must divide $(q^{p - 1} - 1) / p$,
  which implies $p^2 | (q^{p - 1} - 1)$.
\end{proof}

\begin{remark}
  For $q = 2$, this is
  \emph{Wieferich's criterion}, and
  for $q = 3$, this is
  \emph{Mirimanoff's condition}.
\end{remark}

\begin{remark}
  Only two primes ($1093$ and $3511$)
  below $3 \times 10^9$ satisfy
  $2^{p - 1} \equiv 1 \Pmod{p^2}$.
  Neither of these two primes also satisfy
  the congruence
  $3^{p - 1} \equiv 1 \Pmod{p^2}$.
\end{remark}

\begin{remark}
  Class field theory can also be
  considered as studying $1$-dimensional
  representations of the Galois group
  $G_K = \Gal(\overline{K} / K)$. The Langlands
  program studies $n$-dimensional
  representations of $G_K$.
\end{remark}
