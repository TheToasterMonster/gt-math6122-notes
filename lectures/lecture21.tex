\chapter{Mar.~25 --- Completions of Number Fields}

\section{Absolute Values on Number Fields}
\begin{remark}
  Let $K$ be a number field and $\p$ a nonzero
  prime ideal of $\OO_K$. Then for $\alpha \in K^\times$,
  we can define the $\p$-adic valuation
  $v_\p(\alpha) = \ord_\p(\alpha)$, where
  $\ord_\p(\alpha) = k$ if $(\alpha) = \p^k I$ with
  $I$ relatively prime to $\p$, i.e. $I + \p = (1)$.
  We can define the $\p$-adic absolute value
  by $|\alpha|_\p = (N\p)^{-v_\p(\alpha)}$, which is
  non-Archimedean.

  If $\sigma$ is a real embedding
  of $K$, then we can define $|\alpha|_\sigma = |\sigma(\alpha)|_{\R}$,
  where $|\cdot|_{\R}$ is the usual absolute value on
  $\R$. Similarly, if $\{\tau, \overline{\tau}\}$
  is a pair of conjugate complex embeddings, we
  can define $\|\alpha\|_{\{\tau, \overline{\tau}\}} = |\tau(\alpha)|^2_{\C}$,
  where $|\cdot|_{\C}$ is the usual absolute value on
  $\C$.

  Note that $\| \cdot \|_{\{\tau, \overline{\tau}\}}$
  is not actually an absolute value because
  it does not satisfy the triangle inequality,
  we call it the \emph{normalized} absolute value (it
  is squared so that the product formula works).
  We will write
  $\| \cdot \|$ to denote the normalized absolute
  value corresponding to an absolute value $|\cdot|$.

  Recall the following theorems that we stated last
  time:
\end{remark}

\begin{theorem}[Ostrowski for number fields]
  There is a one-to-one correspondence
  \[
    \underbrace{\{\text{places of $K$}\}}_{M_K = M_K^0 \sqcup M_K^\infty} \longleftrightarrow
    \underbrace{\{\text{nonzero prime ideals}\}}_{M_K^0}
    \cup \underbrace{\{\text{real embeddings}\}
    \cup \{\text{pairs of complex embeddings}\}}_{M_K^\infty}.
  \]
\end{theorem}

\begin{theorem}[Product formula]
  For each $v \in M_K$, let $\|\cdot\|_v$ be the
  corresponding normalized absolute value.
  Then for each $\alpha \in K^\times$, we have
  $\prod_{v \in M_K} \|\alpha\|_v = 1$.
\end{theorem}

\begin{proof}
  We have
  \[
    \prod_{v \in M_K} \|\alpha\|_v
    = \prod_{v \in M_K^0} \|\alpha\|_{v}
    \prod_{v \in M_K^\infty} \|\alpha\|_{v}
    = (|N(\alpha)|)^{-1} |N(\alpha)|
    = 1.
  \]
  Note that this is essentially the result
  that $|N((\alpha))| = |N(\alpha)|$.
\end{proof}

\section{Completions of Number Fields}

\begin{remark}
  We can complete $K$ with respect to any
  non-Archimedean absolute value $|\cdot|_{\p}$,
  and we obtain a field $K_\p$ (note that this is not
  a localization). We also have the
  valuation ring
  \[
    \widehat{\OO}_\p = \{x \in K_\p : |x|_{\p} \le 1\}
    = \text{$\p$-adic completion of $\OO_K$}
    = \varprojlim (\OO_K / \p^n).
  \]
  Note that $\widehat{\OO}_\p$ is a local ring
  with unique maximal ideal
  $\widehat{\m}_\p = \{x \in K_\p : |x|_\p < 1\}$.
\end{remark}

\begin{theorem}
  Let $p = \p \cap \Z$. Then we have the following:
  \begin{enumerate}
    \item $[K_\p : \Q_p] < \infty$. In fact,
      $[K_\p : \Q_p] = e(\p / p) f(\p / p)$.
      Moreover, if $K / \Q$ is Galois, then so is
      $K_\p / \Q_p$ and
      $\Gal(K_\p / \Q_p) \cong D_{\p / p}$ in this case.
    \item $\widehat{\OO}_\p / \widehat{\m}_\p^k \cong \OO_K / \p^k$ for all $k \ge 1$ (a particular case
      is $\Z_p / p \Z_p \cong \Z / p\Z$).
  \end{enumerate}
\end{theorem}

\begin{remark}
  An ``explicit'' description of $K_\p$ looks like
  the following: Take the coefficients $S$ to be
  coset representatives for $\OO_K / \p$
  (e.g. $\{0, 1, 2, \dots, p - 1\}$). Then for
  each $x \in K_\p$, we have an expansion
  \[
    x = \sum_{m = -n}^\infty a_m \pi^m, \quad a_m \in S,
  \]
  where we choose a \emph{uniformizer}
  $\pi \in \p \setminus \p^2$.
\end{remark}

\section{Hensel's Lemma}

\begin{prop}
  The element $2$ is not a square in $\Q_5$.
\end{prop}

\begin{proof}
  If $\alpha^2 = 2$ for some $\alpha \in \Q_5$, then
  $|\alpha|_5^2 = |2|_5 = 1$,
  so $|\alpha|_5 = 1$ and thus we have $\alpha \in \Z_5$.
  Then $\overline{\alpha} \in \Z_5 / 5 \Z_5 \cong \Z / 5\Z$
  satisfies $(\overline{\alpha})^2 = \overline{2}$
  in $\Z / 5 \Z$, which is a contradiction.
\end{proof}

\begin{prop}
  The element $-1$ is a square in $\Q_5$. More
  precisely, $x^2 + 1$ has $2$ distinct roots in $\Z_5$,
  where one root in $\Z_5$ is congruent to $2$ mod $5$
  and the other is congruent to $3$ mod $5$.
\end{prop}

\begin{proof}
  Let $\alpha = a_0 + a_1 \cdot 5 + a_2 \cdot 5^2 + \dots$
  for $a_i \in \{0, 1, \dots, 4\}$, and suppose
  $\alpha^2 + 1 = 0$. We must have
  $a_0 = 2$ or $3$ mod $5$ (these are the only solutions
  to $x^2 + 1 \equiv 0 \Pmod{5}$).

  Suppose $a_0 = 2$ (check $a_0 = 3$ as an exercise).
  Then $\alpha^2 + 1 \equiv 0 \Pmod{5^2}$ implies
  \[
    (2 + 5a_1)^2 + 1 \equiv 5 + 20 a_1 \equiv 0 \pmod{25},
  \]
  so $1 + 4 a_1 \equiv 0 \Pmod{5}$, which gives
  $a_1 = 1$. So $\alpha = 2 + 1 \cdot 5 + O(5^2)$.
  Then $\alpha^2 + 1 \equiv 0 \Pmod{5^3}$ implies
  that $(7 + 25 a_2)^2 + 1 \equiv 0 \Pmod{5^3}$,
  which means that
  \[
    50 + 14 \cdot 25 a_2 \equiv 0 \pmod{125},
  \]
  so $2 - a_2 \equiv 0 \Pmod{5}$, which gives
  $a_2 = 2$. So $\alpha = 2 + 1 \cdot 5 + 2 \cdot 5^2 + O(5^3)$, and one can continue.

  In general, let $c_n = a_0 + a_1 \cdot 5^1 + \cdots + a_n \cdot 5^n$.
  Assume $a_0, a_1, \dots, a_n$ have been computed.
  Then
  \[
    (c_n + a_{n + 1} \cdot 5^{n + 1})^2 + 1 \equiv 0 \Pmod{5^{n + 2}},
  \]
  so we get $a_{n + 1} \equiv (1 + c_n^2) / 5^{n + 1} \Pmod{5}$.
  This defines a solution to $x^2 + 1 = 0$ in $\Z_5$
  inductively.
\end{proof}

\begin{prop}[Hensel's lemma, version $2$]
  Let $K$ be a field which is complete with respect
  to a non-Archimedean absolute value $|\cdot|$.
  Suppose $f \in R[x]$, where $R$ is the valuation
  ring of $K$, and that there exists $\alpha_0 \in R$
  such that $|f(\alpha_0)| < |f'(\alpha_0)|^2$.
  Define
  \[
    \alpha_{n + 1} = \alpha_n - \frac{f(\alpha_n)}{f'(\alpha_n)}.
  \]
  Then $\alpha_n \to \alpha$ for some
  $\alpha \in K$ with $f(\alpha) = 0$, and
  $|\alpha - \alpha_0| \le |f(\alpha_0) / f'(\alpha_0)^2| < 1$.
\end{prop}

\begin{proof}
  The idea is to check that $\{\alpha_n\}$ is a
  Cauchy sequence and use the
  Taylor approximation
  \[
    f(\alpha_{n + 1}) = f(\alpha_n - \epsilon_n)
    = f(\alpha_n) - f'(\alpha_n) \epsilon_n + O(\epsilon_n^2)
    = O(\epsilon_n^2)
  ,\]
  where $\epsilon_n = f(\alpha_n) / f'(\alpha_n)$
  (note that $f'(\alpha_n) \epsilon_n = f(\alpha_n)$).
\end{proof}

\begin{remark}
  In practice, we will often choose $\alpha_0$ so that
  $|f(\alpha_0)| < 1$ and $|f'(\alpha_0)| = 1$.
\end{remark}

\begin{example}
  Let $f(x) = x^2 + 1$, $\alpha_0 = 2$, and
  $K = \Q_5$. Note that
  \[
    |f(\alpha_0)| = 1 / 5
    \quad \text{and} \quad
    |f'(\alpha_0)| = |4|_5 = 1,
  \]
  so Hensel's lemma gives a root with
  $\alpha \equiv \alpha_0 \Pmod{5}$ (this holds if and
  only if $|\alpha - \alpha_0|_5 < 1$).
\end{example}

\begin{example}
  We can use Hensel's lemma to show that
  $\Z_p$ contains $p - 1$ different $(p - 1)$-th
  roots of unity. We will solve the equation
  $x^{p - 1} = 1$ in $\Z_p$. Let $f(x) = x^{p - 1} - 1$,
  so that
  \[
    f(x) \equiv (x - 1)(x - 2) \dots (x - (p - 1)) \pmod{p}.
  \]
  So by Hensel's lemma, we have $p - 1$ distinct
  roots of $f(x)$ in $\Z_p$, one for each
  element of $(\Z / p\Z)^\times$.
\end{example}

\begin{theorem}[Baker]
  The world's most complicated group
  $(\Z / p\Z)^\times$ is cyclic.
\end{theorem}

\begin{proof}
  Let $\mu_a$ be the root of unity in $\Z_p$
  corresponding to $a \in (\Z / p\Z)^\times$.
  There is a homomorphism
  \begin{align*}
    (\Z / p\Z)^\times
    &\longrightarrow \Q_p^\times \\
    a &\longmapsto \mu_a.
  \end{align*}
  Let the image of this map be
  $\mu \subseteq \Q_p^\times$. Recall from
  Algebra I that any two splitting fields for
  $f(x)$ over $\Q$ are isomorphic, so
  $\Q(\mu) \subseteq \Q_p$ and
  $\Q(\zeta_{p - 1}) \subseteq \C$ are isomorphic.
  Thus $\mu$ is isomorphic to the $p - 1$ roots of
  unity $\mu_{p - 1} \subseteq \C$.
  But $\mu_{p - 1} = \langle \zeta_{p - 1} \rangle = \langle e^{2\pi i / (p - 1)} \rangle \subseteq \C$,
  is cyclic, so $\mu \cong \Z / p\Z$ is also cyclic.
\end{proof}
