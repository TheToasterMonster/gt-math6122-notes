\chapter{Feb.~13 --- Dirichlet's Unit Theorem}

\section{Dirichlet's Unit Theorem and Proof}

\begin{theorem}[Dirichlet's unit theorem]\label{thm:dirichlet-unit}
  For a number field $K$ of degree $n$, we have
  \[
    \OO_K^\times \cong \Z^{r} \times (\text{finite abelian group})
  \]
  where the finite abelian group is the roots of unity in $K$
  and $r = r_1 + r_2 - 1$ ($\le n - 1$, with equality if and
  only if $K / \Q$ is totally real), where $r_1, r_2$
  are the number of real and (pairs of) complex embeddings
  of $K$.
\end{theorem}

\begin{remark}
  The strategy is the following: Let $L : K^\times \to \R^{r_1 + r_2}$ be the
  homomorphism given by
  \[
    L(\alpha) = (\log |\sigma_1(\alpha)|, \ldots, \log |\sigma_{r_1}(\alpha)|, \log |\tau_1(\alpha)|, \ldots, \log |\tau_{r_2}(\alpha)|),
  \]
  where $\sigma_1, \dots, \sigma_{r_1} : K \hookrightarrow \R$
  are the real embeddings and $\tau_1, \dots, \tau_{r_2} : K \hookrightarrow \C$
  are half of the complex embeddings. We will want
  to show that $L(\OO_K)$ is a lattice.
\end{remark}

\begin{lemma}\label{lem:kernel}
  We have $\ker L|_{\OO_K^\times} = \mu_K$
  and $L(\OO_K^\times) \subseteq H = \{\sum x_i = 0\}$.
\end{lemma}

\begin{proof}
  We can write the kernel of $L_{\OO_K^\times}$ as
  \[
  \ker L|_{\OO_K^\times} = \left\{\alpha \in \OO_K^\times : \substack{\displaystyle\text{absolute values of the conjugates} \\ \displaystyle\text{$\alpha_1, \dots, \alpha_m$ of $\alpha$ are all $1$}}\right\}
    = \mu_K
  \]
  by Kronecker's theorem. Now if $\alpha \in \OO_K^\times$,
  then $|N^K_{\Q} (\alpha)| = 1$, so
  \[
    \left| \prod_i \sigma_i(\alpha) \cdot \prod_j \tau_j(\alpha)^2 \right| = 1,
  \]
  which implies that $\sum \log |\sigma_i(\alpha)| + \sum \log |\tau_j(\alpha)|^2 = 0$, i.e.
  $L(\alpha) \in H$.
\end{proof}

\begin{corollary}
  We have $\OO_K^\times / \mu_K \subseteq H \cong \R^{r_1 + r_2 - 1}$.
\end{corollary}

\begin{lemma}\label{lem:intersection-discrete}
  For any closed ball $B$ around $0$
  in $\R^{r_1 + r_2}$,
  $L(\OO_K^\times) \cap B$ is finite,
  i.e. $L(\OO_K^\times)$ is discrete.
\end{lemma}

\begin{proof}
  We can write the intersection via
  \[
    L(\OO_K^\times) \cap B = \{\alpha \in \OO_K^\times : \text{all conjugates of $\alpha$ have norm bounded by $C$}\}.
  \]
  For any element of the set, there are only finitely
  many possible minimal polynomials, each with only a
  finite number of roots, so the set itself must be
  finite.
\end{proof}

\begin{lemma}\label{lem:lin-alg}
  Let $A = (a_{ij})$ be an $r \times r$ real matrix
  such that:
  \begin{enumerate}
    \item the entries in each row sum to $0$,
    \item the diagonal entries $a_{ii} > 0$, and
    \item the off-diagonal entries $a_{ij} < 0$
      for $i \ne j$.
  \end{enumerate}
  Then $\rank A = r - 1$.
\end{lemma}

\begin{proof}
  It suffices to show that the first $r - 1$ columns
  $v_1, \dots, v_{r - 1}$ are linearly independent.
  Suppose otherwise that we can write
  \[
    \sum_{i = 1}^{r - 1} c_i v_i = 0,
  \]
  where the $c_i$ are not all zero.
  Without loss of generality, assume $c_k = 1$ and
  $c_j \le 1$ for all $j \ne k$. Then
  $a_{kj} < 0$ implies that $c_j a_{jk} \ge a_{jk}$, and
  $\sum_{j = 1}^{r - 1} a_{kj} > \sum_{j = 1}^r a_{kj}$
  since $k \ne r$. Then
  \[
    0 = \sum_{j = 1}^{r - 1} c_j a_{kj}
    \ge \sum_{j = 1}^{r - 1} a_{kj}
    > \sum_{j = 1}^r a_{kj} = 0,
  \]
  which is a contradiction. Since
  $(1)$ imposes one linear condition, we have
  $\rank A = r - 1$.
\end{proof}

\begin{lemma}\label{lem:smaller}
  Fix $k$ with $1 \le k \le r$. Then there is a
  constant $C$ (depending only on $K$) such that
  given $\alpha \in \OO_K \setminus \{0\}$, there
  exists $\beta \in \OO_K \setminus \{0\}$ with:
  \begin{enumerate}
    \item $|N(\beta)| \le C$, and
    \item if $L(\alpha) = (a_1, \dots, a_r)$ and
      $L(\beta) = (b_1, \dots, b_r)$, then
      $b_i < a_i$ for all $i \ne k$.
  \end{enumerate}
\end{lemma}

\begin{proof}
  We claim that taking
  \[
    C = \left(\frac{2}{\pi}\right)^{r_2} \sqrt{|\Delta_K|}
  \]
  works. For convenience, let
  \[
    \epsilon_i =
    \begin{cases}
      1 & \text{if $1 \le i \le r_1$}, \\
      2 & \text{if $r_1 + 1 \le i \le r = r_1 + r_2$}.
    \end{cases}
  \]
  Choose $a_1', \dots, a_r'$ with $a_i' < a_i$
  for each $i$. Define the region
  \[
    E = \{
      x \in \R^{r_1} \times \C^{r_2} : |x_i|^{\epsilon_i} \le C_i
    \},
  \]
  where $C_i = e^{a_i'}$ for $i \ne k$ and $\prod_i C_i = C$.
  It is clear that $E$ is symmetric, compact, convex,
  and
  \[
    \vol(E) = 2^{r_1} \pi^{r_2} \prod_i C_i
    = 2^{r_1} \pi^{r_2} C
    = 2^n \covol{i(\OO_K)}
  \]
  So by Minkowski's theorem,
  $E \cap i(\OO_K) \ne \{0\}$. Thus taking
  $p \in E \cap i(\OO_K)$ and $\beta = i^{-1}(p)$
  works.
\end{proof}

\begin{proof}[Proof of Theorem \ref{thm:dirichlet-unit}]
  Choose $\alpha_0 \in \OO_K$ arbitrarily and fix $k$. By
  Lemma \ref{lem:smaller}, we can find a sequence
  $\alpha_1, \alpha_2, \alpha_3, \dots \in \OO_K \setminus \{0\}$
  such that if $L(\alpha_j) = (a_1(j), \dots, a_r(j))$,
  then $|N(\alpha_j)| \le C$ for $j \ge 1$
  and $a_i(j) < a_i(j - 1)$ for $j \ge 1$, for
  all $i \ne k$.
  There are only a finite number of ideals of norm
  $\le C$, so there exists $j_1 > j_2$ such that
  $(\alpha_{j_1}) = (\alpha_{j_2})$. So
  $u^{(k)} = \alpha_{j_1} / \alpha_{j_2} \in \OO_K^\times$.
  If $L(u^{(k)}) = (a_1, \dots, a_r)$, then
  $a_i < 0$ for $i \ne k$, since
  $a_i = a_i(j_1) - a_i(j_2) < 0$.
  Then $a_k > 0$ since $L(u) \in H$.

  Now applying Lemma \ref{lem:lin-alg} to $u^{(1)}, \dots, u^{(k)}$, we see that
  $\OO_K^\times / \mu_K$ is full rank in
  $H \cong \R^{r_1 + r_2 - 1}$.
\end{proof}

\begin{remark}
  Next time, we will compute the class group and
  unit group of $\Q(\sqrt[3]{11})$.
\end{remark}

\section{Real Quadratic Fields and Continued Fractions}

\begin{example}
  Let $K = \Q(\sqrt{d})$, where $d > 0$ is square-free.
  We know that the unit group $\OO_K^\times$ has rank
  $1$, and (if $d \equiv 2, 3 \Pmod{4}$) the units
  correspond to the integer solutions
  to $x^2 - d y^2 = \pm 1$.

  To solve this equation, one computes the continued
  fraction expansion for $\sqrt{d}$. A \emph{continued
  fraction} is
  \[
    \alpha = a_1 + \frac{1}{a_2 + \displaystyle \frac{1}{a_3 + \dots}}
  \]
  for some positive integers $a_i$, also denoted
  $[a_1, a_2, a_3, \dots]$. The \emph{$n$th convergent}
  for $[a_1, a_2, a_3, \dots]$ is
  \[
    [a_1, a_2, \dots, a_n] =
    a_1 + \frac{1}{a_2 + \displaystyle \frac{1}{\dots + \displaystyle \frac{1}{a_n}}} = \frac{p_n}{q_n}
  \]
  The reason for the name convergent is because
  $p_n / q_n \to \alpha$ as $n \to \infty$.
\end{example}

\begin{theorem}
  Every $\alpha > 1$ has a unique continued fraction
  expansion.
\end{theorem}

\begin{theorem}
  We have the following:
  \begin{enumerate}
    \item The continued fraction expansion of $\alpha$
      is finite if and only if $\alpha \in \Q$.
    \item The continued fraction expansion of $\alpha$
      is pre-periodic if and only if
      $[\Q(\alpha) : \Q] \le 2$.\footnote{A continued fraction expansion is \emph{pre-periodic} if it is periodic after some finite prefix.}
  \end{enumerate}
\end{theorem}

\begin{theorem}
  Let $d \equiv 2, 3 \Pmod{4}$ be square-free and
  positive. Let $\varepsilon > 1$ be the
  fundamental unit of $\Z[\sqrt{d}]$, i.e.
  the unique unit which generates the unit group
  and is $> 1$. Let $k$ be the period of the
  continued fraction expansion of $\sqrt{d}$. Then
  $\varepsilon = p_k + q_k \sqrt{d}$.
\end{theorem}

\begin{example}
  Let $K = \Q(\sqrt{19})$. One can compute that
  \[
    \sqrt{19} = [4, \overline{2, 1, 3, 1, 2, 8}],
  \]
  which has period $6$. The $6$th convergents
  are $p_6 = 170$ and $q_6 = 39$. This says that
  \[
    \sqrt{19} \approx \frac{170}{39}.
  \]
  In fact, $170^2 - 19 \cdot 39^2 = 1$. This also
  means that $170 + 39 \sqrt{19}$ is the fundamental
  unit in $K$.
\end{example}

\begin{remark}
  For $\Q(\sqrt{94})$, the fundamental unit is
  $2143295 + 221064\sqrt{94}$. Furthermore, in
  $\Q(\sqrt{9199})$, the first coefficient for
  the fundamental unit has $88$ decimal digits.
\end{remark}
