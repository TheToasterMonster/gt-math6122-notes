\chapter{Feb.~4 --- Fermat's Last Theorem}

\section{First Case of Fermat's Last Theorem, Continued}

\begin{theorem}[Kronecker]
  If $\alpha \in \C$ is a nonzero algebraic integer,
  all of whose complex conjugates have absolute value
  $1$, then $\alpha$ is a root of unity.
\end{theorem}

\begin{proof}
  Let $f(x) \in \Z[x]$ be the minimal polynomial
  of $\alpha$ (since $\alpha$ is an algebraic integer,
  $f$ is monic):
  \[
    f(x) = x^n + a_{1} x^{n - 1} + \dots + a_{n - 1} x + a_0.
  \]
  Then there exists $C > 0$ such that $|a_i| \le C$
  for all $i$, where $C$ depends only on $n$ (this is
  because each $|a_i| = |\sigma_i(\alpha_1, \dots, \alpha_n)| \le 2^n = C$ by the triangle inequality, where $\sigma_i$ is the $i$th elementary
  symmetric polynomial).  Since the
  $a_i$ are integers, there are only finitely possible
  choices for $f$. Then since each $f$ only has finitely
  many roots, there are only finitely many possible
  choices for $\alpha$.

  The same argument applies
  to $\alpha, \alpha^2, \alpha^3, \dots$, so we see that
  $\{\alpha, \alpha^2, \alpha^3, \dots\}$ is a finite
  set. Thus there exist $i < j$ such that
  $\alpha^i = \alpha^j$, which implies
  that $\alpha^{j - i} = 1$, i.e. $\alpha$ is a
  root of unity.
\end{proof}

\begin{remark}
  The algebraic integer assumption is necessary, e.g.
  consider $(3 / 5) + (4 / 5) i$.
\end{remark}

\begin{lemma}
  If $\alpha \in \Z[\zeta_p]$, then $\alpha^p \equiv a \Pmod{p}$
  for some $a \in \Z$.
\end{lemma}

\begin{proof}
  Let $\zeta = \zeta_p$ and write
  \[
    \alpha = a_0 + a_1 \zeta + \dots + a_{p - 2} \zeta^{p - 2}
  \]
  with $a_i \in \Z$. By the binomial theorem, we have
  (all the cross-terms are divisible by $p$)
  \[
    \alpha^p \equiv a_0^p + (a_1 \zeta)^p + \dots + (a_{p - 2} \zeta^{p - 2})^p
    \equiv a_0 + a_1 + \dots + a_{p - 2} \Pmod{p},
  \]
  where the second step is by
  $\zeta^p = 1$ and Fermat's little theorem.
  So we can take $a = a_0 + \dots + a_{p - 2}$.
\end{proof}

\begin{theorem}[Kummer]
  Let $p$ be an odd prime. If $u \in \Z[\zeta_p]$ is
  a unit, then $u / \overline{u} = \zeta_p^k$ for some
  $k \in \Z$.
\end{theorem}

\begin{proof}
  Let $\alpha = u / \overline{u} \in \Z[\zeta_p] \subseteq \C$.
  Then all conjugates of $\alpha$ have absolute
  value $1$ (the key point is that
  $\Gal(\Q(\zeta_p) / \Q)$ is abelian, so complex
  conjugation commutes with taking the other conjugates).
  So $\alpha$ is a root of unity, which implies
  that $\alpha = \pm \zeta_p^k$ for some $k$. All that
  remains is to show that the sign is $+$.

  Suppose for sake of contradiction that
  $u / \overline{u} = -\zeta_p^k$. Then raising
  both sides to the $k$th power, we find that
  $u^p = -\overline{u}^p$. By the lemma, 
  there is $a \in \Z$ such that $u^p \equiv a \equiv \overline{u}^p \Pmod{p}$ (the second congruence
  comes from taking complex conjugates of both sides).
  This implies that $2 u^p \equiv 0 \Pmod{p}$, so
  we get $p | u^p$ since $p$ is an odd prime. This
  is a contradiction as $u^p$ is a unit.
\end{proof}

\begin{definition}
  A prime $p$ is \emph{regular} if
  $p \nmid |{\Cl(\Z[\zeta_p])}|$, and $p$ is
  \emph{irregular} otherwise.
\end{definition}

\begin{remark}
  Kummer realized that class numbers are related
  to certain congruences of Bernoulli numbers, which
  gives a better criterion for a prime to be regular.
\end{remark}

\begin{remark}
  It is known that there are infinitely many irregular
  primes, with the smallest being $p = 37$. Thus the
  below theorem, due to Kummer, works for all
  primes $p \le 31$.
\end{remark}

\begin{theorem}[Fermat's last theorem for regular primes, first case]
  Let $p \ge 5$ be a regular prime and $p \nmid xyz$.
  Then the equation
  $x^p + y^p = z^p$ has no solutions for
  non-zero integers $x, y, z$.
\end{theorem}

\begin{proof}
  Let $\zeta = \zeta_p$ and write
  (the following equality is on the level of
  ideals)
  \[
    (x + y)(x + \zeta y) \dots (x + y \zeta^{p - 1}) = (z)^p. \tag{$*$}
  \]
  Check as an exercise that the ideals on the left-hand
  side are pairwise relatively prime. Due to
  unique factorization of ideals into prime ideals,
  we must have $(x + y \zeta) = I^p$ for some ideal $I$.
  Now $I$ must be principal since $p$ is regular
  (since $(x + y \zeta)$ is principal but $p$ does
  not divide the class number). So
  \[
    x + y \zeta = u \cdot \alpha^p \quad \text{for some } \alpha \in \Z[\zeta].
  \]
  We claim that $x \equiv y \Pmod{p}$ $(**)$.
  Assuming $(**)$ for now, we have $x \equiv -z \Pmod{p}$
  as well (we can use symmetry to apply the same argument
  to the equation $x^p + (-z)^p = (-y)^p$, since
  $p$ is odd), so
  \[
    2x^p \equiv x^p + u^p \equiv z^p \equiv -x^p \Pmod{p}.
  \]
  This implies that $p | 3x^p$, which is a contradiction
  since $p \ge 5$ and $p \nmid x$.

  Now we prove $(**)$ via the result about
  $u / \overline{u} = \zeta^p$. We have
  $\alpha^p \equiv a \Pmod{p}$ for some $a \in \Z$, so
  \[
    x + y\zeta \equiv u a \Pmod{p}.
  \]
  Then  $\overline{\zeta}_p = \zeta_p^{-1}$
  implies that
  \[
    x + y\zeta^{-1} = \overline{x + y\zeta} \equiv \overline{u} a \Pmod{p}.
  \]
  So $(x + y\zeta) \overline{u} \equiv (x + y\zeta^{-1}) u \Pmod{p}$,
  which implies
  \[
    x + y\zeta \equiv (x + y\zeta^{-1}) \frac{u}{\overline{u}} \Pmod{p}.
  \]
  Thus $x + y\zeta \equiv x \zeta^k + y\zeta^{k - 1} \Pmod{p}$
  for some $0 \le k \le p - 1$ (use
  $u / \overline{u} = \zeta^k$). Show as an exercise
  that this is only possible if $k = 1$ (note that
  the powers of $\zeta$ form an integral basis, so
  a representation in powers of $\zeta$ must be unique).
  This then implies that $x \equiv y \Pmod{p}$, which
  completes the proof.
\end{proof}

\section{More on Cyclotomic Fields}

\begin{definition}
  If $K \subseteq K_1, K_2 \subseteq L$,
  then the \emph{compositum} of $K_1$ and $K_2$ is the
  smallest subfield of $L$ containing both $K_1$ and
  $K_2$.
\end{definition}

\begin{remark}
  If everything is Galois in the above definition,
  $K = K_1 \cap K_2$, and
  $K_1, K_2$ are \emph{linearly disjoint}, i.e.
  that we have
  $[K_1 K_2 : K] = [K_1 : K] [K_2 : K]$, then
  one obtains the result
  \[
    \Gal(K_1 K_2 / K_2) \cong \Gal(K_1 / K).
  \]
\end{remark}

\begin{prop}
  Let $K, K'$ be number fields of degree $n, n'$,
  respectively.
  Assume that
  \begin{enumerate}
    \item $K, K'$ are both Galois over $\Q$,
    \item $K \cap K' = \Q$,
    \item and $(|\Delta_K|, |\Delta_{K'}|) = 1$.
  \end{enumerate}
  Then if $\alpha_1, \dots, \alpha_n$ is an integral
  basis for $\OO_K$ and $\alpha_1', \dots, \alpha_{n'}'$
  is an integral basis for $\OO_{K'}$, then
  $\{\alpha_i \alpha_{j'}\}$ is an integral basis
  for $\OO_{KK'}$.
\end{prop}

\begin{proof}
  Let $\alpha \in \OO_{KK'}$. Field theory shows that
  $\{\alpha_i \alpha_{j'}\}$ is a basis for
  $KK'$ over $\Q$, so we can write
  \[
    \alpha = \sum_{i, j} a_{ij} \alpha_i \alpha_{j}',
    \quad a_{ij} \in \Q.
  \]
  We need to show that $a_{ij} \in \Z$. Let
  $d = |\Delta_K|, d' = |\Delta_{K'}|$. We will
  show that $d a_{ij} \in \Z$ and
  $d' a_{ij} \in \Z$, which will imply that 
  $a_{ij} \in \Z$ since $(d, d') = 1$. Let
  \[
    \beta_j = \sum_i a_{ij} \alpha_i', \quad j = 1, \dots, n.
  \]
  Let $T$ be the $n \times n$ matrix with
  $T_{\ell j} = \sigma_{\ell}(\alpha_j)$, where
  $\sigma_1, \dots, \sigma_n$ are the embeddings of
  $KK'$ over $K'$, i.e. the elements of
  $\Gal(KK' / K')$ since $K, K'$ are Galois over $\Q$.
  Let
  \[
    a =
    \begin{bmatrix}
      \sigma_1(\alpha) \\ \vdots \\ \sigma_n(\alpha)
    \end{bmatrix}
    \quad \text{and} \quad
    b =
    \begin{bmatrix}
      \beta_1 \\ \vdots \\ \beta_n
    \end{bmatrix}.
  \]
  Check as an exercise that $a = Tb$. Multiplying
  this equation by $\adj T$, we find that
  \[
    (\adj T) a = (\det T) b.
  \]
  Now all the entries of $T$, $\adj T$, and $a$
  are algebraic integers, so multiplying the above
  equation by $\det T$ implies that the entries
  of $db$ are algebraic integers. Thus
  $d\beta_j \in \OO_{K'}$ for all $j$. But the
  $\{\alpha_i'\}$ were an integral basis for $\OO_K$
  over $\Z$ by assumption, so in fact $d a_{i, j} \in \Z$
  for all $i, j$, completing the proof.
\end{proof}

\begin{remark}
  Let $K_m = \Q(\zeta_m)$. If $(m, m') = 1$, then
  $(|\Delta_{K_m}|, |\Delta_{K_{m'}}|) = 1$.
  Additionally, some field theory shows that
  if $(m, m') = 1$,
  then $K_m \cap K_{m'} = \Q$ and
  $K_m K_{m'} = K_{mm'}$ in $\C$.

  Using this proposition, along with the previous
  case that $\OO_{\Q(\zeta_m)} = \Z[\zeta_m]$ for
  $m = p^k$ and the Chinese
  remainder theorem,
  one can show that $\OO_{\Q(\zeta_m)} = \Z[\zeta_m]$
  for any $m \ge 1$.
\end{remark}

\begin{corollary}
  For all $m \ge 1$, we have
  $\OO_{\Q(\zeta_m)} = \Z[\zeta_m]$.
\end{corollary}

\section{Geometry of Numbers}

\begin{remark}
  We will discuss Minkowski's geometry of numbers next.
  The goals of this are the following:
  \begin{enumerate}
    \item Improve the constant $M$ from the class
      number computations.
    \item Prove the ``four squares'' theorem, i.e.
      that every $n \ge 1$ is the sum of $4$ integer
      squares.
  \end{enumerate}
\end{remark}
