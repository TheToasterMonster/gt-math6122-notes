\chapter{Apr.~15 --- Class Field Theory}

\section{Introduction to Class Field Theory}

\begin{remark}
  The following are goals of \emph{class field theory}:
  \begin{enumerate}
    \item Classify all (finite) abelian
      extensions of a number field $K$.
    \item Describe $\Gal(K^{\mathrm{ab}} / K)$,
      where $K^{\mathrm{ab}}$ is the maximal
      abelian extension of $K$ (in some
      fixed $\overline{K}$).
  \end{enumerate}
\end{remark}

\begin{lemma}
  If $L_1 / K$, $L_2 / K$ are both abelian,
  then $L_1 L_2 / K$ is also abelian.
\end{lemma}

\begin{proof}
  By Galois theory, $\Gal(L_1 L_2 / K) \le \Gal(L_1 / K) \times \Gal(L_2 / K)$,
  which is abelian.
\end{proof}

\begin{theorem}[Kronecker-Weber]
  $\Q^{\mathrm{ab}} = \Q(\zeta_\infty) = \bigcup_m \Q(\zeta_m)$, i.e.
  every finite abelian extension of $\Q$ is
  contained
  in a cyclotomic extension $\Q(\zeta_m)$ for
  some $m$.
\end{theorem}

\begin{remark}
  Part of the motivation for class field theory
  is to generalize this result to other $K$.

  What is $\Gal(\Q^{\mathrm{ab}} / \Q)$ (this
  is an infinite extension)?
  The answer is that it is the inverse limit
  \[
    \varprojlim \Gal(\Q(\zeta_m) / \Q)
    = \varprojlim (\Z / m\Z)^\times
    = \widehat{\Z}^\times
    = \prod_{p \text{ prime}} \Z_p^\times.
  \]
  Note that $\Z_p^\times \cong \Z / (p - 1)\Z \times \Z_p$
  for $p$ odd and
  $\Z_2^\times \cong \Z / 2\Z \times \Z_2$.
\end{remark}

\begin{example}
  Hilbert studied unramified abelian
  extensions of number fields. For
  $K = \Q$, the maximal unramified (abelian)
  extension is just $\Q$ itself.
  For $K = \Q(\sqrt{-1}), \Q(\sqrt{-2}), \Q(\sqrt{-3})$,
  the maximal unramified abelian extensions
  are also just $K$ itself. For $K = \Q(\sqrt{-5})$, however,
  the extension
  \[
    \Q(\sqrt{-5}, \sqrt{-1}) / \Q(\sqrt{-5})
  \]
  is abelian (it is quadratic) and also
  unramified everywhere.
  For $K = \Q(\sqrt{-31})$, one can show
  $K$ has class number $3$. Letting
  $\alpha$ be a root of $x^3 + x + 1$ (which
  has discriminant $-31$), the extension
  $K(\alpha) / K$
  is abelian of degree $3$ and unramified
  everywhere.

  Hilbert conjectured that
  this is part of a more general phenomenon.
  However, $K = \Q(\sqrt{3})$
  has trivial class group, but the extension
  \[
    \Q(\sqrt{3}, \sqrt{-1}) / \Q(\sqrt{3})
  \]
  is abelian and unramified at $p$ for all
  primes $p$. This seeming contradiction is
  resolved by noting that the extension is
  ``ramified at $\infty$'' (each real embedding
  of $K$ extends to only a single complex
  embedding of $L = \Q(\sqrt{3}, \sqrt{-1})$).
  We want to consider this scenario to have
  ramification index $e = 2$.
\end{example}

\begin{definition}
  An extension $L / K$ is \emph{ramified
  at an infinite place} $v$ of $K$ if
  there exists a complex place $w$ of $L$
  lying over $v$.
\end{definition}

\begin{theorem}[Hilbert?]
  If $K$ is a number field, then the maximal
  abelian extension $H$ of $K$ which is
  unramified at all places (including the
  infinite ones)  is finite and
  $\Gal(H / K) \cong \Cl(K)$.
\end{theorem}

\begin{definition}
  The field $H$ in the above theorem is
  called the \emph{Hilbert class field}.
\end{definition}

\section{Applications of the Hilbert Class Field}
\begin{theorem}
  A prime ideal $\p$ of $\OO_K$ splits
  completely in $H$ if and only if $\p$ is
  principal (if and only if
  the Frobenius element in $\Gal(H / K)$
  is trivial, i.e. $\Frob_\p = \{1\}$).
\end{theorem}

\begin{proof}
  The map (let $I_K$ be the set of fractional
  ideals of $\OO_K$)
  \begin{align*}
    I_K &\longrightarrow \Gal(H / K) \\
    \p &\longmapsto \Frob_\p
  \end{align*}
  is surjective with kernel $P_K$ (the set of
  principal ideals of $\OO_K$), so
  $I_K / P_K \cong \Gal(H / K)$.
\end{proof}

\begin{example}
  The rational primes that split completely in
  $L = \Q(\sqrt{-5}, \sqrt{-1})$ are
  $p \equiv 1, 9 \Pmod{20}$. Let
  $K = \Q(\sqrt{-5})$, and note that $p$
  also has to split in $K$. Then for a prime
  $\p_i$ of $\OO_K$ lying over $p$, we have
  $N(\p_i) = p$, and if $\p_i$ must be
  principal since it splits completely in
  $L$. Writing $\p_i = (x + y \sqrt{5})$,
  we find that $x^2 + 5y^2 = p$ for some $x, y \in \Z$.
  This proves the following statement:
  \begin{quote}
    Let $p$ be a prime. Then
    $p \equiv 1, 9 \Pmod{20}$ if and only if
    $p = x^2 + 5y^2$ for some $x, y \in \Z$.
  \end{quote}
  See D. Cox's \emph{Primes of the form $x^2 + ny^2$} for
  more details.
\end{example}

\begin{remark}
  Kummer considered $K = \Q(\zeta_p)$, and
  the subfield fixed by complex conjugation is
  \[
    \Q(\zeta_p)^+ = \Q(\zeta_p + \zeta_p^{-1}).
  \]
  This can be shown using \emph{Gaussian periods}.
  The extension $\Q(\zeta_p)^+ / \Q(\zeta_p)$ is
  ramified only at $\infty$.
\end{remark}

\begin{theorem}[Kummer]
  $h_{K^+} | h_K$, where $h_K$ denotes the
  class number of $K$ and $K^+$ is the
  subfield of $K$ fixed by complex
  conjugation.
\end{theorem}

\section{Galois Theory for Infinite Extensions}

\begin{definition}
  Let $L / K$ be normal, separable, and
  algebraic (but not necessarily finite), i.e.
  for every $\alpha \in L^\times$, its
  minimal polynomial $f_\alpha \in K[x]$
  splits into distinct linear
  factors. We say $L / K$ is \emph{Galois}, and
  define its \emph{Galois group} as
  \[
    G = \Aut(L / K) = \Gal(L / K)
  \]
\end{definition}

\begin{remark}
  If $L / K$ is finite, then there is a bijection
  between intermediate extensions
  $L / K$ and subgroups of $G$. If
  $[L : K] = \infty$, given
  an intermediate field $L \subseteq M \subseteq K$,
  we can still have $H = \Gal(L / M) \le G$
  and $L^H = M$. But not every subgroup
  $H$ occurs in this manner.

  The solution is to put a topology on $G$,
  and then we get a bijection between
  intermediate fields and closed subgroups
  of $G$ (the topology is essentially defined
  to make this true).
\end{remark}

\begin{definition}
  A basis of open subsets of $G$ consists of
  cosets $gH$, where $H$ is a normal subgroup
  of the form $\Gal(L / M)$ for some
  finite Galois extension $M / K$. This
  defines the \emph{Krull topology} on $G$.
\end{definition}

\begin{remark}
  In the finite setting, the Krull topology
  is discrete, which recovers
  finite Galois theory.
\end{remark}

\begin{remark}
  One can show that (in this case, $\cong$
  denotes homeomorphism)
  \[
    \Gal(L / K) \cong \varprojlim_{\substack{M / K \text{ finite Galois,} \\ K \subseteq M \subseteq L}}
    \underbrace{\Gal(M / K)}_{\text{discrete topological space}}.
  \]
  The Krull topology is the smallest one
  such that the maps
  $\Gal(L / K) \to \Gal(M / K)$ are continuous.
\end{remark}

\begin{prop}
  $G = \Gal(L / K)$ is a compact Hausdorff
  topological group.
\end{prop}

\begin{remark}
  The Galois group $\Gal(L / K)$ is a
  \emph{profinite group}, i.e. an inverse
  limit of finite groups. One gets a
  bijection between intermediate extensions
  $L \subseteq M \subseteq K$ and closed
  subgroups of $G$, as well as a bijection
  between finite extensions $M / K$ and
  open subgroups of $G$ (which also happen to
  be precisely the closed subgroups of finite
  index in $G$).

  Note that if $H \le G$ is open, then
  $\{gH\}$ is an (disjoint) open cover of $G$.
  By compactness, we must have
  $[G : H] < \infty$. Since $H^c$ is the union
  of non-trivial cosets of $H$ (which are
  open since $H$ is), we get that $H^c$ is
  open, hence $H$ is a closed subgroup of
  finite index.
\end{remark}

\begin{example}
  Let $K = \F_q$ and $L = \overline{\F}_q$.
  Then $\Gal(L / K) \cong \widehat{\Z} = \varprojlim (\Z / n\Z) \cong \prod_{p \text{ prime}} \Z_p$.
  There is a canonical generator
  $1 \in \widehat{\Z}$, which corresponds to
  $\Frob_q$ under the isomorphism. Note that
  \[
    \overline{\F}_q = \bigcup_{n = 1}^\infty F_{q^n},
  \]
  and open subgroups of $\widehat{\Z}$ (of
  finite index) are of the form
  $n\widehat{\Z}$ for $n \ge 1$.
\end{example}

\begin{remark}
  Next time, we will generalize this
  idea to $K$
  a finite extension of $\Q_p$ (a \emph{local field}).
  We will define the
  \emph{Artin homomorphism}
  \[
    \Theta : K^\times \longrightarrow \Gal(K^{\mathrm{ab}} / K)
  \]
  which is ``almost'' an isomorphism.
  This will give bijections between
  open subgroups of $K^\times$ of finite index,
  open subgroups of $\Gal(K^{\mathrm{ab}} / K)$
  of finite index, and finite abelian extensions
  $K \subseteq L \subseteq K^{\mathrm{ab}}$).
\end{remark}
