\chapter{Jan.~21 --- Finiteness of the Class Group}

\section{Multiplicativity of the Norm}

\begin{theorem}
  If $I, J \subseteq \OO_K$ are ideals, then
  $N(IJ) = N(I) N(J)$.
\end{theorem}

\begin{proof}
  First observe that if $\mathfrak{a}, \mathfrak{b} \subseteq \OO_K$ are
  relatively prime ideals (i.e. $\mathfrak{a} + \mathfrak{b} = (1)$), then
  \[
    \OO_K / \mathfrak{a} \mathfrak{b} \cong \OO_K / \mathfrak{a} \oplus \OO_K / \mathfrak{b}
  \]
  by the Chinese remainder theorem, and
  the result immediately follows. One can also
  show that if $\p \ne \q$ are nonzero prime ideals in
  $\OO_K$, then
  $\p^s$ and $\q^t$ are relatively prime for every
  $s, t$. Thus by unique factorization of $I, J$ into
  prime ideals,
  it is enough to prove $N(\p^m) = (N(\p))^m$
  for a prime ideal $\p$.

  To do this, observe that we have the chain of inclusions
  \[
    \OO_K \supsetneq \p \supsetneq \p^2 \dots \supsetneq \p^m,
  \]
  and it suffices to show that $[\p^k : \p^{k + 1}] = N(\p)$
  for each $0 \le k < m$. We will show the
  stronger result that $\OO_K / \p \cong \p^k / \p^{k + 1}$
  as abelian groups. To do this, pick
  $\gamma \in \p^k \setminus \p^{k + 1}$ (note
  that $\p^k \ne \p^{k + 1}$ by unique factorization)
  and define $\phi : \OO_K \to \p^k / \p^{k + 1}$
  by $x \mapsto \gamma x$. Since $\gamma x \in \p^{k + 1}$
  whenever $x \in \p$, this induces a map
  $\phi : \OO_K / \p \to \p^k / \p^{k + 1}$, which
  we prove is an isomorphism in Proposition
  \ref{prop:isomorphism}.
\end{proof}

\begin{prop}\label{prop:isomorphism}
  The map $\phi : \OO_K / \p \to \p^k / \p^{k + 1}$
  by $x \mapsto \gamma x$ is an isomorphism
  of abelian groups.
\end{prop}

\begin{proof}
  We will show the following claims:
  \begin{enumerate}
    \item $(\gamma) + \p^{k + 1} = \p^k$. This
      implies that $\phi$ is surjective.
    \item $(\gamma) \cap \p^{k + 1} = \gamma \p$.
      This means that if $\gamma x \in \gamma \p$,
      then $x \in \p$, i.e. $\phi$ is injective.
  \end{enumerate}
  (1) Let $I = (\gamma) + \p^{k + 1}$. Since we already
  know that $\p^k | (\gamma)$, we have $\p^k | I$.
  But $I \supsetneq \p^{k + 1}$, so $I | \p^{k + 1}$,
  and the containment being strict implies that
  we must have $I = \p^k$.

  (2) Let $I' = (\gamma) \cap \p^{k + 1}$. Since
  $\gamma \in \p^k$, we have $\gamma \p \subseteq I'$.
  This is one containment.
  Conversely, let $x \in I'$. Write $x = \gamma y$,
  where $y \in \OO_K$ and $\gamma y \in \p^{k + 1}$.
  Now note that\footnote{Here $\ord_\p(\alpha) = \ord_\p(\mathfrak{a})$ is the largest integer $m$ such that $\p^m | \mathfrak{a}$, where $\mathfrak{a} = (\alpha)$.}
  \[
    \ord_\p(\gamma) + \ord_\p(y) = \ord_\p(\gamma y) \ge k + 1.
  \]
  But $\ord_\p(\gamma) = k$ (since $\gamma \in \p^k \setminus \p^{k + 1}$),
  so $\ord_\p(y) \ge 1$. This implies
  that $\p | (y)$, so $y \in \p$. Since
  $x = \gamma y$, this gives $x \in \gamma \p$.
  This yields the other containment, and so
  $I' = \gamma \p$.
\end{proof}

\begin{corollary}
  Let $[K : \Q] = n$ and $p \in \Z$ be a prime number.
  Write
  \[
    (p) = p\OO_K = \prod_{i = 1}^r \p_i^{e_i},
  \]
  where the $\p_i$ are distinct prime ideals. Then
  $\sum_{i = 1}^r e_i f_i = n$,
  where $N(\p_i) = p^{f_i}$.
\end{corollary}

\begin{proof}
  Since the norm is multiplicative, we have
  \[
    p^n = N(p \OO_K) = N(p) = \prod_{i = 1}^n \sigma_i(p)
  \]
  since each $\sigma_i$ fixes $p$. Then since
  $p \OO_K = \prod_{i = 1}^r \p_i^{e_i}$, we have\footnote{Note that $\mathcal{O}_K / \p_i$ is a finite field (since $\p_i \ne 0$ is prime, hence maximal in $\OO_K$) and a vector space over $\Z / p$ since $(p) \subseteq \p_i$. So $\OO_K / \p_i$ has prime characteristic, hence $N(\p_i) = |\OO_K / \p_i| = p^{f_i}$ for some $f_i$.}
  \[
    p^n = N(p \OO_K) = N(\prod \p_i^{e_i})
    = \prod (N(\p_i))^{e_i}
    = \prod (p^{f_i})^{e_i}
    = \prod p^{e_i f_i}.
  \]
  Thus $n = \sum_{i = 1}^r e_i f_i$, which is the
  desired result.
\end{proof}

\begin{remark}
  In the above case, we will say that the
  $\p_i$ ``lie over'' $p$.
\end{remark}

\section{Finiteness of the Class Group}
\begin{theorem}
  Let $K$ be a number field. Then there exists $M > 0$
  such that every nonzero ideal $I$ of $\OO_K$ contains
  a nonzero element $\alpha$ with
  $|N(\alpha)| \le M \cdot N(I)$. Equivalently,
  $\alpha$ satisfies
  \[
    \inf_{\alpha \in I} \frac{N(\alpha)}{N(I)} \le M,
  \]
  and the above infimum is $1$ if and only if $I$ is
  principal.
\end{theorem}

\begin{proof}
  Choose an integral basis $\alpha_1, \dots, \alpha_n$
  of $\OO_K$, and let $I$ be a nonzero ideal.
  Choose $m$ such that $m^n \le N(I) < (m + 1)^n$.
  Then define the set
  \[
    \Sigma =
    \left\{
      \sum_{i = 1}^n m_j \alpha_j : 0 \le m_j \le m, m_j \in \Z
    \right\}.
  \]
  Note that $\# \Sigma = (m + 1)^n > N(I) = |\OO_K / I|$,
  so by the pigeonhole principle there exist $x \ne y$
  in $\OO_K$ such that $\alpha = x - y \in I$, and
  we can write $\alpha = \sum m_j \alpha_j$ where
  $|m_j| \le m$ for every $j$. Then
  \[
    |N(\alpha)| = \prod_{i = 1}^n |\sigma_i(\alpha)|
    \le \prod_{i = 1}^n \sum_{j = 1}^n |m_j| |\sigma_i(\alpha_j)|
    \le m^n \prod_{i = 1}^n \sum_{j = 1}^n |\sigma_i(\alpha_j)|
    \le N(I) \cdot M,
  \]
  where $M = \prod_{i = 1}^n \sum_{j = 1}^n |\sigma_i(\alpha_j)|$
  is independent of $I$ (but depends on the
  choice of integral basis).
\end{proof}

\begin{corollary}
  Every ideal class in $\OO_K$ contains a nonzero
  ideal of norm at most $M$.
\end{corollary}

\begin{proof}
  Let $C \in \Cl(\OO_K)$, and let $I$ be an ideal
  with $[I] = C^{-1}$. By the above theorem,
  choose $\alpha \in I$ such that $|N(\alpha)| \le M \cdot N(I)$.
  Now $(\alpha) = IJ$ for some $J$, so
  $[J] = [I]^{-1} = C$, and
  \[
    N(J) = \frac{|N(\alpha)|}{N(I)} \le M,
  \]
  which proves the desired result.
\end{proof}

\begin{lemma}
  The set of ideals with norm bounded by $M$
  is finite, i.e. $|\{I : N(I) \le M\}| < \infty$.
\end{lemma}

\begin{proof}
  One way to proceed is to write $I = \prod \p_i^{e_i}$,
  and then use $N(\p_i) = p^{f_i}$.

  Another way to prove this is to note that
  if $|N(I)| = m$, then $mx = 0$ in
  $\OO_K / I$ for every $x \in \OO_K$. So
  $I \supseteq m \OO_K$. But
  $\OO_K / m \OO_K$ is finite, so there are only
  finitely many ideals containing $m \OO_K$.
\end{proof}

\begin{corollary}
  The ideal class group of a number field $\Cl(\OO_K)$ is finite.
\end{corollary}

\begin{proof}
  Each ideal class can be represented by an ideal of
  norm bounded by $M$, and there are only
  finitely many such ideals. Thus there can only be
  finitely many ideal classes.
\end{proof}

\section{Computation of Integral Bases}

\begin{remark}
  Recall that if $[K : \Q] = n$ and
  $\alpha_1, \dots, \alpha_n \in K$ are a basis for
  $K / \Q$, then
  $\Delta(\alpha_1, \dots, \alpha_n) \in \Q^\times$.
  Moreover, if $\alpha_1, \dots, \alpha_n \in \OO_K$,
  then $\Delta(\alpha_1, \dots, \alpha_n) \in \Z$.
  Also, if $\alpha_1, \dots, \alpha_n$ are a $\Z$-basis
  for $\OO_K$, then
  \[
    \Delta(\alpha_1, \dots, \alpha_n) = \Delta_K = \Delta(\OO_K)
  \]
  is independent of the choice of $\Z$-basis. So
  $\Delta_K$ is an invariant of $K$ (or of $\OO_K$),
  called its \emph{discriminant}.
\end{remark}

\begin{prop}
  Let $\alpha_1, \dots, \alpha_n \in \OO_K$ be a basis
  for $K / \Q$, and let
  $d = \Delta(\alpha_1, \dots, \alpha_n)$. Then
  \[
    \Z[\alpha_1, \dots, \alpha_n]
    \subseteq \OO_K \subseteq
    \Z\left[\frac{\alpha_1}{d}, \dots, \frac{\alpha_n}{d}\right].
  \]
\end{prop}

\begin{proof}
  Suppose $\alpha \in \OO_K$, so we can write
  (since $\alpha_1, \dots, \alpha_n$ is a basis for
  $K / \Q$)
  \[
    \alpha = c_1 \alpha_1 + \dots + c_n \alpha_n,
    \quad c_i \in \Q.
  \]
  We want to show that $d c_j \in \Z$. Note that
  $\sigma_i(\alpha) = c_1 \sigma_i(\alpha_1) + \dots + c_n \sigma_i(\alpha_n)$, so
  \[
    \begin{bmatrix}
      \sigma_1(\alpha) \\ \vdots \\ \sigma_1(\alpha)
    \end{bmatrix}
    = T
    \begin{bmatrix}
      c_1 \\ \vdots \\ c_n
    \end{bmatrix},
  \]
  where $T = (\sigma_i(\alpha_j))$. Multiplying
  both sides by $\adj T$, we get
  (note that $T \adj T = \delta I$, where $\delta = \det T$)
  \[
    \begin{bmatrix}
      \beta_1 \\ \vdots \\ \beta_n
    \end{bmatrix}
    = \delta
    \begin{bmatrix}
      c_1 \\ \vdots \\ c_n
    \end{bmatrix},
  \]
  where the $\beta_i \in \OO_K$. Let
  $m_j = \delta \beta_j$, and noting that $\delta^2 = d$
  by definition, we have
  \[
    \begin{bmatrix}
      m_1 \\ \vdots \\ m_n
    \end{bmatrix} = d
    \begin{bmatrix}
      c_1 \\ \vdots \\ c_n
    \end{bmatrix}.
  \]
  This tells us that $d c_i \in \OO_K$ for every $i$.
  But the $c_i$ were also rational, so in fact
  $d c_i \in \OO_K \cap \Q = \Z$.
\end{proof}

\begin{lemma}
  Let $\alpha_1, \dots, \alpha_n \in \OO_K$ be a basis
  for $K / \Q$. Let
  \[
    M = \text{$\Z$-module spanned by $\alpha_1, \dots, \alpha_n$}.
  \]
  Then $\Delta_{K / \Q}(\alpha_1, \dots, \alpha_n) = \Delta_K \cdot |\OO_K / M|^2$.
\end{lemma}

\begin{proof}
  Check this as an exercise; it is a calculation
  involving determinants.
\end{proof}

\begin{corollary}
  Let $\alpha_1, \dots, \alpha_n \in \OO_K$ be a
  basis for $K / \Q$. If
  $\Delta(\alpha_1, \dots, \alpha_n)$ is
  square-free, then the $\alpha_1, \dots, \alpha_n$ form
  an integral basis.
\end{corollary}

\begin{proof}
  If the $\alpha_i$ do not form a basis, then
  the $\Delta$ will contain a
  $|\OO_K / M|^2$ factor by the lemma.
\end{proof}

\begin{example}
  Let $K = \Q(\sqrt{d})$, where $d$ is square-free.
  Then we can see that
  \[
    \Delta_{K / \Q}(1, \sqrt{d}) =
    \det
    \begin{bmatrix}
      1 & \sqrt{d} \\ 1 & -\sqrt{d}
    \end{bmatrix}^2
    = 4d.
  \]
  Thus $4d = \Delta_{K / \Q} \cdot |\OO_K / M|^2$, where
  $M = \Z \cdot 1 + \Z \cdot \sqrt{d}$. Since $d$ is
  square-free, $[\OO_K : M] = 1$ or $2$. Now if
  we have $|\OO_K / (\Z + \Z \sqrt{d})| = 2$, then at
  least one of
  \[
    \frac{1}{2}, \quad \frac{\sqrt{d}}{2}, \quad
    \frac{1 + \sqrt{d}}{2}, \quad \frac{1 - \sqrt{d}}{2}
  \]
  must be an algebraic integer. The first two are
  obviously not algebraic integers, and the third is
  an algebraic integer if and only if the fourth one
  is (since they are conjugates). So the index is $2$
  if and only if $(1 + \sqrt{d}) / 2 \in \OO_K$.
  By looking at the coefficients of the minimal
  polynomial
  \[
    x^2 - x + \frac{1 - d}{4},
  \]
  this happens if and only if $(1 - d) / 4 \in \Z$,
  which is equivalent to $d \equiv 1 \Pmod{4}$.
\end{example}
