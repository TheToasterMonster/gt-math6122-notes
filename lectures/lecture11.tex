\chapter{Feb.~11 --- Lagrange's Four Square Theorem}

\section{Lagrange's Four Square Theorem}
\begin{theorem}[Fermat]
  If $p \equiv 1 \Pmod{4}$ is a prime, then
  there exist $a, b \in \Z$ such that $p = a^2 + b^2$.
\end{theorem}

\begin{proof}[Proof \#1]
  Recall that by Kummer's theorem, factoring
  $(p)$ in $\Z[i]$ is reduced to considering
  $x^2 + 1$ mod $p$. Since $p \equiv 1 \Pmod{4}$,
  we see that $-1$ is a square mod $p$ (e.g. this follows
  by Euler's criterion for the Legendre symbol).
  So $p\Z[i] = \p_1 \p_2$, where each $\p_i$ has norm $p$.
  Since $\Z[i]$ is a UFD, we have $\p_1 = (a + bi)$
  and so $p = \N(\p_1) = N(a + bi) = a^2 + b^2$.
  This proves the theorem.
\end{proof}

\begin{remark}
  There is a way to generalize this proof for sums of
  four squares, but it requires developing unique factorization
  theory in the integer quaternions $\Z[i, j, k]$.
  The proof below generalizes more readily.
\end{remark}

\begin{proof}[Proof \#2]
  As above, we know that $x^2 \equiv -1 \Pmod{p}$
  has a solution, call it $u \in \Z$. Define $\Lambda \subseteq \Z^2$ by
  \[
    \Lambda = \{(a, b) \in \Z^2 \mid b \equiv au \Pmod{p}\}.
  \]
  Then $\Lambda$ is a rank $2$ lattice, and
  $|\Z^2 / \Lambda| = p$ (e.g. one can write an explicit
  isomorphism $\Z^2 / \Lambda \to \Z / p\Z$ by mapping
  $[(a, b)] \in \Z^2 / \Lambda$ to $a \in \Z / p\Z$, noting
  that $a$ completely determines $b$ mod $p$). Let
  \[
    S = \overline{D}(0, r), \quad \text{where } \pi r^2 = 4p.
  \]
  Then $\vol(S) = 4p = 2^2 \cdot \vol(\Lambda)$, so
  by Minkowski's theorem, there exists $(a, b) \ne (0, 0)$
  in $\Lambda \cap S$. So
  \[
    0 < a^2 + b^2 \le r^2 = \frac{4}{\pi} \cdot p < 2p
  \]
  since $\pi > 2$. Since $(a, b) \in \Lambda$, we have
  $b \equiv au \Pmod{p}$, so $b^2 \equiv -a^2 \Pmod{p}$,
  which implies $p | (a^2 + b^2)$. Then
  $a^2 + b^2$ is divisible by $p$ but strictly
  between $0$ an $2p$, so we must have $a^2 + b^2 = p$.
\end{proof}

\begin{theorem}[Lagrange]\label{thm:lagrange-squares}
  Every positive integer $n$ is a sum of four squares.
\end{theorem}

\begin{lemma}
  It suffices to prove Lagrange's four square theorem
  when $n = p$ is prime.
\end{lemma}

\begin{proof}
  There is an identity that says the product of two numbers,
  which are each a sum of four squares, is again a sum of
  four squares. The idea behind the identity is the following:
  In the ring $\mathbb{H}_\Z$ of integral quaternions, we have
  $N(\alpha) = N(\alpha) N(\beta)$, where
  $N(a + bi + cj + dk) = a^2 + b^2 + c^2 + d^2$.
\end{proof}

\begin{remark}
  The norm is also multiplicative in $\C$, where
  $(a + bi)(c + di) = (ac - bd) + (ad + bc)i$
  implies that $(a^2 + b^2) = (ac - bd)^2 + (ad + bc)^2$.
  A similar identity happens for the quaternions.
\end{remark}

\begin{lemma}
  If $p$ is an odd prime, then there exist $u,v \in \Z$
  such that $u^2 + v^2 \equiv -1 \Pmod{p}$.
\end{lemma}

\begin{proof}
  The number of squares in $\F_p = \Z / p\Z$ is
  (e.g. via the primitive element theorem)
  \[
    \frac{p - 1}{2} + 1 = \frac{p + 1}{2}.
  \]
  Define the sets
  \[
    A = \{1 + x^2 : x \in \F_p\} \quad \text{and} \quad
    B = \{-y^2 : y \in \F_p\}.
  \]
  Each of these sets contains $(p + 1) / 2$ elements, so
  $|A| + |B| = p + 1 > \#\F_p$. By the pigeonhole principle,
  $A \cap B \ne \varnothing$, so there exist $u, v \in \F_p$
  such that $1 + u^2 \equiv -v^2 \Pmod{p}$, which
  implies the result.
\end{proof}

\begin{proof}[Proof of Theorem \ref{thm:lagrange-squares}]
  By the first lemma, it suffices to let $n = p$ be prime.
  As
  $2 = 1^2 + 1^2 + 0^2 + 0^2$,
  we can assume that $p$ is an odd prime. By the
  second lemma, choose $u, v$ such that $u^2 + v^2 \equiv -1 \Pmod{p}$.
  Define $\Lambda \subseteq \Z^4$ by
  \[
    \Lambda = \{(a, b, c, d) \in \Z^4 :
    c \equiv ua + vb, d = ub - va \Pmod{p}\}.
  \]
  Note that $\covol(\Lambda) = |\Z^4 / \Lambda| = p^2$
  (similarly we find an isomorphism $\Z^4 / \Lambda \to (\Z / p\Z)^2$ by $(a, b, c, d) \mapsto (a, b)$).
  Once we can find $(a, b, c, d) \ne 0$ in $\Lambda$
  with norm $< 2p$ (claim), we have
  \begin{align*}
    a^2 + b^2 + c^2 + d^2
    &\equiv a^2 + b^2 + (ua + vb) ^2 + (ub - va)^2 \\
    &\equiv a^2 + b^2 + u^2 a^2 + v^2 b^2 + u^2 b^2 + v^2 a^2 \\
    &\equiv a^2 + b^2 + (u^2 + v^2)a^2 + (u^2 + v^2)b^2 \\
    &\equiv a^2 + b^2 - a^2 - b^2 \\
    &\equiv 0 \Pmod{p},
  \end{align*}
  since $u^2 + v^2 \equiv -1 \Pmod{p}$. Then as before,
  $a^2 + b^2 + c^2 + d^2$ is divisible by $p$ but
  lands strictly between $0$ and $2p$, so
  we must have $a^2 + b^2 + c^2 + d^2 = p$. Given the
  claim, this proves the result.

  So it suffices to
  find such a point $(a, b, c, d) \ne 0$. Let
  \[
    B_r = \text{4-dimensional closed ball of radius $r$}.
  \]
  If $\vol(B_r) = 2^4 p^2$, then by Minkowski we
  get a nonzero lattice point in $B_r$. We want $r^2 < 2p$.
  We have
  \[
    \frac{1}{2}\pi^2 r^4 = \vol(B_r) = 16p^2,
  \]
  so $r^2 = \sqrt{32p^2 / \pi^2} = (4\sqrt{2} / \pi) \cdot p$.
  So we need $4 \sqrt{2} / \pi < 2$, which happens if
  and only if $\pi > 2\sqrt{2}$. This is true, since
  $\pi \approx 3.14$ and $2\sqrt{2} \approx 2.83$, which
  completes the proof of the theorem.
\end{proof}

\section{Revisiting Minkowski's Theorem}

\begin{remark}
  Recall that for a number field $K$, every ideal class
  is represented by an ideal of norm $\le M_K$:
  \[
    M_K = \frac{n!}{n^n} \left(\frac{4}{\pi}\right)^{r_2} \sqrt{|\Delta_K|}.
  \]
  This implies that $M_K \ge 1$, since an ideal cannot
  have norm $0$. In the above formula, this gives
  \[
    \sqrt{|\Delta_K|} \ge \frac{n^n}{n!} \left(\frac{\pi}{4}\right)^{n / 2},
  \]
  which is $> 1$ when $n \ge 2$ and goes to $\infty$ as
  $n \to \infty$. This implies that if $K \ne \Q$ is
  any number field, then some $p$ ramifies in $\OO_K$.
  This then means that $\Z$ has no unramified covers,
  which means that $\Z$ has trivial
  ``arithmetic fundamental group.'' The subject discussing
  this is called \emph{\'etale cohomology}.
\end{remark}

\begin{theorem}[Tate?]
  If $E / \Q$ is an elliptic curve, then there is some
  prime of bad reduction.
\end{theorem}

\begin{remark}
  The above theorem from the theory of elliptic curves
  has a similar flavor to our
  discussion above. Fontaine greatly generalizes this result to
  an abelian variety $A / \Q$.
\end{remark}

\section{Dirichlet's Unit Theorem}

\begin{remark}
  Let $K$ be a number, $\OO_K$ its ring of integers, and
  $\OO_K^\times$ the \emph{unit group}.
\end{remark}

\begin{theorem}[Dirichlet, weak]
  The unit group $\OO_K^\times$ is finitely generated. In
  particular,
  \[
    \OO_K \cong \Z^r \times (\text{finite abelian group}),
  \]
  where the finite abeliain group is the roots of unity in $K$.
\end{theorem}

\begin{example}
  Suppose $[K : \Q] = 2$. If $K$ is imaginary, then
  $K = \Q(\sqrt{-d})$, where $d > 0$ is square free. Then
  $u \in \OO_K$ is a unit if and only if $N(u) = 1$,
  Since $u$ is either $a + b \sqrt{-d}$ or $a + b \sqrt{-d} / 2$,
  One can use $N(a + b\sqrt{-d}) = a^2 + db^2 = 1$ to
  easily find the units. The conclusion is
  \[
    \OO_K^\times =
    \begin{cases}
      \{\pm 1, \pm i\}, & \text{if } d = 1, \\
      \{\pm 1, \pm \omega \pm \omega^2\}, & \text{if } d = 3, \\
      \{\pm 1\}, & \text{otherwise},
    \end{cases}
  \]
  so $\OO_K$ has rank $0$. Now consider a real
  quadratic field $K = \Q(\sqrt{d})$.
  For $\Q(\sqrt{2})$, notice that
  \[
    (\sqrt{2} - 1) (\sqrt{2} + 1) = 1.
  \]
  In particular, this means that $(\sqrt{2} - 1)^k$ is a
  unit for each $k \in \Z$, yielding a rank $1$ subgroup.
  In fact,
  \[
    \Z[\sqrt{2}]^\times
    \cong \{\pm 1\} \times \Z
    \cong \{\pm 1\} \times \langle \sqrt{2} - 1 \rangle.
  \]
  One can also view this from the perspective of Diophantine
  equations. Note that $a + b\sqrt{2} \in \Z[\sqrt{2}]$
  is a unit if and only if $a^2 - 2b^2 = \pm 1$. The
  set of solutions to this equation form a group under
  multiplication, and that group is of rank $1$.
  This is the \emph{generalized Pell equation},
  $a^2 - db^2 = \pm 1$. It is known for certain $d$ modulo
  $4$, this equation has
  infinitely many solutions, which are generated by a single
  fundamental solution. This result follows as a special case
  of Dirichlet's unit theorem.
\end{example}

\begin{remark}
  The above Pell equations are difficult to solve, even for
  relatively small values of $d$. For instance, the
  smallest solution to $a^2 - 61 b^2 = 1$ is
  $a \approx 10^6$, $b \approx 10^8$.
\end{remark}

\begin{example}
  Let $K = \Q(\zeta_m)$, where $m = p^k$. Then the unit group
  $\OO_K^\times$ contains
  \[
    \left\{
      \frac{1 - \zeta_m^a}{1 - \zeta_m} : 1 \le a \le p - 1
    \right\}.
  \]
  This generates a subgroup of rank $\phi(m) / 2 - 1 = r_1 + r_2 - 1$ (here $r_1$ is the number of real embeddings
  and $r_2$ is $1 / 2$ the number of complex embeddings).
  Note that this is compatible with the quadratic case,
  since $r_1 + r_2 - 1$ is $1$ if $K$ is real and $0$ if $K$
  is complex.
\end{example}

\begin{theorem}[Dirichlet's unit theorem]
  For a number field $K$ of degree $n$, we have
  \[
    \OO_K^\times \cong \Z^{r} \times (\text{finite abelian group})
  \]
  where the finite abelian group is the roots of unity in $K$
  and $r = r_1 + r_2 - 1$ ($\le n - 1$, with equality if and
  only if $K / \Q$ is totally real), where $r_1, r_2$
  are the number of real and (pairs of) complex embeddings
  of $K$.
\end{theorem}
