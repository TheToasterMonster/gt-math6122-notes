\chapter{Jan.~30 --- Computing Ideal Class Groups and Applications}

\section{Computing Ideal Class Groups}

\begin{example}
  Let $K = \Q(\sqrt{2})$. We know that
  $\OO_K = \Z[\sqrt{2}]$. Recall that every ideal
  class in $\Cl(\OO_K)$ contains an ideal of
  norm $\le M$, where (if
  $\OO_K = \Z \alpha_1 \oplus \dots \oplus \Z \alpha_n$
  as $\Z$-modules)
  \[
    M = \prod_{i = 1}^n \sum_{j = 1}^n |\sigma_{i}(\alpha_j)|.
  \]
  Thus in this case, we have
  $M = (1 + \sqrt{2})^2 \approx 5.8 < 6$. Thus
  every ideal class contains an ideal of norm
  $\le 5$. We will want to factor the
  ideals $(2), (3), (5)$ in $\OO_K$ via Kummer's
  theorem. We need to factor
  \[
    x^2 - 2 \Pmod{p}, \quad p = 2, 3, 5.
  \]
  Mod $2$, we have $x^2 - 2 \equiv x^2 \Pmod{2}$,
  so $(2) = \p_2^2$. Now $x^2 - 2$ is irreducible mod
  both $3$ and $5$ since $2$ is not a quadratic residue
  mod $3$ or $5$. Thus $(3) = \p_3$ and
  $(5) = \p_5$ with $f_3 = f_5 = 2$. Thus
  \[
    N(\p_2) = 2, \quad N(\p_3) = 9, \quad N(\p_5) = 25.
  \]
  So any nonzero ideal in $\OO_K$ is equivalent to
  $\p_2$ or $(1)$ (since $\p_2^2 = (2)$ is principal).
  Thus
  \[
    \Cl(\OO_K) = \{(1), [\p_2]\},
  \]
  which is isomorphic to either
  $\{1\}$ if $\p_2$ is principal or
  $\Z / 2\Z$ if $\p_2$ is not principal. But
  $\p_2 = (\sqrt{2})$, so
  \[\Cl(\Z[\sqrt{2}]) = \{1\}.\]
  This also implies that $\Z[\sqrt{2}]$ is a PID,
  and hence a UFD.
\end{example}

\begin{example}
  Let $K = \Q(\sqrt{-5})$, where we have
  seen that $\OO_K = \Z[\sqrt{-5}]$. By a similar
  reasoning as above, we have
  $M = (1 + \sqrt{5})^2 < 11$. So we want to find
  all nonzero prime ideals of norm $\le 10$. We consider
  $x^2 + 5 \Pmod{p}$ for $p = 2, 3, 5, 7$, where
  we can factor
  \begin{align*}
    p = 2 & : x^2 + 5 \equiv x^2 + 1 \equiv (x + 1)^2 \Pmod{2}, \\
    p = 3 & : x^2 + 5 \equiv x^2 - 1 \equiv (x - 1)(x + 1) \Pmod{3}, \\
    p = 5 & : x^2 + 5 \equiv x^2 \Pmod{5}, \\
    p = 7 & : x^2 + 5 \equiv (x + 3)(x + 4) \Pmod{7}.
  \end{align*}
  Thus we have
  $(2) = \p_2^2$, $(3) = \p_3 \p_3'$, $(5) = \p_5^2 = (\sqrt{-5})^2 = (\sqrt{-5})$,
  and $(7) = \p_7 \p_7'$. Since $(5)$ is principal,
  we only need to consider $(2), (3), (7)$, which
  each have norm $\le 10$. Thus
  \[
    \Cl(\OO_K) = \langle [\p_2], [\p_3], [\p_7] \rangle,
  \]
  where we note that $\p_3'$ and $\p_7'$
  are inverses to $\p_3$ and $\p_7$, respectively,
  since their product is principal, so we do not
  need to include them as generators.
  Note that $\OO_K^\times = \{\pm 1\}$, so we do
  not need to worry much about units. If
  any of the $\p_p$ is principal, then it is generated
  by an element $\alpha = a + b\sqrt{-5}$ with
  \[
    p = N(\alpha) = a^2 + 5b^2.
  \]
  This cannot happen for $p = 2, 3, 7$, so
  $\p_2, \p_3, \p_7$ are not principal. Now
  $N(1 + \sqrt{-5}) = 6$, so
  \[
    (1 + \sqrt{-5}) = \p_2 \p_3 \text{ or } \p_2 \p_3'.
  \]
  This means that one of $\p_3$ or $\p_3'$ is an
  inverse to $\p_2$ in $\Cl(\OO_K)$, so in fact
  \[
    \Cl(\OO_K) = \langle [\p_2], [\p_7] \rangle.
  \]
  Also note that $[\p_2]^2 = 1$ since $\p_2^2 = (2)$
  is principal. We can also see that
  $N(3 + \sqrt{-5}) = 14$, so
  \[
    (3 + \sqrt{-5}) = \p_2 \p_7 \text{ or } \p_2 \p_7'.
  \]
  Thus we also do not need $\p_7$ as a generator, so
  $\Cl(\OO_K) = \langle [\p_2] \rangle$. Since
  $[\p_2]^2 = 1$ and $\p_2$ is not principal,
  \[
    \Cl(\Z[\sqrt{-5}]) = \Z / 2\Z.
  \]
\end{example}

\section{Applications of Class Group Computations}

\begin{theorem}
  The Diophantine equation $y^2 = x^3 - 5$ has
  no integer solutions.
\end{theorem}

\begin{proof}
  Assume that we have a solution $x, y$ to the
  above equation. Writing $x^3 = y^2 + 5$, we
  can factor
  \[
    x^3 = (y + \sqrt{-5})(y - \sqrt{-5}).
  \]
  in $\Z[\sqrt{-5}]$. By looking at the equation
  mod $4$, we see that $x$ must be odd.
  Also, $(x, y) = 1$ since otherwise $(x, y) = 5$
  and one can derive a contradiction with the
  equation $y^2 = x^3 - 5$.

  Now we claim that
  $(y + \sqrt{-5})$ and $(y - \sqrt{-5})$ are
  coprime (equivalent to comaximal in a Dedekind
  domain) ideals. To see this, suppose otherwise
  that $\p$ divides both. Then $\p | (x^3) = (x)^3$,
  so we must have $\p | (x)$ by unique factorization.
  Also, $\p | (2y)$. But $\p | (x)$ means that
  $N(\p)$ is odd, so $\p \nmid (2)$. Thus
  $\p | (y)$. Then $(x, y) = 1$ implies that
  $\p | (1)$, a contradiction. Thus
  $(y + \sqrt{-5})$ and $(y - \sqrt{-5})$ are
  coprime.

  Thus we may write
  \[
    (x)^3 = (y + \sqrt{-5})(y - \sqrt{-5}).
  \]
  Since the above ideals are relatively prime,
  unique factorization implies that each of
  $(y \pm \sqrt{-5})$ is the cube of some ideal. 
  Write $(y + \sqrt{-5}) = \mathfrak{a}^3$ for some ideal
  $\mathfrak{a}$. Then $[\mathfrak{a}]^3 = 1$
  in $\Cl(\Z[\sqrt{-5}])$, so
  $\Cl(\Z[\sqrt{-5}]) = \Z / 2\Z$ implies that
  $[\mathfrak{a}] = 1$. Thus $\mathfrak{a} = (\alpha)$
  for some $\alpha = a + b \sqrt{-5}$, so
  \[
    (a + b \sqrt{-5})^3
    = \alpha^3 = \pm (y + \sqrt{-5})
  \]
  as elements. Finish the proof and derive a
  contradiction from here as an exercise.
\end{proof}

\section{Cyclotomic Fields}

\begin{theorem}\label{thm:cyclotomic-prime-powers}
  Let $m = p^k$ for $k \ge 1$. Then the ring of
  integers of $\Q(\zeta_m)$ is $\Z[\zeta_m]$.
\end{theorem}

\begin{remark}
  Recall that the minimal polynomial for
  $\zeta_m$ over $\Q$ is
  \[
    \Phi_m(x) = \frac{x^{p^k} - 1}{x^{p^{k - 1}} - 1}
    = x^{(p - 1)p^{k - 1}} + x^{(p - 2)p^{k - 1}} + \dots + x^{p^{k - 1}} + 1.
  \]
  This polynomial is irreducible
  over $\Z$ (e.g. $\Phi_m(x + 1)$ is Eisenstein at $p$),
  with
  \[
    \deg \Phi_m(x) = \phi(m) = (p - 1)p^{k - 1}.
  \]
  Note that $\Q(\zeta_m) / \Q$ is Galois
  with Galois group isomorphic to $(\Z / m\Z)^\times$.
\end{remark}

\begin{lemma}
  For any $m \ge 1$, the
  discriminant $\Delta(\zeta_m) = \disc(\Phi_m) = \Delta(1, \zeta_m, \zeta_m^2, \dots)$
  divides $m^{\phi(m)}$.
\end{lemma}

\begin{proof}
  Since $\Phi_m | (x^m - 1)$, we can write
  $x^m - 1 = \Phi_m(x) g(x)$ for a polynomial $g$.
  Taking
  a derivative,
  \[
    mx^{m - 1} = \Phi_m'(x) g(x) + \Phi_m(x) g'(x).
  \]
  Plugging in $x = \zeta_m^j$ for $(j, m) = 1$ (these
  are the conjugates of $\zeta_m$),
  \[
    m\zeta_m^{j(m - 1)} = \Phi_m'(\zeta_m^j) g(\zeta_m^j)
  \]
  since $\Phi_m(\zeta_m^j) = 0$. Taking the norm in
  $\Q(\zeta_m) / \Q$, we have
  \[
    m^{\phi(m)}
    = N(m \zeta_m^{m - 1})
    = N(\Phi_m'(\zeta_m^j)) N(g(\zeta_m^j))
    = \pm \Delta(\zeta_m) N(g(\zeta_m^j)),
  \]
  where $N(g(\zeta_m^j)) \in \Z$
  since an integer polynomial evaluated at an algebraic
  integer is an algebraic integer, and the norm of
  an algebraic integer is an integer. Also,
  $g(\zeta_m^j) \ne 0$, so $\Delta(\zeta_m) | m^{\phi(m)}$.
\end{proof}

\begin{proof}[Proof of Theorem \ref{thm:cyclotomic-prime-powers}]
  Since $\Phi_m(x + 1)$ is Eisenstein at $p$,
  we have $p \nmid [\OO_{\Q(\zeta_m)} : \Z[\zeta_m]]$.
  But
  \[
    \Delta(\zeta_m) = \Delta(\Q(\zeta_m)) \cdot [\OO_{\Q(\zeta_m)} : \Z[\zeta_m]]^2
  \]
  and $\Delta(\zeta_m)$ is a power of $p$ by the
  lemma, so $[\OO_{\Q(\zeta_m)} : \Z[\zeta_m]]$ is a power of $p$.
  So $[\OO_{\Q(\zeta_m)} : \Z[\zeta_m]] = 1$.
\end{proof}

\begin{remark}
  Using field/Galois theory, we will show
  that $\OO_{\Q(\zeta_m)} = \Z[\zeta_m]$
  for all $m \ge 1$ next time.
\end{remark}

\section{First Case of Fermat's Last Theorem}

\begin{remark}
  Our current goal is: If
  $p$ is an odd prime and $u \in \Z[\zeta_p]^\times$,
  then $u / \overline{u} = \zeta_p^k$ for some
  $k \in \Z$.
\end{remark}

\begin{lemma}
  If $m$ is a positive integer, then the roots
  of unity in $\Q(\zeta_m)$ are
  \[
    \begin{cases}
      \text{primitive $m$th roots of $1$} & \text{if $m$ is even}, \\
      \text{primitive $(2m)$-th roots of $1$} & \text{if $m$ is odd}.
    \end{cases}
  \]
\end{lemma}

\begin{proof}
  If $m$ is odd, then $\Q(\zeta_m) = \Q(\zeta_{2m})$.
  So assume without loss of generality that $m$ is even.
  Suppose $\zeta \in \Q(\zeta_m)$
  and $\zeta^k = 1$ for some $k$. We want to show
  that $k | m$. Assume without loss of generality that
  $\zeta = e^{2\pi i / k}$. Since
  $\zeta_k, \zeta_m \in \Q(\zeta_{m})$, one can check
  as an exercise that
  $\zeta_r \in \Q(\zeta_{m})$ where $r = \lcm(k, m)$. Then
  $\Q(\zeta_r) \subseteq \Q(\zeta_m)$, so taking
  degrees implies that $\phi(r) \le \phi(m)$. Also
  $m | r$, which one can check
  implies that $\phi(m) \le \phi(r)$.
  This means that $\phi(m) = \phi(r)$, so
  $m = r$ and thus $k | m$.
\end{proof}
