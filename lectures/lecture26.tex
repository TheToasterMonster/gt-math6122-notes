\chapter{Apr.~17 --- Class Field Theory, Part 2}

\section{Local Class Field Theory}

\begin{definition}
  Let $G$ be a topological group. Its
  \emph{profinite completion} is
  \[
    \widehat{G} = \varprojlim_U G / U,
  \]
  where $U$ ranges over the open normal
  subgroups of $G$ of finite index, and
  $G / U$ is given the discrete topology.
  The \emph{universal property of the profinite completion}
  is that any map from $G$ to a profinite
  group $H$ factors uniquely through the
  profinite completion $\widehat{G}$.
\end{definition}

\begin{remark}
  The finite-index open normal subgroups
  of $G$ are in bijection with the
  finite-index open normal subgroups of
  its profinite completion $\widehat{G}$.
\end{remark}

\begin{example}
  We have $\widehat{\Z} = \varprojlim_n (\Z / n\Z)$.
\end{example}

\begin{theorem}
  Let $K$ be a local field (i.e. a finite
  extension $K / \Q_p$ or $K / \R$, or
  equivalently, $k_v$ for a number field
  $k$ and a place $v$ of $k$). Then
  there exists an injective homomorphism
  \[
    \Theta : K^\times \longrightarrow
    \Gal(K^\mathrm{ab} / K)
  \]
  which induces an isomorphism
  $\widehat{K^\times} \overset{\cong}{\longrightarrow} \Gal(K^\mathrm{ab} / K)$.
\end{theorem}

\begin{remark}
  What does $\widehat{K^\times}$ look like?
  If $K$ is Archimedean, then either $K = \C$
  or $K = \R$. If $K = \C$, then
  the left-hand side is $\widehat{C^\times} = \varprojlim \C^\times / \C^\times = \{1\}$
  (since $\C^\times$ has no non-trivial
  finite-index open subgroups)
  and the right-hand side is also $\{1\}$
  (since $\C$ is algebraically closed).
  For $K = \R$, the left-hand side is
  $\Z / 2\Z$ (there is a non-trivial
  open subgroup $\R_{> 0}$), and the
  right-hand side is $\Gal(\C / \R) \cong \Z / 2\Z$.

  If $K$ is non-Archimedean, then
  $K^\times \cong \Z \times \OO_K^\times$ since
  there is a split exact sequence
  \begin{center}
    \begin{tikzcd}
      0 \ar[r] & \OO_K \ar[r] & K^\times \ar[r, "v_p"] & \Z \ar[r] & 0
    \end{tikzcd}
  \end{center}
  where $v_p : K^\times \to \Z$ is the
  $p$-adic valuation. So we get that
  (note that $\OO_K^\times$ is already profinite)
  \[
    \widehat{K^\times} \cong \widehat{\Z} \times \OO_K^\times
    \cong \prod_{p} \Z_p \times \OO_K^\times.
  \]
  We also have the following:
  \begin{center}
    \begin{tikzcd}
      0 \ar[r] & \OO_K^\times \ar[d, equal] \ar[r] & K^\times \ar[d, "\Theta"] \ar[r, "v_p"] & \Z \ar[d, hook] \ar[r] & 0 \\
      0 \ar[r] & \Gal(K^{\mathrm{ab}} / K^{\mathrm{unr}})
      \ar[r] & \Gal(K^{\mathrm{ab}} / K)
      \ar[r] & \Gal(K^{\mathrm{unr}} / K)
      \ar[r] & 0
    \end{tikzcd}
  \end{center}
  where $\Gal(K^\mathrm{unr} / K) \cong \Gal(\overline{k} / k) \cong \widehat{\Z}$.
  For $\pi \in \OO_K$ with
  $v(\pi) = 1$, i.e. $\pi$ is a uniformizer,
  $\Theta(\pi)$ is
  the Frobenius automorphism.
  Also,
  $\Theta(\OO_K^\times) = \Gal(K^\mathrm{ab} / K^\mathrm{unr})$,
  which is the inertia subgroup of $G$.
\end{remark}

\begin{theorem}
  Let $L / K$ be a finite abelian extension, and
  $\Theta_{L / K}$ be the composite map
  \begin{center}
    \begin{tikzcd}
      K^\times \ar[r] & \Gal(K^\mathrm{ab} / K)
      \ar[r] & \Gal(L / K)
    \end{tikzcd}
  \end{center}
  Then $\Theta_{L / K}$ is surjective, and
  $\ker \Theta_{L / K} = N_{L / K}(L^\times) \le K^\times$.
  In particular, $K^\times / N_{L / K}(L^\times) \cong \Gal(L / K)$.
\end{theorem}

\begin{theorem}[Local class field theory]
  There is an inclusion-reversing
  correspondence between the finite
  abelian extensions of a local field $K$ and
  finite-index open subgroups of $K^\times$
  given by
  \begin{center}
    \begin{tikzcd}
      L / K \text{ abelian} \ar[r] &
      \ar[l] N_{L / K}(L^\times)
    \end{tikzcd}
  \end{center}
\end{theorem}

\section{Adeles and Ideles}

\begin{definition}
  Let $K$ be a number field. For a place
  $v$ of $K$, note that the completion $K_v$ is
  a local field, and
  let $\OO_v$ is the valuation ring in
  $K_v$ (in the non-Archimedean case, or
  $K_v$ itself in the Archimedean case).
  Define the \emph{adele ring} of $K$ to be
  ($\prod'$ denotes the \emph{restricted product})
  \[
    \mathbb{A}_K = \prod_v' (K_v, \OO_v)
    = \left\{(a_v) \in \prod_v K_v : a_v \in \OO_v \text{ for all but finitely many } v\right\}.
  \]
  Consider the basis of open neighborhoods
  of $0$ given by $\prod_v U_v$, where
  $U_v$ is open in $K_v$ and $U_v = \OO_v$
  for almost all $v$. This turns
  $\mathbb{A}_K$ into a \emph{topological ring}.
\end{definition}

\begin{theorem}
  The adele ring $\mathbb{A}_K$ is locally
  compact.
\end{theorem}

\begin{remark}
  Note that it is important for $\mathbb{A}_K$
  to be locally compact, e.g. for harmonic
  analysis. Also,
  there is an embedding $K \hookrightarrow \mathbb{A}_K$
  given by
  $v \mapsto \prod_v \alpha_v$, where
  $i_v(\alpha) = \alpha_v$ for the inclusion
  $i_v : K \hookrightarrow K_v$.
\end{remark}

\begin{theorem}
  $K$ is discrete in $\mathbb{A}_K$, and
  $\mathbb{A}_K / K$ is compact.
\end{theorem}

\begin{example}
  Consider the case $\Z \hookrightarrow \R$,
  where $\Z$ is discrete and $\R / \Z$ is compact.
\end{example}

\begin{definition}
  The group of \emph{ideles} of $K$ is
  $J_K = \mathbb{A}_K^\times = \prod_v' (K_v^\times, \OO_v^\times)$.
  We take a basis of open neighborhoods
  of $1$ to be $\prod_v U_v$ for
  open $U_v \subseteq K_v^\times$, such that
  $U_v = \OO_v^\times$ for almost all $v$. Note that this is \emph{not} the subspace topology
  from $\mathbb{A}_K^\times \subseteq \mathbb{A}_K$.
\end{definition}

\begin{theorem}
  $J_K$ is a locally compact abelian topological
  group.
\end{theorem}

\begin{definition}
  There is a norm $\| \cdot \| : J_K \to \R_{> 0}$
  given by
  $\|\alpha\| = \prod_v \|\alpha_v\|_v$,
  where $\|\cdot\|_v : K_v \to \R_{> 0}$. The
  product formula says that
  $K^\times \subseteq J_K^1$ (the ideles of
  norm $1$). Define the quotient
  $J_K / K^\times = C_K$ to be the \emph{idele class group} of $K$, and
  $J^1_K / K^\times = C_K^1 \le C_K$.
\end{definition}

\begin{theorem}
  $C_K^1$ is compact.
\end{theorem}

\begin{proof}
  This follows from the finiteness of
  $\Cl(\OO_K)$ and Dirichlet's unit theorem.
\end{proof}

\section{Global Class Field Theory}

\begin{remark}
  Chevalley realized that adeles and ideles
  give a natural way to state global class
  field theory.
\end{remark}

\begin{remark}
  In global class field theory, $C_K$ will
  play that role that $K^\times$ played in
  local class field theory.
\end{remark}

\begin{theorem}[Artin]
  There exists a homomorphism
  \[
    \Theta : C_K \longrightarrow \Gal(K^\mathrm{ab} / K)
  \]
  inducing an isomorphism
  $\widehat{C}_K \overset{\cong}{\longrightarrow} \Gal(K^\mathrm{ab} / K)$.
\end{theorem}

\begin{remark}
  How do we define the global
  Artin map $\Theta$? Pick $L / K$ finite
  abelian. Define a map
  \begin{align*}
    \Theta_{L / K} : J_K
    &\longrightarrow \Gal(L / K) \\
    (a_v) &\longmapsto \prod_v \Theta_v(a_v),
  \end{align*}
  where $\Theta_v$ is the local Artin map.
  Note that if $v$ is
  unramified in $L$ and $a_v \in \OO_v^\times$,
  then $\Theta_v(a_v) = 1$, so the
  above product is indeed well-defined. Then
  take the inverse limit over all such $L$,
  and we get a map
  \[
    \Theta : J_K \longrightarrow \Gal(K^\mathrm{ab} / K).
  \]
\end{remark}

\begin{theorem}[Artin's reciprocity law]
  $K^\times \subseteq \ker \Theta$, i.e.
  for all $\alpha \in K^\times$, we have
  $\prod_v \Theta_v(\alpha) = 1$.
\end{theorem}

\begin{example}
  Let $p, q$ be distinct odd primes, and
  $K = \Q$, $L = \Q(\sqrt{p^*})$ for
  $p^* = p (-1)^{(p - 1) / 2}$. Then for
  $\alpha = q$, Artin's reciprocity
  law says
  \[
    \left(\frac{p^*}{q}\right)
    \left(\frac{q}{p^*}\right) = 1,
  \]
  which recovers the law of quadratic
  reciprocity.
\end{example}

\begin{theorem}[Global class field theory]
  There is a bijection between the
  open finite-index subgroups of $C_K$
  and the finite abelian
  extensions of $K$
  given by the inclusion-reversing
  correspondence
  \[
    N_{L / K}(C_L) \longleftrightarrow L.
  \]
\end{theorem}
