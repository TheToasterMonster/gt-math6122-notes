\chapter{Mar.~6 --- Factorization in Cyclotomic Fields}

\section{Factorization in Cyclotomic Fields}

\begin{remark}
  Let $\Z[\zeta_m] \subseteq \Q(\zeta_m)$ and
  $p \nmid m$ prime. We can factor
  \[
    p \Z[\zeta_m] = \p_1 \dots \p_r,
  \]
  where $p \nmid m$ implies that the
  $\p_i$ are distinct, i.e. $e_i = 1$ for each $i$.
  Since $\Q(\zeta_m) / \Q$ is Galois, we have
  $f_1 = \dots = f_r = f$ and thus
  $fr = \phi(m)$.
\end{remark}

\begin{prop}
  In the above situation, we have
  $f = \text{order of $[p]$ in $(\Z / m\Z)^\times$}$.
\end{prop}

\begin{proof}
  Note that $G = \Gal(\Q(\zeta_m) / \Q) \cong (\Z / m\Z)^\times$
  by the isomorphism
  \begin{align*}
    (\Z / m\Z)^\times
    &\longrightarrow \Gal(\Q(\zeta_m) / \Q) \\
    a
    &\longmapsto
    (\sigma_a : \zeta_m \mapsto \zeta_m^a).
  \end{align*}
  We have $f = [\Z[\zeta_m] / \p : \F_p]$, but this
  is hard to deal with. Instead, Kummer's theorem
  says that
  \[
    f = \text{degree of any irreducible factor of $\Phi_m \Mod{p}$}.
  \]
  Let $g(x)$ be an irreducible factor of
  $\overline{\Phi_m(x)}$, so that $\deg g = f$.
  Then $\F_p[x] / (g(x))$ is a finite field of
  order $p^f$, so its Galois group over $\F_p$
  is generated by $x \mapsto x^p$. By the
  proof of Kummer's theorem,
  \[
    \Z[\zeta_m] / \p \cong \F_p[x] / (g(x)),
  \]
  so the Galois group of
  $\F_p[x] / (g(x))$ is $\Gal(\ell / k)$, where
  $\ell$ is $\Z[\zeta_m] / \p$ and $k$ is $\F_p$.
  Then $D_{\p} \cong \Gal(\ell / k)$ is a subgroup
  of $G$, and the generator $x \mapsto x^p$ in
  $\Gal(\ell / k)$ corresponds to
  $p \in (\Z / m\Z)^\times$. Since $\ell$ has
  order $p^f$ (so $x \mapsto x^p$ has order
  $f$ in $\Gal(\ell / k)$), we see that the order of
  $[p]$ in $(\Z / m\Z)^\times$ is $f$.
\end{proof}

\begin{example}
  Let $m = 15$, so that $\phi(15) = 8$. Then
  \[
    \Phi_{15}(x)
    = \frac{(x^{15} - 1)(x - 1)}{(x^5 - 1)(x^3 - 1)}
    = x^8 - x^7 + x^5 - x^4 + x^3 - x + 1.
  \]
  Mod $2$, we can factor
  \[
    \Phi_{15}(x) \equiv (x^4 + x^3 + 1)(x^4 + x + 1) \pmod{2},
  \]
  and notice that $2^4 \equiv 1 \Pmod{15}$. On the
  other hand, we have $31 \equiv 1 \Pmod{15}$ and
  \[
    \Phi_{15}(x)
    \equiv (x + 3)(x + 11)(x + 12)(x + 13)(x + 17)(x + 21)(x + 22)(x + 24)
    \pmod{15}.
  \]
\end{example}

\section{Law of Quadratic Reciprocity}
\begin{remark}
  Let $p$ be an odd prime and let
  $K_p = \Q(\zeta_p)$. Note that since
  $\Gal(K_p / \Q) \cong (\Z / p\Z)^\times$ and $p$ is
  odd, $\Gal(K / \Q)$ is cyclic of even order, hence
  there is a unique subgroup $H$ of index $2$. By Galois
  theory, this subgroup corresponds to a unique
  quadratic subfield $L = K_p^H \subseteq K_p$.
\end{remark}

\begin{lemma}
  The unique subgroup $H$ of index $2$ is the subgroup of
  squares mod $p$.
\end{lemma}

\begin{proof}
  Half of the residues mod $p$ are squares, so
  this is a subgroup of index $2$, hence it is $H$.
\end{proof}

\begin{remark}
  One way to calculate $L$ is to use Gaussian
  periods (Gauss sums). We will do this later, and
  we proceed with a different method for now.
  Because the ramification index $e$ is multiplicative
  in towers, and because $p$ is the only prime
  ramifying in $K_p$, it follows that $p$ is the
  only prime ramifying in $L$.

  We know that $L = \Q(\sqrt{d})$ where $d$ is
  square-free. We have previously seen that
  \[
    \Delta_{\Q(\sqrt{d})} =
    \begin{cases}
      d & \text{if $d \equiv 1 \Pmod{4}$}, \\
      4d & \text{if $d \equiv 2, 3 \Pmod{4}$}.
    \end{cases}
  \]
  Note that we have to be in the first case
  since $2$ does not ramify in $L$, and so
  $d = \pm p$ since $d$ is square-free and $p$ is
  the only prime dividing it. We also have $d \equiv 1 \Pmod{4}$,
  so $d = p^*$, where
  \[
    p^* = (-1)^{(p - 1) / 2} p =
    \begin{cases}
      p & \text{if $p \equiv 1 \Pmod{4}$}, \\
      -p & \text{if $p \equiv 3 \Pmod{4}$}.
    \end{cases}
  \]
  Now suppose that $p, q$ are distinct odd primes
  (so $q$ is unramified in $L$). We will figure out
  how $q$ factors in $L$, i.e. how $q \OO_L$ factors
  into prime ideals. We will do this in two different
  ways.

  We first use Kummer's criterion. We look at
  $x^2 - p^* \Pmod{q}$ (note that
  $q \nmid [\OO_L : \Z[\sqrt{p^*}]] = 2$ since $q$ is
  an odd prime), which
  splits into linear factors if and only if
  $(\frac{p^*}{q}) = 1$.

  On the other hand, let $\q$ be a prime ideal of
  $\Z[\zeta_p]$ lying over $q$.
  Let $D = D_{\q / q} \subseteq (\Z / p\Z)^\times$, where
  $D$ corresponds to $\langle q \rangle$. Then
  $K_p^D$ is the largest
  subfield of $K_p$ in which $q$ splits completely
  (since $K_p / \Q$ is Galois). But
  $L = K_p^H$, so Galois theory implies that
  \[
    \text{$q$ splits in $L$}
    \iff L \subseteq K_p^D \iff D \subseteq H
    \iff q \in H
    \iff \left(\frac{q}{p}\right) = 1.
  \]
\end{remark}

\begin{corollary}[Law of quadratic reciprocity]\label{cor:quadratic-reciprocity}
  Let $p, q$ be distinct odd primes. Then
  \[
    \left(\frac{p}{q}\right)
    = (-1)^{(p - 1) / 2} (-1)^{(q - 1) / 2}
    \left(\frac{q}{p}\right)
  \]
\end{corollary}

\begin{proof}
  Recall Euler's criterion that
  $(\frac{a}{q}) \equiv a^{(q - 1) / 2} \Pmod{q}$.
  Then by the above calculation,
  \[
    \left(\frac{q}{p}\right)
    = \left(\frac{p^*}{q}\right)
    = \left(\frac{(-1)^{(p - 1) / 2}}{q}\right)
    \left(\frac{p}{q}\right)
    = ((-1)^{(p - 1) / 2})^{(q - 1) / 2}
    \left(\frac{p}{q}\right),
  \]
  which is the desired formula.
\end{proof}

\begin{remark}
  The above argument can be generalized more
  readily than other more elementary proofs of
  quadratic reciprocity.
  Let $L / K$ be a finite
  abelian extension of number fields (i.e.
  $\Gal(L / K)$ is abelian).
  One can embed $L$
  into some analogue $M$ of $K_p$ (this is called a
  \emph{ray class field}), so
  $K \subseteq L \subseteq M$.

  One then analyzes
  how $\p$ factors directly in $L / K$ and also
  using the $L = M^H$ perspective.
  This results in
  \emph{Artin's reciprocity law}, which
  is a result of \emph{class field theory}.
  As a subject, class field theory
  tries to understand the abelian extensions of a
  number field.
\end{remark}

\begin{theorem}[Hilbert]
  Let $K$ be a number field. Then there exists a
  maximal finite abelian extension $H$ of $K$ which is
  unramified at all primes (and $\infty$). Moreover,
  \[
    \Gal(H / K) \cong \Cl(\OO_K).
  \]
\end{theorem}

\section{Gauss Sums}

\begin{definition}
  Let $\zeta = \zeta_p = e^{2\pi i / p}$. Define the
  \emph{Gauss sum}
  \[
    g = \sum_{t \in (\Z / p\Z)^\times} \left(\frac{t}{p}\right) \zeta_p^t.
  \]
\end{definition}

\begin{lemma}
  We have $g^2 = p^*$.
\end{lemma}

\begin{proof}
  We sketch the proof. Define
  \[
    g_a = \sum_{t \in (\Z / p\Z)^\times} \left(\frac{t}{p}\right)\zeta_p^{at}.
  \]
  Note that we have the identity $g_a = (\frac{a}{p}) g$:
  \[
    g = \sum_{b \in (\Z / p\Z)^\times} \left(\frac{b}{p}\right) \zeta_p^b
    = \sum_{t \in (\Z / p\Z)^\times} \left(\frac{at}{p}\right) \zeta_p^{at}
    = \left(\frac{a}{p}\right)\sum_{t \in (\Z / p\Z)^\times} \left(\frac{t}{p}\right) \zeta_p^{at}
  \]
  since multiplication by $a$ is a permutation on
  $(\Z / p\Z)^\times$. Now we calculate
  $\sum_a g_a g_{-a}$ in two different ways:
  \[
    g_a g_{-a}
    = \left(\frac{a}{p}\right)\left(\frac{-a}{p}\right) g^2
    = \left(\frac{-1}{p}\right) g^2,
  \]
  so that $\sum_a g_{a} g_{-a} = (\frac{-1}{p}) (p - 1) g^2$.
  On the other hand, we can also write
  \[
    g_a g_{-a} =
    \sum_{x, y} \left(\frac{xy}{p}\right) \zeta_p^{a(x - y)},
  \]
  so $\sum_a g_a g_{-a} = \sum_{x, y} (\frac{xy}{p}) \sum_a \zeta_p^{a(x - y)} = p(p - 1)$.
  Comparing the two formulas gives
  $g^2 = p^*$.
\end{proof}

\begin{proof}[Proof of Corollary \ref{cor:quadratic-reciprocity} using Gauss sums]
  Let $p, q$ be distinct primes. First we have
  \[
    g^q \equiv g_q = \left(\frac{q}{p}\right) g
    \pmod{q \Z[\zeta_p]},
  \]
  so we have
  \[
    p^* (g^2)^{(q - 1) / 2}
    = g^2 (g^2)^{(q - 1) / 2}
    = g^{q + 1} \equiv \left(\frac{q}{p}\right) g^2
    = \left(\frac{q}{p}\right) p^* \pmod{q \Z[\zeta_p]}.
  \]
  Since $(g^2)^{(q - 1) / 2} = (\frac{p^*}{q})$,
  comparing the above expressions
  yields $(\frac{p^*}{q}) = (\frac{q}{p})$,
  as desired.
\end{proof}

\begin{corollary}
  We have $\Q(\sqrt{p^*}) \subseteq \Q(\zeta_p)$.
\end{corollary}

\begin{remark}
  Gauss actually computed the value of $g$, in addition
  to $g^2$. Gauss found that
  \[
    g =
    \begin{cases}
      \sqrt{p} & \text{if $p \equiv 1 \Pmod{4}$}, \\
      i \sqrt{p} & \text{if $p \equiv 3 \Pmod{4}$}.
    \end{cases}
  \]
  One can also formulate a similar question for
  cubes, but the answer is much more complicated.
\end{remark}
