\documentclass[12pt, letterpaper, oneside]{book}
\usepackage[margin={0.6in, 0.75in}]{geometry}
\usepackage{microtype}
% \usepackage{kpfonts}
\usepackage{amsmath, amssymb, amsthm}
\usepackage{parskip}
\usepackage[many]{tcolorbox}
\usepackage{footnote}
\usepackage{cancel}
\usepackage{titlesec}
\usepackage{pgffor}
\usepackage[shortlabels, inline]{enumitem}
\usepackage{hyperref}
\usepackage{tikz-cd}

\usepackage[overload]{textcase}

\renewcommand{\chaptername}{Lecture}
\newtheorem{axiom}{Axiom}[chapter]
\newtheorem{theorem}{Theorem}[chapter]
\newtheorem{prop}{Proposition}[chapter]
\newtheorem{corollary}{Corollary}[theorem]
\newtheorem{lemma}{Lemma}[chapter]
\theoremstyle{definition}
\newtheorem{definition}{Definition}[chapter]
\newtheorem{exercise}{Exercise}[chapter]
\newtheorem{example}{Example}[definition]
\newtheorem*{remark}{Remark}

\tcbset{sharp corners, breakable, enhanced, parbox=false}
\newtcolorbox{mybox}[3][]
{
  colframe = #2!150,
  colback  = #2!5,
  coltitle = #2!0!white,  
  title    = {#3},
  #1,
}

\titleformat{\chapter}[display]
    {\normalfont\huge\bfseries}{\chaptertitlename\ \thechapter}{20pt}{\Huge}
\titlespacing*{\chapter}{0pt}{0pt}{40pt}

\newcommand{\R}{\mathbb{R}}
\newcommand{\N}{\mathbb{N}}
\newcommand{\Z}{\mathbb{Z}}
\newcommand{\C}{\mathbb{C}}
\newcommand{\Q}{\mathbb{Q}}
\newcommand{\F}{\mathbb{F}}
\newcommand{\OO}{\mathcal{O}}
\newcommand{\p}{\mathfrak{p}}
\newcommand{\q}{\mathfrak{q}}
\newcommand{\m}{\mathfrak{m}}
\newcommand{\sphere}{\mathbb{S}}
\newcommand{\ZF}{\mathsf{ZF}}
\newcommand{\ZFC}{\mathsf{ZFC}}
\newcommand{\AC}{\mathsf{AC}}
\newcommand{\Mod}[1]{\ {\mathrm{mod}\ #1}}
\newcommand{\Pmod}[1]{\ (\mathrm{mod}\ #1)}

\newcommand{\T}{\mathcal{T}}
\newcommand{\B}{\mathcal{B}}

\DeclareMathOperator{\vol}{vol}
\DeclareMathOperator{\Int}{int}
\DeclareMathOperator{\area}{area}
\DeclareMathOperator{\diag}{diag}
\DeclareMathOperator{\id}{id}
\DeclareMathOperator{\Cl}{Cl}
\DeclareMathOperator{\ord}{ord}
\DeclareMathOperator{\adj}{adj}
\DeclareMathOperator{\disc}{disc}
\DeclareMathOperator{\lcm}{lcm}
\DeclareMathOperator{\Gal}{Gal}
\DeclareMathOperator{\re}{Re}
\DeclareMathOperator{\im}{Im}
\DeclareMathOperator{\GL}{GL}
\DeclareMathOperator{\covol}{covol}
\DeclareMathOperator{\rank}{rank}
\DeclareMathOperator{\Spec}{Spec}
\DeclareMathOperator{\Frac}{Frac}

\title{MATH 6122: Algebra II}
\author{Frank Qiang\\Instructor: Matthew Baker}
\date{Georgia Institute of Technology\\Spring 2025}

\begin{document}
  \maketitle

  \begingroup
  \let\cleardoublepage\clearpage
  \tableofcontents
  \endgroup

  % \foreach \i in {00, 01, 02, 03, 04, ..., 50} {%
  %   \edef\FileName{lectures/lecture\i.tex}%     The % here are necessary to eliminate any
  %   \IfFileExists{\FileName}{%  spurious spaces that may get inserted
  %      \input{\FileName}%       at these points
  %   }
  % }
  \chapter{Jan.~7 --- Motivation for Algebraic Number Theory}

\section{Motivation: Fermat's Last Theorem}

\begin{theorem}[Fermat's last theorem\footnote{This problem was finally resolved by Wiles-Taylor in 1995.}]
  $x^n + y^n = z^n$
  has no nonzero integer solutions when $n \ge 3$.
\end{theorem}

\begin{remark}
  The $n = 3$ case was likely solved by Fermat, and
  Euler and Gauss had work for $n = 4$. Thus we will
  assume
  $n \ge 5$. We can also assume $n$ is prime, since
  if $n = pm$, then we can equivalently consider
  \[
    (x^m)^p + (y^m)^p = (z^m)^p.
  \]
  So let $p \ge 5$ be prime. Let
  $\zeta = \zeta_p$ be a primitive $p$th root of $1$.
  Then consider
  \[
    x^p + y^p = (x + y) (x + \zeta y) (x + \zeta^2 y) \dots (x + \zeta^{p-1} y) = z^p.
  \]
  Note that $x + \zeta^j y \in \Z[\zeta] \subseteq \C$.
  Let us pretend for the moment that $\Z[\zeta]$ is
  a UFD.\footnote{It is far from it, and this is likely the mistake that Fermat originally made.}
  One can check that
  \[
    \gcd(x + \zeta^j y, x + \zeta^k y) = 1
  \]
  whenever $j \ne k$. If $\Z[\zeta]$ were a UFD, then
  we could conclude that
  \[
    x + y \zeta = u \alpha^p
  \]
  for some $u \in \Z[\zeta]^\times$ and
  $\alpha \in \Z[\zeta]$.\footnote{In a UFD, if a product of relatively prime elements is a $p$th power, then each factor must itself be a $p$th power.}
  For the sake of illustration, suppose $u = \pm \zeta^j$
  for some $j$.
  Then
  \[
    \alpha = a_0 + a_1 \zeta + \dots + a_{p - 2} \zeta^{p - 2}
  \]
  for $a_i \in \Z$. This gives
  \[
    \alpha^p = a_0 + a_1 + \dots + a_{p - 2} \pmod{p},
  \]
  using Fermat's little theorem, $\zeta^p = 1$,
  and the binomial theorem. So
  $\alpha^p = a \text{ (mod $p$)}$ with $z \in \Z$,
  and
  \[
    x + y \zeta = \pm a \zeta^j \pmod{p}
  \]
  for some $0 \le j \le p - 1$. Note that
  $\zeta^p = -(1 + \zeta + \dots + \zeta^{p - 2})$,
  and one can check as an exercise that this implies
  $p | x$ or $p | y$. This would have proved the
  ``first case'' of Fermat's last theorem.
\end{remark}

\begin{remark}
  However, Kummer (c.~1850) observed that $\Z[\zeta]$ is
  rarely a UFD (in fact, $\Z[\zeta]$ is a UFD if and
  only if $p \le 19$).\footnote{Kummer made the first real progress on Fermat's last theorem in quite a while.} Also, when
  $p \ge 5$, the unit group of $\Z[\zeta]$ is
  always infinite (so that $\Z[\zeta]^\times \ne \{\pm \zeta^j\}$).
\end{remark}

\begin{theorem}[Kummer]
  Fermat's last theorem holds for all ``regular'' primes.\footnote{A prime $p$ is \emph{regular} if $p$ does not divide the order of the \emph{ideal class group} of $\Z[\zeta]$.}
\end{theorem}

\begin{remark}
  The first irregular prime is $37$, so Kummer's method
  works for $3 \le n \le 36$.
\end{remark}

\section{Algebraic Integers}

\begin{remark}
  To resolve these issues,
  Kummer realized that one can replace elements
  of $\Z[\zeta]$ by ``ideal elements.'' Later on,
  Dedekind look at Kummer's work and introduced
  the modern notion of an ideal. We will be working
  towards the \emph{unique factorization of ideals into
  prime ideals} in certain cases.
\end{remark}

\begin{remark}
  We will work at the level of generality of Dedekind
  rings (as opposed to just number rings). This is
  because there is an analogue of such a unique
  factorization of ideals
  for function fields of curves in algebraic geometry,
  and this framework is general enough to capture
  both cases.
\end{remark}

\begin{definition}
  Let $K / \Q$ be a finite extension (i.e. a \emph{number field}).
  Then $\alpha \in K$ is an \emph{algebraic integer}
  if there exists a monic polynomial $f \in \Z[x]$
  such that $f(\alpha) = 0$.
\end{definition}

\begin{theorem}
  Let $A \subseteq B$ be rings and let $b \in B$.
  Then the following are equivalent:
  \begin{enumerate}
    \item $b$ is integral over $A$ (i.e. there exists
      a monic $f \in A[x]$ such that $f(b) = 0$).
    \item $A[b]$ is a finitely generated $A$-module.\footnote{Here $A[b]$ is the smallest subring of $B$ containing $A$ and $b$, so $A[b] = \{a_0 + a_1 b + a_2 b^2 + \dots + a_k b_k : a_i \in A\}$.}
    \item $A[b]$ is contained in a subring $C \subseteq B$
      which is finitely generated as an $A$-module.
  \end{enumerate}
\end{theorem}

\begin{proof}
  $(1 \Rightarrow 2)$ This direction is standard, one
  only needs powers up to $\deg f$ since $f(b) = 0$.

  $(2 \Rightarrow 3)$ This direction is clear
  since $A[b]$ itself satisfies the desired conditions.

  $(3 \Rightarrow 1)$ The idea is to argue via
  determinants and use the Cayley-Hamilton theorem for
  modules.
\end{proof}

\begin{corollary}
  Integrality is transitive, i.e. if $B$ is integral
  over $A$ and $C$ is integral over $B$, then $C$ is
  integral over $A$.\footnote{We say that $B$ is \emph{integral over $A$} if every $b \in B$ is integral over $A$.}
\end{corollary}

\begin{proof}
  A finitely generated module over a finitely
  generated module is finitely generated.
\end{proof}

\begin{corollary}
  If $\alpha, \beta$ are integral over $A$, then
  $\alpha \pm \beta, \alpha \beta$ are also integral
  over $A$.
\end{corollary}

\begin{proof}
  This is because
  $\alpha \pm \beta, \alpha \beta \subseteq C = A[\alpha][\beta]$.
\end{proof}

\begin{theorem}
  The set of all algebraic integers in $K$ (denoted
  $\mathcal{O}_K$) forms a
  subring of $K$.\footnote{This theorem is not obvious: Given $f(\alpha) = 0$ and
  $g(\beta) = 0$, one must find a polynomial $h$ such
  that $h(\alpha + \beta) = 0$.}
\end{theorem}

  \chapter{Jan.~9 --- Algebraic Integers and Dedekind Domains}

\section{More on Algebraic Integers}
\begin{prop}
  Suppose $\alpha, \beta \in \overline{\Z} \subseteq \C$,
  then
  $\alpha + \beta, \alpha \beta \in \overline{\Z}$.\footnote{Here $\overline{\Z}$ is the set of algebraic integers.}
\end{prop}

\begin{proof}
  First, note that every algebraic integer is an
  eigenvalue of some integer matrix (e.g. take the
  companion matrix for the minimal polynomial).
  So take linear maps $T_\alpha : V_\alpha \ to V_\alpha$
  and $T_\beta : V_\beta \to V_\beta$ which have
  $\alpha$ and $\beta$ as eigenvalues, respectively.
  Then one can check that the map on the direct sum
  \[
    T_\alpha \oplus T_\beta : V_\alpha \oplus V_\beta \to V_\alpha \oplus V_\beta
  \]
  has $\alpha + \beta$ as an eigenvalue. Similarly,
  by looking at the map on the tensor product
  \[
    T_\alpha \otimes T_\beta : V_\alpha \otimes V_\beta \to V_\alpha \otimes V_\beta
  \]
  has $\alpha \beta$ as an eigenvalue. Hence we see
  that
  $\alpha + \beta, \alpha \beta \in \overline{\Z}$ as
  well.
\end{proof}

\begin{remark}
  This is a constructive proof of what we showed via
  finitely generated modules last time.
\end{remark}

\begin{lemma}
  Let $\alpha \in K$ be an algebraic number. Then
  $\alpha$ is an algebraic integer, i.e.
  $\alpha \in \mathcal{O}_K$, if and only if the
  minimal polynomial of $\alpha$ over $\Q$, call it
  $f_\alpha \in \Q[x]$, has integer coefficients.
\end{lemma}

\begin{proof}
  $(\Leftarrow)$ This direction is clear by the
  definition of an algebraic integer.

  $(\Rightarrow)$ We need to show that if
  $\alpha \in \mathcal{O}_K$, then $f_\alpha \in \Z[x]$.
  By assumption, there exists some monic integer
  polynomial $h \in \Z[x]$ such that $h(\alpha) = 0$.
  From this, we know that $f_\alpha | h$ in
  $\Q[x]$.\footnote{Note that it suffices to show that $f_\alpha | h$ in $\Z[x]$, so from here, a suitable version of Gauss's lemma immediately implies the desired result.}
  Let $\alpha_1, \dots, \alpha_n$ be the roots of
  $f_\alpha$ with $\alpha_1 = \alpha$. Since
  $f_\alpha | h$, we know that
  $h(\alpha_i) = 0$ for every $i$, so $h \in \Z[x]$
  implies that $\alpha_i \in \overline{\Z}$ for each $i$.
  Thus the coefficients of $f_\alpha$ are elementary
  symmetric functions of the $\alpha_i$,\footnote{These operations preserve the notion of being an algebraic integer.} so
  \[
    f_\alpha \in (\overline{Z} \cap \Q)[x].
  \]
  Thus it suffices to show that $\overline{\Z} \cap \Q = \Z$ to conclude the result.
  For this, suppose $r / s \in \Q$ is the root of
  \[
    x^n + a_{n-1} x^{n-1} + \dots + a_1 x + a_0 \in \Z[x].
  \]
  We can assume $(r, s) = 1$ without loss of generality.\footnote{Here we write $(r, s)$ to denote $\gcd(r, s)$.}
  Plugging in, we obtain
  \[
    (r / s)^n + a_{n-1} (r / s)^{n-1} + \dots + a_1 (r / s) + a_0 = 0.
  \]
  Clearly denominators by multiplying by $s^n$, we
  obtain
  \[
    r^n + a_{n - 1} sr^{n - 1} + \dots + a_1 s^{n - 1} r + a_0 s^n = 0
  \]
  The right-hand side is divisible by $s$ and every
  term on the left-hand side except $r^n$ is divisible
  by $s$, so we must have $s | r^n$. Since $(r, s) = 1$,
  this implies that $s = \pm 1$, i.e. $r / s \in \Z$.
\end{proof}

\begin{example}
  For $K = \Q$, we have $\mathcal{O}_K = \Z$. This
  follows from the previous lemma since
  the minimal polynomial of $a \in \Q$ is $x - a$, which
  has integer coefficients precisely when $a \in \Z$.
\end{example}

\begin{example}
  Let $K = \Q(\sqrt{d})$, i.e. $K$ is \emph{quadratic number field}.
  Clearly $\Z[\sqrt{d}] \subseteq \mathcal{O}_K$,
  but this is not always an equality. For example,
  \[
    \phi = \frac{1 + \sqrt{5}}{2} \notin \Z[\sqrt{5}],
  \]
  but $x^2 - x - 1$ has $\phi$ as a root.
\end{example}

\begin{exercise}
  Let $d$ be a square-free integer and
  $K = \Q(\sqrt{d})$. Show that
  \[
    \mathcal{O}_K =
    \begin{cases}
      \Z[\sqrt{d}] & \text{if $d \equiv 2, 3 \Pmod{4}$}, \\
      \Z[(1 + \sqrt{d}) / 2] & \text{if $d \equiv 1 \Pmod{4}$}.
    \end{cases}
  \]
\end{exercise}

\begin{definition}
  Let $S$ be a ring.
  If $R \subseteq S$ is a subring, then we say that
  $R$ is \emph{integrally closed} in $S$ if whenever
  $\alpha \in S$ is integral over $R$, then
  $\alpha \in R$.
\end{definition}

\begin{remark}
  Recall that for a domain $R$, its \emph{field of fractions} $K$ is the
  localization
  \[
    K = S^{-1} R
  \]
  where $S = R \setminus \{0\}$. There is a natural
  embedding of $R$ into $K$ via $r \mapsto r / 1$.
\end{remark}

\begin{lemma}
  The fraction field of $\mathcal{O}_K$ is $K$. More
  precisely, for every $\alpha \in K$, there exists
  $m \in \Z$, $m \ne 0$, such that $m\alpha \in \mathcal{O}_K$.
\end{lemma}

\begin{proof}
  Since $\alpha$ is algebraic, there exists some monic
  polynomial $f_\alpha \in \Q[x]$ such that
  $f_\alpha(\alpha)$. By clearing denominators, there
  exists $m \in \Z$ such that $mf_\alpha \in \Z[x]$.
  So we have
  \[
    m \alpha^n + b_{n - 1} \alpha^{n - 1} + \dots + b_1 \alpha + b_0 = 0,
  \]
  and multiplying by $m^{n - 1}$ on both sides, we obtain
  \[
    m^n \alpha^n + m^{n - 1} b_{n - 1} \alpha^{n - 1} + \dots + m^{n - 1} b_1 \alpha + m^{n - 1} b_0 = 0,
  \]
  which implies
  \[
    (m\alpha)^n + b_{n - 1} (m\alpha)^{n - 1} + \dots + m^{n - 2} b_1 (m\alpha) + m^{n - 1} b_0 = 0.
  \]
  This shows that $m\alpha$ is integral, i.e.
  $m\alpha \in \mathcal{O}_K$.
\end{proof}

\begin{theorem}
  The ring of integers $\mathcal{O}_K$ is
  integrally closed (in its fraction field).
\end{theorem}

\begin{proof}
  Transitivity of integrality implies that $\mathcal{O}_K$
  is integrally closed in $K$. The theorem then
  follows from the fact that $K$ is the fraction field
  of $\mathcal{O}_K$.
\end{proof}

\begin{remark}
  The theorem says that (it implies the second equality)
  \[
    \mathcal{O}_K
    = \{\alpha \in K \mid \text{$\alpha$ is integral over $\Z$}\}
    = \{\alpha \in K \mid \text{$\alpha$ is integral over $\mathcal{O}_K$}\}.
  \]
\end{remark}

\section{Dedekind Domains}

\begin{definition}
  A \emph{Dedekind domain} is a Noetherian
  integrally closed domain of dimension $1$.
\end{definition}

\begin{remark}
  Recall that all rings in this class are commutative
  and have a $1$. A dimension $1$ domain is a domain
  which is not a field and every nonzero prime ideal
  is maximal. In general, the dimension of a ring $R$
  is the maximum length of a chain of prime ideals
  of the form
  \[
    \mathfrak{p}_0 \subsetneq \mathfrak{p}_1 \subsetneq \dots \subsetneq \mathfrak{p}_n.
  \]
  In dimension $1$, this corresponds to
  $(0) \subsetneq \mathfrak{p}$ being the maximum
  chain for every nonzero prime
  ideal $\mathfrak{p}$, which is equivalent to the
  other definition.
\end{remark}

\begin{remark}
  Our goal for now will be to show that $\mathcal{O}_K$
  is a Dedekind domain.
\end{remark}

\begin{definition}
  Let $k$ be either $\Q$ or $\R$ and
  $V$ be a finite-dimensional $k$-vector
  space. A \emph{complete lattice} in $V$ is a
  discrete additive subgroup $\Lambda$ of $V$ which
  spans $V$, where discrete means that any
  bounded subset of $\Lambda$ is finite (equivalent to
  being discrete in the sense of topology).
\end{definition}

\begin{prop}
  Let $V$ be as above (dimension $n$ over $k$) and
  $\Lambda \subseteq V$ an additive subgroup which
  spans $V$. Then the following are equivalent:
  \begin{enumerate}
    \item $\Lambda$ is discrete.
    \item $\Lambda$ is generated by $n$ elements.
    \item $\Lambda \cong \Z^n$ as $\Z$-modules.
  \end{enumerate}
\end{prop}

\begin{proof}
  $(2 \Leftrightarrow 3)$ This follows by the structure
  theorem.

  $(1 \Rightarrow 2)$ Suppose $\Lambda$ is discrete, and
  let $x_1, \dots, x_n \in \Lambda$ be a basis for $V$.
  Let $\Lambda_0$ be the $\Z$-module which is
  spanned by $x_1, \dots, x_n$. We claim that
  $\Lambda / \Lambda_0$ is finite, which implies that
  $\Lambda$ is also generated by $n$ elements (exercise).
  To see the claim, we note that there exists
  $M > 0$ such that if
  $x = \sum \lambda_i x_i \in \Lambda$ with
  $\lambda_i \in k$ and all $|\lambda_i| < 1 / M$,
  then $x = 0$. This is standard and follows from
  all norms being equivalent in a finite-dimensional
  vector space and the assumption that $\Lambda$ is
  discrete.

  Now let $y_1, y_2, \dots$ be coset representatives
  for $\Lambda / \Lambda_0$. Without loss of generality
  (by translating in the coset),
  assume each $y_i \in C$, where $C$ is the unit cube.
  Cover $C$ by $M^n$ boxes of the form
  \[
    \frac{m_i}{M} \le \lambda_i < \frac{m_i + 1}{M}
  \]
  with $m_i \in \Z$ and $0 \le m_i < M$. We must have
  $|\Lambda / \Lambda_0| \le M^n$, since otherwise
  we end up with two $y_i \ne y_j$ in the same box by the
  pigeonhole principle, and
  $y_i - y_j \in C[1 / M] \cap \Lambda = \{0\}$ leads
  to a contradiction.

  $(2 \Rightarrow 1)$ This proof is to be finished next
  class.
\end{proof}

\begin{theorem}
  If $I$ is a nonzero ideal in a number ring
  $\mathcal{O}_K$, then $\mathcal{O}_K / I$ is finite.
\end{theorem}

\begin{proof}
  The strategy is to show that if $[K : \Q] = n$, then
  $\mathcal{O}_K \cong \Z^n$ and $I \cong \Z^n$
  as $\Z$-modules. This will imply that
  $\mathcal{O}_K / I$ is finite, which follows from
  the proof of the structure theorem. In fact, we will
  show the that $I$ and $\mathcal{O}_K$
  are lattices in $K \cong \Q^n \subseteq \R^n$.
  Note
  that it suffices to show that $\mathcal{O}_K$ is
  a lattice, since it immediately follows that
  $I \subseteq \mathcal{O}_K$ is also discrete, hence
  also a lattice as $I$ is an additive subgroup.

  The proof is to be finished next class.
\end{proof}

\begin{corollary}
  A number ring $\mathcal{O}_K$ is Noetherian.
\end{corollary}

\begin{proof}
  Suppose that we have an ascending chain of ideals
  \[
    I = I_0 \subseteq I_1 \subseteq I_2 \subseteq \dots.
  \]
  Suppose without loss of generality that $I_0 \ne 0$.
  Since $\mathcal{O}_K / I$ is finite, by an isomorphism
  theorem we see that there are only finitely many ideals
  in $\mathcal{O}_K$ containing $I$. This implies that
  the chain must eventually stabilize, i.e. that
  $\mathcal{O}_K$ is Noetherian.
\end{proof}

\begin{corollary}
  A number ring $\mathcal{O}_K$ is 1-dimensional.
\end{corollary}

\begin{proof}
  Verify as an exercise that $\mathcal{O}_K$ is not
  a field.
  Let $\mathfrak{p}$ be a nonzero prime ideal, so that
  $\mathcal{O}_K / \mathfrak{p}$ is a finite domain,
  hence a field. This implies that $\mathfrak{p}$ is
  maximal, so $\mathcal{O}_K$ is 1-dimensional.
\end{proof}

  \chapter{Jan.~14 --- Unique Factorization of Ideals}

\section{Norm in a Field}

\begin{remark}
  Let $K / \Q$ be a finite extension of degree $n$.
  Our goal will be to define a \emph{norm}
  $N_{K / \Q} : K \to \Q$ which also sends
  $\mathcal{O}_K \to \Z$. Note that there are $n$
  distinct embeddings $\sigma_1, \dots, \sigma_n : K \to \C$, e.g.
  choose a primitive element $\theta \in K$
  (so that $K = \Q(\theta)$) with minimal polynomial
  $f$ of degree $n$ and define
  $\sigma : K \to \C$ by sending $\theta$ to
  some root of $f$, of which there are $n$ choices.\footnote{As an example of having $n$ embeddings, consider $\Q(\sqrt{2}) \subseteq \R \subseteq \C$, where we can send $\sqrt{2} \mapsto \pm \sqrt{2}$.}
\end{remark}

\begin{definition}
  Given a finite extension $K / \Q$, define the
  \emph{norm} $N_{K / \Q} : K \to \Q$ by
  \[
    N_{K / \Q}(x) = \prod_{i = 1}^n \sigma_i(x),
  \]
  where $\sigma_1, \dots, \sigma_n : K \to \C$
  are the $n$ distinct embeddings of $K$ into $\C$.
\end{definition}

\begin{exercise}
  Show that in fact $N_{K / \Q}(\gamma) \in \Q$. (Hint: One way is via Galois theory.)
\end{exercise}

\begin{exercise}
  Define $[\gamma] : K \to K$ by $x \mapsto \gamma x$,
  which is a $\Q$-linear map. Show that
  $N_{K / \Q}(\gamma) = \det [\gamma]$.
\end{exercise}

\begin{prop}
  We have the following properties of the norm $N = N_{K / \Q}$:
  \begin{enumerate}
    \item $N(\gamma) = 0$ if and only if $\gamma = 0$;
    \item if $\gamma \in \mathcal{O}_K$, then
      $N(\gamma) \in \Z$.
  \end{enumerate}
\end{prop}

\begin{proof}
  Check these properties as an exercise.
\end{proof}

\begin{theorem}
  A number ring $\mathcal{O}_K$ is a complete
  lattice in $K \cong \Q^n \subseteq \R^n$.
\end{theorem}

\begin{proof}
  We need to show that $\mathcal{O}_K$ is discrete.
  Note that there exists a basis $\alpha_1, \dots, \alpha_n$
  for $K / \Q$ such that $\alpha_i \in \OO_K$
  for every $i$.
  Now suppose otherwise that $\OO_K$ is not discrete,
  so there
  are arbitrarily small $\lambda_1, \dots, \lambda_n \in \Q$
  such that $\alpha = \sum \lambda_i \alpha_i$
  is nonzero and in $\OO_K$. Then
  \[
    N_{K / \Q}(\alpha)
    = \phi(\lambda_1, \dots, \lambda_n)
  \]
  for some homogeneous polynomial $\phi$ of
  degree $n$ (since each
  $\sigma(\alpha) = \sum \lambda_i \sigma(\alpha_i)$).
  Thus if $|\lambda_i| \ll 1$, the polynomial
  $\phi$ also gets small and we can obtain
  $0 < |N_{K / \Q}(\alpha)| < 1$, a contradiction
  since $N_{K / \Q}(\alpha) \in \Z$.
\end{proof}

\begin{corollary}
  If $I \subseteq \OO_K$ is a nonzero ideal, then
  $I$ is also a complete lattice in $\R^n$.
\end{corollary}

\begin{proof}
  One needs to show that $I$ contains a basis for
  $K / \Q$. Choose any nonzero $c \in I$ and consider
  $c \alpha_1, \dots, c \alpha_n \in I$ (since $I$
  is an ideal). This will also be a basis for
  $K / \Q$ since $c \ne 0$.
\end{proof}

\begin{corollary}
  We have $|\OO_K / I| < \infty$ for every
  nonzero ideal $I \subseteq \OO_K$.
\end{corollary}

\begin{proof}
  This is because $\OO_K \cong I \cong \Z^n$
  as $\Z$-modules, so the result follows by the
  structure theorem.
\end{proof}

\begin{remark}
  These details complete the proof from last time that
  $\OO_K$ is a Dedekind domain.
\end{remark}

\begin{remark}
  The following is a preview of what we will do
  later in the class: We will define the
  \emph{norm} of an ideal to be $N(I) = |\OO_K / I|$.
  One can show that if $I = (\gamma)$, then
  $N(I) = N(\gamma)$. An extension of the previous
  techniques then leads to a proof of the finiteness
  of the \emph{ideal class group}.
\end{remark}

\section{Unique Factorization of Ideals}

\begin{remark}
  Recall that for ideals $I = (\alpha_1, \dots, \alpha_k)$
  and $J = (\beta_1, \dots, \beta_\ell)$, their
  \emph{product} is
  $IJ = (\alpha_i \beta_j)_{i, j}$.
\end{remark}

\begin{example}
  Consider $R = \Z[\sqrt{-5}]$, which is the
  ring of integers $\OO_K$ in $K = \Q(\sqrt{-5})$.
  Note that
  \[
    6 = 2(3) = (1 + \sqrt{-5})(1 - \sqrt{-5})
  \]
  and these elements are irreducible and not associates,
  so $R$ is not a UFD. However, let
  \[
    \p_1 = (2, 1 + \sqrt{-5}),
    \quad \p_2 = (2, 1 - \sqrt{-5}),
    \quad \p_3 = (3, 1 + \sqrt{-5}),
    \quad \p_4 = (3, 1 - \sqrt{-5}).
  \]
  None of these ideals are principal, but they are
  all prime ideals. One can check that
  \[
    \p_1 \p_2 = (4, 2 - 2\sqrt{-5}, 2 + 2 \sqrt{-5}, 6) = (2),
  \]
  that
  $\p_3 \p_4 = (3)$, that
  \[
    \p_1 \p_3 = (6, 2 + 2 \sqrt{-5}, 3 + 3\sqrt{-5}, 6)
    = (1 + \sqrt{-5}),
  \]
  and finally that $\p_2 \p_4 = (1 - \sqrt{-5})$. At
  the level of ideals, the original equation then becomes
  \[
    (6) = (2)(3) = (\p_1 \p_2) (\p_3 \p_4)
    = (\p_1 \p_3) (\p_2 \p_4)
    = (1 + \sqrt{-5})(1 - \sqrt{-5}).
  \]
  In fact, the previous nonunique factorization is
  now the same factorization in the language of ideals.
\end{example}

\begin{lemma}
  Let $I_1, \dots, I_n$ be ideals in a commutative
  ring $R$, and let $\p$ be a prime ideal. Suppose that
  $I_1 I_2 \dots I_n \subseteq \p$. Then
  $I_j \subseteq \p$ for some $j$.
\end{lemma}

\begin{proof}
  Check this as an exercise, it follows from the
  definition of a prime ideal.
\end{proof}

\begin{lemma}
  Let $R$ be a Noetherian ring, and $I \subseteq R$
  be a nonzero ideal. Then there exist
  nonzero prime ideals $\p_1, \dots, \p_r$ such that
  $\p_1 \p_2 \dots \p_r \subseteq I$.
\end{lemma}

\begin{proof}
  Let $\Sigma$ be the set of all $I$ for which
  the lemma is false. If $\Sigma \ne \varnothing$, then
  since $R$ is Noetherian, $\Sigma$ has a maximal
  element (pick $I_1 \in \Sigma$, if it is not maximal,
  then we can find $I_2 \in \Sigma$ with
  $I_1 \subsetneq I_2$, and we obtain
  $I_1 \subsetneq I_2 \subsetneq \dots$ by continuing;
  this chain must terminate since $R$ is Noetherian).
  Let $J$ be such a maximal element. Now $J$ cannot
  be prime, so there exist $a, b \in R$ such that
  $ab \in J$ but $a, b \notin J$. Let
  \[
    \mathfrak{a} = (J, a) \supsetneq J \quad
    \text{and} \quad \mathfrak{b} = (J, b) \supsetneq J.
  \]
  Then $\mathfrak{a} \supseteq \p_1 \p_2 \dots \p_m$
  and $\mathfrak{b} \supseteq \mathfrak{q}_1 \mathfrak{q}_2 \dots \mathfrak{q}_n$.
  Since $\mathfrak{a} \mathfrak{b} = (J^2, Ja, Jb, ab) \subseteq J$, we obtain
  \[
    J \supseteq \mathfrak{a} \mathfrak{b} \supseteq \p_1 \dots \p_m \mathfrak{q}_1 \dots \mathfrak{q}_n,
  \]
  which is a contradiction. Thus we must have
  $\Sigma = \varnothing$, so the lemma holds for
  every nonzero ideal $I$.
\end{proof}

\section{Inverse Ideals}

\begin{example}
  Consider the problem of finding $(2)^{-1}$ in $\Z$.
  Logically, the answer should be something like $(1 / 2) = (1 / 2) \Z \subseteq \Q$,
  which is not an ideal in $\Z$.\footnote{Note that this is not an ideal of $\Q$ either since it is not closed under multiplication by elements of $\Q$. The inverse ideal $(2)^{-1}$ is instead a $\Z$-submodule of $\Q$, viewed as a $\Z$-module.}
  This will satisfy $2 ((1 / 2)\Z) = \Z$.
\end{example}

\begin{definition}
  Let $R$ be an integral domain with fraction field $K$,
  and let $I$ be a nonzero ideal in $R$. Then
  the \emph{inverse ideal} $I^{-1}$ of $I$ is
  \[
    I^{-1} = \{x \in K \mid xI \subseteq R\}.
  \]
\end{definition}

\begin{example}
  Let $R = \Z$ and $I = (2)$. Then we can
  see that
  \[
    I^{-1} = \{x \in \Q \mid x (2) \subseteq \Z\}
    = \frac{1}{2} \Z.
  \]
\end{example}

\begin{remark}
  Our goal at this point is to show that if $R$ is
  Dedekind, then $I I^{-1} = R$. Note that if
  $M, N$ are two $R$-submodules of $K$, then their
  product is well-defined:
  \[
    MN = \text{$R$-submodule of $K$ generated by } \{xy \mid x \in M, y \in N\},
  \]
  e.g. $((1 / 2)\Z)((1 / 3)\Z) = (1 / 6)\Z$. This is
  how we will make sense of the product $II^{-1}$.
\end{remark}

\begin{lemma}
  If $I = (a)$, then $I^{-1} = (a^{-1})$ and
  $I I^{-1} = (1) = R$.
\end{lemma}

\begin{proof}
  Check this as an exercise.
\end{proof}

\begin{prop}
  If $R$ is Dedekind, $I \ne 0$ is an ideal, and
  $\p \ne 0$ is a prime ideal, then $\p^{-1} I \ne I$.
\end{prop}

\begin{proof}
  First consider the special case $I = R$, and we
  want to show that $\p^{-1} \ne R$. We will find
  $x \in \p^{-1}$ which is not in $R$. To do this,
  we will take $x = a^{-1} b = b / a$ for some $a, b \in R$.
  We want $(b / a) \p \subseteq R$, so we should look
  for $b \p \subseteq (a)$ with $b \notin (a)$.
  Let $a \in \p$ be any nonzero element, and
  we will find a suitable $b$.

  Since $R$ is Noetherian, there exist prime ideals
  $\p_i \ne 0$ such that
  $\p_1 \dots \p_r \subseteq (a) \subseteq \p$.
  Without loss of generality, we can assume $r$ is minimal.
  This then implies that $\p_i \subseteq \p$ for
  some $i$, which implies $\p_i = \p$ since
  $R$ is $1$-dimensional. Assume without loss of
  generality that $i = 1$, so $\p_1 = \p$.

  If $r = 1$, then $\p = (a)$, so that
  $\p^{-1} = (a^{-1}) \ne R$ since $a$ is not a unit.
  So now assume $r \ge 2$. Then
  \[
    \p_2 \dots \p_r \not\subseteq (a)
  \]
  by the minimality of $r$, so there exists
  $b \in \p_2 \dots \p_r$ such that $b \notin (a)$.
  But $b \p = b \p_1 \subseteq (a)$, so the element
  $x = b / a \in \p^{-1}$
  but is not in $R$. This proves the statement
  when $I = R$.

  In the general case, using the hypothesis that $R$
  is Noetherian, we can write $I = (\alpha_1, \dots, \alpha_n)$.
  Assume otherwise that $\p^{-1} I = I$.
  Then for $x \in \p^{-1}$, we can write
  \[
    x \alpha_i = \sum_{j = 1}^n a_{ij} \alpha_j,
    \quad a_{ij} \in R.
  \]
  Let $A = (a_{ij})$ and define
  $T = x I_n - A$. Check as an exercise that
  $\det T = 0$. Since $\det T$ is a monic polynomial
  in $x$ with coefficients in $R$, we see that
  $x$ is integral over $R$. Since $R$ is integrally
  closed, we must have $x \in R$, so we get
  $\p^{-1} = R$. This contradicts the above special
  case.
\end{proof}

\begin{remark}
  The key idea of the proof is Cayley-Hamilton for
  modules: Let $R$ be a commutative ring and $M$ a
  finitely generated $R$-module. Then if
  $JM = M$, there exists $a$ with $1 - a \in J$
  such that $aM = M$. The proof above uses a similar
  strategy to the proof of this statement.
\end{remark}

  \chapter{Jan.~16 --- Ideal Class Group}

\section{Unique Factorization of Ideals, Continued}
The following is a corollary of Proposition
\ref{prop:dedekind}:

\begin{corollary}
  If $R$ is Dedekind and
  $\p \ne 0$ is a prime ideal, then
  $\p^{-1} \p = R = (1)$.
\end{corollary}

\begin{proof}
  First note that we have $\p \subseteq \p^{-1} \p \subseteq R$
  since $R \subseteq \p^{-1}$ by the definition
  of $\p^{-1}$. Furthermore, $\p^{-1}$ is an
  $R$-submodule of $K$, so $\p^{-1} \p$ is an
  $R$-submodule of $R$, i.e. an ideal of $R$. Also,
  by Proposition \ref{prop:dedekind},
  $\p^{-1} \p \ne \p$. Now $R$ being $1$-dimensional
  implies that $\p$ is maximal, so we must have
  $\p^{-1} \p = R$.
\end{proof}

\begin{prop}
  A Dedekind domain $R$ admits unique factorization
  of ideals into prime ideals.
\end{prop}

\begin{proof}
  For uniqueness, suppose that
  $I = \p_1 \cdots \p_r = \q_1 \cdots \q_s$.
  Then $\q_1 \dots \q_s \subseteq \p_1$, so we must
  have some $\q_i \subseteq \p_1$. Without loss
  of generality, assume $\q_1 \subseteq \p_1$, so
  that $\q_1 = \p_1$. Now multiplying by
  $\p_1^{-1}$, we get
  \[
    \p_2 \dots \p_r = \q_2 \dots \q_s.
  \]
  Proceeding by induction finishes the proof for
  uniqueness.

  Now we argue for existence. Let $\Sigma$ be the set
  of all proper ideals of $R$ which cannot be written
  as a product of prime ideals. If $\Sigma$ is
  nonempty, then the Noetherian property of $R$ implies
  that $\Sigma$ has a maximal element $J$. Then
  $J \subsetneq \p$ for some maximal ideal $\p$, which
  is equivalently a nonzero prime ideal since
  $R$ is one-dimensional. Since
  $R \subseteq \p^{-1}$, we have the chain of inclusions
  \[
    J \subsetneq J \p^{-1} \subsetneq \p \p^{-1} = R.
  \]
  Since $J$ was maximal in $\Sigma$, we must have
  $J \p^{-1} \notin \Sigma$, so we can write
  $J \p^{-1} = \p_1 \p_2 \dots \p_r$.
  But then we have $J = \p \p_1 \p_2 \dots \p_r$
  which is a contradiction with $J \in \Sigma$.
\end{proof}

\section{Ideal Class Group}

\begin{prop}
  In a Dedekind ring $R$, to contain is to divide,
  i.e. $I \subseteq J$ if and only if $J | I$.\footnote{We say that $J$ \emph{divides} $I$, written $J | I$, if $I = J J'$ for some ideal $J'$.}
\end{prop}

\begin{proof}
  $(\Rightarrow)$ If $I \subseteq J$, then
  $I J^{-1} \subseteq J J^{-1} = R$.\footnote{Note that we have technically only proved this property for prime ideals, but any ideals factors as prime ideals and we can argue via this factorization.} Then $J' = IJ^{-1}$
  is an ideal and satisfies $I = J J'$.

  $(\Leftarrow)$ This is the easier direction, verify
  this as an exercise.
\end{proof}

\begin{definition}
  Let $R$ be an integral domain.
  A \emph{fractional ideal} of $R$ is an $R$-submodule
  $J$ of $K$ such that $aJ$ is an ideal for some
  $a \in R$.
\end{definition}

\begin{exercise}
  If $I \subseteq R$ is an ideal, then show that
  $I^{-1}$ is a fractional ideal.
\end{exercise}

\begin{exercise}
  If $J$ is an $R$-submodule of $K$, then show that
  $J$ is a fractional ideal if and only if $J$ is
  finitely generated as an $R$-module.
\end{exercise}

\begin{exercise}
  Show that set of nonzero fractional ideals in a
  Dedekind domain $R$ forms a group under multiplication.
\end{exercise}

\begin{remark}
  In fact, one can actually show that
  \[
    I(R) = \{\text{nonzero fractional ideals}\}
    = \{\p_1^{k_1} \p_2^{k_2} \dots \p_r^{k_r} \mid k_i \in \Z\}.
  \]
  Due to unique factorization, this is actually the
  free abelian group on the set of nonzero prime ideals.
  We can also define
  \[
    P(R) = \{\text{principal fractional ideals}\}
    = \{aR \mid a \in K\}.
  \]
\end{remark}

\begin{definition}
  The \emph{ideal class group} of a Dedekind domain $R$
  is 
  the quotient $\Cl(R) = I(R) / P(R)$.\footnote{As a shorthand, we may write ``the class group of
  a number field $K$'' to mean $\Cl(\mathcal{O}_K)$.}
\end{definition}

\begin{exercise}
  Show that $\Cl(R)$ is also the equivalence classes
  of ideals under $\sim$, where $I \sim J$ if
  there exist $a, b \in R$ such that $aI = bJ$.
\end{exercise}


\begin{remark}
  Our goal now will be to show that if $R = \OO_K$
  and $[K : \Q] < \infty$, then $\Cl(R)$ is finite.
  The key tool will be the norm
  $N : \{\text{ideals of R}\} \to \N$, where
  $\N$ contains $0$.
\end{remark}

\begin{definition}
  We define the \emph{norm} of an ideal $I \subseteq R$
  to be $N(I) = |R/I|$.
\end{definition}

\begin{remark}
  To prove the finiteness of $\Cl(\OO_K)$ where
  $K$ is a number field,
  we will need to show the following properties of the
  norm $N$:
  \begin{itemize}
    \item $N((\alpha)) = N^{K}_\Q(\alpha)$.
    \item $N(IJ) = N(I) N(J)$.
  \end{itemize}
  Then, we will proceed to show the following:
  \begin{itemize}
    \item There exists $M \ge 0$ such that
      $\{\text{ideals } I \mid N(I) \le M\}$ is finite.
    \item Letting $\nu(I) = \min_{\alpha \in I} \{N(I) / N(\alpha)\}$,
      there exists $M$ such that
      $\nu(I) \le M$ for every $I$.
      Moreover, $\nu(I) = 1$ if and only if
      $I$ is principal.
      Note that
      $\nu(I) \in \Z$ by the multiplicative property
      of $N$.
  \end{itemize}
\end{remark}

\section{Discriminants}

\begin{definition}
  Let $L / K$ be a finite separable field extension,
  where $[L : K] = n$.
  Fix a Galois closure $M$ of $L / K$, so there are
  $n$ distinct embeddings
  $\sigma_1, \dots, \sigma_n : L \to M$ fixing $K$. The
  \emph{norm} of $\alpha \in L$ is
  \[
    N^L_{K}(\alpha) = \sigma_1(\alpha) \dots \sigma_n(\alpha) \in K.
  \]
  Now let $\alpha_1, \dots, \alpha_n \in L$.
  The \emph{discriminant} of $\alpha_1, \dots, \alpha_n$
  is
  \[
    \Delta(\alpha_1, \dots, \alpha_n) = \det
    \begin{bmatrix}
      \sigma_1(\alpha_1) & \cdots & \sigma_1(\alpha_n) \\
      \vdots & \ddots & \vdots \\
      \sigma_n(\alpha_1) & \cdots & \sigma_n(\alpha_n)
    \end{bmatrix}^2 = (\det T)^2.
  \]
\end{definition}

\begin{lemma}
  For $\alpha_1, \dots, \alpha_n \in L$,
  the discriminant
  $\Delta(\alpha_1, \dots, \alpha_n) \in K$
  and is nonzero if and
  only if $\alpha_1, \dots, \alpha_n$ form a basis
  for $L / K$.
\end{lemma}

\begin{proof}
  $(\Rightarrow)$ One can show the contrapositive
  that if $\alpha_1, \dots, \alpha_n$ are linearly
  dependent, then $\Delta = 0$.

  $(\Leftarrow)$ Let $\alpha_1, \dots, \alpha_n$
  be a basis for $L / K$.
  By the primitive element theorem, there exists
  $\theta \in L$ such that $L = K(\theta)$, so
  that $1, \theta, \theta^2, \dots, \theta^{n - 1}$
  form a basis for $L / K$. Then we have
  \[
    \begin{bmatrix}
      \alpha_1 \\ \vdots \\ \alpha_n
    \end{bmatrix}
    = M
    \begin{bmatrix}
      1 \\ \vdots \\ \theta^{n - 1}
    \end{bmatrix}
  \]
  for some matrix $M \in M_{n \times n}(K)$ with
  $\det M \ne 0$.
  This implies that
  \[
    \begin{bmatrix}
      \sigma_i(\alpha_1)  \\ \vdots \\ \sigma_i(\alpha_n)
    \end{bmatrix}
    = M
    \begin{bmatrix}
      1 \\ \vdots \\ \sigma_i(\theta^{n - 1})
    \end{bmatrix}.
  \]
  Thus if we define
  \[
    T' =
    \begin{bmatrix}
      \sigma_1(1) & \cdots & \sigma_1(\theta^{n - 1}) \\
      \vdots & \ddots & \vdots \\
      \sigma_n(1) & \cdots & \sigma_n(\theta^{n - 1})
    \end{bmatrix}
    =
    \begin{bmatrix}
      \sigma_1(1) & \cdots & \sigma_1(\theta)^{n - 1} \\
      \vdots & \ddots & \vdots \\
      \sigma_n(1) & \cdots & \sigma_n(\theta)^{n - 1}
    \end{bmatrix}
  \]
  and $\Delta' = (\det T')^2$, then
  $T = T' M^t$ implies $\Delta = \Delta' (\det M)^2$.
  Now $T'$ is a Vandermonde matrix, so
  \[
    (\det T')^2 = \prod_{i \ne j} (\sigma_i(\theta) - \sigma_j(\theta)) \ne 0.
  \]
  We can also see $\Delta' = (\det T')^2 \in K^\times$ (via Galois theory)
  and $(\det M)^2 \in K^\times$, so
  $\Delta \in K^\times$ as well.
\end{proof}

\begin{theorem}
  Let $K$ be a number field and $\alpha \in \OO_K$.
  Then $N((\alpha)) = N(\alpha)$.
\end{theorem}

\begin{proof}
  Let $\omega_1, \dots, \omega_n$ be a $\Z$-basis for
  $\OO_K$, and let
  $\alpha_i = \alpha \omega_i$. Then
  $\alpha_1, \dots, \alpha_n$ is a $\Z$-basis for
  $\mathfrak{a} = (\alpha)$. Thus we may write
  \[
    \begin{bmatrix}
      \alpha_1 \\ \vdots \\ \alpha_n
    \end{bmatrix}
    = A
    \begin{bmatrix}
      \omega_1 \\ \vdots \\ \omega_n
    \end{bmatrix}
  \]
  for some matrix $A \in M_{n \times n}(\Z)$.
  Now the theory of finitely generated modules over
  a PID implies that $N(\mathfrak{a}) = |{\det A}|$. (This
  is because we have two free $\Z$-modules of
  rank $n$: $\mathcal{O}_K$ and $\mathfrak{a} \subseteq \OO_K$. So if $A \sim A'$ where $A' = \diag(d_1, \dots, d_n)$ is in Smith normal form, then
  $|\OO_K / \mathfrak{a}| = |(\Z / d_1) \times \dots \times (\Z / d_n)|$, so we see that
  $N(\mathfrak{a}) = |\OO_K / \mathfrak{a}| = |d_1 \dots d_n| = |{\det A'}| = |{\det A}|$.)
  Thus we have
  \[
    \Delta(\alpha_1, \dots, \alpha_n)
    = (\det A)^2 \Delta(\omega_1, \dots, \omega_n).
  \]
  But we can also see that
  \begin{align*}
    \Delta(\alpha_1, \dots, \alpha_n)
    = \Delta(\alpha \omega_1, \dots, \alpha \omega_n)
    &= \det
    \begin{bmatrix}
      \sigma_1(\alpha \omega_1) & \cdots & \sigma_1(\alpha \omega_n) \\
      \vdots & \ddots & \vdots \\
      \sigma_n(\alpha \omega_1) & \cdots & \sigma_n(\alpha \omega_n)
    \end{bmatrix}^2 \\
    &= (\sigma_1(\alpha) \dots \sigma_n(\alpha))^2 \Delta(\omega_1, \dots, \omega_n)
    = N(\alpha)^2 \Delta(\omega_1, \dots, \omega_n).
  \end{align*}
  This shows that $N(\mathfrak{a})^2 = (\det A)^2 = N(\alpha)^2$,
  so that $N(\mathfrak{a}) = N(\alpha)$
  since these values are positive.
\end{proof}

  \chapter{Jan.~21 --- Finiteness of the Class Group}

\section{Multiplicativity of the Norm}

\begin{theorem}
  If $I, J \subseteq \OO_K$ are ideals, then
  $N(IJ) = N(I) N(J)$.
\end{theorem}

\begin{proof}
  First observe that if $\mathfrak{a}, \mathfrak{b} \subseteq \OO_K$ are
  relatively prime ideals (i.e. $\mathfrak{a} + \mathfrak{b} = (1)$), then
  \[
    \OO_K / \mathfrak{a} \mathfrak{b} \cong \OO_K / \mathfrak{a} \times \OO_K / \mathfrak{b}
  \]
  by the Chinese remainder theorem, and
  the result immediately follows. One can also
  show that if $\p \ne \q$ are nonzero prime ideals in
  $\OO_K$, then
  $\p^s$ and $\q^t$ are relatively prime for every
  $s, t$. Thus by unique factorization of $I, J$ into
  prime ideals,
  it is enough to prove $N(\p^m) = (N(\p))^m$
  for a prime ideal $\p$.

  To do this, observe that we have the chain of inclusions
  \[
    \OO_K \supsetneq \p \supsetneq \p^2 \dots \supsetneq \p^m,
  \]
  and it suffices to show that $[\p^k : \p^{k + 1}] = N(\p)$
  for each $0 \le k < m$. We will show the
  stronger result that $\OO_K / \p \cong \p^k / \p^{k + 1}$
  as abelian groups. To do this, pick
  $\gamma \in \p^k \setminus \p^{k + 1}$ (note
  that $\p^k \ne \p^{k + 1}$ by unique factorization)
  and define $\phi : \OO_K \to \p^k / \p^{k + 1}$
  by $x \mapsto \gamma x$. Since $\gamma x \in \p^{k + 1}$
  whenever $x \in \p$, this induces a map
  $\phi : \OO_K / \p \to \p^k / \p^{k + 1}$, which
  we prove is an isomorphism in Proposition
  \ref{prop:isomorphism}.
\end{proof}

\begin{prop}\label{prop:isomorphism}
  The map $\phi : \OO_K / \p \to \p^k / \p^{k + 1}$
  by $x \mapsto \gamma x$ is an isomorphism
  of abelian groups.
\end{prop}

\begin{proof}
  We will show the following claims:
  \begin{enumerate}
    \item $(\gamma) + \p^{k + 1} = \p^k$. This
      implies that $\phi$ is surjective.
    \item $(\gamma) \cap \p^{k + 1} = \gamma \p$.
      This means that if $\gamma x \in \gamma \p$,
      then $x \in \p$, i.e. $\phi$ is injective.
  \end{enumerate}
  (1) Let $I = (\gamma) + \p^{k + 1}$. Since we already
  know that $\p^k | (\gamma)$, we have $\p^k | I$.
  But $I \supsetneq \p^{k + 1}$, so $I | \p^{k + 1}$,
  and the containment being strict implies that
  we must have $I = \p^k$.

  (2) Let $I' = (\gamma) \cap \p^{k + 1}$. Since
  $\gamma \in \p^k$, we have $\gamma \p \subseteq I'$.
  This is one containment.
  Conversely, let $x \in I'$. Write $x = \gamma y$,
  where $y \in \OO_K$ and $\gamma y \in \p^{k + 1}$.
  Now note that\footnote{Here $\ord_\p(\alpha) = \ord_\p(\mathfrak{a})$ is the largest integer $m$ such that $\p^m | \mathfrak{a}$, where $\mathfrak{a} = (\alpha)$.}
  \[
    \ord_\p(\gamma) + \ord_\p(y) = \ord_\p(\gamma y) \ge k + 1.
  \]
  But $\ord_\p(\gamma) = k$ (since $\gamma \in \p^k \setminus \p^{k + 1}$),
  so $\ord_\p(y) \ge 1$. This implies
  that $\p | (y)$, so $y \in \p$. Since
  $x = \gamma y$, this gives $x \in \gamma \p$.
  This yields the other containment, and so
  $I' = \gamma \p$.
\end{proof}

\begin{corollary}
  Let $[K : \Q] = n$ and $p \in \Z$ be a prime number.
  Write
  \[
    (p) = p\OO_K = \prod_{i = 1}^r \p_i^{e_i},
  \]
  where the $\p_i$ are distinct prime ideals. Then
  $\sum_{i = 1}^r e_i f_i = n$,
  where $N(\p_i) = p^{f_i}$.
\end{corollary}

\begin{proof}
  Since the norm is multiplicative, we have
  \[
    p^n = N(p \OO_K) = N(p) = \prod_{i = 1}^n \sigma_i(p)
  \]
  since each $\sigma_i$ fixes $p$. Then since
  $p \OO_K = \prod_{i = 1}^r \p_i^{e_i}$, we have\footnote{Note that $\mathcal{O}_K / \p_i$ is a finite field (since $\p_i \ne 0$ is prime, hence maximal in $\OO_K$) and a vector space over $\Z / p$ since $(p) \subseteq \p_i$. So $\OO_K / \p_i$ has prime characteristic, hence $N(\p_i) = |\OO_K / \p_i| = p^{f_i}$ for some $f_i$.}
  \[
    p^n = N(p \OO_K) = N(\prod \p_i^{e_i})
    = \prod (N(\p_i))^{e_i}
    = \prod (p^{f_i})^{e_i}
    = \prod p^{e_i f_i}.
  \]
  Thus $n = \sum_{i = 1}^r e_i f_i$, which is the
  desired result.
\end{proof}

\begin{remark}
  In the above case, we will say that the
  $\p_i$ ``lie over'' $p$.
\end{remark}

\section{Finiteness of the Class Group}
\begin{theorem}
  Let $K$ be a number field. Then there exists $M > 0$
  such that every nonzero ideal $I$ of $\OO_K$ contains
  a nonzero element $\alpha$ with
  $|N(\alpha)| \le M \cdot N(I)$. Equivalently,
  $\alpha$ satisfies
  \[
    \inf_{\alpha \in I} \frac{N(\alpha)}{N(I)} \le M,
  \]
  and the above infimum is $1$ if and only if $I$ is
  principal.
\end{theorem}

\begin{proof}
  Choose an integral basis $\alpha_1, \dots, \alpha_n$
  of $\OO_K$, and let $I$ be a nonzero ideal.
  Choose $m$ such that $m^n \le N(I) < (m + 1)^n$.
  Then define the set
  \[
    \Sigma =
    \left\{
      \sum_{i = 1}^n m_j \alpha_j : 0 \le m_j \le m, m_j \in \Z
    \right\}.
  \]
  Note that $\# \Sigma = (m + 1)^n > N(I) = |\OO_K / I|$,
  so by the pigeonhole principle there exist $x \ne y$
  in $\OO_K$ such that $\alpha = x - y \in I$, and
  we can write $\alpha = \sum m_j \alpha_j$ where
  $|m_j| \le m$ for every $j$. Then
  \[
    |N(\alpha)| = \prod_{i = 1}^n |\sigma_i(\alpha)|
    \le \prod_{i = 1}^n \sum_{j = 1}^n |m_j| |\sigma_i(\alpha_j)|
    \le m^n \prod_{i = 1}^n \sum_{j = 1}^n |\sigma_i(\alpha_j)|
    \le N(I) \cdot M,
  \]
  where $M = \prod_{i = 1}^n \sum_{j = 1}^n |\sigma_i(\alpha_j)|$
  is independent of $I$ (but depends on the
  choice of integral basis).
\end{proof}

\begin{corollary}
  Every ideal class in $\OO_K$ contains a nonzero
  ideal of norm at most $M$.
\end{corollary}

\begin{proof}
  Let $C \in \Cl(\OO_K)$, and let $I$ be an ideal
  with $[I] = C^{-1}$. By the above theorem,
  choose $\alpha \in I$ such that $|N(\alpha)| \le M \cdot N(I)$.
  Now $(\alpha) = IJ$ for some $J$, so
  $[J] = [I]^{-1} = C$, and
  \[
    N(J) = \frac{|N(\alpha)|}{N(I)} \le M,
  \]
  which proves the desired result.
\end{proof}

\begin{lemma}
  The set of ideals with norm bounded by $M$
  is finite, i.e. $|\{I : N(I) \le M\}| < \infty$.
\end{lemma}

\begin{proof}
  One way to proceed is to write $I = \prod \p_i^{e_i}$,
  and then use $N(\p_i) = p^{f_i}$.

  Another way to prove this is to note that
  if $|N(I)| = m$, then $mx = 0$ in
  $\OO_K / I$ for every $x \in \OO_K$. So
  $I \supseteq m \OO_K$. But
  $\OO_K / m \OO_K$ is finite, so there are only
  finitely many ideals containing $m \OO_K$.
\end{proof}

\begin{corollary}
  The ideal class group of a number field $\Cl(\OO_K)$ is finite.
\end{corollary}

\begin{proof}
  Each ideal class can be represented by an ideal of
  norm bounded by $M$, and there are only
  finitely many such ideals. Thus there can only be
  finitely many ideal classes.
\end{proof}

\section{Computation of Integral Bases}

\begin{remark}
  Recall that if $[K : \Q] = n$ and
  $\alpha_1, \dots, \alpha_n \in K$ are a basis for
  $K / \Q$, then
  $\Delta(\alpha_1, \dots, \alpha_n) \in \Q^\times$.
  Moreover, if $\alpha_1, \dots, \alpha_n \in \OO_K$,
  then $\Delta(\alpha_1, \dots, \alpha_n) \in \Z$.
  Also, if $\alpha_1, \dots, \alpha_n$ are a $\Z$-basis
  for $\OO_K$, then
  \[
    \Delta(\alpha_1, \dots, \alpha_n) = \Delta_K = \Delta(\OO_K)
  \]
  is independent of the choice of $\Z$-basis. So
  $\Delta_K$ is an invariant of $K$ (or of $\OO_K$),
  called its \emph{discriminant}.
\end{remark}

\begin{prop}
  Let $\alpha_1, \dots, \alpha_n \in \OO_K$ be a basis
  for $K / \Q$, and let
  $d = \Delta(\alpha_1, \dots, \alpha_n)$. Then
  \[
    \Z[\alpha_1, \dots, \alpha_n]
    \subseteq \OO_K \subseteq
    \Z\left[\frac{\alpha_1}{d}, \dots, \frac{\alpha_n}{d}\right].
  \]
\end{prop}

\begin{proof}
  Suppose $\alpha \in \OO_K$, so we can write
  (since $\alpha_1, \dots, \alpha_n$ is a basis for
  $K / \Q$)
  \[
    \alpha = c_1 \alpha_1 + \dots + c_n \alpha_n,
    \quad c_i \in \Q.
  \]
  We want to show that $d c_j \in \Z$. Note that
  $\sigma_i(\alpha) = c_1 \sigma_i(\alpha_1) + \dots + c_n \sigma_i(\alpha_n)$, so
  \[
    \begin{bmatrix}
      \sigma_1(\alpha) \\ \vdots \\ \sigma_1(\alpha)
    \end{bmatrix}
    = T
    \begin{bmatrix}
      c_1 \\ \vdots \\ c_n
    \end{bmatrix},
  \]
  where $T = (\sigma_i(\alpha_j))$. Multiplying
  both sides by $\adj T$, we get
  (note that $T \adj T = \delta I$, where $\delta = \det T$)
  \[
    \begin{bmatrix}
      \beta_1 \\ \vdots \\ \beta_n
    \end{bmatrix}
    = \delta
    \begin{bmatrix}
      c_1 \\ \vdots \\ c_n
    \end{bmatrix},
  \]
  where the $\beta_i \in \OO_K$. Let
  $m_j = \delta \beta_j$, and noting that $\delta^2 = d$
  by definition, we have
  \[
    \begin{bmatrix}
      m_1 \\ \vdots \\ m_n
    \end{bmatrix} = d
    \begin{bmatrix}
      c_1 \\ \vdots \\ c_n
    \end{bmatrix}.
  \]
  This tells us that $d c_i \in \OO_K$ for every $i$.
  But the $c_i$ were also rational, so in fact
  $d c_i \in \OO_K \cap \Q = \Z$.
\end{proof}

\begin{lemma}
  Let $\alpha_1, \dots, \alpha_n \in \OO_K$ be a basis
  for $K / \Q$. Let
  \[
    M = \text{$\Z$-module spanned by $\alpha_1, \dots, \alpha_n$}.
  \]
  Then $\Delta_{K / \Q}(\alpha_1, \dots, \alpha_n) = \Delta_K \cdot |\OO_K / M|^2$.
\end{lemma}

\begin{proof}
  Check this as an exercise; it is a calculation
  involving determinants.
\end{proof}

\begin{corollary}
  Let $\alpha_1, \dots, \alpha_n \in \OO_K$ be a
  basis for $K / \Q$. If
  $\Delta(\alpha_1, \dots, \alpha_n)$ is
  square-free, then the $\alpha_1, \dots, \alpha_n$ form
  an integral basis.
\end{corollary}

\begin{proof}
  If the $\alpha_i$ do not form a basis, then
  the $\Delta$ will contain a
  $|\OO_K / M|^2$ factor by the lemma.
\end{proof}

\begin{example}
  Let $K = \Q(\sqrt{d})$, where $d$ is square-free.
  Then we can see that
  \[
    \Delta_{K / \Q}(1, \sqrt{d}) =
    \det
    \begin{bmatrix}
      1 & \sqrt{d} \\ 1 & -\sqrt{d}
    \end{bmatrix}^2
    = 4d.
  \]
  Thus $4d = \Delta_{K / \Q} \cdot |\OO_K / M|^2$, where
  $M = \Z \cdot 1 + \Z \cdot \sqrt{d}$. Since $d$ is
  square-free, $[\OO_K : M] = 1$ or $2$. Now if
  we have $|\OO_K / (\Z + \Z \sqrt{d})| = 2$, then at
  least one of
  \[
    \frac{1}{2}, \quad \frac{\sqrt{d}}{2}, \quad
    \frac{1 + \sqrt{d}}{2}, \quad \frac{1 - \sqrt{d}}{2}
  \]
  must be an algebraic integer. The first two are
  obviously not algebraic integers, and the third is
  an algebraic integer if and only if the fourth one
  is (since they are conjugates). So the index is $2$
  if and only if $(1 + \sqrt{d}) / 2 \in \OO_K$.
  By looking at the coefficients of the minimal
  polynomial
  \[
    x^2 - x + \frac{1 - d}{4},
  \]
  this happens if and only if $(1 - d) / 4 \in \Z$,
  which is equivalent to $d \equiv 1 \Pmod{4}$.
\end{example}

  \chapter{Jan.~23 --- Computing Rings of Integers}

\section{More on Computing of Rings of Integers}
\begin{lemma}
  Let $\alpha_1, \dots, \alpha_n \in \OO_K$ be a
  basis for $K / \Q$, and suppose that
  $\OO_K / (\Z \alpha_1 \oplus \dots \oplus \Z \alpha_n)$
  has exponent $m$, i.e. $m \alpha \in \Z \alpha_1 \oplus \dots \oplus \Z \alpha_n$
  for every $\alpha \in \OO_K$. Then
  \[
    \OO_K \subseteq \Z \frac{\alpha_1}{m} \oplus \dots \oplus \Z \frac{\alpha_n}{m}.
  \]
  Moreover, if $\OO_K \ne \Z \alpha_1 \oplus \dots \oplus \Z \alpha_n$,
  then there exist $0 \le m_i \le m - 1$, not all
  zero, such that
  \[
    m_1 \frac{\alpha_1}{m} + \dots + m_n \frac{\alpha_n}{m} \in \OO_K.
  \]
\end{lemma}

\begin{proof}
  The idea is the following: Let
  $M = \Z \alpha_1 \oplus \dots \oplus \Z \alpha_n$,
  so $m \OO_K \subseteq M$. Then
  $\OO_K \subseteq (1 / m) M$, which proves the
  first part. Now if $\OO_K \ne M$, then there
  exists a nonzero element of $(1 / m) M$ which is
  not in $M$. This gives a nonzero element in the
  quotient:
  \[
    \frac{(1 / m) M}{M}
    = \left[m_1 \frac{\alpha_1}{m} + \dots + m_n \frac{\alpha_n}{m}\right]
  \]
  for some $m_i$, as in the statement.
\end{proof}

\begin{example}
  We can apply this lemma to our example from last
  lecture: Let $K = \Q(\sqrt{d})$, where
  $d$ is square-free. Then $\Delta(1, \sqrt{d}) = 4d$,
  so $\Delta_K | 4d$. Then we have
  \[
    \frac{4d}{\Delta_K} = [\OO_K : (\Z \oplus \Z \sqrt{d})]^2
    = 1^2 \text{ or } 2^2.
  \]
  Thus by the lemma, either $\OO_K = \Z[\sqrt{d}]$
  or one of
  \[
    \frac{1}{2}, \quad \frac{\sqrt{d}}{2}, \quad \frac{1 + \sqrt{d}}{2}
  \]
  is in $\OO_K$. The first two are obvious not in
  $\OO_K$, and $(1 + \sqrt{d}) / 2 \in \OO_K$ if
  and only if $d \equiv 1 \Pmod{4}$. Thus if
  $d \equiv 1 \Pmod{4}$, then
  $1, (1 + \sqrt{d}) / 2$ is an integral basis for
  $\OO_K$.
\end{example}

\begin{prop}
  Let $[K : \Q] = n$ and $\alpha \in \OO_K$
  with minimal polynomial of degree $n$. Suppose
  further that the minimal polynomial of $\alpha$ is
  Eisenstein at $p$. Then
  $p \nmid [\OO_K : \Z[\alpha]]$.\footnote{Recall that $\Z[\alpha] = \Z \oplus \Z \alpha \oplus \dots \oplus \Z \alpha^{n - 1}$.}
\end{prop}

\begin{proof}
  Let the minimal polynomial of $\alpha$ be
  \[
    f(x) = x^n + a_{n - 1} x^{n - 1} + \dots + a_1 x + a_0,
  \]
  with $p | a_i$ for every $i$ and $p^2 \nmid a_0$
  (since $f$ is Eisenstein at $p$). Suppose
  otherwise that $p | [\OO_K : \Z[\alpha]]$. Then
  by Cauchy's theorem, there exists  $\xi \in \OO_K$
  such that $[\xi] \in \OO_K / \Z[\alpha]$ has
  order $p$. Then
  \[
    p \xi = b_0 + b_1 \alpha + \dots + b_{n - 1} \alpha^{n - 1}
  \]
  where $b_i \in \Z$, not all divisible by $p$.
  Let $j$ be the smallest index such that $p \nmid b_j$.
  Then
  \[
    \OO_K \ni \eta = \xi - \left(\frac{b_0}{p} + \frac{b_1}{p} \alpha + \dots + \frac{b_{j - 1}}{p} \alpha^{j - 1}\right)
    = \frac{b_j}{p} \alpha^j + \frac{b_{j + 1}}{p} \alpha^{j + 1} + \dots + \frac{b_n}{p} \alpha^n.
  \]
  So we have
  \[
    \OO_K \ni \eta \alpha^{n - j - 1} = \frac{b^j}{p} \alpha^{n - 1} + \frac{\alpha^n}{p} (b_{j + 1} + b_{j + 2} \alpha + \dots).
  \]
  Also notice that
  \[
    \frac{\alpha^n}{p} = -\frac{a_0 + a_1 \alpha + \dots + a_{n - 1} \alpha^{n - 1}}{p} \in \OO_K
  \]
  since $f$ was Eisenstein at $p$. Since
  $(b_{j + 1} + b_{j + 2} \alpha + \dots) \in \OO_K$,
  we see that $b_j \alpha^{n - 1} / p \in \OO_K$
  and $p \nmid b_j$. So
  \[
    \Z \ni N^K_\Q\left(\frac{b_j}{p} \alpha^{n - 1}\right)
    = \frac{b_j^n}{p^n} N(\alpha^{n - 1})
    = \frac{b_j^n}{p^n} a_0^{n - 1}.
  \]
  Now $p \nmid b_j$ and $p^2 \nmid a_0$ so we
  have at most $n - 1$ factors of $p$ in the numerator,
  a contradiction.
\end{proof}

\begin{prop}
  For $K = \Q(\sqrt[3]{2})$, we have
  $\OO_K = \Z[\sqrt[3]{2}]$.
\end{prop}

\begin{proof}
  Let $\alpha = \sqrt[3]{2}$ and
  $M = \Z[\alpha] = \Z \oplus \Z \alpha \oplus \Z \alpha^2$.
  Let $m = |\OO_K / M|$. Then
  \[
    m^2 \Delta(\OO_K)
    = \Delta(1, \alpha, \alpha^2)
    = \Delta(f_\alpha),
  \]
  where $f_\alpha$ is the minimal polynomial of
  $\alpha$ (check that $\Delta(1, \alpha, \alpha^2) = \Delta(f_\alpha)$).
  Recall that up to signs,
  \[
    \Delta(f) = \prod_{\text{roots } \alpha_i} f(\alpha_i).
  \]
  For a cubic polynomial $f(x) = x^3 + ax + b$, the
  discriminant is $\Delta f = -4 a^3 - 27b^2$ (for a quadratic
  $f(x) = x^2 + bx + c$, it is $\Delta f = b^2 - 4c$).
  Thus for $f_\alpha(x) = x^3 - 2$, we have
  \[
    m^2 \Delta(\OO_K) = \Delta(f_\alpha) = -108
    = -6^2 \cdot 3.
  \]
  Thus the index $m$ divides $6$, and since
  $f_\alpha$ is Eisenstein at $2$, we have
  $2 \nmid m$. Now notice that
  \[
    \Z[\alpha] = \Z[\beta], \quad \text{where } \beta = \alpha - 2.
  \]
  The minimal polynomial of $\beta$ is
  $g(x) = (x + 2)^3 - 2 = x^3 + 6x^2 + 12x + 6$. Then
  $g$ is Eisenstein at $3$, so $3 \nmid m$. Thus
  we must have $m = 1$, which proves that
  $\OO_K = \Z[\alpha]$.
\end{proof}

\begin{remark}
  Later on, we will show that if
  $K = \Q(\zeta_n)$ for some $n \ge 1$ (where
  $\zeta_n$ is a primitive $n$th root of unity), then
  $\OO_K = \Z[\zeta_n]$. The proof will largely
  involve similar types of ideas.
\end{remark}

\begin{remark}
  In general if $K = \Q(\theta)$ and we only work
  with $\Z[\theta]$ instead of $\OO_K$, we may run
  into trouble since $\Z[\theta]$ may not be
  integrally closed (hence we may not have unique
  factorization of ideals).
\end{remark}

\section{Computing Factorizations of Ideals}
\begin{prop}
  Let $K = \Q(\sqrt{d})$, where $d$ is square-free.
  Let $p$ be an odd prime with $p \nmid d$. Then:
  \begin{enumerate}[(a)]
    \item if $(\frac{d}{p}) = 1$,
      then $p \OO_K = \p_1 \p_2$ where
      $\p_1 = (p, a + \sqrt{d}) \ne \p_2 = (p, a - \sqrt{d})$, and $a^2 \equiv d \Pmod{p}$;
    \item if $(\frac{d}{p}) = -1$, then $p \OO_K = \p$
      is prime in $\OO_K$.
  \end{enumerate}
  In the above, $(\frac{d}{p})$ is the
  \emph{Legendre symbol}.
\end{prop}

\begin{proof}
  $(a)$ Note that
  \[
    \p_1 \p_2 = (p^2, p(a + \sqrt{d}), p(a - \sqrt{d}), a^2 - d) \subseteq (p)
  \]
  since each of the above terms is divisible by $p$.
  But $p^2 \in \p_1 \p_2$ and
  $p(a + \sqrt{a}) + p(a - \sqrt{d}) = 2ap$ so
  \[
    (p) = (\gcd(p^2, 2ap)) \subseteq \p_1 \p_2.
  \]
  This gives the equality $\p_1 \p_2 = (p)$. Now we
  show that $\p_i$ is prime. Note that
  \[
    N(\p_1) N(\p_2) = N(p) = p^2.
  \]
  It is enough to show that $a + \sqrt{d} \notin (p)$
  (this implies $\p_1 \ne (p)$, so
  $N(\p_1) = |\OO_K / \p_1| < |\OO_K / (p)| = p^2$ and
  we must have $N(\p_1) = p$).
  Now if $p | (a + \sqrt{d})$, then
  $p | (a - \sqrt{d})$ as well, so $p | 2a$. This is a
  contradiction. Thus $N(\p_i) = p$, which implies
  that $\p_i$ is a prime ideal (otherwise the norm
  would also factor). It only remains to show that
  $\p_1 \ne \p_2$, which is left as an exercise.

  $(b)$ It is enough to show that there is no prime
  ideal $\p \subseteq \OO_K$ such that $N(\p) = p$.
  Equivalently, it suffices to show that if $\p$ is a
  prime ideal, then
  $\OO_K / \p \ncong \Z / p\Z$.  Note that $x^2 - d$
  has a root in $\OO_K$ and thus in $\OO_K / \p$, so $\OO_K / \p \cong \Z / p\Z$ would imply that
  $d$ is a square modulo $p$, contradicting $(\frac{d}{p}) = -1$.
\end{proof}

\begin{exercise}
  Show that if $p | d$, then $p \OO_K = \p^2$ for some
  prime ideal $\p$.
\end{exercise}

\begin{theorem}[Kummer]
  Let $K = \Q(\theta)$ with $\theta \in \OO_K$.
  Suppose $p$ is a prime such that $p \nmid [\OO_K : \Z[\theta]]$.
  Let $g$ be the minimal polynomial of $\theta$.
  Factor $g \Mod p$ as
  \[
    g(x) \equiv g_1(x)^{e_1} \dots g_r(x)^{e_r} \Pmod{p},
  \]
  where $g_i(x) \in \Z[x]$, $\overline{g_i(x)}$ is
  irreducible over $\mathbb{F}_p$, and the
  $\overline{g_i}$ are pairwise distinct.\footnote{Here $\overline{g(x)}$ denotes the reduction of $g(x)$ modulo $p$.} Then
  \[
    p \OO_K = \p_1^{e_1} \dots \p_r^{e_r},
  \]
  where $\p_i = (p, g_i(\theta))$ is a prime ideal,
  $N(\p_i) = p^{f_i}$ where
  $f_i = \deg g_i$, and the $\p_i$ are distinct.
\end{theorem}

\begin{remark}
  Note that this generalizes the quadratic case:
  $x^2 - d \Mod p$ factors if and only if $d$ is a
  square modulo $p$, and
  the ideals are $(p, g_i(\theta))$ for
  $g_1 = x - a$ and $g_2 = x + a$.
\end{remark}

  \chapter{Jan.~28 --- Kummer's Theorem}

\section{Kummer's Theorem}

\begin{lemma}
  Let $\theta \in \OO_K$ and assume
  $p \nmid [\OO_K : \Z[\theta]]$. Then
  \[
    \OO_K / p \OO_K \cong \Z[\theta] / p \Z[\theta].
  \]
\end{lemma}

\begin{proof}
  Consider the map $\psi: \Z[\theta] \hookrightarrow \OO_K \twoheadrightarrow \OO_K / p \OO_K$.
  Note that we have $p\Z[\theta] \subseteq \ker \psi$,
  so $\psi$ induces a map $\overline{\psi}: \Z[\theta] / p \Z[\theta] \to \OO_K / p \OO_K$
  on the quotient. We will show that
  $\overline{\psi}$ is an isomorphism, by checking:
  \begin{enumerate}
    \item $\ker \psi = p \Z[\theta]$.

  Let $\alpha \in \ker \psi$.
  Then $\alpha \in \Z[\theta] \cap p \OO_K$, so
  $\alpha = p \beta$ for some $\beta \in \OO_K$.
  Then $\overline{\beta} \in \OO_K / \Z[\theta]$
  has order dividing $p$ since $p \overline{\beta} = \overline{\alpha} = 0$.
  Therefore $\overline{\beta} = 0$, so
  $\beta \in \Z[\theta]$. This gives $\alpha \in p\Z[\theta]$, so
  $\ker \psi = p \Z[\theta]$.

  \item $\psi$ is surjective.

  Note that if $(|G|, p) = 1$ where $G$ is a
  finite abelian group, then $[p] : G \to G$ is injective
  and hence bijective. So let $\gamma \in \OO_K$,
  so that $\overline{\gamma} \in \OO_K / \Z[\theta]$ is
  a multiple of $p$, i.e. $\overline{\gamma} = p \overline{\gamma'}$ for some $\gamma' \in \OO_K$.
  Then $\gamma - p \gamma' \in \Z[\theta]$,
  so $\psi(\gamma - p\gamma') = \gamma$.
  Since $\gamma - p \gamma' \in \Z[\theta]$, this
  shows that $\psi$ is surjective.
  \end{enumerate}

  Thus $\overline{\psi}$ is bijective, so it
  is an isomorphism $\Z[\theta] / p \Z[\theta] \to \OO_K / p \OO_K$.
\end{proof}

\begin{theorem}[Kummer]
  Let $K = \Q(\theta)$ and $p$ be a prime.
  Assume that $p \nmid [\OO_K : \Z[\theta]]$, and let
  $g(x)$ be the minimal polynomial of $\theta$.
  Write (let $\overline{g}$ denote the reduction of
  $g$ modulo $p$)
  \[
    \overline{g} = \prod_{i = 1}^r (\overline{g_i})^{e_i},
  \]
  where $g_i(x) \in \Z[x]$ and $\overline{g_i} \in \F_p[x]$
  is irreducible and monic, with
  $g_1, \dots, g_r$ distinct. Then
  \[
    p \OO_K = \p_1^{e_1} \dots \p_r^{e_r},
  \]
  where $N(\p_i) = p^{f_i}$ for $f_i = \deg g_i$, and
  $\p_i = (p, g_i(\theta))$ are distinct prime ideals.
\end{theorem}

\begin{proof}
  Let $\p_i = (p, g_i(\theta))$ as in the statement.
  Then
  \[
    \OO_K / \p_i = \OO_K / (p, g_i(\theta))
    \cong \Z[\theta] / (p, g_i(\theta))
    \cong \Z[x] / (p, g_i(x))
  \]
  The first isomorphism follows from the lemma, which
  holds only when $p \nmid [\OO_K : \Z[\theta]]$.
  Note that
  \[
    \F_p[\theta] / (\overline{g_i}(\theta)) \cong \Z[\theta] / (p, g_i(\theta))
    \cong \Z[x] / (p, g_i(x))
    \cong \F_p[x] / (\overline{g_i}(x)).
  \]
  Since $\overline{g_i}$ is irreducible of
  degree $f_i$, the quotient is a field of size
  $p^{f_i}$. This proves that $N(\p_i) = p^{f_i}$
  and also that $\p_i$ is a maximal ideal (so in
  particular, a prime ideal).
  Now if $n = [K : \Q]$, then
  \[
    \sum_{i = 1}^r e_i f_i = \deg \overline{g} = n.
  \]
  Check as an exercise that the $\p_i$ are distinct
  (use the fact that $\overline{g_i}$ and
  $\overline{g_j}$ are relatively prime, so that
  $(\overline{g_i}, \overline{g_j}) = 1$ in $\F_p[x]$).
  Now we will show that $p \OO_K \cong \p_1^{e_1} \dots \p_r^{e_r}$.
  First observe that
  \[
    \p_1^{e_1} \dots \p_r^{e_r} = (p, g_1(\theta))^{e_1} \dots (p, g_r(\theta))^{e_r}
    \subseteq (p, g_1(\theta)^{e_1} \dots g_r(\theta)^{e_r}) = (p).
  \]
  (Check the above inclusion as an exercise. Note that
  for $\overline{g}(x) = \overline{g_1}(x) \overline{g_2}(x)$, we can find $h$ such that $g_1 + g_2 = 1 + ph$, so
  that $(p, g_1(\theta))(p, g_2(\theta)) = (p^2, p g_1(\theta), p g_2(\theta), g_1(\theta) g_2(\theta)) = (p)$ as
  $p(1 + ph) = p + p^2 h$ and $g(\theta) = 0$).
  Thus $p \OO_K | \p_1^{e_1} \dots \p_r^{e_r}$, which
  implies that
  \[
    p \OO_K = \p_1^{e_1'} \dots \p_r^{e_r'}
  \]
  with $0 \le e_i' \le e_i$. But $n = \sum e_i' f_i = \sum e_i f_i$, so $e_i' = e_i$ for all $i$,
  which completes the proof.
\end{proof}

\section{Ramification}

\begin{definition}
  Let $\p_i, e_i, f_i, r$ be defined as in the statement
  of the previous
  theorem. We say that the $\p_i$ are prime ideals
  \emph{lying over} $p$, and $e_i$ is called
  the \emph{ramification index} of $\p_i$ over $p$.

  If $e_i = 1$, then we say that $\p_i$ is
  \emph{unramified} over $p$. Otherwise if
  $e_i > 1$, we say that $p$ is \emph{ramified}.
  Finally if $e_i = n$, then we say that $\p$ is
  \emph{totally ramified}, i.e. $p\OO_K = \p^n$.

  If $p \OO_K$ is prime, then we say that $p$ is
  \emph{inert}. If $r = n$, i.e. if
  $p \OO_K = \p_1 \dots \p_n$ for distinct $\p_i$,
  then we say that $p$ \emph{splits completely} in
  $\OO_K$ (or in $K$). The $f_i$
  is called the \emph{residue degree}.
\end{definition}

\begin{corollary}
  If the minimal polynomial of $\theta \in \OO_K$ is
  Eisenstein at $p$ and $K = \Q(\theta)$, then 
  $p$ is totally ramified in $\OO_K$.
\end{corollary}

\begin{proof}
  We have previously shown that
  $p \nmid [\OO_K : \Z[\theta]]$, so Kummer's theorem
  applies. Let $g(x)$ be the minimal polynomial of
  $\theta$. Then $\overline{g(x)} = x^n$ in
  $\F_p[x]$ since $g(x)$ is Eisenstein at $p$. Thus
  by Kummer's theorem, we have
  $p \OO_K = \p^n$ where $\p = (p, \theta)$, i.e.
  $p$ is totally ramified.
\end{proof}

\begin{corollary}
  Only finitely many primes ramify in any number
  field $K / \Q$. More specifically, if
  $p \nmid [\OO_K : \Z[\theta]]$ for some $\theta \in \OO_K$, then
  $p$ ramifies in $K$ if and only if
  $p | \Delta_K$.
\end{corollary}

\begin{proof}
  Note that $p$ ramifies in $\OO_K$ if and only if
  $\overline{g}$ has a multiple root in $\F_p[x]$,
  if and only if $\Delta(\overline{g}) = 0$ in $\F_p[x]$.
  Now recall that we have
  \[
    \Delta_K = \frac{\Delta_{\Z[\theta]}}{[\OO_K : \Z[\theta]]^2},
  \]
  so $p | \Delta_K$ if and only if
  $p | \Delta_{\Z[\theta]}$ (since
  $p \nmid [\OO_K : \Z[\theta]]$ by hypothesis).
  So we look at $\Delta_{\Z[\theta]}$ instead.
  Taking $g$ to be the minimal polynomial of $\theta$,
  this happens if and only if $p | \Delta(g) \equiv \Delta_K$.
\end{proof}

\begin{remark}
  The above corollary holds in greater generality
  (without the hypothesis that $p \nmid [\OO_K : \Z[\theta]]$),
  but the proof requires some more advanced tools.
\end{remark}

\section{More Computing of Rings of Integers}

\begin{remark}
  Recall that $\OO_{\Q(\sqrt[3]{2})} = \Z[\sqrt[3]{2}]$.
  We will now generalize this result.
\end{remark}

\begin{theorem}
  Let $p$ be a prime and $a \ne 0, \pm 1$ be a
  square-free integer such that $(p, a) = 1$.
  Let $\theta = \sqrt[p]{a}$. Letting
  $K = \Q(\theta) = \Q[x] / (x^p - a)$, we have
  $\OO_K = \Z[\theta]$
  if and only if $a^{p - 1} \not\equiv 1 \Pmod{p^2}$.
\end{theorem}

\begin{proof}
  $(\Leftarrow)$
  Let $K = \Q(\theta)$ and assume that
  $a^p \not\equiv a \Pmod{p^2}$. The discriminant
  of $x^p - a$ is
  \[
    \Delta(\theta) = \pm p^p a^{p - 1}
    = \Delta_K \cdot [\OO_K : \Z[\theta]]^2.
  \]
  Note that $x^p - a$ is Eisenstein at every prime
  dividing $a$. Now observe that
  \[
    (x + a)^p - a
  \]
  is Eisenstein at $p$ (since $p^2 \nmid (a^p - a)$
  by hypothesis), and $\Z[\theta] = \Z[\theta - a]$,
  so $p \nmid [\OO_K : \Z[\theta]]$.

  $(\Rightarrow)$ Suppose that $\OO_K \ne \Z[\theta]$.
  Kummer's theorem implies that $p \OO_K = \p^p$ where
  $\p = (p, \theta - a)$. Note that
  \[
    x^p - a \equiv (x - a)^p \pmod{p}
  \]
  by Fermat's little theorem, and that $N(\p) = p$.
  Now $\theta - a \in \p$ (and $\theta - a \notin \p^2$), so
  \[
    p \in \p^2 = (p^2, p(\theta - a), (\theta - a)^2)
  \]
  since $\p^2 | (p) = \p^p$ and $p \ge 2$, so
  $(p) \subseteq \p^2$. Thus we have
  $(\theta - a) = \p \mathfrak{a}$
  for some ideal $\mathfrak{a}$ which is relatively
  prime to $\p$. Now
  $(p, N(\mathfrak{a})) = 1$ since
  $N(\mathfrak{a}) = \prod q_i^{e_i f_i}$ where
  $\mathfrak{a} = \q_1^{e_1} \dots \q_r^{e_r}$ for
  $\q_i \ne \p$, so $q_i \ne p$. Then
  \[
    a^p - a = |N(\theta - a)| = N(\p \mathfrak{a})
    = p N(\mathfrak{a})
  \]
  where $N(\mathfrak{a})$ is relative prime to $p$,
  so $p^2$ does not divide $a^p - a$.
\end{proof}

\begin{remark}
  Next class, we will show that $\Q(\sqrt{-5})$
  has class number $2$, which we will use to solve
  the Diophantine equation $y^2 = x^3 - 5$ in $\Z$
  via arithemtic in $\Z[\sqrt{-5}]$, by writing
  \[
    x^3 = y^2 + 5 = (y + \sqrt{-5})(y - \sqrt{-5}).
  \]
  We will fix our previous issue by arguing
  that if a product of ideals is
  a cube, then each ideal is a cube when the class
  number is not a multiple of $3$. This is
  similar to Kummer's work on Fermat's last theorem.
\end{remark}

  \chapter{Jan.~30 --- Computing Ideal Class Groups and Applications}

\section{Computing Ideal Class Groups}

\begin{example}
  Let $K = \Q(\sqrt{2})$. We know that
  $\OO_K = \Z[\sqrt{2}]$. Recall that every ideal
  class in $\Cl(\OO_K)$ contains an ideal of
  norm $\le M$, where (if
  $\OO_K = \Z \alpha_1 \oplus \dots \oplus \Z \alpha_n$
  as $\Z$-modules)
  \[
    M = \prod_{i = 1}^n \sum_{j = 1}^n |\sigma_{i}(\alpha_j)|.
  \]
  Thus in this case, we have
  $M = (1 + \sqrt{2})^2 \approx 5.8 < 6$. Thus
  every ideal class contains an ideal of norm
  $\le 5$. We will want to factor the
  ideals $(2), (3), (5)$ in $\OO_K$ via Kummer's
  theorem. We need to factor
  \[
    x^2 - 2 \Pmod{p}, \quad p = 2, 3, 5.
  \]
  Mod $2$, we have $x^2 - 2 \equiv x^2 \Pmod{2}$,
  so $(2) = \p_2^2$. Now $x^2 - 2$ is irreducible mod
  both $3$ and $5$ since $2$ is not a quadratic residue
  mod $3$ or $5$. Thus $(3) = \p_3$ and
  $(5) = \p_5$ with $f_3 = f_5 = 2$. Thus
  \[
    N(\p_2) = 2, \quad N(\p_3) = 9, \quad N(\p_5) = 25.
  \]
  So any nonzero ideal in $\OO_K$ is equivalent to
  $\p_2$ or $(1)$ (since $\p_2^2 = (2)$ is principal).
  Thus
  \[
    \Cl(\OO_K) = \{(1), [\p_2]\},
  \]
  which is isomorphic to either
  $\{1\}$ if $\p_2$ is principal or
  $\Z / 2\Z$ if $\p_2$ is not principal. But
  $\p_2 = (\sqrt{2})$, so
  \[\Cl(\Z[\sqrt{2}]) = \{1\}.\]
  This also implies that $\Z[\sqrt{2}]$ is a PID,
  and hence a UFD.
\end{example}

\begin{example}
  Let $K = \Q(\sqrt{-5})$, where we have
  seen that $\OO_K = \Z[\sqrt{-5}]$. By a similar
  reasoning as above, we have
  $M = (1 + \sqrt{5})^2 < 11$. So we want to find
  all nonzero prime ideals of norm $\le 10$. We consider
  $x^2 + 5 \Pmod{p}$ for $p = 2, 3, 5, 7$, where
  we can factor
  \begin{align*}
    p = 2 & : x^2 + 5 \equiv x^2 + 1 \equiv (x + 1)^2 \Pmod{2}, \\
    p = 3 & : x^2 + 5 \equiv x^2 - 1 \equiv (x - 1)(x + 1) \Pmod{3}, \\
    p = 5 & : x^2 + 5 \equiv x^2 \Pmod{5}, \\
    p = 7 & : x^2 + 5 \equiv (x + 3)(x + 4) \Pmod{7}.
  \end{align*}
  Thus we have
  $(2) = \p_2^2$, $(3) = \p_3 \p_3'$, $(5) = \p_5^2 = (\sqrt{-5})^2 = (\sqrt{-5})$,
  and $(7) = \p_7 \p_7'$. Since $(5)$ is principal,
  we only need to consider $(2), (3), (7)$, which
  each have norm $\le 10$. Thus
  \[
    \Cl(\OO_K) = \langle [\p_2], [\p_3], [\p_7] \rangle,
  \]
  where we note that $\p_3'$ and $\p_7'$
  are inverses to $\p_3$ and $\p_7$, respectively,
  since their product is principal, so we do not
  need to include them as generators.
  Note that $\OO_K^\times = \{\pm 1\}$, so we do
  not need to worry much about units. If
  any of the $\p_p$ is principal, then it is generated
  by an element $\alpha = a + b\sqrt{-5}$ with
  \[
    p = N(\alpha) = a^2 + 5b^2.
  \]
  This cannot happen for $p = 2, 3, 7$, so
  $\p_2, \p_3, \p_7$ are not principal. Now
  $N(1 + \sqrt{-5}) = 6$, so
  \[
    (1 + \sqrt{-5}) = \p_2 \p_3 \text{ or } \p_2 \p_3'.
  \]
  This means that one of $\p_3$ or $\p_3'$ is an
  inverse to $\p_2$ in $\Cl(\OO_K)$, so in fact
  \[
    \Cl(\OO_K) = \langle [\p_2], [\p_7] \rangle.
  \]
  Also note that $[\p_2]^2 = 1$ since $\p_2^2 = (2)$
  is principal. We can also see that
  $N(3 + \sqrt{-5}) = 14$, so
  \[
    (3 + \sqrt{-5}) = \p_2 \p_7 \text{ or } \p_2 \p_7'.
  \]
  Thus we also do not need $\p_7$ as a generator, so
  $\Cl(\OO_K) = \langle [\p_2] \rangle$. Since
  $[\p_2]^2 = 1$ and $\p_2$ is not principal,
  \[
    \Cl(\Z[\sqrt{-5}]) = \Z / 2\Z.
  \]
\end{example}

\section{Applications of Class Group Computations}

\begin{theorem}
  The Diophantine equation $y^2 = x^3 - 5$ has
  no integer solutions.
\end{theorem}

\begin{proof}
  Assume that we have a solution $x, y$ to the
  above equation. Writing $x^3 = y^2 + 5$, we
  can factor
  \[
    x^3 = (y + \sqrt{-5})(y - \sqrt{-5}).
  \]
  in $\Z[\sqrt{-5}]$. By looking at the equation
  mod $4$, we see that $x$ must be odd.
  Also, $(x, y) = 1$ since otherwise $(x, y) = 5$
  and one can derive a contradiction with the
  equation $y^2 = x^3 - 5$.

  Now we claim that
  $(y + \sqrt{-5})$ and $(y - \sqrt{-5})$ are
  coprime (equivalent to comaximal in a Dedekind
  domain) ideals. To see this, suppose otherwise
  that $\p$ divides both. Then $\p | (x^3) = (x)^3$,
  so we must have $\p | (x)$ by unique factorization.
  Also, $\p | (2y)$. But $\p | (x)$ means that
  $N(\p)$ is odd, so $\p \nmid (2)$. Thus
  $\p | (y)$. Then $(x, y) = 1$ implies that
  $\p | (1)$, a contradiction. Thus
  $(y + \sqrt{-5})$ and $(y - \sqrt{-5})$ are
  coprime.

  Thus we may write
  \[
    (x)^3 = (y + \sqrt{-5})(y - \sqrt{-5}).
  \]
  Since the above ideals are relatively prime,
  unique factorization implies that each of
  $(y \pm \sqrt{-5})$ is the cube of some ideal. 
  Write $(y + \sqrt{-5}) = \mathfrak{a}^3$ for some ideal
  $\mathfrak{a}$. Then $[\mathfrak{a}]^3 = 1$
  in $\Cl(\Z[\sqrt{-5}])$, so
  $\Cl(\Z[\sqrt{-5}]) = \Z / 2\Z$ implies that
  $[\mathfrak{a}] = 1$. Thus $\mathfrak{a} = (\alpha)$
  for some $\alpha = a + b \sqrt{-5}$, so
  \[
    (a + b \sqrt{-5})^3
    = \alpha^3 = \pm (y + \sqrt{-5})
  \]
  as elements. Finish the proof and derive a
  contradiction from here as an exercise.
\end{proof}

\section{Cyclotomic Fields}

\begin{theorem}
  Let $m = p^k$ for $k \ge 1$. Then the ring of
  integers of $\Q(\zeta_m)$ is $\Z[\zeta_m]$.
\end{theorem}

\begin{remark}
  Recall that the minimal polynomial for
  $\zeta_m$ over $\Q$ is
  \[
    \Phi_m(x) = \frac{x^{p^k} - 1}{x^{p^{k - 1}} - 1}
    = x^{(p - 1)p^{k - 1}} + x^{(p - 2)p^{k - 1}} + \dots + x^{p^{k - 1}} + 1.
  \]
  This polynomial is irreducible
  over $\Z$ (e.g. $\Phi_m(x + 1)$ is Eisenstein at $p$),
  with
  \[
    \deg \Phi_m(x) = \phi(m) = (p - 1)p^{k - 1}.
  \]
  Note that $\Q(\zeta_m) / \Q$ is Galois
  with Galois group isomorphic to $(\Z / m\Z)^\times$.
\end{remark}

\begin{lemma}
  For any $m \ge 1$, the
  discriminant $\Delta(\zeta_m) = \disc(\Phi_m) = \Delta(1, \zeta_m, \zeta_m^2, \dots)$
  divides $m^{\phi(m)}$.
\end{lemma}

\begin{proof}
  Since $\Phi_m | (x^m - 1)$, we can write
  $x^m - 1 = \Phi_m(x) g(x)$ for a polynomial $g$.
  Taking
  a derivative,
  \[
    mx^{m - 1} = \Phi_m'(x) g(x) + \Phi_m(x) g'(x).
  \]
  Plugging in $x = \zeta_m^j$ for $(j, m) = 1$ (these
  are the conjugates of $\zeta_m$),
  \[
    m\zeta_m^{j(m - 1)} = \Phi_m'(\zeta_m^j) g(\zeta_m^j)
  \]
  since $\Phi_m(\zeta_m^j) = 0$. Taking the norm in
  $\Q(\zeta_m) / \Q$, we have
  \[
    m^{\phi(m)}
    = N(m \zeta_m^{m - 1})
    = N(\Phi_m'(\zeta_m^j)) N(g(\zeta_m^j))
    = \pm \Delta(\zeta_m) N(g(\zeta_m^j)),
  \]
  where $N(g(\zeta_m^j)) \in \Z$
  since an integer polynomial evaluated at an algebraic
  integer is an algebraic integer, and the norm of
  an algebraic integer is an integer. Also,
  $g(\zeta_m^j) \ne 0$, so $\Delta(\zeta_m) | m^{\phi(m)}$.
\end{proof}

\begin{remark}
  Using field/Galois theory, we will show
  that $\OO_{\Q(\zeta_m)} = \Z[\zeta_m]$
  for all $m \ge 1$ next time.
\end{remark}

\section{First Case of Fermat's Last Theorem}

\begin{remark}
  The goal for now is to show that if
  $p$ is an odd prime and $u \in \Z[\zeta_p]^\times$,
  then
  \[
    \frac{u}{\overline{u}} = \zeta_p^k, \quad
    \text{for some integer $k$}.
  \]
\end{remark}

\begin{lemma}
  If $m$ is a positive integer, then the roots
  of unity in $\Q(\zeta_m)$ are
  \[
    \begin{cases}
      \text{primitive $m$th roots of $1$} & \text{if $m$ is even}, \\
      \text{primitive $(2m)$-th roots of $1$} & \text{if $m$ is odd}.
    \end{cases}
  \]
\end{lemma}

\begin{proof}
  If $m$ is odd, then $\Q(\zeta_m) = \Q(\zeta_{2m})$.
  So assume without loss of generality that $m$ is even.
  Suppose $\zeta \in \Q(\zeta_m)$
  and $\zeta^k = 1$ for some $k$. We want to show
  that $k | m$. Assume without loss of generality that
  $\zeta = e^{2\pi i / k}$. Since
  $\zeta_k, \zeta_m \in \Q(\zeta_{m})$, one can check
  as an exercise that
  $\zeta_r \in \Q(\zeta_{m})$ where $r = \lcm(k, m)$. Then
  $\Q(\zeta_r) \subseteq \Q(\zeta_m)$, so taking
  degrees implies that $\phi(r) \le \phi(m)$. Also
  $m | r$, which one can check
  implies that $\phi(m) \le \phi(r)$.
  This means that $\phi(m) = \phi(r)$, so
  $m = r$ and thus $k | m$.
\end{proof}

  \chapter{Feb.~4 --- Fermat's Last Theorem}

\section{First Case of Fermat's Last Theorem, Continued}

\begin{theorem}[Kronecker]
  If $\alpha \in \C$ is a nonzero algebraic integer,
  all of whose complex conjugates have absolute value
  $1$, then $\alpha$ is a root of unity.
\end{theorem}

\begin{proof}
  Let $f(x) \in \Z[x]$ be the minimal polynomial
  of $\alpha$ (since $\alpha$ is an algebraic integer,
  $f$ is monic):
  \[
    f(x) = x^n + a_{1} x^{n - 1} + \dots + a_{n - 1} x + a_0.
  \]
  Then there exists $C > 0$ such that $|a_i| \le C$
  for all $i$, where $C$ depends only on $n$ (this is
  because each $|a_i| = |\sigma_i(\alpha_1, \dots, \alpha_n)| \le 2^n = C$ by the triangle inequality, where $\sigma_i$ is the $i$th elementary
  symmetric polynomial).  Since the
  $a_i$ are integers, there are only finitely possible
  choices for $f$. Then since each $f$ only has finitely
  many roots, there are only finitely many possible
  choices for $\alpha$.

  The same argument applies
  to $\alpha, \alpha^2, \alpha^3, \dots$, so we see that
  $\{\alpha, \alpha^2, \alpha^3, \dots\}$ is a finite
  set. Thus there exist $i < j$ such that
  $\alpha^i = \alpha^j$, which implies
  that $\alpha^{j - i} = 1$, i.e. $\alpha$ is a
  root of unity.
\end{proof}

\begin{remark}
  The algebraic integer assumption is necessary, e.g.
  consider $(3 / 5) + (4 / 5) i$.
\end{remark}

\begin{lemma}
  If $\alpha \in \Z[\zeta_p]$, then $\alpha^p \equiv a \Pmod{p}$
  for some $a \in \Z$.
\end{lemma}

\begin{proof}
  Let $\zeta = \zeta_p$ and write
  \[
    \alpha = a_0 + a_1 \zeta + \dots + a_{p - 2} \zeta^{p - 2}
  \]
  with $a_i \in \Z$. By the binomial theorem, we have
  (all the cross-terms are divisible by $p$)
  \[
    \alpha^p \equiv a_0^p + (a_1 \zeta)^p + \dots + (a_{p - 2} \zeta^{p - 2})^p
    \equiv a_0 + a_1 + \dots + a_{p - 2} \Pmod{p},
  \]
  where the second step is by
  $\zeta^p = 1$ and Fermat's little theorem.
  So we can take $a = a_0 + \dots + a_{p - 2}$.
\end{proof}

\begin{theorem}[Kummer]
  Let $p$ be an odd prime. If $u \in \Z[\zeta_p]$ is
  a unit, then $u / \overline{u} = \zeta_p^k$ for some
  $k \in \Z$.
\end{theorem}

\begin{proof}
  Let $\alpha = u / \overline{u} \in \Z[\zeta_p] \subseteq \C$.
  Then all conjugates of $\alpha$ have absolute
  value $1$ (the key point is that
  $\Gal(\Q(\zeta_p) / \Q)$ is abelian, so complex
  conjugation commutes with taking the other conjugates).
  So $\alpha$ is a root of unity, which implies
  that $\alpha = \pm \zeta_p^k$ for some $k$. All that
  remains is to show that the sign is $+$.

  Suppose for sake of contradiction that
  $u / \overline{u} = -\zeta_p^k$. Then raising
  both sides to the $k$th power, we find that
  $u^p = -\overline{u}^p$. By the lemma, 
  there is $a \in \Z$ such that $u^p \equiv a \equiv \overline{u}^p \Pmod{p}$ (the second congruence
  comes from taking complex conjugates of both sides).
  This implies that $2 u^p \equiv 0 \Pmod{p}$, so
  we get $p | u^p$ since $p$ is an odd prime. This
  is a contradiction as $u^p$ is a unit.
\end{proof}

\begin{definition}
  A prime $p$ is \emph{regular} if
  $p \nmid |{\Cl(\Z[\zeta_p])}|$, and $p$ is
  \emph{irregular} otherwise.
\end{definition}

\begin{remark}
  Kummer realized that class numbers are related
  to certain congruences of Bernoulli numbers, which
  gives a better criterion for a prime to be regular.
\end{remark}

\begin{remark}
  It is known that there are infinitely many irregular
  primes, with the smallest being $p = 37$. Thus the
  below theorem, due to Kummer, works for all
  primes $p \le 31$.
\end{remark}

\begin{theorem}[Fermat's last theorem for regular primes, first case]
  Let $p \ge 5$ be a regular prime and $p \nmid xyz$.
  Then the equation
  $x^p + y^p = z^p$ has no solutions for
  non-zero integers $x, y, z$.
\end{theorem}

\begin{proof}
  Let $\zeta = \zeta_p$ and write
  (the following equality is on the level of
  ideals)
  \[
    (x + y)(x + \zeta y) \dots (x + y \zeta^{p - 1}) = (z)^p. \tag{$*$}
  \]
  Check as an exercise that the ideals on the left-hand
  side are pairwise relatively prime. Due to
  unique factorization of ideals into prime ideals,
  we must have $(x + y \zeta) = I^p$ for some ideal $I$.
  Now $I$ must be principal since $p$ is regular
  (since $(x + y \zeta)$ is principal but $p$ does
  not divide the class number). So
  \[
    x + y \zeta = u \cdot \alpha^p \quad \text{for some } \alpha \in \Z[\zeta].
  \]
  We claim that $x \equiv y \Pmod{p}$ $(**)$.
  Assuming $(**)$ for now, we have $x \equiv -z \Pmod{p}$
  as well (we can use symmetry to apply the same argument
  to the equation $x^p + (-z)^p = (-y)^p$, since
  $p$ is odd), so
  \[
    2x^p \equiv x^p + u^p \equiv z^p \equiv -x^p \Pmod{p}.
  \]
  This implies that $p | 3x^p$, which is a contradiction
  since $p \ge 5$ and $p \nmid x$.

  Now we prove $(**)$ via the result about
  $u / \overline{u} = \zeta^p$. We have
  $\alpha^p \equiv a \Pmod{p}$ for some $a \in \Z$, so
  \[
    x + y\zeta \equiv u a \Pmod{p}.
  \]
  Then  $\overline{\zeta}_p = \zeta_p^{-1}$
  implies that
  \[
    x + y\zeta^{-1} = \overline{x + y\zeta} \equiv \overline{u} a \Pmod{p}.
  \]
  So $(x + y\zeta) \overline{u} \equiv (x + y\zeta^{-1}) u \Pmod{p}$,
  which implies
  \[
    x + y\zeta \equiv (x + y\zeta^{-1}) \frac{u}{\overline{u}} \Pmod{p}.
  \]
  Thus $x + y\zeta \equiv x \zeta^k + y\zeta^{k - 1} \Pmod{p}$
  for some $0 \le k \le p - 1$ (use
  $u / \overline{u} = \zeta^k$). Show as an exercise
  that this is only possible if $k = 1$ (note that
  the powers of $\zeta$ form an integral basis, so
  a representation in powers of $\zeta$ must be unique).
  This then implies that $x \equiv y \Pmod{p}$, which
  completes the proof.
\end{proof}

\section{More on Cyclotomic Fields}

\begin{definition}
  If $K \subseteq K_1, K_2 \subseteq L$,
  then the \emph{compositum} of $K_1$ and $K_2$ is the
  smallest subfield of $L$ containing both $K_1$ and
  $K_2$.
\end{definition}

\begin{remark}
  If everything is Galois in the above definition,
  $K = K_1 \cap K_2$, and
  $K_1, K_2$ are \emph{linearly disjoint}, i.e.
  that we have
  $[K_1 K_2 : K] = [K_1 : K] [K_2 : K]$, then
  one obtains the result
  \[
    \Gal(K_1 K_2 / K_2) \cong \Gal(K_1 / K).
  \]
\end{remark}

\begin{prop}
  Let $K, K'$ be number fields of degree $n, n'$,
  respectively.
  Assume that
  \begin{enumerate}
    \item $K, K'$ are both Galois over $\Q$,
    \item $K \cap K' = \Q$,
    \item and $(|\Delta_K|, |\Delta_{K'}|) = 1$.
  \end{enumerate}
  Then if $\alpha_1, \dots, \alpha_n$ is an integral
  basis for $\OO_K$ and $\alpha_1', \dots, \alpha_{n'}'$
  is an integral basis for $\OO_{K'}$, then
  $\{\alpha_i \alpha_{j'}\}$ is an integral basis
  for $\OO_{KK'}$.
\end{prop}

\begin{proof}
  Let $\alpha \in \OO_{KK'}$. Field theory shows that
  $\{\alpha_i \alpha_{j'}\}$ is a basis for
  $KK'$ over $\Q$, so we can write
  \[
    \alpha = \sum_{i, j} a_{ij} \alpha_i \alpha_{j}',
    \quad a_{ij} \in \Q.
  \]
  We need to show that $a_{ij} \in \Z$. Let
  $d = |\Delta_K|, d' = |\Delta_{K'}|$. We will
  show that $d a_{ij} \in \Z$ and
  $d' a_{ij} \in \Z$, which will imply that 
  $a_{ij} \in \Z$ since $(d, d') = 1$. Let
  \[
    \beta_j = \sum_i a_{ij} \alpha_i', \quad j = 1, \dots, n.
  \]
  Let $T$ be the $n \times n$ matrix with
  $T_{\ell j} = \sigma_{\ell}(\alpha_j)$, where
  $\sigma_1, \dots, \sigma_n$ are the embeddings of
  $KK'$ over $K'$, i.e. the elements of
  $\Gal(KK' / K')$ since $K, K'$ are Galois over $\Q$.
  Let
  \[
    a =
    \begin{bmatrix}
      \sigma_1(\alpha) \\ \vdots \\ \sigma_n(\alpha)
    \end{bmatrix}
    \quad \text{and} \quad
    b =
    \begin{bmatrix}
      \beta_1 \\ \vdots \\ \beta_n
    \end{bmatrix}.
  \]
  Check as an exercise that $a = Tb$. Multiplying
  this equation by $\adj T$, we find that
  \[
    (\adj T) a = (\det T) b.
  \]
  Now all the entries of $T$, $\adj T$, and $a$
  are algebraic integers, so multiplying the above
  equation by $\det T$ implies that the entries
  of $db$ are algebraic integers. Thus
  $d\beta_j \in \OO_{K'}$ for all $j$. But the
  $\{\alpha_i'\}$ were an integral basis for $\OO_K$
  over $\Z$ by assumption, so in fact $d a_{i, j} \in \Z$
  for all $i, j$, completing the proof.
\end{proof}

\begin{remark}
  Let $K_m = \Q(\zeta_m)$. If $(m, m') = 1$, then
  $(|\Delta_{K_m}|, |\Delta_{K_{m'}}|) = 1$.
  Additionally, some field theory shows that
  if $(m, m') = 1$,
  then $K_m \cap K_{m'} = \Q$ and
  $K_m K_{m'} = K_{mm'}$ in $\C$.

  Using this proposition, along with the previous
  case that $\OO_{\Q(\zeta_m)} = \Z[\zeta_m]$ for
  $m = p^k$ and the Chinese
  remainder theorem,
  one can show that $\OO_{\Q(\zeta_m)} = \Z[\zeta_m]$
  for any $m \ge 1$.
\end{remark}

\begin{corollary}
  For all $m \ge 1$, we have
  $\OO_{\Q(\zeta_m)} = \Z[\zeta_m]$.
\end{corollary}

\section{Motivation for Geometry of Numbers}

\begin{remark}
  We will discuss Minkowski's geometry of numbers next.
  The goals of this are the following:
  \begin{enumerate}
    \item Improve the constant $M$ from the class
      number computations.
    \item Prove the ``four squares'' theorem, i.e.
      that every $n \ge 1$ is the sum of $4$ integer
      squares.
  \end{enumerate}
\end{remark}

  \chapter{Feb.~6 --- Geometry of Numbers}

\section{Complete Lattices and Covolume}
\begin{remark}
  Let $K$ be a number field with $[K : \Q] = n$.
  Recall that $\OO_K \cong \Z^n$ as $\Z$-modules, but
  we will use a different method.
  Let $\sigma_1, \dots, \sigma_n : K \hookrightarrow \C$
  be the $n$ embeddings of $K$ into $\C$.
  Call an embedding $\sigma$ \emph{real} if $\sigma(K) \subset \R$, and
  \emph{complex} otherwise. Note that if
  $\sigma : K \hookrightarrow \C$ is complex, then $\overline{\sigma} : K \hookrightarrow \C$
  satisfies
  \[
    \re(\sigma) = \re(\overline{\sigma}) \quad \text{and} \quad \im(\sigma) = -\im(\overline{\sigma}),
  \]
  so $\sigma, \overline{\sigma}$ are kind of dependent.
  This indicates that we should consider pairs of
  complex embeddings instead. Suppose we
  have $r_1$ real embeddings
  $\sigma_1, \dots, \sigma_{r_1} : K \hookrightarrow \R$ and $r_2$ pairs of complex embeddings
  $\tau_1, \dots, \tau_{r_2} : K \hookrightarrow \C$,
  so that $r_1 + 2r_2 = n$. Then we have the
  embedding
  \[
    (\sigma_1, \dots, \sigma_{r_1}, \tau_1, \dots, \tau_{r_2})
    : K \overset{i}{\hookrightarrow} \R^{r_1} \times \C^{r_2} \cong
    \R^{r_1} \times (\R^2)^{r_2} \cong \R^{r_1 + 2r_2} \cong \R^n. \tag{$*$}
  \]
\end{remark}

\begin{definition}
  Let $\Lambda$ be a complete lattice in $\R^n$. A
  \emph{fundamental domain} $F$ is a subset
  $F \subseteq \R^n$ such that for all
  $x \in \R^n$, there is a unique $y \in F$ such that
  $x - y \in \Lambda$.
\end{definition}

\begin{remark}
  If $x_1, \dots, x_n$ form a
  $\Z$-basis for $\Lambda$, i.e.
  $\Lambda = \Z x_1 + \cdots + \Z x_n$, then
  \[
    F = \{a_1 x_1 + \dots + a_n x_n : 0 \le x_i \le 1\}
  \]
  is a fundamental domain, and every fundamental
  domain arises in this way. In particular, this means
  that if $F, F'$ are fundamental domains,
  there exists $T \in \GL_n(\Z)$ such that
  $F' = T(F)$.
\end{remark}

\begin{definition}
  The \emph{covolume} of a complete lattice
  $\Lambda \subseteq \R^n$, denoted
  $\covol(\Lambda)$, is the volume of any fundamental
  domain. The above discussion says that $\covol(\Lambda)$
  is well-defined.\footnote{Note that we can think of $\covol{\Lambda}$ as $\vol(\R^n / \Lambda)$, which motivates the name ``covolume.''}
\end{definition}

\begin{theorem}
  Let $i : K \to \R^n$ be defined as in $(*)$. Then
  \begin{enumerate}
    \item $i(\OO_K)$ is a complete lattice in $\R^n$, and
    \item $\covol(i(\OO_K)) = 2^{-r_2} \sqrt{|\Delta_K|}$.
  \end{enumerate}
\end{theorem}

\begin{proof}
  One can show that $i(\OO_K)$ is discrete, so
  it is a complete lattice in $\R^n$ (since $\OO_K$ is
  of rank $n$ and $i$ is an embedding).
  Let $\alpha_1, \dots, \alpha_n$ be an integral
  basis for $\OO_K$. Let
  \[
    B = \begin{bmatrix}
      \sigma_1(\alpha_1) & \dots & \sigma_1(\alpha_n) \\
      \vdots & \ddots & \vdots \\
      \sigma_{r_1}(\alpha_1) & \dots & \sigma_{r_1}(\alpha_n)
    \end{bmatrix},
  \]
  and note that $|{\det B}| = \sqrt{|\Delta_K|}$.
  Now define the matrix
  \[
    A = \begin{bmatrix}
      \sigma_1(\alpha_1) & \dots & \sigma_1(\alpha_n) \\
      \vdots & \ddots & \vdots \\
      \sigma_{r_1}(\alpha_1) & \dots & \sigma_{r_1}(\alpha_n) \\
      \re \tau_{1}(\alpha_1) & \dots & \re \tau_{1}(\alpha_n) \\
      \im \tau_{1}(\alpha_1) & \dots & \im \tau_{1}(\alpha_n) \\
      \vdots & \ddots & \vdots \\
      \re \tau_{r_2}(\alpha_1) & \dots & \re \tau_{r_2}(\alpha_n) \\
      \im \tau_{r_2}(\alpha_1) & \dots & \im \tau_{r_2}(\alpha_n)
    \end{bmatrix},
  \]
  and one can compute that $\covol(i(\OO_K)) = |{\det A}| = 2^{-r_2} |{\det B}| = 2^{-r_2} \sqrt{|\Delta_K|}$.
\end{proof}

\begin{prop}
  Let $\Lambda' \subseteq \Lambda$ be a finite
  index sublattice. Then
  \[
    \covol(\Lambda') = [\Lambda : \Lambda'] \cdot \covol(\Lambda).
  \]
\end{prop}

\begin{proof}
  A fundamental domain for $\Lambda'$ contains
  $[\Lambda : \Lambda']$ copies of a fundamental domain
  for $\Lambda$.
\end{proof}

\begin{corollary}
  For an ideal $I \subseteq \OO_K$, we have
  $\covol(i(I)) = N(I) \cdot 2^{-r_2} \sqrt{|\Delta_K|}$.
\end{corollary}

\begin{remark}
  We have seen before that for every ideal $I$, there
  exists $\alpha \in I$ such that
  $|N(\alpha)| \le M \cdot N(I)$.
  We will now try to improve this constant $M$ using
  the above ideas. To do this, we will need to find a
  norm function on $\R^n$ which is compatible with
  $N$ on $\OO_K$.
\end{remark}

\begin{definition}
  Define the \emph{norm} $\mathcal{N} : \R^n \to \R$ by
  \[
    \mathcal{N}(a_1, \dots, a_{r_1}, x_1, y_1, \dots, x_{r_2}, y_{r_2}) = a_1 \dots a_{r_1} (x_1^2 + y_1^2) \dots (x_{r_2}^2 + y_{r_2}^2),
  \]
  for $(a_1, \dots, a_{r_1}) \in \R^{r_1}$ and
  $(x_1, y_1, \dots, x_{r_2}, y_{r_2}) \in \C^{r_2} \cong \R^{2r_2}$.
  Note that $N(\alpha) = \mathcal{N}(i(\alpha))$.
\end{definition}

\section{Minkowski's Theory of the Geometry of Numbers}

\begin{definition}
  A set $S \subseteq \R^n$ is \emph{(centrally) symmetric} when
  $x \in S$ if and only if $-x \in S$.
\end{definition}

\begin{lemma}[Geometric pigeonhole principle]
  Let $S \subseteq \R^n$ is a bounded measurable
  set. If $T : S \to \R^n$ is piecewise volume-preserving
  and $\vol(S) > \vol(T(S))$, then $T$ is not injective.
\end{lemma}

\begin{proof}
  Since $T$ is piecewise volume-preserving, we can
  write $S = \bigsqcup S_i$ (disjoint union) such that
  \[
    \vol(T(S_i)) = \vol(S_i) \quad \text{for every } i.
  \]
  If $T$ is injective, then $T(S) = \bigsqcup T(S_i)$,
  which implies that
  \[\vol(T(S)) = \sum \vol(T(S_i)) = \sum \vol(S_i) = \vol(S),\]
  which contradicts the hypothesis that
  $\vol(S) > \vol(T(S))$.
\end{proof}

\begin{example}\label{example:lattice-map}
  Let $\Lambda \subseteq \R^n$ be a lattice and
  $F$ be a fundamental domain for $\Lambda$. Let
  $T : \R^n \to F$ send $x \in \R^n$ to the
  unique $y \in F$ such that $x - y \in \Lambda$.
  Then $T$ is piecewise volume-preserving.
\end{example}

\begin{theorem}[Minkowski's convex body theorem]
  Let $\Lambda \subseteq \R^n$ be a complete lattice,
  and let $S \subseteq \R^n$ be a convex, symmetric,
  bounded set. If
  $\vol(S) > 2^n \covol(\Lambda)$, then
  $S$ contains a nonzero element of $\Lambda$.\footnote{Note from measure theory that any convex set is (Lebesgue) measurable.}
\end{theorem}

\begin{proof}
  Consider the lattice $\Lambda' = 2\Lambda \subseteq \Lambda$, so
  $\covol(\Lambda') = 2^n \covol(\Lambda)$. Let
  $F'$ be a fundamental domain for $\Lambda'$. Let
  $T : \R^n \to F'$ be as in Example \ref{example:lattice-map},
  which is piecewise volume-preserving. Then
  \[
    \vol(S) > 2^n \covol(\Lambda) = \covol(\Lambda')
    = \vol(F') \ge \vol(T(S))
  \]
  since $T(S) \subseteq F'$. Thus by the geometric
  pigeonhole principle, $T$ is not injective, i.e.
  there exist distinct
  $x', y' \in S$ such $T(x') = T(y')$.
  So $p' = x' - y' \in \Lambda'$. Write $p' = 2p$
  with $p \in \Lambda \setminus \{0\}$. Since
  $S$ is symmetric, $-y' \in S$, and convexity implies
  \[
    p = \frac{1}{2}x' + \frac{1}{2}(-y') \in S.
  \]
  Thus $p$ is a nonzero lattice point in $S$, which
  completes the proof.
\end{proof}

\begin{remark}
  The $\vol(S) > 2^n \covol(\Lambda)$ condition is
  sharp: Let $\Lambda = \Z^n$ and
  $S = (-1, 1)^n$. Note that if $S$ is closed (so compact since
  $S$ is bounded), then the same conclusion holds
  when $\vol(S) = 2^n \covol(\Lambda)$.
\end{remark}

\section{Applications to Class Group Computations}
\begin{remark}
  We will apply Minkowski's convex body theorem
  to a compact, convex, symmetric
  \[
    S \subseteq \{x \in \R^n : |\mathcal{N}(x)| \le 1\}.
  \]
  Note that $\{x \in \R^n : |\mathcal{N}(x)| \le 1\}$ is
  not in general convex. For instance, consider the case
  where $K / \Q$ is a real quadratic field and
  $\mathcal{N}(x, y) = xy$. One can try to instead
  consider a subset $S$ with a
  diamond shape lying inside the
  hyperbola shape, and consider homogeneous
  scalings of $S$. In general, set
  \[
    S = \left\{x \in \R^n : |a_1| + \dots + |a_{r_1}| + 2\left(\sqrt{x_1^2 + y_1^2} + \dots + \sqrt{x_{r_2}^2 + y_{r_2}^2} \le n\right)\right\},
  \]
  where $x = (a_1, \dots, a_{r_1}, x_1, y_1, \dots, x_{r_2}, y_{r_2})$.
  One can check that $S \subseteq \{x \in \R^n : |\mathcal{N}(x)| \le 1\}$
  (via tools like the AM-GM inequality, etc.), and
  that $S$ is compact, convex, symmetric.
  One can also explicitly compute via calculus that
  \[
    \vol(S) = \frac{n^n}{n!} 2^{r_1} \left(\frac{\pi}{2}\right)^{r_2}.
  \]
\end{remark}

\begin{corollary}
  Let Minkowski's constant be
  \[
    M_K = \frac{n^n}{n!} \left(\frac{4}{\pi}\right)^{r_2} \sqrt{|\Delta_K|}.
  \]
  Then every ideal class in $\OO_K$ contains
  a nonzero ideal of norm $\le M_K$. Equivalently,
  every ideal $I$ in $\OO_K$ contains an element
  $\alpha \in I$ with $|N(\alpha)| \le M_K \cdot N(I)$.
\end{corollary}

\begin{remark}
  This is a significant improvement over the old method.
  For $\Q(\sqrt{-5})$, the old bound gives
  $M = 10$, whereas this method gives
  $M_k = 4\sqrt{5} / \pi < 3$.
\end{remark}

  \chapter{Feb.~11 --- Lagrange's Four Square Theorem}

\section{Lagrange's Four Square Theorem}
\begin{theorem}[Fermat]
  If $p \equiv 1 \Pmod{4}$ is a prime, then
  there exist $a, b \in \Z$ such that $p = a^2 + b^2$.
\end{theorem}

\begin{proof}[Proof \#1]
  Recall that by Kummer's theorem, factoring
  $(p)$ in $\Z[i]$ is reduced to considering
  $x^2 + 1$ mod $p$. Since $p \equiv 1 \Pmod{4}$,
  we see that $-1$ is a square mod $p$ (e.g. this follows
  by Euler's criterion for the Legendre symbol).
  So $p\Z[i] = \p_1 \p_2$, where each $\p_i$ has norm $p$.
  Since $\Z[i]$ is a UFD, we have $\p_1 = (a + bi)$
  and so $p = \N(\p_1) = N(a + bi) = a^2 + b^2$.
  This proves the theorem.
\end{proof}

\begin{remark}
  There is a way to generalize this proof for sums of
  four squares, but it requires developing unique factorization
  theory in the integer quaternions $\Z[i, j, k]$.
  The proof below generalizes more readily.
\end{remark}

\begin{proof}[Proof \#2]
  As above, we know that $x^2 \equiv -1 \Pmod{p}$
  has a solution, call it $u \in \Z$. Define $\Lambda \subseteq \Z^2$ by
  \[
    \Lambda = \{(a, b) \in \Z^2 \mid b \equiv au \Pmod{p}\}.
  \]
  Then $\Lambda$ is a rank $2$ lattice, and
  $|\Z^2 / \Lambda| = p$ (e.g. one can write an explicit
  isomorphism $\Z^2 / \Lambda \to \Z / p\Z$ by mapping
  $[(a, b)] \in \Z^2 / \Lambda$ to $a \in \Z / p\Z$, noting
  that $a$ completely determines $b$ mod $p$). Let
  \[
    S = \overline{D}(0, r), \quad \text{where } \pi r^2 = 4p.
  \]
  Then $\vol(S) = 4p = 2^2 \cdot \vol(\Lambda)$, so
  by Minkowski's theorem, there exists $(a, b) \ne (0, 0)$
  in $\Lambda \cap S$. So
  \[
    0 < a^2 + b^2 \le r^2 = \frac{4}{\pi} \cdot p < 2p
  \]
  since $\pi > 2$. Since $(a, b) \in \Lambda$, we have
  $b \equiv au \Pmod{p}$, so $b^2 \equiv -a^2 \Pmod{p}$,
  which implies $p | (a^2 + b^2)$. Then
  $a^2 + b^2$ is divisible by $p$ but strictly
  between $0$ an $2p$, so we must have $a^2 + b^2 = p$.
\end{proof}

\begin{theorem}[Lagrange]\label{thm:lagrange-squares}
  Every positive integer $n$ is a sum of four squares.
\end{theorem}

\begin{lemma}
  It suffices to prove Lagrange's four square theorem
  when $n = p$ is prime.
\end{lemma}

\begin{proof}
  There is an identity that says the product of two numbers,
  which are each a sum of four squares, is again a sum of
  four squares. The idea behind the identity is the following:
  In the ring $\mathbb{H}_\Z$ of integral quaternions, we have
  $N(\alpha) = N(\alpha) N(\beta)$, where
  $N(a + bi + cj + dk) = a^2 + b^2 + c^2 + d^2$.
\end{proof}

\begin{remark}
  The norm is also multiplicative in $\C$, where
  $(a + bi)(c + di) = (ac - bd) + (ad + bc)i$
  implies that $(a^2 + b^2) = (ac - bd)^2 + (ad + bc)^2$.
  A similar identity happens for the quaternions.
\end{remark}

\begin{lemma}
  If $p$ is an odd prime, then there exist $u,v \in \Z$
  such that $u^2 + v^2 \equiv -1 \Pmod{p}$.
\end{lemma}

\begin{proof}
  The number of squares in $\F_p = \Z / p\Z$ is
  (e.g. via the primitive element theorem)
  \[
    \frac{p - 1}{2} + 1 = \frac{p + 1}{2}.
  \]
  Define the sets
  \[
    A = \{1 + x^2 : x \in \F_p\} \quad \text{and} \quad
    B = \{-y^2 : y \in \F_p\}.
  \]
  Each of these sets contains $(p + 1) / 2$ elements, so
  $|A| + |B| = p + 1 > \#\F_p$. By the pigeonhole principle,
  $A \cap B \ne \varnothing$, so there exist $u, v \in \F_p$
  such that $1 + u^2 \equiv -v^2 \Pmod{p}$, which
  implies the result.
\end{proof}

\begin{proof}[Proof of Theorem \ref{thm:lagrange-squares}]
  By the first lemma, it suffices to let $n = p$ be prime.
  As
  $2 = 1^2 + 1^2 + 0^2 + 0^2$,
  we can assume that $p$ is an odd prime. By the
  second lemma, choose $u, v$ such that $u^2 + v^2 \equiv -1 \Pmod{p}$.
  Define $\Lambda \subseteq \Z^4$ by
  \[
    \Lambda = \{(a, b, c, d) \in \Z^4 :
    c \equiv ua + vb, d = ub - va \Pmod{p}\}.
  \]
  Note that $\covol(\Lambda) = |\Z^4 / \Lambda| = p^2$
  (similarly we find an isomorphism $\Z^4 / \Lambda \to (\Z / p\Z)^2$ by $(a, b, c, d) \mapsto (a, b)$).
  Once we can find $(a, b, c, d) \ne 0$ in $\Lambda$
  with norm $< 2p$ (claim), we have
  \begin{align*}
    a^2 + b^2 + c^2 + d^2
    &\equiv a^2 + b^2 + (ua + vb) ^2 + (ub - va)^2 \\
    &\equiv a^2 + b^2 + u^2 a^2 + v^2 b^2 + u^2 b^2 + v^2 a^2 \\
    &\equiv a^2 + b^2 + (u^2 + v^2)a^2 + (u^2 + v^2)b^2 \\
    &\equiv a^2 + b^2 - a^2 - b^2 \\
    &\equiv 0 \Pmod{p},
  \end{align*}
  since $u^2 + v^2 \equiv -1 \Pmod{p}$. Then as before,
  $a^2 + b^2 + c^2 + d^2$ is divisible by $p$ but
  lands strictly between $0$ and $2p$, so
  we must have $a^2 + b^2 + c^2 + d^2 = p$. Given the
  claim, this proves the result.

  So it suffices to
  find such a point $(a, b, c, d) \ne 0$. Let
  \[
    B_r = \text{4-dimensional closed ball of radius $r$}.
  \]
  If $\vol(B_r) = 2^4 p^2$, then by Minkowski we
  get a nonzero lattice point in $B_r$. We want $r^2 < 2p$.
  We have
  \[
    \frac{1}{2}\pi^2 r^4 = \vol(B_r) = 16p^2,
  \]
  so $r^2 = \sqrt{32p^2 / \pi^2} = (4\sqrt{2} / \pi) \cdot p$.
  So we need $4 \sqrt{2} / \pi < 2$, which happens if
  and only if $\pi > 2\sqrt{2}$. This is true, since
  $\pi \approx 3.14$ and $2\sqrt{2} \approx 2.83$, which
  completes the proof of the theorem.
\end{proof}

\section{Revisiting Minkowski's Theorem}

\begin{remark}
  Recall that for a number field $K$, every ideal class
  is represented by an ideal of norm $\le M_K$:
  \[
    M_K = \frac{n!}{n^n} \left(\frac{4}{\pi}\right)^{r_2} \sqrt{|\Delta_K|}.
  \]
  This implies that $M_K \ge 1$, since an ideal cannot
  have norm $0$. In the above formula, this gives
  \[
    \sqrt{|\Delta_K|} \ge \frac{n^n}{n!} \left(\frac{\pi}{4}\right)^{n / 2},
  \]
  which is $> 1$ when $n \ge 2$ and goes to $\infty$ as
  $n \to \infty$. This implies that if $K \ne \Q$ is
  any number field, then some $p$ ramifies in $\OO_K$.
  This then means that $\Z$ has no unramified covers,
  which means that $\Z$ has trivial
  ``arithmetic fundamental group.'' The subject discussing
  this is called \emph{\'etale cohomology}.
\end{remark}

\begin{theorem}[Tate?]
  If $E / \Q$ is an elliptic curve, then there is some
  prime of bad reduction.
\end{theorem}

\begin{remark}
  The above theorem from the theory of elliptic curves
  has a similar flavor to our
  discussion above. Fontaine greatly generalizes this result to
  an abelian variety $A / \Q$.
\end{remark}

\section{Dirichlet's Unit Theorem}

\begin{remark}
  Let $K$ be a number, $\OO_K$ its ring of integers, and
  $\OO_K^\times$ the \emph{unit group}.
\end{remark}

\begin{theorem}[Dirichlet, weak]
  The unit group $\OO_K^\times$ is finitely generated. In
  particular,
  \[
    \OO_K \cong \Z^r \times (\text{finite abelian group}),
  \]
  where the finite abeliain group is the roots of unity in $K$.
\end{theorem}

\begin{example}
  Suppose $[K : \Q] = 2$. If $K$ is imaginary, then
  $K = \Q(\sqrt{-d})$, where $d > 0$ is square free. Then
  $u \in \OO_K$ is a unit if and only if $N(u) = 1$,
  Since $u$ is either $a + b \sqrt{-d}$ or $a + b \sqrt{-d} / 2$,
  One can use $N(a + b\sqrt{-d}) = a^2 + db^2 = 1$ to
  easily find the units. The conclusion is
  \[
    \OO_K^\times =
    \begin{cases}
      \{\pm 1, \pm i\}, & \text{if } d = 1, \\
      \{\pm 1, \pm \omega \pm \omega^2\}, & \text{if } d = 3, \\
      \{\pm 1\}, & \text{otherwise},
    \end{cases}
  \]
  so $\OO_K$ has rank $0$. Now consider a real
  quadratic field $K = \Q(\sqrt{d})$.
  For $\Q(\sqrt{2})$, notice that
  \[
    (\sqrt{2} - 1) (\sqrt{2} + 1) = 1.
  \]
  In particular, this means that $(\sqrt{2} - 1)^k$ is a
  unit for each $k \in \Z$, yielding a rank $1$ subgroup.
  In fact,
  \[
    \Z[\sqrt{2}]^\times
    \cong \{\pm 1\} \times \Z
    \cong \{\pm 1\} \times \langle \sqrt{2} - 1 \rangle.
  \]
  One can also view this from the perspective of Diophantine
  equations. Note that $a + b\sqrt{2} \in \Z[\sqrt{2}]$
  is a unit if and only if $a^2 - 2b^2 = \pm 1$. The
  set of solutions to this equation form a group under
  multiplication, and that group is of rank $1$.
  This is the \emph{generalized Pell equation},
  $a^2 - db^2 = \pm 1$. It is known for certain $d$ modulo
  $4$, this equation has
  infinitely many solutions, which are generated by a single
  fundamental solution. This result follows as a special case
  of Dirichlet's unit theorem.
\end{example}

\begin{remark}
  The above Pell equations are difficult to solve, even for
  relatively small values of $d$. For instance, the
  smallest solution to $a^2 - 61 b^2 = 1$ is
  $a \approx 10^6$, $b \approx 10^8$.
\end{remark}

\begin{example}
  Let $K = \Q(\zeta_m)$, where $m = p^k$. Then the unit group
  $\OO_K^\times$ contains
  \[
    \left\{
      \frac{1 - \zeta_m^a}{1 - \zeta_m} : 1 \le a \le p - 1
    \right\}.
  \]
  This generates a subgroup of rank $\phi(m) / 2 - 1 = r_1 + r_2 - 1$ (here $r_1$ is the number of real embeddings
  and $r_2$ is $1 / 2$ the number of complex embeddings).
  Note that this is compatible with the quadratic case,
  since $r_1 + r_2 - 1$ is $1$ if $K$ is real and $0$ if $K$
  is complex.
\end{example}

\begin{theorem}[Dirichlet's unit theorem]
  For a number field $K$ of degree $n$, we have
  \[
    \OO_K^\times \cong \Z^{r} \times (\text{finite abelian group})
  \]
  where the finite abelian group is the roots of unity in $K$,
  and $r = r_1 + r_2 - 1$ ($\le n - 1$, with equality if and
  only if $K / \Q$ is totally real), where $r_1, r_2$
  are the number of real and (pairs of) complex embeddings
  of $K$.
\end{theorem}

  \chapter{Feb.~13 --- Dirichlet's Unit Theorem}

\section{Dirichlet's Unit Theorem and Proof}

\begin{theorem}[Dirichlet's unit theorem]\label{thm:dirichlet-unit}
  For a number field $K$ of degree $n$, we have
  \[
    \OO_K^\times \cong \Z^{r} \times (\text{finite abelian group})
  \]
  where the finite abelian group is the roots of unity in $K$
  and $r = r_1 + r_2 - 1$ ($\le n - 1$, with equality if and
  only if $K / \Q$ is totally real), where $r_1, r_2$
  are the number of real and (pairs of) complex embeddings
  of $K$.
\end{theorem}

\begin{remark}
  The strategy is the following: Let $L : K^\times \to \R^{r_1 + r_2}$ be the
  homomorphism given by
  \[
    L(\alpha) = (\log |\sigma_1(\alpha)|, \ldots, \log |\sigma_{r_1}(\alpha)|, \log |\tau_1(\alpha)|, \ldots, \log |\tau_{r_2}(\alpha)|),
  \]
  where $\sigma_1, \dots, \sigma_{r_1} : K \hookrightarrow \R$
  are the real embeddings and $\tau_1, \dots, \tau_{r_2} : K \hookrightarrow \C$
  are half of the complex embeddings. We will want
  to show that $L(\OO_K)$ is a lattice.
\end{remark}

\begin{lemma}\label{lem:kernel}
  We have $\ker L|_{\OO_K^\times} = \mu_K$
  and $L(\OO_K^\times) \subseteq H = \{\sum x_i = 0\}$.
\end{lemma}

\begin{proof}
  We can write the kernel of $L_{\OO_K^\times}$ as
  \[
  \ker L|_{\OO_K^\times} = \left\{\alpha \in \OO_K^\times : \substack{\displaystyle\text{absolute values of the conjugates} \\ \displaystyle\text{$\alpha_1, \dots, \alpha_m$ of $\alpha$ are all $1$}}\right\}
    = \mu_K
  \]
  by Kronecker's theorem. Now if $\alpha \in \OO_K^\times$,
  then $|N^K_{\Q} (\alpha)| = 1$, so
  \[
    \left| \prod_i \sigma_i(\alpha) \cdot \prod_j \tau_j(\alpha)^2 \right| = 1,
  \]
  which implies that $\sum \log |\sigma_i(\alpha)| + \sum \log |\tau_j(\alpha)|^2 = 0$, i.e.
  $L(\alpha) \in H$.
\end{proof}

\begin{corollary}
  We have $\OO_K^\times / \mu_K \subseteq H \cong \R^{r_1 + r_2 - 1}$.
\end{corollary}

\begin{lemma}\label{lem:intersection-discrete}
  For any closed ball $B$ around $0$
  in $\R^{r_1 + r_2}$,
  $L(\OO_K^\times) \cap B$ is finite,
  i.e. $L(\OO_K^\times)$ is discrete.
\end{lemma}

\begin{proof}
  We can write the intersection via
  \[
    L(\OO_K^\times) \cap B = \{\alpha \in \OO_K^\times : \text{all conjugates of $\alpha$ have norm bounded by $C$}\}.
  \]
  For any element of the set, there are only finitely
  many possible minimal polynomials, each with only a
  finite number of roots, so the set itself must be
  finite.
\end{proof}

\begin{lemma}\label{lem:lin-alg}
  Let $A = (a_{ij})$ be an $r \times r$ real matrix
  such that:
  \begin{enumerate}
    \item the entries in each row sum to $0$,
    \item the diagonal entries $a_{ii} > 0$, and
    \item the off-diagonal entries $a_{ij} < 0$
      for $i \ne j$.
  \end{enumerate}
  Then $\rank A = r - 1$.
\end{lemma}

\begin{proof}
  It suffices to show that the first $r - 1$ columns
  $v_1, \dots, v_{r - 1}$ are linearly independent.
  Suppose otherwise that we can write
  \[
    \sum_{i = 1}^{r - 1} c_i v_i = 0,
  \]
  where the $c_i$ are not all zero.
  Without loss of generality, assume $c_k = 1$ and
  $c_j \le 1$ for all $j \ne k$. Then
  $a_{kj} < 0$ implies that $c_j a_{jk} \ge a_{jk}$, and
  $\sum_{j = 1}^{r - 1} a_{kj} > \sum_{j = 1}^r a_{kj}$
  since $k \ne r$. Then
  \[
    0 = \sum_{j = 1}^{r - 1} c_j a_{kj}
    \ge \sum_{j = 1}^{r - 1} a_{kj}
    > \sum_{j = 1}^r a_{kj} = 0,
  \]
  which is a contradiction. Since
  $(1)$ imposes one linear condition, we have
  $\rank A = r - 1$.
\end{proof}

\begin{lemma}\label{lem:smaller}
  Fix $k$ with $1 \le k \le r$. Then there is a
  constant $C$ (depending only on $K$) such that
  given $\alpha \in \OO_K \setminus \{0\}$, there
  exists $\beta \in \OO_K \setminus \{0\}$ with:
  \begin{enumerate}
    \item $|N(\beta)| \le C$, and
    \item if $L(\alpha) = (a_1, \dots, a_r)$ and
      $L(\beta) = (b_1, \dots, b_r)$, then
      $b_i < a_i$ for all $i \ne k$.
  \end{enumerate}
\end{lemma}

\begin{proof}
  We claim that taking
  \[
    C = \left(\frac{2}{\pi}\right)^{r_2} \sqrt{|\Delta_K|}
  \]
  works. For convenience, let
  \[
    \epsilon_i =
    \begin{cases}
      1 & \text{if $1 \le i \le r_1$}, \\
      2 & \text{if $r_1 + 1 \le i \le r = r_1 + r_2$}.
    \end{cases}
  \]
  Choose $a_1', \dots, a_r'$ with $a_i' < a_i$
  for each $i$. Define the region
  \[
    E = \{
      x \in \R^{r_1} \times \C^{r_2} : |x_i|^{\epsilon_i} \le C_i
    \},
  \]
  where $C_i = e^{a_i'}$ for $i \ne k$ and $\prod_i C_i = C$.
  It is clear that $E$ is symmetric, compact, convex,
  and
  \[
    \vol(E) = 2^{r_1} \pi^{r_2} \prod_i C_i
    = 2^{r_1} \pi^{r_2} C
    = 2^n \covol{i(\OO_K)}
  \]
  So by Minkowski's theorem,
  $E \cap i(\OO_K) \ne \{0\}$. Thus taking
  $p \in E \cap i(\OO_K)$ and $\beta = i^{-1}(p)$
  works.
\end{proof}

\begin{proof}[Proof of Theorem \ref{thm:dirichlet-unit}]
  Choose $\alpha_0 \in \OO_K$ arbitrarily and fix $k$. By
  Lemma \ref{lem:smaller}, we can find a sequence
  $\alpha_1, \alpha_2, \alpha_3, \dots \in \OO_K \setminus \{0\}$
  such that if $L(\alpha_j) = (a_1(j), \dots, a_r(j))$,
  then $|N(\alpha_j)| \le C$ for $j \ge 1$
  and $a_i(j) < a_i(j - 1)$ for $j \ge 1$, for
  all $i \ne k$.
  There are only a finite number of ideals of norm
  $\le C$, so there exists $j_1 > j_2$ such that
  $(\alpha_{j_1}) = (\alpha_{j_2})$. So
  $u^{(k)} = \alpha_{j_1} / \alpha_{j_2} \in \OO_K^\times$.
  If $L(u^{(k)}) = (a_1, \dots, a_r)$, then
  $a_i < 0$ for $i \ne k$, since
  $a_i = a_i(j_1) - a_i(j_2) < 0$.
  Then $a_k > 0$ since $L(u) \in H$.

  Now applying Lemma \ref{lem:lin-alg} to $u^{(1)}, \dots, u^{(k)}$, we see that
  $\OO_K^\times / \mu_K$ is full rank in
  $H \cong \R^{r_1 + r_2 - 1}$.
\end{proof}

\begin{remark}
  Next time, we will compute the class group and
  unit group of $\Q(\sqrt[3]{11})$.
\end{remark}

\section{Real Quadratic Fields and Continued Fractions}

\begin{example}
  Let $K = \Q(\sqrt{d})$, where $d > 0$ is square-free.
  We know that the unit group $\OO_K^\times$ has rank
  $1$, and (if $d \equiv 2, 3 \Pmod{4}$) the units
  correspond to the integer solutions
  to $x^2 - d y^2 = \pm 1$.

  To solve this equation, one computes the continued
  fraction expansion for $\sqrt{d}$. A \emph{continued
  fraction} is
  \[
    \alpha = a_1 + \frac{1}{a_2 + \displaystyle \frac{1}{a_3 + \dots}}
  \]
  for some positive integers $a_i$, also denoted
  $[a_1, a_2, a_3, \dots]$. The \emph{$n$th convergent}
  for $[a_1, a_2, a_3, \dots]$ is
  \[
    [a_1, a_2, \dots, a_n] =
    a_1 + \frac{1}{a_2 + \displaystyle \frac{1}{\dots + \displaystyle \frac{1}{a_n}}} = \frac{p_n}{q_n}
  \]
  The reason for the name convergent is because
  $p_n / q_n \to \alpha$ as $n \to \infty$.
\end{example}

\begin{theorem}
  Every $\alpha > 1$ has a unique continued fraction
  expansion.
\end{theorem}

\begin{theorem}
  We have the following:
  \begin{enumerate}
    \item The continued fraction expansion of $\alpha$
      is finite if and only if $\alpha \in \Q$.
    \item The continued fraction expansion of $\alpha$
      is pre-periodic if and only if
      $[\Q(\alpha) : \Q] \le 2$.\footnote{A continued fraction expansion is \emph{pre-periodic} if it is periodic after some finite prefix.}
  \end{enumerate}
\end{theorem}

\begin{theorem}
  Let $d \equiv 2, 3 \Pmod{4}$ be square-free and
  positive. Let $\varepsilon > 1$ be the
  fundamental unit of $\Z[\sqrt{d}]$, i.e.
  the unique unit which generates the unit group
  and is $> 1$. Let $k$ be the period of the
  continued fraction expansion of $\sqrt{d}$. Then
  $\varepsilon = p_k + q_k \sqrt{d}$.
\end{theorem}

\begin{example}
  Let $K = \Q(\sqrt{19})$. One can compute that
  \[
    \sqrt{19} = [4, \overline{2, 1, 3, 1, 2, 8}],
  \]
  which has period $6$. The $6$th convergents
  are $p_6 = 170$ and $q_6 = 39$. This says that
  \[
    \sqrt{19} \approx \frac{170}{39}.
  \]
  In fact, $170^2 - 19 \cdot 39^2 = 1$. This also
  means that $170 + 39 \sqrt{19}$ is the fundamental
  unit in $K$.
\end{example}

\begin{remark}
  For $\Q(\sqrt{94})$, the fundamental unit is
  $2143295 + 221064\sqrt{94}$. Furthermore, in
  $\Q(\sqrt{9199})$, the first coefficient for
  the fundamental unit has $88$ decimal digits.
\end{remark}

  \chapter{Feb.~18 --- Computing Unit Groups}

\section{Computing Unit Groups}

\begin{exercise}
  Show that the sign of $\Delta_K$ is
  $(-1)^{r_2}$.
\end{exercise}

\begin{remark}
  If $n = 3$ and $r_1 = r_2 = 1$, then
  by Dirichlet's unit theorem we know that
  $\OO_K^\times$ has rank
  $1$. The \emph{fundamental unit} is the unique
  generator $\varepsilon \in \OO_K^\times$ with
  $\varepsilon > 1$. This means that
  \[
    \OO_K^\times = \{\pm \varepsilon^k : k \in \Z\}.
  \]
  (Note that when $K$ has a real embedding, the
  only roots of unity are $\pm 1$.) We want a lower
  bound for $\varepsilon$.
\end{remark}

\begin{lemma}
  Let $K$ be a cubic number field with negative
  discriminant. Then
  \[
    \varepsilon > \sqrt[3]{\frac{|\Delta_K| - 24}{4}}.
  \]
  Equivalently, $|\Delta_K| < 4 \varepsilon^3 + 24$.
\end{lemma}

\begin{proof}
  Let $\varepsilon = \varepsilon_1$ and
  $\varepsilon_2, \varepsilon_3$ be its other two
  conjugates. Since $K$ has a pair of complex embeddings,
  we have $\varepsilon_3 = \overline{\varepsilon_2}$.
  Write $\varepsilon = u^2$ with $u > 1$ in $\R$.
  Then since $N(\varepsilon) = 1$, we have
  \[
    |\varepsilon_2|^2 = \frac{1}{\varepsilon} = u^{-2}.
  \]
  Thus $\varepsilon_2 = u^{-1} e^{i\theta}$, with
  $0 \le \theta \le \pi$ (by exchanging the two
  conjugates, if necessary). Then
  \[
    |\Delta(\varepsilon)|^{1 / 2}
    = |\Delta(1, \varepsilon, \varepsilon^2)|^{1 / 2}
    =
    \det
    \begin{bmatrix}
      1 & \varepsilon & \varepsilon^2 \\
      1 & \varepsilon_2 & \varepsilon_2^2 \\
      1 & \varepsilon_3 & \varepsilon_3^2
    \end{bmatrix}
    = 2(u^3 + u^{-3} - 2\cos \theta) \sin \theta.
  \]
  Note that $K = \Q(\varepsilon)$, so
  $|\Delta_K| \le |\Delta(\varepsilon)|$. Thus it
  suffices to bound $|\Delta(\varepsilon)|$. We claim
  that
  \[
    2(u^3 + u^{-3} - 2\cos \theta) \sin \theta
    \le (4 \varepsilon^3 + 24)^{1 / 2} = (4u^6 + 24)^{1 / 2},
  \]
  which would prove the lemma. Set $2a = u^3 + u^{-3}$,
  so that
  \[
    |\Delta(\varepsilon)|^{1 / 2}
    = 4(a - \cos \theta) \sin \theta.
  \]
  For fixed $a$, this is maximized when
  $a \cos \theta = 2 \cos^2 \theta - 1$.
  Let $x = \cos \theta$ and $g(x) = 2x^2 - ax - 1$, so
  that $\cos \theta$ is a root of $g$.
  Note that $u > 1$, so $a > 1$ and thus
  $g(1) = 1 - a < 0$. Clearly, $g(x) > 0$ for $x$
  sufficiently large, so $g$ has a root $> 1$.
  But this is not the root we want.
  We also have $g(-1 / 2u^3) < 0$ and
  $g(-1) > 0$,
  where $-1 / 2u^3 \in (-1 / 2, 0)$, so this gives
  a root $x_0 \in (-1, -1 / 2u^3) \subseteq (0, 1)$. Then
  \[
    |\Delta(\varepsilon)|^{1 / 2} \le 4(a - x_0)(1 - x_0^2)^{1 / 2},
  \]
  which gives the bound
  \[
    |\Delta(\varepsilon)| \le 16(a^2 + 1 - x_0^2 - x_0^{4})
    < 4u^6 + 24
  \]
  since $x_0 \in (-1, -1 / 2u^3)$. This proves the
  claim.
\end{proof}

\begin{remark}
  The smallest value of $|\Delta_K|$ over all cubic
  number fields $K$ is $23$.
\end{remark}

\begin{example}
  We will find the unit group for $K = \Q(\sqrt[3]{2})$.
  Let $\alpha = \sqrt[3]{2}$, which satisfies
  \[
    1 + \alpha + \alpha^2 = \frac{\alpha^3 - 1}{\alpha - 1} = \frac{1}{\alpha - 1}.
  \] 
  In particular, $u = 1 + \alpha + \alpha^2 \in \OO_K^\times$,
  and we claim that
  $\varepsilon = u$. Note that
  \[
    \Delta_K = \Delta(x^3 - 2) = -108.
  \]
  So the lemma implies that
  $\varepsilon^3 > (108 - 24) / 4 = 21$, i.e.
  $\varepsilon > \sqrt[3]{21}$. One can compute that
  \[
    1 < u < 7 < (21)^{2 / 3},
  \]
  which implies that $1 < u < \varepsilon^2$.
  Since $u = \varepsilon^m$ for $m \ge 1$ (as $u$
  is positive), we must have $m = 1$.
\end{example}

\begin{example}
  Let $K = \Q(\sqrt[3]{11})$. We will find
  $\OO_K^\times$ and $\Cl(\OO_K)$. Note that the
  calculus argument (lower bound for $\varepsilon$)
  will not work here: It only tells us that a certain
  unit $u$ is either $\varepsilon$ or $\varepsilon^2$.
  Note that
  \[11^2 \not\equiv 1 \pmod{9},\]
  so $\OO_K = \Z[\sqrt[3]{11}]$. Also note that
  $\Delta_K = -3 \cdot 11^2$, so Minkowski's constant is
  \[
    M_K = \frac{3!}{3^3} \left(\frac{4}{\pi}\right) \sqrt{3^3 \cdot 11^2}
    < 17.
  \]
  Thus we want to factor $2, 3, 5, 7, 11, 13$ in
  $\OO_K$. We can factor $x^3 - 11 \Pmod{p}$ by:
  \begin{center}
    \begin{tabular}{c|c}
      $p$ & $x^3 - 11 \Pmod{p}$ \\
      \hline
      $2$ & $(x - 1)(x^2 + x + 1)$ \\
      $3$ & $(x + 1)^3$ \\
      $5$ & $(x - 1)(x^2 + x + 1)$ \\
      $7$ & $x^3 - 4$ \\
      $11$ & $x^3$ \\
      $13$ & $x^3 - 2$
    \end{tabular}
  \end{center}
  Thus by Kummer's theorem, we have
  $(2) = \p_2 \p_2'$ with $N(\p_2) = 2$
  and $N(\p_2') = 4$, $(3) = \p_3^3$,
  $(5) = \p_5 \p_5'$ with $N(\p_5) = 5$, and
  $(11) = \p_{11}^3$ where $\p_{11} = (\alpha)$
  for $\alpha = \sqrt[3]{11}$.
  Thus the class group is generated via
  \[
    \Cl(\OO_K) = \langle [\p_2], [\p_3], [\p_5] \rangle.
  \]
  Now we want to find some elements of $\OO_K$ with
  small norm.

  Note that $\alpha$ has minimal polynomial
  $x^3 - 11$, so $\alpha - t$ has minimal polynomial
  $(x - t)^3 - 11$
  for $t \in \Z$, so $N(\alpha - t) = t^3 - 11$.
  For $t = 1$,
  \[
    N(\alpha - 1) = -10,
  \]
  so $(\alpha - 1)$ has norm $10$. Thus
  $(\alpha - 1) = \p_2 \p_5$, which allows us to remove
  $\p_5$ as a generator. For $t = 2$,
  \[
    N(\alpha - 2) = -3,
  \]
  so $(\alpha - 2) = \p_3$ and we can remove
  $\p_3$ as a generator. So
  $\Cl(\OO_K) = \langle [\p_2] \rangle$, and it suffices
  to find the order of $\p_2$. Set $t = -1$, so
  \[
    N(\alpha + 1) = -12.
  \]
  Thus $(\alpha + 1) = \p_3 \p_2^2$ or
  $\p_3 \p_2'$. But $\p_2 = (2, \alpha - 1)$
  contains $\alpha + 1$, so $\p_2$ divides
  $(\alpha + 1)$. Thus
  \[
    (\alpha + 1) = \p_3 \p_2^2,
  \]
  so $\p_2^2$ is principal. So $\p_2$ has order
  dividing $2$, and $\Cl(\OO_K) \cong \Z / 2\Z$
  or $\{1\}$.

  To see which one it is, we will need some
  information about the unit group. We begin
  by finding some nontrivial unit.
  To do this, note that $\p_3 = (\alpha - 2)$ and
  $\p_3 \p_2^2 = (\alpha + 1)$. Then $\p_2^2 = (\beta)$
  for
  \[
    \beta = \frac{\alpha + 1}{\alpha - 2} = \alpha^2 + 2\alpha + 5.
  \]
  Using $t = 3$ from before, we have
  $N(\alpha - 3) = 16$, so $(\alpha - 3) = \p_2^4$ or
  $\p_2^2 \p_2'$ since $\alpha - 3 \in \p_2$.
  But $\p_2 \p_2' = (2)$ and $2$ does not divide
  $\alpha - 3$, so we must have $(\alpha - 3) = \p_2^4$.
  Then $\p_2^4 = (\beta^2)$, so
  \[
    u = -\frac{\beta^2}{\alpha - 3} \approx 266.99 > 1
  \]
  is a unit. The lower bound gives
  $\varepsilon > 9.34$, so $\varepsilon^3 > u$.
  Thus either $u = \varepsilon$ or $\varepsilon^2$.

  The new idea from this point is the following:
  We will construct a homomorphism
  $\Z[\alpha] \to \F_p$ for suitable $p$, such that
  the image of $u$ is not a square (this will imply
  that $u$ itself cannot be a square). Try
  \[
    \p_5 = (5, \alpha - 1)
  \]
  with norm $5$, so reduction
  mod $\p_5$ gives a homomorphism
  $\Z[\alpha] \to \F_5 = \Z[\alpha] / \p_5$ which
  maps $\alpha \mapsto 1$. Using
  $u = -\beta^2 / (\alpha - 3) = -(\alpha^2 + 2\alpha + 5) / (\alpha - 3)$,
  we have $u \mapsto 2$, which is not a
  square.
  So $\varepsilon = u$.

  Now we claim that $\p_2$ is not principal.
  If it were, then $\p_2 = (\gamma)$ and
  \[
    (\beta) = \p_2^2 = (\gamma^2).
  \]
  Then for $v = u^{-1} = -2\alpha^2 + 4\alpha + 1$,
  we can write $\pm v^m \beta = \gamma^2$ for some
  $m$. Without loss of generality, we can assume
  $m = 0$ or $1$ (by absorbing powers of $v$ into
  $\gamma$). So one of $\beta$, $-\beta$, $v\beta$, or
  $-v\beta$ is a square in $\OO_K$.
  As before, we can find homomorphisms $\Z[\alpha] \to \F_p$ for various
  $p$ such that each of these elements map to
  non-squares, which
  will give a contradiction. Note that $19$ splits
  completely in $K$, so we get three homomorphisms
  $\Z[\alpha] \to \F_{19}$, with $\alpha \mapsto \{5, -3, -2\}$.
  Choose the one which maps $\alpha \mapsto 5$.
  This one sends $\beta \to 2$ and $v\beta \to -1$,
  which are non-squares. Choosing the one
  with $\alpha \mapsto -2$, we find
  $-\beta \mapsto -5$ and $-v\beta \mapsto -1$,
  which are non-squares mod $19$. Thus
  $\p_2$ is not principal, so $\p_2$ has order $2$.

  This shows that $\Cl(\OO_K) \cong \Z / 2\Z$
  and $\OO_K^\times = \{\pm \langle v \rangle\}$.
\end{example}

  \chapter{Feb.~20 --- Localization}

\section{Motivation for Localization}

\begin{remark}
So far we have been dealing with the following
scenario of a number field $K$:
\begin{center}
  \begin{tikzcd}
    K \ar[dash, r] \ar[dash, d, "n"] & \OO_K \ar[dash, d, "n"] \\
    \Q \ar[dash, r] & \Z
  \end{tikzcd}
\end{center}
We would now like to consider relative extensions,
for a finite extension $L / K$:
\begin{center}
  \begin{tikzcd}
    L \ar[dash, r] \ar[dash, d, "n"] & \OO_L \ar[dash, d, "n"] \\
    K \ar[dash, r] & \OO_K
  \end{tikzcd}
\end{center}
For instance, that if prime ideals $\q_1, \dots, \q_r \subseteq \OO_L$ lie
over $\p \subseteq \OO_K$, then $[L : K] = \sum_{i = 1}^r e_i f_i$.
One of our goals will be to show that a Noetherian
domain $R$ is Dedekind if and only if $R_\p$ is a
PID for every prime ideal $\p$.
The second goal will be to prove
``Dirichlet's $S$-unit theorem.''
\end{remark}

\section{Localization}

\begin{definition}
  Let $R$ be a domain and $K$ its field of fractions.
  Let $S$ be a \emph{multiplicative subset} of $R$,
  i.e. $0 \notin S$, $1 \in S$, and
  $a, b \in S$ implies $ab \in S$. Then the
  \emph{localization of $R$ at $S$} is the subring
  \[
    S^{-1} R = \left\{ \frac{a}{b} : a \in R, b \in S \right\} \subseteq K.
  \]
\end{definition}

\begin{remark}
  The equivalence relation of $a / b \sim c / d$
  if $ad = bc$ is not necessary anymore since
  $S^{-1} R \subseteq K$.
\end{remark}

\begin{remark}
  Note that $R \subseteq S^{-1} R \subseteq K$. The
  idea is that $S^{-1} R$ will have some of the nice
  properties of $K$ while still retaining enough
  of the information of $R$.
\end{remark}

\begin{example}
  For $S = R \setminus \{0\}$, we have
  $S^{-1} R = K$, and for $S = \{1\}$, we have
  $S^{-1} R = R$.
\end{example}

\begin{example}
  If $\p \subseteq R$ is a prime ideal, then
  $S = R \setminus \p$ is a multiplicative subset.
  Thus we can define $R_\p = S^{-1} R$, which we
  will call the \emph{localization of $R$ at $\p$}.
\end{example}

\begin{remark}
  One can definition localization in a more general
  context, e.g. for a ring which is not a domain.
  This is necessary for algebraic geometry, but we do
  not need this, so we will avoid it.
\end{remark}

\begin{prop}
  The prime ideals of $S^{-1} R$ are in
  (inclusion-preserving) bijection with
  the prime ideals of $R$ disjoint from $S$.
\end{prop}

\begin{proof}
  Denote by $\Spec R$ the set of prime ideals of $R$.
  We will show the following bijection:
  \begin{align*}
    \{\q \in \Spec(R) : \q \cap S = \varnothing\}
    &\longleftrightarrow \Spec(S^{-1} R) = \Spec(R') \\
    \q &\longmapsto S^{-1} \q = \left\{\frac{a}{b} : a \in \q, b \in S\right\} \\
    \q' \cap R &\mathrel{\reflectbox{\ensuremath{\longmapsto}}} \q'
  \end{align*}
  First we claim that $S^{-1} \q$ is a prime ideal in
  $R'$. Note that $\q \cap S = \varnothing$ is
  equivalent to $1 \notin S^{-1} \q$. Check as an
  exercise that $S^{-1} \q$ is in fact an ideal
  in $R'$. To see that it is prime, suppose that
  $(a_1 / b_1)(a_2 / b_2) \in S^{-1} \q$, where
  $a_1, a_2 \in R$ and $b_1, b_2 \in S$. Then
  we can see that
  \[
    \frac{a_1 a_2}{b_1 b_2} = \frac{a}{b}, \quad a \in \q, b \in S,
  \]
  so $a_1 a_2 b = a b_1 b_2 \in \q$. As
  $b \in S$ and $q \cap S = \varnothing$, we have
  $b \notin \q$. Thus $a_1 \in \q$ or $a_2 \in \q$, i.e.
  $S^{-1} \q$ is prime.

  The other direction is easier and is mostly
  left as an exercise. To see that
  $(\q' \cap R) \cap S = \varnothing$, suppose that
  $s \in S \cap \q'$. Then $s \cdot (1 / s) = 1 \in \q'$,
  where $s \in \q'$ and $(1 / s) \in R$, which is
  impossible since $\q' \ne R$.

  It only remains to show that these maps are
  inverses of each other, which is left as an exercise.
\end{proof}

\begin{corollary}
  The prime ideals of $R_\p$ are in bijection
  with the prime ideals of $R$ contained in $\p$.
\end{corollary}

\begin{corollary}
  The localization $R_\p$ is a local ring,
  i.e. it has a unique maximal ideal.
\end{corollary}

\begin{example}
  We can write the localization $\Z_{(2)}$ as
  \[
    \Z_{(2)} = \left\{ \frac{a}{b} \in \Q : \text{$b$ is odd} \right\}.
  \]
  The unique maximal ideal is $2 \Z_{(2)} = \{2 a / b : \text{$b$ is odd}\}$, and
  $\Z_{(2)} \setminus 2\Z_{(2)} = \Z_{(2)}^\times$. We
  also have
  \[
    \Z_{(2)} / 2\Z_{(2)} = \Z / 2\Z.
  \]
  In general, we will see that
  $\OO_K / \p^m \OO_K \cong (\OO_K)_\p / \p^m (\OO_K)_\p$,
  so we can study the localization instead.
\end{example}

\begin{lemma}
  Let $R$ be a ring and $\m$ a maximal ideal in $R$.
  If $s \notin \m$, then $\m^n + (s) = (1)$ for all
  $n \ge 1$.
  Equivalently, $\overline{s}$ is a unit in
  $R / \m^n$ for every $n \ge 1$.
\end{lemma}

\begin{proof}
  We induct on $n$. The case $n = 1$ is clear since
  $\m$ is maximal. Now suppose
  $(1) = \m^{n - 1} + (s)$, so
  \[
    \m = \m^n + s \m \subsetneq \m^n + (s)
  \]
  since $s \notin \m$. But $\m$ is a maximal ideal, so
  we must have $\m^n + (s) = (1)$.
\end{proof}

\begin{remark}
  We will denote $\m_\p = \p R_\p$, the unique
  maximal ideal in the localization $R_\p$.
\end{remark}

\begin{lemma}
  Let $R$ be an integral domain and $\p$ a maximal
  ideal. Then for all $n \ge 1$, the natural map
  \[
    \phi : R / \p^n \to R_\p / \m_\p^n
  \]
  is an isomorphism. (In particular,
  $R / \p \cong R_\p / \m_\p$.)
\end{lemma}

\begin{proof}
  The natural map is the map induced by the
  inclusion $R \hookrightarrow R_\p$ on the quotient
  $R / \p^n$.

  First we show that $\phi$ is injective.
  Suppose $x \in R \cap \m_\p^n$, and we will show
  that $x \in \p^n$. Write
  \[
    x = \frac{y}{s}, \quad y \in \p^n, s \notin \p.
  \]
  By the lemma, we have
  $\overline{s} \in (R / \p^n)^\times$. But
  $sx = y \in \p^n$, which implies that $\overline{s} \cdot \overline{x} = 0$
  in $R / \p^n$. Since $\overline{s}$ is a unit,
  this means that $\overline{x} = 0$, so $x \in \p^n$.
  This shows that $\phi$ is injective.

  Now we show that $\phi$ is surjective.
  Let $r / s \in R_\p$ with $r \in R$ and
  $s \notin \p$. By the lemma, there exists $r' \in R$
  such that $r \equiv r' s \Pmod{\p^n}$ (e.g.
  take $r' = s^{-1} r \in R / \p^n$). Then
  $r / s \equiv r' \Pmod{\m_\p^n}$, so we have
  \[
    \phi(r') = \frac{r}{s} + \m_\p^n.
  \]
  Since $r / s \in R_\p$ was arbitrary, this shows
  surjectivity. Thus $\phi$ is an isomorphism.
\end{proof}

\section{Dedekind Domains and Localization}

\begin{lemma}
  If $R$ is a Noetherian domain, then so is
  $S^{-1} R$.
\end{lemma}

\begin{proof}
  The key observation is that every ideal of
  $S^{-1} R$ is of the form $S^{-1} I$ for some ideal
  $I$ of $R$ (check this as an exercise).
  So if $I$ is generated by $x_1, \dots, x_r$,
  then $S^{-1} I$ is generated by $x_1 / 1, \dots, x_r / 1$.
\end{proof}

\begin{lemma}
  If $R$ is integrally closed, then $S^{-1} R$ is also
  integrally closed.
\end{lemma}

\begin{proof}
  Suppose $\alpha = a / b \in K$ satisfies
  a monic polynomial $f \in (S^{-1} R)[x]$. Then we need
  to show that $\alpha \in S^{-1} R$. Write
  \[
    f(x) = x^n + a_{n - 1} x^{n - 1} + \dots + a_0,
    \quad a_i = \frac{r_i}{s_i} \in S^{-1} R.
  \]
  Let $s = s_0 s_1 \dots s_{n - 1}$ and multiply the
  equation $f(\alpha) = 0$ by $s^n$ to get
  \[
    (s\alpha)^n + a_{n - 1} s (s \alpha)^{n - 1}
    + \dots + a_1 s^{n - 1} (s\alpha) + a_0 s^n = 0.
  \]
  The coefficients $a_{n - 1} s, \dots, a_0 s^n$
  are in $R$, so $s \alpha$ is integral over $R$.
  But $R$ is integrally closed, so $s \alpha \in R$.
  Since $s \in S$, we have $\alpha \in S^{-1} R$, which
  proves that $S^{-1} R$ is integrally closed.
\end{proof}

\begin{prop}
  If $R$ is a Dedekind domain and $S$ is a multiplicative
  set, then $S^{-1} R$ is either a Dedekind domain or a field.
\end{prop}

\begin{proof}
  From the above lemmas, it suffices to show
  $\dim(S^{-1} R) \le 1$. There is an bijection from
  chains of prime ideals in $S^{-1} R$ to
  chains of prime ideals in $R$ disjoint from $S$, so
  $\dim(S^{-1} R) \le \dim R = 1$.
\end{proof}

\begin{corollary}
  If $R$ is Dedekind, then $R_\p$ is a PID
  for every nonzero prime ideal $\p$.
\end{corollary}

\begin{proof}
  Note that $R_\p$ is a local Dedekind ring with
  prime ideals $(0)$ and $\m_\p$. By unique
  factorization, every nonzero ideal of $R_\p$ is
  of the form $\m_\p^k$. So it suffices to show that
  $\m_\p$ is principal.

  To do this, choose any $\pi \in \p \setminus \p^2$,
  and we claim that $\m_\p = \pi R_\p$.
  Since $\pi R_\p$ is an ideal, we have
  $\pi R_\p = \m_\p^k$ for some $k \ge 1$.
  If $k \ge 2$, then $\pi \in \m_\p^2 \cap R = (\m_\p \cap R)^2 = \p^2$,
  a contradiction. So $k = 1$.
\end{proof}

\begin{definition}
  A local PID is called a \emph{discrete valuation ring (DVR)}.
\end{definition}

  \chapter{Feb.~25 --- Localization, Part 2}

\section{Valuations}

\begin{example}
  Recall that $R_\p$ is a local PID, also known
  as a discrete valuation ring. Consider
  \[
    \Z_{(p)} = \left\{
      \frac{a}{b} \in \Q : p \nmid b
    \right\} \subseteq \Q.
  \]
  There is a \emph{$p$-adic valuation}
  $v_p : \Z_{(p)} \to \N \cup \{\infty\}$ given by
  $v_p(x) = k$ if $x = p^k \cdot a / b$ where
  $a, b \in \Z$, $p \nmid a$, $p \nmid b$, and
  $v_p(0) = \infty$.
  Now $v_p$ extends to $v_p : \Q \to \Z \cup \{\infty\}$,
  and we can recover $\Z_{(p)}$ as
  \[
    \Z_{(p)} = \{x \in \Q : v_p(x) \ge 0\}.
  \]
  This is known as the \emph{valuation ring} associated
  to $v_p$. This is discrete since
  $\Z \cup \{\infty\}$ is discrete.
\end{example}

\begin{definition}
  A \emph{(real) valuation} on a field $k$ is a function
  $v : k \to \R \cup \{\infty\}$ such that
  \begin{itemize}
    \item $v(x) = \infty$ if and only if $x = 0$;
    \item $v(ab) = v(a) + v(b)$;
    \item $v(a + b) \ge \min\{v(a), v(b)\}$.
  \end{itemize}
\end{definition}
Given such a valuation $v$, there
\emph{valuation ring associated to $v$} is
$\OO_v = \{x \in k : v(x) \ge 0\}$.

\begin{lemma}
  For all $x \in k$, either $x \in \OO_v$ or
  $x^{-1} \in \OO_v$.
\end{lemma}

\begin{definition}
  A \emph{valuation ring} is a subring $\OO$
  of a field $k$ such that for all $x \in k$,
  $x \in \OO$ or $x^{-1} \in \OO$.
\end{definition}

\section{Dedekind Domains and Localization, Continued}

\begin{lemma}\label{lem:int-int-closed}
  If $R$ is an integral domain, then
  \[
    R = \bigcap_{\p \text{ prime}} R_\p
    = \bigcap_{\m \text{ maximal}} R_\m.
  \]
\end{lemma}

\begin{proof}
  Since every maximal ideal is prime, we prove
  the statement for maximal ideals. We need to show that
  $\bigcap_\p R_\p \subseteq R$. Choose
  $a / b \in \bigcap_\p R_\p$. Define the ideal
  \[
    I = \{y \in R : ay \in bR\}
  \]
  We will show that $I = R$, which implies that
  $a \in b R$ and thus $a / b \in R$ since
  $1 \in I$. Check as an exercise that $I$ is indeed
  an ideal. Now since $a / b \in R_\p$, we can
  write $a / b = x / y$ with $x, y \in R$ and
  $y \notin \p$. Then $ay = bx$, so $y \in I$.
  Since $y \notin \p$, this means that $I \nsubseteq \p$.
  We can do this for every maximal ideal $\p$. But every
  $I \ne R$ is contained in a maximal ideal, so 
  we must have $I = R$.
\end{proof}

\begin{theorem}
  If $R$ is a Noetherian integral domain, then $R$ is
  Dedekind if and only if $R_\p$ is a PID for
  every nonzero prime ideal $\p$ of $R$.
\end{theorem}

\begin{proof}
  $(\Rightarrow)$ This was Corollary \ref{cor:local-pid}.

  $(\Leftarrow)$ The tricky part to
  show is that $R$ is integrally closed. This part
  follows from Lemma \ref{lem:int-int-closed} since the
  intersection
  of integrally closed subrings is again
  integrally closed.
\end{proof}

\section{\texorpdfstring{$S$}{S}-Integers and \texorpdfstring{$S$}{S}-Units}

\begin{definition}
  Let $S$ be a finite set of nonzero prime ideals in
  a domain $R$ and $K = \Frac R$. Define
  \[
    R^S = \left\{
      \frac{x}{y} \in K : x, y \in R,\, y \notin \p \text{ for all } \p \notin S
    \right\}
    = T^{-1} R,
  \]
  where $T = \{x \in R : x \notin \bigcup_{\p \notin S} \p\}$.
\end{definition}

\begin{remark}
  If $R$ is Dedekind, this means $(y)$ is divisible
  only by primes in $S$.
\end{remark}

\begin{example}
  Let $S = \{(2), (3)\} \subseteq \Spec \Z$. Then
  \[
    \Z^S = \left\{
      \frac{x}{y} \in \Q : x, y \in \Z,\, y = \pm 2^a 3^b
    \right\}.
  \]
  In some sense, $\Z_{(2)}$ is close to $\Q$ while
  $\Z^{(2)}$ is close to $\Z$. Note that
  $(\Z^{(2)})^\times = \pm \{2^k\}$, which is of
  rank $1$.
\end{example}

\begin{remark}
  The following are some facts and definitions
  regarding $R^S$:
  \begin{itemize}
    \item $R^S$ is Dedekind.
    \item $\Cl(R^S)$ is finite.
    \item $(R^S)^\times$ is finitely generated of
      rank $|S| + r_1 + r_2 - 1$.
  \end{itemize}
\end{remark}

\begin{exercise}
  Let $K$ be a number field, $R = \OO_K$, and $S$ a
  finite set of nonzero prime ideals. Show that
  \begin{center}
    \begin{tikzcd}
      1 \ar[r] & R^\times \ar[r] & (R^S)^\times \ar[r] & \bigoplus_{\p \in S} (K^\times / R_\p^\times) \ar[r] & \Cl(R) \ar[r] & \Cl(R^S) \ar[r] & 0
    \end{tikzcd}
  \end{center}
  is an exact sequence, where
  $K^\times / R_\p^\times \cong \Z$.
\end{exercise}

\begin{remark}
  Note that the above shows that $\Cl(R^S)$
  is in fact a quotient of $\Cl(R)$, hence it must be
  finite.
\end{remark}

\begin{prop}
  If $K$ is a number field, then there exists a finite
  set $S$ of nonzero prime ideals such that
  $\OO_K^S$ is a PID (equivalently, $\Cl(\OO_K^S) = \{0\}$).
\end{prop}

\begin{proof}
  There is a map $\rho : \Cl(R) \to \Cl(R^S)$ given by
  $[I] \mapsto [I R^S]$, which one can verify as
  an exercise is surjective. Let $I_1, \dots, I_t$ be
  ideals which generate $\Cl(R)$. Define
  \[
    S = \text{all prime ideals of $R$ dividing some $I_k$}.
  \]
  For $\p \in S$, we have $\p R^S = (1)$. So
  $\rho([I_k]) = 0$ for all $k$. Thus we must have
  $\Cl(R^S) = 0$.
\end{proof}

\begin{remark}
  The above proposition says that we may get a PID
  $\OO_K^S$ in place of $\OO_K$ (which is often not a
  PID), at the cost of increasing the rank of the unit
  group by $|S|$.
\end{remark}

\section{Applications to Elliptic Curves}

\begin{theorem}[Siegel]
  Let $K$ be a number field and $S$ a finite set
  of nonzero primes in $\OO_K$. Let $f(x) \in K[x]$
  be a separable polynomial of degree $\ge 3$.
  Then the Diophantine equation
  \[
    Y^2 = f(X)
  \]
  has only finitely many solutions with
  $X, Y \in \OO_K^S$. (As a special case, this holds when
  $S = \varnothing$, so that $\OO_K^S = \OO_K$. In particular, if $K = \Q$ and $S = \varnothing$, then $\OO_K^S = \Z$. The case $\deg f = 3$ is an elliptic curve.)
\end{theorem}

\begin{proof}
  We will use the following result without proof:
  \begin{quote}
    \textbf{Theorem} (Siegel, Mahler)\textbf{.}
    Let $K, S$ as before. Then the equation $x + y = 1$
    has only finitely many solutions with $x, y \in (\OO_K^S)^\times$.\footnote{For instance, $x + y = 1$ has finitely solutions with $x, y \in \pm \{2^a 3^b\}$. This is not obvious, e.g. $1 = 3^2 - 2^3 = 3 - 2 = 2^2 - 3 = \dots$.}
  \end{quote}
  Without loss of generality, assume that $f$ splits over
  $K$ (by enlarging $K$ if necessary), so
  \[
    f(X) = a(X - \alpha_1) \dots (X - \alpha_n), \quad a, \alpha_j \in K
  \]
  for distinct $\alpha_1, \dots, \alpha_n$ and
  $n \ge 3$. Let $K^S = (\OO_K^S)^\times$.
  By enlarging $S$ if necessary, we
  can assume:
  \begin{enumerate}
    \item $a \in K^S$, $\alpha_1, \dots, \alpha_n \in \OO_K^S$;
    \item $\alpha_i - \alpha_j \in K^S$ for all $i \ne j$;
    \item $\OO_K^S$ is a PID.
  \end{enumerate}
  By Dirichlet's $S$-unit theorem, $K^S$ is
  finitely generated, and so $K^S / (K^S)^2$ is finite.
  So if
  \[
    L = \text{compositum of $K(\sqrt{u})$ for all $u \in K^S$},
  \]
  then $L / K$ is finite. Now define the set
  \[
    T = \text{prime ideals of $\OO_L$ containing some element of $S$}.
  \]
  We will work with $\OO_L^T$, the
  $T$-integers in $\OO_L$. Now suppose
  $y^2 = f(x)$ with $x, y \in \OO_K^S$, so that
  \[
    y^2 = a(x - \alpha_1) \dots (x - \alpha_n).
  \]
  Then any prime ideal $\p$ of $\OO_K^S$ divides
  at most one of these terms (otherwise it divides
  $\alpha_i - \alpha_j$ for $i \ne j$, which we assumed
  lies in $K^S$). Since $\p$ must divides $y^2$, it
  must divide $y^2$ twice, and since
  \[
    (x - \alpha_1) \dots (x - \alpha_n) = a^{-1} y^2,
  \]
  we must have $x - \alpha_i = u_i a_i^2$
  for each $i$, where $u_i \in K^S$ and
  $a_i \in \OO_K^S$. Then
  $(x - \alpha_i) = \mathfrak{a}_i^2$, where $\mathfrak{a}_i = (a_i)$.
  Let $u_i = v_i^2$ with $v_i \in L^T$, then
  \[
    x - \alpha_i = v_i^2 a_i^2 = w_i^2, \quad w_i \in \OO_L^T.
  \]
  The trick is then the following: We can write
  \[
    \alpha_j - \alpha_i = w_i^2 - w_j^2
    = (w_i - w_j)(w_i + w_j).
  \]
  Since $\alpha_j - \alpha_i \in K^S$, we must have
  $w_i - w_j, w_i + w_j \in L^T$. Because
  $n \ge 3$, we have \emph{Siegel's identities}:
  \[
    \frac{w_1 - w_2}{w_1 - w_3} + \frac{w_2 - w_3}{w_1 - w_3} = 1 \quad \text{and} \quad
    \frac{w_1 + w_2}{w_1 - w_3} - \frac{w_2 + w_3}{w_1 - w_3} = 1.
  \]
  All of the above quotients lie in $L^T$, so there
  are only finitely many choices for the quotients
  by the theorem of Siegel and Mahler.
  One can show that there are then only finitely
  many choices for $w_1$, hence there can only be
  finitely many choices for $x = \alpha_1 + w_1^2$,
  since $\alpha_1$ is fixed.
\end{proof}

\end{document}
